% !TEX root = paper.tex

Die Edition unseres WunderPasses soll als Property auf die anfangs geforderte Möglichkeit einer gewissen Individualisierung des WunderPasses durch seinen Besitzer einzahlen. Zu individuell darf eine solche NFT-Property aber auch nicht sein, da der NFT zwingend seinen Eigentümer wechseln können soll, da das ganze Unterfangen mit dem NFT-Pass andernfalls ad absurdum führte.

Um die Edition-Property noch etwas interessanter zu gestalten, sollen Exemplare jeder Edition nicht endlos verfügbar sein, sondern stattdessen irgendwann einmal \textit{aufgebraucht}. In solch einem Fall soll sich der User aber nicht einfach irgendeiner anderer Edition bedienen, sondern erhält die \textit{"Oberedition"} (Parent) seiner ursprünglich gewünschten Edition. Und sollte auch diese \textit{aufgebraucht} sein, dann wiederum die \textit{"Oberedition"} der \textit{"Oberedition"} usw. 

\vspace{0.2cm}

\begin{NFT-Prop}[Edition]

Als Ausprägung der WunderPass-NFT-\textbf{Edition} haben wir uns für \textbf{Städte} der Welt entschieden. Die \textbf{Parent-Edition} einer Stadt ist das dazugehörige \textbf{Land}, deren 
Parent-Edition wiederum der entsprechende \textbf{Kontinent} und als \textbf{oberste Editions-Ebene} dann die \textbf{Welt-Edition}. Letztere unterliegt folglich dann auch keiner stückweisen Obergrenze mehr.

\vspace{0.2cm}

\underline{\textbf{\textit{Beispiel einer Edition-Kette:}}}

\vspace{0.2cm}

\begin{equation*}
\textrm{Berlin } \rightarrow \textrm{ Germany } \rightarrow \textrm{ Europe } \rightarrow \textrm{ World }
\end{equation*} 

\vspace{0.2cm}

Es gilt das folgende grobe Regel-Set, was jedoch explizit, auch nach Launch, modifizierbar bleiben soll:

\begin{itemize}
    \item Die möglichen Editionen werden von uns bestimmt. Diese müssen nicht zwingend beim Launch des NFT vollständig benannt werden, sondern können stattdessen auch nachträglich eingepflegt werden. User-Wünsche (in welcher Form auch immer) sind dabei explizit erwünscht.
    \item Jede berücksichtigte \textit{Städte-Edition} ist genau \textbf{100} Mal verfügbar. Sind alle 100 Exemplare einer \textit{Städte-Edition} bereits gemintet (verbraucht), erhält die nächste Mint-Anfrage nach einem WunderPass derselben Edition automatisch die zu dieser Städte-Edition gehörende \textit{Landes-Edition}.
    \item Die \textit{Landes-Editionen} sind in einer maximalen Stückzahl von je \textbf{10.000} pro berücksichtigtem Land verfügbar. Sind auch diese aufgebraucht, wird die durch den User ausgewählte Stadt auf die ihrem Land übergeordnete \textit{Kontinent-Edition} gemappt.
    \item Die \textit{Kontinent-Editionen} sind in einer maximalen Stückzahl von je \textbf{1.000.000} für jeden Kontinent (außer der Antarktis) vorgesehen. Sollte sich auch diese Menge irgendwann erschöpfen, greifen wir zu der übergeordneten \textit{Welt-Edition}.
\end{itemize} 

\end{NFT-Prop}

\vspace{0.4cm}

\underline{\textbf{Quantitative Daten zu den Editionen:}}

\begin{itemize}
    \item Nach aktuellem Stand sind mindestens 693 \textit{Städte-Edition} vorgesehen.
    \item Die genannten \textit{Städte-Editionen} verteilen sich dabei aktuell auf 179 \textit{Landes-Editionen}.
    \item Die unterschiedlichen \textit{Kontinent-Editionen} belaufen sich auf 6 (Nord- und Südamerika, Europa, Afrika, Asien und Australien).
    \item Die übergeordnete \textit{Welt-Edition} ist in ihrer Stückzahl unbegrenzt. 
    \item Die Auswahl der angebotenen \textit{Städte-Editionen} folgt (mit Augenmaß) in etwa folgender Logik:
    \begin{itemize}
    	\item Die Hauptstadt eines jeden mit einer \textit{Landes-Edition} versehenen Landes ist gleichzeitig auch eine verfügbare \textit{Städte-Edition}.
    	\item Mit Ausnahme der Hauptstädte erfordert die Größe einer Stadt (nach Einwohnern) ein Mindestmaß $m_1$, um als \textit{Städte-Edition} aufgenommen zu werden.
    	\item Sofern es das vorige Kriterium hergibt, sollen nach Möglichkeit für jedes Land mit einer eigenen \textit{Landes-Edition} mindestens dessen 5 größten Städte mit einer eigenen \textit{Städte-Edition} versehen werden.
    	\item Überschreiten die $n$ größten Städte eines in die \textit{Landes-Editionen} aufgenommenen Landes eine bestimmte Mindestgröße $m_2$ (nach Einwohnern), werden alle $n$ Städte in die verfügbaren \textit{Städte-Editionen} aufgenommen. Dieses Kriterium wird aufgrund des vorigen ausschließlich für $n > 5$ relevant.
    	\item Städte der G7-Länder werden (ungeachtet etwaiger Mindestgröße) vermehrt in die \textit{Städte-Editionen} aufgenommen (bis zu 25 \textit{Städte-Editionen} pro G7-Land).
    \end{itemize} 
    \item Einzelne Städte können bei Bedarf auch bei Missachtung aller vorigen Kriterien aufgenommen werden.
\end{itemize}

\vspace{0.5cm}




