% !TEX root = paper.tex

\subsubsection{Beispielhafte Analyse der Collection}

\vspace{0.2cm}

Um ein besseres Gefühl über die formulierte Logik unseres NFTs zu bekommen, wollen wir ein Beispiel mit konkreten Zahlen rechnen und begeben uns dazu eine gute Weile in die Zukunft - zu einem Zeitpunkt, zu dem bereits genau 316.157 NFT-Pässe gemintet wurden. Ich als potenzieller Interessent an einem Pass-NFT möchte verstehen, welchen Pass ich als nächsten in etwa zu erwarten hätte.

Wir analysieren die 316.157 bereits geminteten Pässe.

\vspace{0.2cm} 

\underline{\textbf{Status:}}

\begin{itemize}
  \item Es wurden 200 Pässe des Status \textit{diamond} gemintet.
  \item Es wurden 1.600 Pässe des Status \textit{black} gemintet.
  \item Es wurden 12.800 Pässe des Status \textit{pearl} gemintet.
  \item Es wurden 102.400 Pässe des Status \textit{platin} gemintet.
  \item Es wurden 199.157 Pässe des Status \textit{ruby} gemintet.
  \item Von den insgesamt 819.200 vorgesehenen \textit{ruby} Pässen sind demnach noch 620.043 noch verfügbar.
\end{itemize}

\vspace{0.2cm}

\textit{\textbf{Unser Pass wird also definitiv den Status 'Ruby' haben!}}

\vspace{0.3cm}


\underline{\textbf{Hologramm:}}

\vspace{0.2cm}

Hinsichtlich der Hologramme können wir nur über die ersten 315.904 der 316.157 bisher geminteten Pässe eine definitive Aussage treffen. Die übrigen 253 folgen einer gewissen Wahrscheinlichkeitsverteilung. Zunächst zu den ersten 315.904:

\begin{itemize}
  \item Es wurden 159.186 Pässe mit dem Hologramm der \textit{Jesus-Statue} gemintet.
  \item Es wurden 78.976 Pässe mit dem Hologramm des \textit{Maj Mahal} gemintet.
  \item Es wurden 39.488 Pässe mit dem Hologramm des \textit{Machu Picchu} gemintet.
  \item Es wurden 19.744 Pässe mit dem Hologramm der \textit{Chichén Itzá} gemintet.
  \item Es wurden 9.872 Pässe mit dem Hologramm des \textit{Kolosseum} gemintet.
  \item Es wurden 4.936 Pässe mit dem Hologramm der \textit{Petra} gemintet.
  \item Es wurden 2.468 Pässe mit dem Hologramm der \textit{Chinesischen Mauer} gemintet.
  \item Es wurden 1.234 Pässe mit dem Hologramm den \textit{Pyramiden von Gizeh} gemintet. noch verfügbar.
\end{itemize}

\vspace{0.3cm}

Die Evaluierung der übrigen 253 ist insofern recht dankbar, als dass die 253 schon sehr nah an der zyklischen 256 liegt ($= 2^{n}$, wobei $n=8$ für die acht verfügbaren Hologramme steht). Damit beschränkt sich die Analyse eigentlich lediglich auf die nächsten drei Pässe, von denen der erste unserer ist. 











\vspace{0.5cm}

\begin{itemize}
  \item vorrechnet, welche ersten 316.157 NFT-Pässe schon weggemintet sein könnten und Wahrscheinlichkeiten für den neu zu mintenden NFT-Pass erklären.
  \item neuen NFT-Pass unter Einbindung der Wahrscheinlichkeiten und vorgegaukelten Zufalls errechnet.
  \item geminteten neuen NFT-Pass als exakte Grafik in unserem Design hier abbilden.
\end{itemize}

\vspace{0.3cm}