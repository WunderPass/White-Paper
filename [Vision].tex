% !TEX root = paper.tex
\section{Vision}
\label{sec:vision}

\todo{TODO: Einleitung ausformulieren}

\vspace{0.3cm}

\textit{Wenn Personen auch im virtuellen Raum mehr sein wollen als Warenempfänger,
Zahlende oder "Nicknames", nämlich individuelle und facettenreiche Kommunikationspartner, dann muss sich die Komplexität des digitalen Identitätskonzeptes derjenigen
des realen annähern.}

\textit{Im Bereich der realen Identität ist aber weder ein fest abgegrenzter Raum von zu berücksichtigenden Bereichen oder Themen, welche einer Identität zuzuweisen wären, benennbar, noch sind Standards definiert, auf denen der Informationsaustausch zwischen
Individuen stattfindet. Dies macht ein umfassendes Konzept erforderlich mit den Möglichkeiten, die notwendige Flexibilität einerseits und eine Vereinheitlichung oder einen Abgleich des Informationsflusses andererseits zu gewährleisten. Dieses Konzept muss
dem durch die Individualität der Identitäten gegebenen Mangel an Kompatibilität entgegentreten.} 

\textit{Die Umsetzung des realen Identitätskonzeptes in ein digitales Identitätskonzept muss [...] Umfassende \textbf{Vernetzung mit direkten Verbindungen} und
die Möglichkeit zur Nutzung von Daten in maschinenlesbarer Form erscheinen für einen möglichen Einsatz als vorteilhaft. Um dem Anwender keine Barriere in den Weg
zu legen, ist es notwendig, möglichst viele Aspekte – gerade solche von struktureller und
organisatorischer Natur – \textbf{so weit wie möglich transparent} zu halten. \textbf{Wenn der Anwender auf der einen Seite durch keinen oder nur einen minimalen Mehraufwand, aber auf der anderen Seite verschiedene Vorteile, Erleichterungen oder neue Dienste erlangt, die auf dem Konzept digitaler Identitäten aufsetzen, wäre dies eine Bewältigung eines ansonsten sicherlich auftretenden Akzeptanzproblems.}} [\href{https://vsis-www.informatik.uni-hamburg.de/getDoc.php/thesis/47/DA_Gordian_Kaulbarsch.pdf}{Auszug aus der Arbeit "Identitäten und ihre Schnittstellen auf Basis von Ontologien in einer dezentralen Umgebung"}]

\vspace{0.3cm}

\todo{TODO: Aufgreifend aus vorigen Kapitel (verlinken)}

\vspace{0.3cm}

\begin{Solution}[fehlende Eindeutigkeit]

\todo{TODO: ausformulieren}
Kryptografische Identität ist eindeutig $\rightarrow$ private key

\end{Solution}

\vspace{0.3cm}


\begin{Solution}[Redundanz und fehlerbehaftete Daten]

\todo{TODO: ausformulieren}
Eine einzige globale Tabelle als Identity-Management-Service in der Blockchain

\end{Solution}

\vspace{0.3cm}


\begin{Solution}[mangelhafte UX]

\todo{TODO: ausformulieren}
Einen private key muss man nicht mehrfach registrieren.

\end{Solution}

\vspace{0.3cm}


\begin{Solution}[Datenschutz]

\todo{TODO: ausformulieren}
\begin{itemize}
  \item meine Daten liegen an einer einzigen Stelle gespeichert
  \item Daten sind verschlüsselt und unhackbar
  \item Derart ließen sich auch Identitätsdaten auf automatisierte Weise kontrolliert weitergeben. 'Kontrolliert' in diesem Zusammenhang bedeutet die Möglichkeit für den Anwender, selbst zu entscheiden, an wen er welche Daten wann und zu welchen Bedingungen übermittelt.
\end{itemize}

\end{Solution}

\vspace{0.3cm}


\begin{Solution}[Datenmissbrauch/Bereicherung]

\todo{TODO: ausformulieren}
Ich werde an der Verwendung meiner Daten monetär beteiligt (Token-Economics)

\end{Solution}

\vspace{0.3cm}


\begin{Solution}[Abhängigkeit von Big Tech]

\todo{TODO: ausformulieren}
Bei Self-Sovereign Identity (SSI) oder selbstbestimmter Identität kontrollieren und besitzen Nutzer ihre digitalen Identitäten und weitere verifizierbare digitale Nachweise (Verifiable Credentials (VC)), ohne hierfür auf eine zentrale Stelle, wie etwa Facebook oder Google, angewiesen zu sein. Sie sind somit komplett unabhängig von Dritt-Instanzen und entscheiden vollkommen eigenständig, wer welche Identitätsdaten zur Verfügung gestellt bekommt, da alle Identitätsdaten ausschließlich bei ihnen gespeichert werden. Dadurch ist ein einfacher, flexibler, sicherer und vertrauenswürdiger Austausch von manipulationssicheren digitalen Nachweisen zwischen Nutzer und Anwendungen möglich.

\end{Solution}

\vspace{0.3cm}


\begin{Solution}[Ungenutzte Möglichkeiten]

\todo{TODO: ausformulieren}
Daten-Querverweise $\rightarrow$ Beispiel anführen 

(zB aus \href{https://norbert-pohlmann.com/glossar-cyber-sicherheit/self-sovereign-identity-ssi/}{Vorlesung zu SSI})

\end{Solution}

\vspace{0.5cm}