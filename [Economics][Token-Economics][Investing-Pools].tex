% !TEX root = paper.tex

\begin{itemize}
	\item Das Pool-Projekt wird als \textbf{Curation Market} implementiert und bekommt seinen eigenen Token (IPT).
	\item Der IPT wird mittels \textbf{(Augmented) Bonding-Curves} implementiert, ist also gegen eine Einlage für jeden und immer mintbar.
	\item Die Einlage für den IPT ist in WUNDER zu erbringen \todo{(Es ist noch unklar, wie man an WUNDER kommt, wenn es vorher keinen Token-Sale gegeben hat. Ob der WUNDER ebenfalls mittels Bonding-Curves abzubilden wäre, sei hier erst einmal mehr als unklar.)}
	\item Der erste und größere Investor für das Pool-Projekt wäre WunderPass selbst. Für die erfolgte Einlage in den Projekt-Pool bekäme WunderPass IPT, die es für Incentivierungen und Rewards für die Nutzung von Pools verwenden könnte. Dieses Invest könnte (im Gegensatz zu den Einlagen anderer Investoren) zB. auch einem Locking unterliegen, um eine gewisse Preisstabilität des IPS zu gewährleisten.
	\item Der Pool-Initiator müsste bei der Pool-Eröffnung IPT staken, die er unter bestimmten Umständen verlieren könnte, wenn sein Pool zB. ungenutzt bleibt. So könnte man sicherstellen, dass ernste Absichten hinter den Pools stecken und diese auch genutzt werden. 
	\item Der Initiator wäre damit gleichzeitig auch Investor in das gesamte Pool-Projekt (da er ja für die Poolerstellung IPT kaufen muss).
	\item Gleichzeitig müsste der Initiator jedoch auch für sein Staken (ins Risiko gehen) belohnt werden, falls der Pool läuft und genutzt wird. Diese Belohnung würde in Form von zusätzlichen IPT (Stake-Rewards) erfolgen und z.B durch WunderPass und/oder den anderen Poolteilnehmern als eine Art Gebühr getragen werden und dabei folgenden Faktoren folgen:
	\begin{itemize}
		\item Pool-Lifetime
		\item Anzahl Teilnehmer
		\item Pool-Einsatz/-Umsatz
		\item etwaiger Gewinn aus Invests
		\item NFT-Pass-Status
		\item Upvoting durch andere IPT-Holder (als ein Art \textit{Master Pool-Creator})
	\end{itemize}
	\item In jedem Fall sollte der Staker im Normalfall (falls er nicht irgendwie Scheiße baut) bei der Auflösung des Pools mindestens seinen Einsatz zurückerhalten (also keinerlei Gebühren für die Nutzung des Pool-Service zahlen). In aller Regel sollte er mit mehr als dem ursprünglich gestakten Betrag rausgehen.
	\item Der nötige Staking-Betrag könnte fix pro Pool sein oder aber variabel und dabei von folgenden Kriterien abhängen:
	\begin{itemize}
		\item geplante Pool-Lifetime
		\item Anzahl Teilnehmer (min/max)
		\item Pool-Einsatz pro Teilnehmer (min/max)
		\item etwaige abgegeben Garantien seitens Pool-Creator (\textit{der Pool muss mindestens x, y und z erfüllen...}, bei deren Verfehlungen der Staker bestraft und im Erfolgsfall besonders entlohnt wird)
		\item NFT-Pass-Status
		\item Reputation in der Community (als ein Art \textit{Master Pool-Creator}
	\end{itemize}
	Die entscheidende Frage beim zu entrichtenden Stake-Betrag, ist die Klärung, ob es im Interesse des Stakers (Pool-Creator) sei, besonders viel (um größere Staking-Rewards zu erhalten) oder besonders wenig (um kein Risiko zu tragen) zu staken / staken zu müssen.
	\item Ein weiterer sehr essenzieller Faktor für die Größe des zu stakenden Betrags könnte der Kurs des IPTs sein. Denn laut der \textbf{Bonding-Curves}-Implementierung würde der IPT-Preis mit steigender Zirkulation steigen, was mit der Zunahme von existierende Pools geschähe. Damit wäre die Erstellung neuer Pools mit ihrer zahlenmäßigen Zunahme stets kapital-intensiver (aber nicht gleichbedeutend teurer). \textbf{Die Frage hierbei ist also, ob der zu erbringende Stake des Pool-Creators auf den \textit{Total-Supply des IPTs} normiert werden sollte oder nicht}, die gänzlich mit der obigen Fragestellung einhergeht, ob der Pool-Creator eigentlich staken möchte oder das nur tun muss.
	\begin{itemize}
		\item Gegen eine Normierung spricht die Annahme/Hoffnung, ein Pool-Creator sei gleichzeitig auch ein großer Supporter des gesamten Projekt und glaube daran. Wenn der IPT-Preis steigt, ist dies gleichbedeutend mit der Zunahme an genutzten Pools, an denen der Pool-Creator als Staker, Besitzer von IPT und damit Projekt-Investor auch selbst (finanziell) profitiert.
		\item Für eine Normierung spricht dagegen die potenzielle Gefahr, neue oder bestehende User durch eine zu hohe finanzielle Sicherheitseinlage davon abzuschrecken neue Pools zu erstellen.
	\end{itemize}
	Die Antwort auf diese Fragestellung könnte auch darin liegen, ob wir uns besonders viele oder lieber weniger aber besonders Teilnehmer-starke Pools wünschen.	
	\item Alle anderen Pool-Teilnehmer müssen eine Gebühr für ihre Teilnahme am Pool (und die Nutzung des Service) erbringen. Auch das hat in IPT zu erfolgen. \todo{Ob die Gebühr bei Pool-Beitritt gewissermaßen als \textit{prepaid} zu erbringen ist oder aber \textit{on demand} für eine anfallende Aktion bleibt zunächst unklar}. Die Gebühren könnten sich nach folgenden Faktoren bzw. Features richten und könnten sowohl voraussehbar sein (dann beim Pool-Beitritt zu entrichten) als auch \textit{on demand} anfallen:
	\begin{itemize}
		\item pro Zeiteinheit (Vorauszahlung für einen gewissen Zeitraum als prepaid; danach Zahlungsaufforderung um im Pool zu bleiben
		\item abhängig vom Einsatz (multiplikativ zum ersten Punkt)
		\item abhängig vom etwaigen Gewinn aus Invests (on demand)
		\item abhängig vom Pool-Creator (Community-/Staking-Status als Invest-Guru; prepaid)
		\item bei Aufstockung des Invests (on demand)
		\item bei Vorzeitigem Ausbezahlen und Verlassen des Pools (on demand)
		\item gekoppelt an Beteiligung am Staken (prozentual auf die vorigen Punkte anzurechnen)
		\item abhängig vom NFT-Pass-Status (prozentual auf die vorigen Punkte anzurechnen)
	\end{itemize} 
	\item Die Teilnehmer können bei dieser Logik aber nicht wie nicht wie die Staker zusätzlich als Projekt-Investoren angesehen werden, weil sie IPTs kaufen, da die gekauften IPTs direkt als Gebühr entrichtet werden. Für die Pool-Teilnehmer stellt der IPT also eher einen Utility- bzw. Purpose-Token dar weshalb die Höhe der zu entrichtenden Gebühr zweifelsfrei auf Basis von \textit{Total-Supply des IPTs} normiert werden muss (die Gebühr darf keinesfalls mit Zunahme von Pools steigen).
	\item Die entrichteten \textbf{Gebühren gehen zu einem relevanten Teil in die Treasury des gemeinschaftlichen Investing-Pools-Contracts} und stellen damit eine gemeinschaftliche Projekt-Erwirtschaftung dar. Der verbleibende Teil der Gebühren geht an den Pool-Staker. Die Aufschlüsslung der Verteilung auf Project-Treasury und Pool-Staker könnte halbwegs komplex werden und folgende Festlegungen folgen bzw. Gegebenheiten berücksichtigen:
	\begin{itemize}
		\item Verteilung nach einem simplen prozentualen Schlüssel (zB. 50-50)
		\item Absolute Mindest- und Obergrenzen des Projekt-Pool-Anteils (mindestens Betrag x geht an den Projekt-Pool; wenn es nicht reicht, alles; ab der Mindestgrenze erfolgt eine prozentuale Verteilung bis zu einer Maximalgrenze y für den Projekt-Pool; alles darüber geht an den Staker)
		\item progressive Verteilung abhängig des erbrachten Staking-Betrags (so könnte der Staker pro gestaktem IPT einen prozentualem Anteil $x \in [0; 1]$ pro IPT an Gebühren für sich beanspruchen, wobei das $x$ mit Größe des gestakten Betrags progressiv stiege)
		 \item Begünstigung des Stakers in Abhängigkeit seines NFT-Pass-Status.
	\end{itemize}
	\item Folgende Auswahl sollte vermutlich bei der Pool-Erstellung angeboten werden:
	\begin{itemize}
		\item $[$Pool-Creator erbringt den Stake; alle anderen Teilnehmer bezahlen die Gebühren$]$
		\item $[$Alle Pool-Teilnehmer teilen sich den zu erbringenden Stake und alle anfallenden Gebühren$]$
	\end{itemize} 
	\item Bedingt durch den Gebühren-Mechanismus erwirtschaften das Pool-Projekt Einnahme für die gemeinschaftliche Pool-Treasury, womit auch der IPT auf natürliche Weise im Wert steigt (ohne dass neue Investoren hinzukommen müssen, die einen höheren Tokenpreis bezahlen müssen). Jeder IPT-Holder - also auch insbesondere die Pool-Staker - profitiert also an der Erstellung neuer Pools (der Staker also indirekt an seinem eigenen Pool als auch an den fremden, was additiv zu seinen direkten Stake-Rewards hinzukommt). Das fördert also Word-of-Mouth, was sowohl auf die Preissteigerung des IPTs durch neue IPT-Holder einzahlt, als auch die Pool-Treasury durch neue Pools und die dafür anfallenden Gebühren füttert, was wiederum den IPT-Kurs befeuert.
	\item
\end{itemize}

\vspace{0.5cm}
