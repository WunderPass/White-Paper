\documentclass[11pt]{scrartcl}


%---------Konfiguration------


%-----Sprache und Zeichen----
\usepackage[utf8]{inputenc}
% \usepackage{ucs}
% \usepackage[T1]{fontenc}

% Zeilenumbrüche der deutschen Sprache
% \usepackage[ngerman]{babel}


%---------Farben-------------
\usepackage[RGB]{xcolor}
\definecolor{dunkelgruen}{RGB}{0 136 0}
\newcommand\todo[1]{\textcolor{red}{#1}}


%---------Links/Refs---------
\usepackage[colorlinks=true, urlcolor=blue]{hyperref}


%---------Grafiken-----------
\usepackage{graphicx}
\usepackage{subfig}



%---------Sonstiges----------
\usepackage{parcolumns}
\usepackage{enumitem}



%---------Mathe--------------
\usepackage{amsthm, amssymb, mathtools}
% \usepackage{amsmath, amssymb, amstext ,lipsum}



%---------Umgebungen---------
\usepackage[framemethod=tikz]{mdframed}

%---------Business---------
\mdtheorem[
  linecolor=dunkelgruen,
  frametitlefont=\sffamily\bfseries\color{white},
  frametitlebackgroundcolor=dunkelgruen,
]{Business-Def}{Definition}

\mdtheorem[
  linecolor=dunkelgruen,
  frametitlefont=\sffamily\bfseries\color{white},
  frametitlebackgroundcolor=dunkelgruen,
]{Hypothese}{Hypothese}

\mdtheorem[
  linecolor=gray,
  frametitlefont=\sffamily\bfseries\color{white},
  frametitlebackgroundcolor=gray,
]{Praemisse}{Prämisse}

\mdtheorem[
  linecolor=red,
  frametitlefont=\sffamily\bfseries\color{white},
  frametitlebackgroundcolor=red,
]{Abgrenzung}{Abgrenzung}

\mdtheorem[
  linecolor=violet,
  frametitlefont=\sffamily\bfseries\color{white},
  frametitlebackgroundcolor=dunkelgruen,
]{Umsetzung}{Umsetzung}

\mdtheorem[
  linecolor=violet,
  frametitlefont=\sffamily\bfseries\color{white},
  frametitlebackgroundcolor=gray,
]{Ausblick}{Ausblick}

\mdtheorem[
  linecolor=blue,
  frametitlefont=\sffamily\bfseries\color{white},
  frametitlebackgroundcolor=blue,
]{Fazit}{Conclusion}

\mdtheorem[
  linecolor=red,
  frametitlefont=\sffamily\bfseries\color{white},
  frametitlebackgroundcolor=red,
]{Problem}{Problem}

\mdtheorem[
  linecolor=cyan,
  frametitlefont=\sffamily\bfseries\color{white},
  frametitlebackgroundcolor=dunkelgruen,
]{Solution}{Lösung}

\mdtheorem[
  linecolor=dunkelgruen,
  frametitlefont=\sffamily\bfseries\color{white},
  frametitlebackgroundcolor=blue,
]{Konzept}{Konzept}





%---------Mathe------------
\mdtheorem[
  linecolor=gray,
  frametitlefont=\sffamily\bfseries\color{white},
  frametitlebackgroundcolor=gray,
]{Def}{Definition}


\mdtheorem[
  linecolor=red,
  frametitlefont=\sffamily\bfseries\color{white},
  frametitlebackgroundcolor=red,
]{Assumption}{Annahme}

\mdtheorem[
  linecolor=violet,
  frametitlefont=\sffamily\bfseries\color{white},
  frametitlebackgroundcolor=violet,
]{Example}{Beispiele}

\mdtheorem[
  linecolor=cyan,
  frametitlefont=\sffamily\bfseries\color{black},
  frametitlebackgroundcolor=cyan,
]{Algo}{Algorithmus}




%---------Dokument-----------

%---------Titel--------------
\title{WunderPools (Summary)}
\author{G. Fricke, M. Löchner, S.Tschurilin}
\date{\today{}, Berlin}
 
%---------Inhalt-------------
\begin{document}


\maketitle

% \tableofcontents{}


% !TEX root = paper.tex

\begin{Problem}[Bedarf nach Gemeinschaftskassen und und deren Verwahrung]

\todo{Problem: BaFin-Lizenz}

Die Idee hinter den sogenannten \textit{Wunder-Pools} ist das Bündeln von Liquidität mehrerer User/Teilnehmer bzw. eine Art 'Treuehandverwahrung' in einem gemeinsamen Pool. 

\end{Problem}

\vspace{0.3cm}


\begin{Example}[Anwendungsfälle sind zahlreich]

\begin{itemize}
  \item Gemeinsame Invests in (Crypto-)Assets.
  \item Pool für ein gemeinsames (Geburtstags-)Geschenk.
  \item Kicktipp-Pool (der über die gesamte Saison verwahrt werden muss).
  \item Wetten unter Freunden (z. B. Sportereinisse wie ein WM-Finale).
  \item Ausgleichspool für Auslagen von Geld an Freunde (Splitwise).
\end{itemize}

\end{Example}

\vspace{0.3cm}


\begin{Solution}[Smart-Contracts brauchen keine BaFin-Lizenz]

\todo{TODO}

\begin{itemize}
  \item \todo{Kernaussage beleuchten}
  \item Grobe Beschreibung und Verlinkungen ins ausführliche Pool-Paper
  \begin{itemize}
	\item Pool-Erzeugung
	\item Pool-Lifetime
	\item Pool-Liquidierung
  \end{itemize}
\end{itemize}


\end{Solution}

\vspace{0.3cm}


\begin{Hypothese}[Abstraktion mittels Payout-Oracles]

\todo{TODO}

\end{Hypothese}

\vspace{0.3cm}







\begin{Problem}[Bedarf nach Gemeinschaftskassen und und deren Verwahrung]

\todo{TODO}

\end{Problem}

\vspace{0.3cm}


\vspace{0.5cm}    % binde die Datei '[Pools][Zusammenfassung].tex' ein






\end{document}

