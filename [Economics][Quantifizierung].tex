% !TEX root = paper.tex
\subsection{Quantifizierung}
\label{sec:eco_zahlen}
\todo{Einleitung - Start}

Wir wollen den Mehrwert von User-Provider-Connections mittels Wunderpass einen bezifferbaren Mehrwert verleihen und diesen fundiert argumentieren. Dazu müssen wir diesen Value messen und beziffern können. Die Ergebnisse dieses Kapitels werden insbesondere für das im Kapitel \ref{sec:wpt_reward_pool} beleuchteten "Reward-Pools" von großer Bedeutung sein. Bzw. sogar im gesamten übergeordneten Kapitel \ref{sec:eco_wpt}.
\todo{Einleitung - Ende}

% !TEX root = paper.tex
\subsubsection{Grundlegende Definitionen}
\label{sec:eco_zahlen_def}

Sei $t_0$ der initiale Zeitpunkt all unserer Messungen und Betrachtungen (vermutlich der Zeitpunkt des MVP-Launches).

Darauf aufbauend betrachten wir das künftige Zeitintervall $T$, welches einzig an Relevanz für unser Vorhaben und alle in diesem Kapitel getätigten Ausführungen besitzt:

\begin{equation*}
  T = [t_0; \infty[
\end{equation*}
Der Zeitstrahl muss nicht zwingend unendlich sein. Er muss ebenfalls nicht zwingend infinitesimal fortlaufend sein und kann stattdessen je nach Kontext endlich und/oder diskret betrachtet werden. Also z. B. auch wahlweise als 

\begin{equation*}
  T = [t_0; t_{ende}]
\end{equation*}

\begin{equation*}
  T = [t_0; t_1;...; t_{ende}]
\end{equation*}
definiert sein. In letzteren beiden Fällen wird jedoch $t_{ende}$ in aller Regel eine kontextbezogene (unverzichtbare) Bedeutung haben, die eine solche Definition des Zeitstrahls unverzichtbar macht. So könnte $t_{ende}$ z. B. für eine mathematisch quantifizierbare Erreichung unserer Vision stehen. \\

Sei $\mathbf{t \in T}$ fortan stets ein beliebiger Zeitpunkt, zu welchem wir eine Aussage treffen möchten. \\


Wir definieren die Anzahl aller zum Zeitpunkt $t$ potenziellen User $U^{(t)}$ überhaupt und ihre (maximale) Anzahl $n^{(t)}$ als \\

\begin{Def}\label{defU}
\begin{equation*}
  U^{(t)} = \left\{ u^{(t)}_1; u^{(t)}_2;...; u^{(t)}_{n} \right\}
\end{equation*}
\end{Def} 

\vspace{0.3cm}


Und ganz analog dazu ebenfalls die potenziellen Service-Provider $S^{(t)}$ und ihre (maximale) Anzahl $m^{(t)}$ als \\

\begin{Def}\label{defS}
\begin{equation*}
  S^{(t)} = \left\{ s^{(t)}_1; s^{(t)}_2;...; s^{(t)}_{m}\right\}
\end{equation*}
\end{Def}

\vspace{0.3cm}

Man beachte, dass die definierten Mengen $U^{(t)}$ und $S^{(t)}$ bzw. ihre Größe gewissermaßen den Fortschritt der Digitalisierung insgesamt beschreiben (potenzielle User brauchen einen Zugang zum digitalen Ökosystem und potenzielle Provider sind unabhängige Service-Dienstleister, die eigenmächtig darüber entscheiden, zu solchen zu werden) und in keiner Weise im Einfluss Wunderpasses stehen. Viel mehr beschreiben sie die "Umstände der Welt", mit denen WunderPass (wie alle anderen) "arbeiten" müssen.  

\vspace{0.6cm}


Nun definieren den \textbf{\textit{Connection-Koeffizienten}} zwischen den eben definierten potenziellen Usern $\mathbf{U^{(t)}}$ und den Service-Providern $\mathbf{S^{(t)}}$ zum Zeitpunkt $t$ als boolesche Funktion $\mathbf{\alpha^{(t)}}$, die über über die Tatsache \textit{"is connected"} bzw. \textit{"is not connected"} entscheidet: \\

\begin{Def}\label{defKoeff}

\begin{equation*}
  \alpha^{(t)} : U^{(t)} \times S^{(t)} \rightarrow \{0; 1\} 
\end{equation*}

\[
\alpha^{(t)}(u, s):=\left\{%
\begin{array}{ll}
    1, & \hbox{falls User $u \in U^{(t)}$ mit mit Provider $s \in S^{(t)}$ connectet ist} \\
    0, & \hbox{andernfalls} \\
\end{array}%
\right.
\]

\vspace{1cm}

Bzw. wenn man die diskreten Auslegungen der Pools $U^{(t)} = \left\{ u^{(t)}_1; u^{(t)}_2;...; u^{(t)}_{n} \right\}$ und $S^{(t)} = \left\{ s^{(t)}_1; s^{(t)}_2;...; s^{(t)}_{m} \right\}$ heranzieht, alternativ als

\[
\alpha^{(t)}_{ij}:=\left\{%
\begin{array}{ll}
    1, & \hbox{falls User $u^{(t)}_i \in U^{(t)}$ mit mit Provider $s^{(t)}_j \in S^{(t)}$ connectet ist} \\
    0, & \hbox{andernfalls} \\
\end{array}%
\right.
\]

\end{Def}

\vspace{0.6cm}

Man beachte, dass wir bei den diskreten/Aufzählungs-basierten Definitionen oben, der Übersicht halber etwas "geschlampt" haben, indem wir - klar zeitbedingte - Indizes stillschweigend als $n$ und $m$ bezeichnet haben, gleichwohl diese korrekterweise $n^{(t)}$ und $m^{(t)}$ lauten müssten. Nur verwirrt eben ein Ausdruck wie $u^{(t)}_{n^{(t)}}$ mehr, als dieser in seiner pedantischen Korrektheit einen Mehrwert generiert. Wir werden genannte Ungenauigkeit zudem im weiteren Verlauf in gleicher Weise fortführen und gehen davon aus, der Leser wisse damit umzugehen. 

\vspace{0.3cm}


    % binde die Datei '[Economics][Quantifizierung][Definitionen].tex' ein

\subsubsection{Zustandsbeschreibung der digitalen Welt}
\label{sec:eco_zahlen_zustand_digitalisierung}

Mit diesen geschaffenen Formalisierungs-Werkzeugen lassen sich nun einige Dinge formal deutlich besser greifen. Und zwar zum einen im Folgenden die übergeordneten "Umstände der digitalen Welt" (auf die WunderPass bestenfalls sehr geringfügig Einfluss üben kann) aber zum anderen ebenfalls unser gesamtes Vorhaben inklusive der übergeordneten WunderPass-Vision, die in den darauf folgenden Kapitels beleuchtet wird. 

\vspace{0.3cm}

Aufgrund der bereits weiter oben erwähnten nicht möglichen Einflussnahme auf die Mengen $U^{(t)}$ und $S^{(t)}$ benötigen wir noch ein weiteres Hilfsmittel, dessen Existenz wir im Folgenden einfach voraussetzen möchten - und diese mit Möglichkeiten der Markt-Analyse rechtfertigen.

\vspace{0.3cm}

\begin{Assumption}[Digitalisierungs-Orakel]\label{assumptionOrakel}

Sei $t \in T$. Anstatt die (nicht wirklich berechtigte) Kenntnis der Mengen $U^{(t)}$ und $S^{(t)}$ vorzugeben, wollen wir lieber die (realistischere) Existenz einer "Schätzfunktion" $dP^{(t)}$ (digital progress) annehmen. Wir definieren $dP^{(t)}$ als

\begin{align*}
dP &: T \rightarrow \mathbb{N} \times \mathbb{N}  \\
dP^{(t)} &:= \left(n^{(t)}, m^{(t)}\right)
\end{align*}
wobei $n^{(t)} = |U^{(t)}|$ und $m^{(t)} = |S^{(t)}|$ darstellen sollen, ohne dafür zwingend die exakten Mengen $U^{(t)}$ und $S^{(t)}$ kennen zu müssen.

\end{Assumption}

\vspace{0.3cm}

Und auf der letzten Annahme aufbauend der Vollständigkeit halber die aus praktischer Sicht vollkommen alternativlose Annahme ergänzen, laut der Service-Provider stets eine große Anzahl an Users ansprechen/bedienen und damit zahlenmäßig den Usern stark unterlegen sind.

\vspace{0.3cm}

\todo{[TODO1][Annahme 2 ist noch buggy]}

\begin{Assumption}[Verhältnismäßigkeit der Teilnehmer]\label{assumptionRatio}
Für alle $t \in T$ mit $\left(n^{(t)}, m^{(t)}\right) = dP^{(t)}$ gilt:

\begin{equation*}
m^{(t)} << n^{(t)} \tag{i}
\end{equation*}

\vspace{0.3cm}

Diese Aussage mag zahlenmäßig noch etwas "griffiger" formuliert werden. Dafür möchten wir das Verhältnis der Größen $n^{(t)}$ und $m^{(t)}$ abschätzen: Für unseren Zeithorizont, an dessen Ende - einem ausreichend späten, aber auch nicht in unabsehbar fernen Zukunft liegenden Zeitpunkt $t_{end} \in T$ - wir von einer WunderWelt sprechen, sei die Annahme

\begin{align*}
n^{(t_{end})} &\thickapprox 10 Mrd. \textrm{ und } m^{(t_{end})} \thickapprox 10.000 \textrm{ bzw.} \\
dP^{(t_{end})} &\thickapprox (10^{10}, 10^{4}) = 10^{4} * (10^{6}, 1) \tag{ii}
\end{align*}
nicht ganz abwegig. Genauso wenig unvernünftig scheint die Annahme, WunderPass begänne seine Welteroberung mit einem MVP mit lediglich einem einzigen Service-Provider - z. B. dem Guard (siehe Kap. \ref{sec:guard}] - und einer überschaubaren Anzahl an angepeilten Usern, also

\begin{align*}
n^{(t_0)} &\thickapprox 1.000 \textrm{ und } m^{(t_0)} = 1 \textrm{ bzw.} \\
dP^{(t_0)} &\thickapprox (1.000, 1) \tag{iii}
\end{align*}

\vspace{0.3cm}

Mit den beiden zuletzt getroffenen (quantitativen) Annahmen (ii) und (iii) lässt sich auch die initiale (qualitative) Annahme (i) ebenfalls quantifizieren:

\vspace{0.3cm}

Für alle $t \in ]t_0; t_{end}[$ mit $\left(n^{(t)}, m^{(t)}\right) = dP^{(t)}$ gilt:

\begin{equation*}
1.000 = \frac{n^{(t_0)}}{m^{(t_0)}} < \frac{n^{(t)}}{m^{(t)}} < \frac{n^{(t_{end})}}{m^{(t_{end})}} = 1.000.000 \tag{iv}
\end{equation*}

\end{Assumption}

\vspace{0.3cm}

Wir fassen Annahme \ref{assumptionRatio} in einer abschließenden Definition zusammen:

\vspace{0.3cm}

\begin{Def}\label{defRatio}

Seien $t \in T$ und $dP^{(t)} = \left(n^{(t)}, m^{(t)}\right)$ wie in Annahme \ref{assumptionOrakel} beschrieben. Wie definieren die "Verhältnismäßigkeit der Teilnehmer" als

\begin{align*}
\sigma &: T \rightarrow \mathbb{Q} \\
\sigma^{(t)} &= \frac{n^{(t)}}{m^{(t)}}
\end{align*}

\vspace{0.3cm}

Zudem halten wir fest, Annahme \ref{assumptionRatio} lege nahe, man könne in der Praxis stets von

\begin{equation*}
1.000 < \sigma^{(t)} < 1.000.000
\end{equation*}
ausgehen.

\end{Def}

\todo{[ende TODO1]}

\vspace{0.6cm}



\subsubsection{Zustandsbeschreibung WunderPass - simple Betrachtung}
\label{sec:eco_zahlen_zustand_wp}

% !TEX root = paper.tex

\paragraph{Status quo} 
\label{sec:eco_zahlen_zustand_wp_now}
\textrm{ }

\vspace{0.3cm}

Aufbauend auf die bisher erzielten Ergebnisse, wollen wir nun auch dem Stand von WunderPass für einen beliebigen Zeitpunkt $t \in T$ einen formalisierten Charakter verleihen und definieren zunächst einmal mittels der in Def \ref{defKoeff} beschriebenen Koeffizienten $\alpha^{(t)}_{ij}$ die sogenannten "connected Pools" von Usern und Service-Providern zum Zeitpunkt $t \in T$:

\vspace{0.3cm}

\begin{Def}\label{defPools}

Wir definieren den "connected User-Pool" $\widehat{U}^{(t)} \subseteq U^{(t)}$ und den "connected Service-Provider-Pool" $\widehat{S}^{(t)} \subseteq S^{(t)}$ als

\begin{align}
\widehat{U}^{(t)}:&= \left\{u \in U^{(t)} \mid \exists s^{*} \in S^{(t)} \textrm{ mit } \alpha^{(t)}(u, s^{*}) = 1 \right\} \tag{i} \\ 
\widehat{S}^{(t)}:&= \left\{s \in S^{(t)} \mid \exists u^{*} \in U^{(t)} \textrm{ mit } \alpha^{(t)}(u^{*}, s) = 1 \right\} \tag{ii}
\end{align}

\vspace{0.3cm}

Für die diskrete/sortierte Variante ist dies wieder gleichbedeutend mit
\begin{align}
\widehat{U}^{(t)} &= \left\{ \widehat{u}^{(t)}_1; \widehat{u}^{(t)}_2;...; \widehat{u}^{(t)}_{\widehat{n}} \right\} \tag{iii} \\ 
\widehat{S}^{(t)} &= \left\{ \widehat{s}^{(t)}_1; \widehat{s}^{(t)}_2;...; \widehat{s}^{(t)}_{\widehat{m}}\right\} \tag{iv}
\end{align}

\vspace{0.6cm}

Der Wert $\widehat{n} \leq n$ beschreibt die Größe des connecteten User-Pools - also die Anzahl $\widehat{n}$ der tatsächlich mit WunderPass connecteten User unter den $n$ potenziellen Usern. Analog steht $\widehat{m} \leq m$ für die Anzahl der tatsächlich mit WunderPass connecteten Prividern. Der Vollständigkeit halber übertragen wir das aus Def \ref{defKoeff} stammende Verständnis der Connection-Koeffizienten auch auf die eben definierten "connected Pools"

\[
\widehat{\alpha}^{(t)}_{ij}:=\left\{%
\begin{array}{ll}
    1, & \hbox{falls User $\widehat{u}^{(t)}_i \in \widehat{U}^{(t)}$ mit mit Provider $\widehat{s}^{(t)}_j \in \widehat{S}^{(t)}$ connectet ist} \\
    0, & \hbox{andernfalls} \\
\end{array}%
\right. \tag{v}
\] 

\end{Def}

\vspace{0.6cm}

Man beachte bei den diskreten/sortierten Schreibweisen der definierten Mengen $U^{(t)}$, $\widehat{U}^{(t)}$, $S^{(t)}$ und $\widehat{S}^{(t)}$, dass in aller Regel $u^{(t)}_i \neq \widehat{u}^{(t)}_i$ und $s^{(t)}_j \neq \widehat{s}^{(t)}_j$ gelten. Die sich teils trivial aus den letzten Definitionen ergebenden Zusammenhänge fallen wir in Form eines Theorems zusammen:

\vspace{0.3cm}

\input{[Economics][Quantifizierung][theoremPools]}    % binde die Datei '[Economics][Quantifizierung][theoremPools].tex' ein


Es ist klar, dass WunderPass sich in gewisser Weise an den definierten numerischen Messgrößen ihrer angebundenen Teilnehmer $\widehat{n}$ und $\widehat{m}$ messen können wird. Zusätzlich dazu möchten wir ein - womöglich deutlich relevanteres - numerisches Maß formalisieren. Nämlich die intuitive und sehr simple KPI "Gesamtzahl bestehender User-to-Provider-Connections" zum Zeitpunkt $t \in T$.

\vspace{0.3cm}

\begin{Def}\label{defGamma}

\begin{equation*}
  \Gamma : T \rightarrow \mathbb{N} 
\end{equation*}

\begin{equation*}
  \Gamma(t):= \sum_{i=1}^n \sum_{j=1}^m \alpha^{(t)}_{ij} \textrm{ mit } (n, m) = \left(n^{(t)}, m^{(t)}\right) = dP^{(t)}
\end{equation*}

\end{Def}

\vspace{0.6cm}

Nun beweisen wir folgende sich ergebende Zusammenhänge:

\begin{Theorem}\label{theremConnectionsCount}

Sei $t \in T$ ein beliebiger Zeitpunkt, $(n, m) = dP^{(t)}$ und $\widehat{U}^{(t)} = \left\{ \widehat{u}^{(t)}_1; \widehat{u}^{(t)}_2;...; \widehat{u}^{(t)}_{\widehat{n}} \right\}$ und $\widehat{S}^{(t)} = \left\{ \widehat{s}^{(t)}_1; \widehat{s}^{(t)}_2;...; \widehat{s}^{(t)}_{\widehat{m}} \right\}$ die connecteten Teilnehmer-Pools mit $\widehat{n} < n$ sowie $\widehat{m} < m$.

Zudem soll in Anlehnung an Annahme \ref{assumptionRatio} $\widehat{m} << \widehat{n}$ gelten. Dann gelten zusätzlich auch folgende Aussagen:

\vspace{0.3cm}

\begin{align*}
\Gamma(t)&= \sum_{i=1}^{\widehat{n}} \sum_{j=1}^{\widehat{m}} \widehat{\alpha}^{(t)}_{ij} \tag{i} \label{theremConnectionsCount_1} \\ 
\widehat{n} &\leq \Gamma(t) \leq \widehat{n} * \widehat{m} \tag{ii} \\
\widehat{n} = \Gamma(t) &\Leftrightarrow \forall i \in \left\{1;...; \widehat{n} \right\}
\textrm{ gilt } \sum_{j=1}^{\widehat{m}} \widehat{\alpha}^{(t)}_{ij} = 1 \tag{iii} \\
\Gamma(t) = \widehat{n} * \widehat{m} &\Leftrightarrow \widehat{\alpha}^{(t)}_{ij} = 1 \textrm{  } \forall i \in \left\{1;...; \widehat{n} \right\} \textrm{ und } \forall j \in \left\{1;...; \widehat{m} \right\} \tag{iv}
\end{align*}

\vspace{0.3cm}

Man beachte, Aussage (ii) impliziert insbesondere

\begin{equation*}
\Gamma(t) = 0 \Leftrightarrow \widehat{n} = 0 \Leftrightarrow \widehat{U}^{(t)} = \emptyset  \Leftrightarrow \widehat{S}^{(t)} = \emptyset
\end{equation*}

\vspace{0.3cm}

Aussage (iv) beschreibt dagegen quasi eine "\textbf{\textit{Voll-Vernetzung}}" der aktuell connecteten Teilnehmer!

\end{Theorem}

\vspace{0.3cm}


\begin{proof}[Beweis] \textrm{ }

\vspace{0.3cm}

Die Aussage (i) ist intuitiv nahezu trivial. Das explizite Vorrechnen dagegen etwas aufwendig, erfolgt aber in Grunde sehr ähnlich wie der Beweis der Aussagen (v) und (vi) des Theorems \ref{theoremPools}.

\vspace{0.4cm}

zu (ii): 

$\Gamma(t) \leq \widehat{n} * \widehat{m}$ ergibt sich aus
\begin{equation*}
  \Gamma(t) = \sum_{i=1}^{\widehat{n}} \sum_{j=1}^{\widehat{m}} \widehat{\alpha}^{(t)}_{ij} \leq \sum_{i=1}^{\widehat{n}} \sum_{j=1}^{\widehat{m}} 1 = \widehat{n} * \widehat{m}
\end{equation*}

\vspace{0.3cm}

Nun zeigen wir $\widehat{n} \leq \Gamma(t)$. Für $n = 0$ ergibt sich die Aussage aus Punkt (iv) aus Theorem \ref{theoremPools}. Sei also $n > 0$. Dann ist

\begin{align*}
\Gamma(t) &\overset{\text{(i)}}{=} \sum_{i=1}^{\widehat{n}} \sum_{j=1}^{\widehat{m}} \widehat{\alpha}^{(t)}_{ij} \\
&\overset{(Def \ref{defPools})}{\geq} \sum_{i=1}^{\widehat{n}} 1 = \widehat{n}
\end{align*}

\vspace{0.4cm}

zu (iii): 
"$\Leftarrow$" ist trivial. 

\vspace{0.4cm}

Zu "$\Rightarrow$": Sei $\widehat{n} = \Gamma(t)$. Angenommen es gäbe ein $i^{*} \in \left\{1;...; \widehat{n} \right\}$ mit $\sum_{j=1}^{\widehat{m}} \widehat{\alpha}^{(t)}_{i^{*}j} > 1$. Dann müsste es aufgrund der Annahme aber auch ein $i^{**} \in \left\{1;...; \widehat{n} \right\}$ mit $\sum_{j=1}^{\widehat{m}} \widehat{\alpha}^{(t)}_{i^{**}j} < 1$ also $\sum_{j=1}^{\widehat{m}} \widehat{\alpha}^{(t)}_{i^{**}j} = 0$ geben. In diesem Fall wäre aber $\widehat{u}^{(t)}_{i^{**}} \notin \widehat{U}^{(t)}$ und somit auch $i^{**} \notin \left\{1;...; \widehat{n} \right\}$. Widerspruch!

\vspace{0.4cm}

zu (iv): 
"$\Leftarrow$" ist wieder trivial.

\vspace{0.4cm} 

Zu "$\Rightarrow$": Es gelte also $\Gamma(t) = \widehat{n} * \widehat{m}$. Angenommen es gäbe ein $i^{*} \in \{1,...,\widehat{n}\}$ und ein $j^{*} \in \{1,...,\widehat{m}\}$, sodass $\alpha^{(t)}_{i^{*}j^{*}} = 0$. Dann wäre unter Gültigkeit der Aussage (i)

\begin{align*}
\widehat{n} * \widehat{m} &= \Gamma(t) = \sum_{i=1}^{\widehat{n}} \sum_{j=1}^{\widehat{m}} \widehat{\alpha}^{(t)}_{ij} \\
&\leq \left(\sum_{i=1}^{\widehat{n}} \sum_{j=1}^{\widehat{m}} 1 \right) - 1 = \widehat{n} * \widehat{m} - 1 < \widehat{n} * \widehat{m} \\
\end{align*}
Widerspruch!
  
\end{proof}
\vspace{0.6cm}


 % binde die Datei '[Economics][Quantifizierung][WP][Status quo].tex' ein

% !TEX root = paper.tex

\paragraph{Messbarkeit Status quo} 
\label{sec:eco_zahlen_zustand_wp_nowVlaue}
\textrm{ }

\vspace{0.3cm}

Kommend von der intuitiven Annahme, die Größe der definierten "connected Pools" $\widehat{U}^{(t)}$ und $\widehat{S}^{(t)}$ sei irgendwie erstrebenswert in unserem Sinne, definierten wir im vorangehenden Abschnitt das - aus unserer Sicht fundierteres und geeigneteres - Maß $\Gamma(t)$, um dem Verständnis von "erstrebenswerter Zustand" besser gerecht zu werden.

In diesem Abschnitt wollen wir die - bisher eher wertfrei/objektiv formulierten -  Ergebnisse des vorigen Abschnitts in den Kontext der "Erstrebenswertigkeit" stellen. Also eine formale und quantifizierbare Vergleichbarkeit unserer - ohnehin beim Lesen des letzten Abschnitts mitschwingender - Intuition schaffen, die Werte 
\begin{itemize}
  \item $\widehat{n} = |\widehat{U}^{(t)}|$, 
  \item $\widehat{m} = |\widehat{S}^{(t)}|$ und vor allem 
  \item $\Gamma(t)$ 
\end{itemize}
seien umso besser je größer sie seien. Alle der eben genannten Größen, denen wir hier eine intuitiv spürbare "Erstrebenswertigkeit" beimessen, besitzen einen klaren Zeitbezug. Daher überrascht es kaum, wir strebten die genannte quantifizierbare Vergleichbarkeit für je zwei beliebige Zeitpunkte $t_1, t_2 \in T$ an. Formale Vergleichbarkeit schreit nur so nach der mathematisch verstandenen "Ordnungsrelation":

\vspace{0.3cm}

\begin{Def}\label{defRelation}

Wir bedienen uns der in Definition \ref{defGamma} beschriebenen Funktion $\Gamma(t)$, um damit eine \href{https://de.wikipedia.org/wiki/Ordnungsrelation}{Ordnungsrelation} 
auf unserem Zeitstrahl $T$ für je zwei beliebige Zeitpunkte $t_1, t_2 \in T$ zu erhalten: 

\vspace{0.3cm}

\begin{equation*}
  R_{\preceq} \subseteq T \times T \textrm{ mit}
\end{equation*}

\begin{equation*}
  R_{\preceq}:= \left\{ (t_1, t_2) \in T \times T \mid \Gamma(t_1) \leq \Gamma(t_2) \right\}
\end{equation*}
\vspace{1cm}
Mittels $R_{\preceq}$ erhalten wir eine Ordnung unseres Zeitstahls $T$ und erklären zudem insbesondere, was "erstrebenswert" bedeutet. Ein beliebiger Zeitpunkt $t_1 \in T$ ist nämlich verbal genau dann "nicht weniger erstrebenswert" in Sinne unserer Vision als ein beliebiger anderer Zeitpunkt $t_2 \in T$, falls $(t_1, t_2) \in R_{\preceq}$ gilt.

\vspace{0.3cm}

\begin{equation*}
  \textrm{Wir schreiben fortan statt } (t_1, t_2) \in R_{\preceq} \textrm{ lieber } t_1 \preceq t_2 
\end{equation*}

\end{Def}

\vspace{1cm}

Man beachte, dass es sich bei der definierten Ordnungsrelation gar um eine \href{https://de.wikipedia.org/wiki/Ordnungsrelation#Totalordnung}{Totalordnung} handelt!
Der Form halber ergänzen wir an der Stelle noch um zwei weitere - schematisch induzierte - Relationen auf unserem Zeitstrahl $T$:

\vspace{0.3cm}

\begin{Def}\label{defRelationen}

Um zusätzlich zur in Def \ref{defRelation} definierten Ordnungsrelation "$\preceq$", auch dem Verständnis von "echt besser" und "gleich gut" Rechnung zu tragen, definieren wir die beiden Relationen "$\prec$" und "$\simeq$"

\vspace{0.3cm}

\begin{equation*}
  R_{\prec}:= \left\{ (t_1, t_2) \in T \times T \mid \Gamma(t_1) < \Gamma(t_2) \right\}
\end{equation*}

\begin{equation*}
  R_{\simeq}:= \left\{ (t_1, t_2) \in T \times T \mid \Gamma(t_1) = \Gamma(t_2) \right\}
\end{equation*}

\vspace{1cm}

Bei $R_{\prec}$ handelt es sich im Übrigen wieder um eine Ordnungsrelation. Bei $R_{\simeq}$ dagegen nicht.

\end{Def}

\vspace{0.3cm}

Auch für die letzten beiden Relationen wollen wir fortan die vereinfachte Schreibweise $t_1 \prec t_2$ und $t_1 \simeq t_2$ nutzen.

\vspace{0.6cm}

 % binde die Datei '[Economics][Quantifizierung][WP][Messbarkeit].tex' ein

% !TEX root = [Economics][Quantifizierung].tex

\paragraph{Fazit} 
\label{sec:eco_zahlen_zustand_wp_fazit}
\textrm{ }

\vspace{0.3cm}

Ungeachtet des Werts der bisher erzielten erfolgreichen Ergebnisse hinsichtlich der quantitativen Einordnung des WunderPass-Fortschritts zu einem Zeitpunkt $t \in T$, besitzt der Zusatz "...simple Betrachtung" innerhalb der Überschrift des gegenwärtigen Kapitels durchaus seine Rechtfertigung.

Wir haben zwar die Größe $\Gamma(t)$ als sehr gut geeigneten Gradmesser für den Fortschritt WunderPasses herausgearbeitet und dieses ebenfalls in Abhängigkeit der intuitiven Erfolgsmesser $\widehat{n}$ und $\widehat{m}$ gesetzt sowie nach unten und oben abgeschätzt. Jedoch scheint unser Ökosystem zu komplex und unsere bisherige Betrachtungsweise zu global geprägt, als dass wir guten Gewissens den besagten Zusatz "...simple Betrachtung" in der Überschrift des gegenwärtigen Kapitels weglassen könnten. Den geäußerten Zweifel verdeutlicht folgendes 

\vspace{0.3cm}

\begin{Example*}
Wir nehmen den Zustand zum Zeitpunkt $t \in T$ mit $\widehat{m} = 5$ angebundenen Service-Providern und als durch $\Gamma(t) \approx 50.000$ beschrieben an und schauen uns drei Szenarien an, die allesamt die getroffene Annahme hergeben:

\vspace{0.3cm}

\begin{enumerate}
  \item Wir könnten von $\widehat{n} = 50.000$ angebundenen Usern ausgehen, von denen je 10.000 mit je einem einzigen der $\widehat{m} = 5$ Provider connectet wären und keinem anderem. 
  \item Genauso könnten dieselben $\widehat{n} = 50.000$ angebundene User so verteilt sein, dass 49.996 (quasi alle) mit demselben einzelnen Provider connectet sind, und die restlichen 4 (also quasi niemand) User mit je einem anderen der verbleibenden 4 Provider verbunden sind.
  \item Ein ganz anderes Szenario wäre der Fall von $\widehat{n} \approx 25.000$, von denen jeder mit denselben zwei unserer fünf Service-Providern connectet wäre (und keinem anderen) und zudem ein paar vereinzelte zusätzliche User mit je einem der verbleibenden drei unserer fünf Provider.
\end{enumerate}

\vspace{0.3cm}

Rein an den Größen $\widehat{n}$, $\widehat{m}$, $\Gamma(t)$ gemessen, scheint Fall (3) aufgrund von $\widehat{n} = 25.000$ der schlechteste zu sein. Rein intuitiv scheint genau dieser Fall aber der beste zu sein. Dies ist aber nur ein Gefühl. Es lassen sich ebenso gute Argumente finden, warum Fall (1) oder Fall (2) der beste sein könnten. Es kommt eben darauf an...Gleichwohl für alle der Fälle $\Gamma(t) = 50.000$ gilt, lässt sich zweifelsfrei entscheiden, welcher zwingend der beste sein soll.

Was sich jedoch objektiv beurteilen lässt, ist die Tatsache, dass in Fall (2) vier der fünf Service-Provider quasi "wertlos" sind. Und in Fall (3) immer noch drei von fünf!

\vspace{0.3cm}

Wir könnten also unsere Gegenüberstellung der drei angeführten Cases auch zur folgenden quantitativen Beurteilung stellen:

\vspace{0.3cm}

\begin{enumerate}
  \item $\Gamma_1(t) = 50.000$, $\widehat{n}_1 = 50.000$ und $\widehat{m}_1 = 5$
  \item $\Gamma_2(t) = 50.000$, $\widehat{n}_2 = 50.000$ und $\widehat{m}_2 = 1$
  \item $\Gamma_3(t) = 50.000$, $\widehat{n}_3 = 25.000$ und $\widehat{m}_3 = 2$
\end{enumerate}

\vspace{0.3cm}

Was ist also besser?

\end{Example*}

\vspace{0.6cm}

 % binde die Datei '[Economics][Quantifizierung][WP][Fazit].tex' ein




\subsubsection{Zustandsbeschreibung WunderPass - detaillierte Sicht}
\label{sec:eco_zahlen_zustand_wp_advanced}


\paragraph{Teilnehmer} 
\label{sec:eco_zahlen_zustand_wp_advanced_teilnehmer}
\textrm{ }

\vspace{0.3cm}

\todo{WIP}
\vspace{1cm}

\todo{[TODO4]["individuelle" User- und Provider-Pools]}
\vspace{0.3cm}

Nicht alle Teilnehmer innerhalb der WunderPass-Netzwerks sind gleichbedeutend. Dies ist zweifelsfrei klar hinsichtlich der Unterscheidung zwischen connecteten Usern $\widehat{u} \in \widehat{U}^{(t)}$ und Service-Providern $\widehat{s} \in \widehat{S}^{(t)}$. Jedoch bestehen ebenfalls signifikante Unterschiede jeweils innerhalb der beiden Teilnehmerklassen $\widehat{U}^{(t)}$ und $\widehat{S}^{(t)}$ (siehe \todo{[TODO4]["Sättigung"]}). Um dieser Unterscheidung unserer Teilnehmer gerecht zu werden, definieren wir "connected Pools" pro individuellen Teilnehmer als

\vspace{0.3cm}

\begin{Def}\label{defTeilnehmerPool}

Sei $t \in T$, $\widehat{U}^{(t)}$ und $\widehat{S}^{(t)}$ die übergeordneten "connected" User- und Provider-Pools und $\widehat{u}_{*} \in \widehat{U}^{(t)}$ und $\widehat{s}_{*} \in \widehat{S}^{(t)}$ die entsprechenden Teilnehmer, für deren individuelle "connected Pools" wir uns an dieser Stelle interessieren. Wir definieren die genannten "connected Pools" als

\begin{align*}
&accounts : \widehat{U}^{(t)} \rightarrow \mathcal{P}\left(\widehat{S}^{(t)}\right) \\
&accounts^{(t)}(\widehat{u}_{*}) = \left\{\widehat{s} \in \widehat{S}^{(t)} \textrm{ } | \textrm{ } \alpha^{(t)}(\widehat{u}_{*}, \widehat{s}) = 1 \right\}
\end{align*}
und

\begin{align*}
&userbase : \widehat{S}^{(t)} \rightarrow \mathcal{P}\left(\widehat{U}^{(t)}\right) \\
&userbase^{(t)}(\widehat{s}_{*}) = \left\{\widehat{u} \in \widehat{U}^{(t)} \textrm{ } | \textrm{ } \alpha^{(t)}(\widehat{u}, \widehat{s}_{*}) = 1 \right\}
\end{align*}

\end{Def}

\vspace{0.6cm}

\begin{Lemma}\label{lemmaPools}

\begin{align}
\bigcup_{\widehat{u} \in \widehat{U}^{(t)}} \left(accounts^{(t)} (\widehat{u})\right) = \widehat{S}^{(t)} \tag{i} \\
\bigcup_{\widehat{s} \in \widehat{S}^{(t)}} \left(userbase^{(t)} (\widehat{s})\right) = \widehat{U}^{(t)} \tag{ii}
\end{align}

\end{Lemma}

\vspace{0.3cm}

\begin{proof}[Beweis] \textrm{ }

\vspace{0.3cm}

zu (i): Es ist

\begin{align*}
\bigcup_{\widehat{u} \in \widehat{U}^{(t)}} \left(accounts^{(t)} (\widehat{u})\right) &\overset{\text{Def \ref{defTeilnehmerPool}}}{=} \bigcup_{\widehat{u} \in \widehat{U}^{(t)}} \left\{\widehat{s} \in \widehat{S}^{(t)} \textrm{ } | \textrm{ } \alpha^{(t)}\left(\widehat{u}, \widehat{s}\right) = 1 \right\} \\
&\overset{(*)}{=} \left\{\widehat{s} \in \widehat{S}^{(t)} \mid \exists \widehat{u} \in \widehat{U}^{(t)} \textrm{ mit } \alpha^{(t)}\left(\widehat{u}, \widehat{s}\right) = 1 \right\} \\
&\overset{\text{Th. \ref{theoremPools} (\ref{theoremPools_6})}}{=} \widehat{S}^{(t)}
\end{align*}
Die Gleichheit zu (*) ergibt aus der Tatsache, Mengen-Vereinigungen ignorierten Doppelzählungen! 

\vspace{0.3cm}

Aussage (ii) folgt ganz analog!

\end{proof}

\vspace{0.3cm}


\begin{Theorem}\label{theremPoolsCount}

\begin{equation*}
\sum_{\widehat{u} \in \widehat{U}^{(t)}} |accounts^{(t)} (\widehat{u})| = \sum_{\widehat{s} \in \widehat{S}^{(t)}} |userbase^{(t)} (\widehat{s})| = \Gamma(t)
\end{equation*}

\end{Theorem}

\vspace{0.3cm}

\begin{proof}[Beweis] \textrm{ }

\vspace{0.3cm}

Es ist

\begin{align*}
\Gamma(t)&\overset{\text{Th. \ref{theremConnectionsCount} (\ref{theremConnectionsCount_1})}}{=} \sum_{i=1}^{\widehat{n}} \sum_{j=1}^{\widehat{m}} \widehat{\alpha}^{(t)}_{ij} \\
&= \sum_{\widehat{u} \in \widehat{U}^{(t)}} \textrm{  } \sum_{\widehat{s} \in \widehat{S}^{(t)}} \alpha^{(t)}\left(\widehat{u}, \widehat{s}\right) \\
&= \sum_{\widehat{u} \in \widehat{U}^{(t)}} \textrm{  } \sum_{\widehat{s} \in \widehat{S}^{(t)} \textrm{ mit } \alpha^{(t)}\left(\widehat{u}, \widehat{s}\right) = 1} 1 \\
&\overset{\text{Def \ref{defTeilnehmerPool}}}{=} \sum_{\widehat{u} \in \widehat{U}^{(t)}} \textrm{  } \sum_{s \in accounts^{(t)} (\widehat{u})} 1 \\
&= \sum_{\widehat{u} \in \widehat{U}^{(t)}} |accounts^{(t)} (\widehat{u})|
\end{align*}

\vspace{0.3cm}

Die zweite Gleichheit folgt analog, falls man die Kommutativität der Def \ref{defGamma} beachtet: 

\begin{equation*}
  \Gamma(t)= \sum_{i=1}^n \sum_{j=1}^m \alpha^{(t)}_{ij} = \sum_{j=1}^m \sum_{i=1}^n \alpha^{(t)}_{ij}
\end{equation*}

\end{proof}


\vspace{0.6cm}
\todo{[ende TODO4]}
\vspace{1cm}\todo{[TODO4]["individuelle" User- und Provider-Pools]}
\vspace{0.3cm}

Nicht alle Teilnehmer innerhalb der WunderPass-Netzwerks sind gleichbedeutend. Dies ist zweifelsfrei klar hinsichtlich der Unterscheidung zwischen connecteten Usern $\widehat{u} \in \widehat{U}^{(t)}$ und Service-Providern $\widehat{s} \in \widehat{S}^{(t)}$. Jedoch bestehen ebenfalls signifikante Unterschiede jeweils innerhalb der beiden Teilnehmerklassen $\widehat{U}^{(t)}$ und $\widehat{S}^{(t)}$ (siehe \todo{[TODO4]["Sättigung"]}). Um dieser Unterscheidung unserer Teilnehmer gerecht zu werden, definieren wir "connected Pools" pro individuellen Teilnehmer als

\vspace{0.3cm}

\begin{Def}\label{defTeilnehmerPool}

Sei $t \in T$, $\widehat{U}^{(t)}$ und $\widehat{S}^{(t)}$ die übergeordneten "connected" User- und Provider-Pools und $\widehat{u}_{*} \in \widehat{U}^{(t)}$ und $\widehat{s}_{*} \in \widehat{S}^{(t)}$ die entsprechenden Teilnehmer, für deren individuelle "connected Pools" wir uns an dieser Stelle interessieren. Wir definieren die genannten "connected Pools" als

\begin{align*}
&accounts : \widehat{U}^{(t)} \rightarrow \mathcal{P}\left(\widehat{S}^{(t)}\right) \\
&accounts^{(t)}(\widehat{u}_{*}) = \left\{\widehat{s} \in \widehat{S}^{(t)} \textrm{ } | \textrm{ } \alpha^{(t)}(\widehat{u}_{*}, \widehat{s}) = 1 \right\}
\end{align*}
und

\begin{align*}
&userbase : \widehat{S}^{(t)} \rightarrow \mathcal{P}\left(\widehat{U}^{(t)}\right) \\
&userbase^{(t)}(\widehat{s}_{*}) = \left\{\widehat{u} \in \widehat{U}^{(t)} \textrm{ } | \textrm{ } \alpha^{(t)}(\widehat{u}, \widehat{s}_{*}) = 1 \right\}
\end{align*}

\end{Def}

\vspace{0.6cm}

\begin{Lemma}\label{lemmaPools}

\begin{align}
\bigcup_{\widehat{u} \in \widehat{U}^{(t)}} \left(accounts^{(t)} (\widehat{u})\right) = \widehat{S}^{(t)} \tag{i} \\
\bigcup_{\widehat{s} \in \widehat{S}^{(t)}} \left(userbase^{(t)} (\widehat{s})\right) = \widehat{U}^{(t)} \tag{ii}
\end{align}

\end{Lemma}

\vspace{0.3cm}

\begin{proof}[Beweis] \textrm{ }

\vspace{0.3cm}

zu (i): Es ist

\begin{align*}
\bigcup_{\widehat{u} \in \widehat{U}^{(t)}} \left(accounts^{(t)} (\widehat{u})\right) &\overset{\text{Def \ref{defTeilnehmerPool}}}{=} \bigcup_{\widehat{u} \in \widehat{U}^{(t)}} \left\{\widehat{s} \in \widehat{S}^{(t)} \textrm{ } | \textrm{ } \alpha^{(t)}\left(\widehat{u}, \widehat{s}\right) = 1 \right\} \\
&\overset{(*)}{=} \left\{\widehat{s} \in \widehat{S}^{(t)} \mid \exists \widehat{u} \in \widehat{U}^{(t)} \textrm{ mit } \alpha^{(t)}\left(\widehat{u}, \widehat{s}\right) = 1 \right\} \\
&\overset{\text{Th. \ref{theoremPools} (\ref{theoremPools_6})}}{=} \widehat{S}^{(t)}
\end{align*}
Die Gleichheit zu (*) ergibt aus der Tatsache, Mengen-Vereinigungen ignorierten Doppelzählungen! 

\vspace{0.3cm}

Aussage (ii) folgt ganz analog!

\end{proof}

\vspace{0.3cm}


\begin{Theorem}\label{theremPoolsCount}

\begin{equation*}
\sum_{\widehat{u} \in \widehat{U}^{(t)}} |accounts^{(t)} (\widehat{u})| = \sum_{\widehat{s} \in \widehat{S}^{(t)}} |userbase^{(t)} (\widehat{s})| = \Gamma(t)
\end{equation*}

\end{Theorem}

\vspace{0.3cm}

\begin{proof}[Beweis] \textrm{ }

\vspace{0.3cm}

Es ist

\begin{align*}
\Gamma(t)&\overset{\text{Th. \ref{theremConnectionsCount} (\ref{theremConnectionsCount_1})}}{=} \sum_{i=1}^{\widehat{n}} \sum_{j=1}^{\widehat{m}} \widehat{\alpha}^{(t)}_{ij} \\
&= \sum_{\widehat{u} \in \widehat{U}^{(t)}} \textrm{  } \sum_{\widehat{s} \in \widehat{S}^{(t)}} \alpha^{(t)}\left(\widehat{u}, \widehat{s}\right) \\
&= \sum_{\widehat{u} \in \widehat{U}^{(t)}} \textrm{  } \sum_{\widehat{s} \in \widehat{S}^{(t)} \textrm{ mit } \alpha^{(t)}\left(\widehat{u}, \widehat{s}\right) = 1} 1 \\
&\overset{\text{Def \ref{defTeilnehmerPool}}}{=} \sum_{\widehat{u} \in \widehat{U}^{(t)}} \textrm{  } \sum_{s \in accounts^{(t)} (\widehat{u})} 1 \\
&= \sum_{\widehat{u} \in \widehat{U}^{(t)}} |accounts^{(t)} (\widehat{u})|
\end{align*}

\vspace{0.3cm}

Die zweite Gleichheit folgt analog, falls man die Kommutativität der Def \ref{defGamma} beachtet: 

\begin{equation*}
  \Gamma(t)= \sum_{i=1}^n \sum_{j=1}^m \alpha^{(t)}_{ij} = \sum_{j=1}^m \sum_{i=1}^n \alpha^{(t)}_{ij}
\end{equation*}

\end{proof}


\vspace{0.6cm}
\todo{[ende TODO4]}
\vspace{1cm}





















\subsubsection{Other Stuff}
\label{sec:eco_zahlen_zustand_todo}

\vspace{2cm}
\todo{[TODO2][Abschätzung $\frac{\widehat{n}}{\widehat{m}}$]}
\vspace{0.3cm}

Aussagen aus Annahme \ref{assumptionRatio} und Theorem \ref{theremConnectionsCount} - Aussage (ii) - verwerten und Annahme \ref{assumptionRatio} deutlich verschärfen.

\todo{[ende TODO2]}
\vspace{1cm}


\todo{[TODO3]["Verdichtung"]}
\vspace{0.3cm}

Die Maße $\widehat{n}$, $\widehat{m}$ und $\Gamma(t)$ sind sehr objektiv und teils zielführend. Sie scheinen aber nicht zu reichen. So kann es z. B. User $\widehat{u} \in \widehat{U}^{(t)}$ geben, die im worst case ausschließlich zu einem einzigen Provider $\widehat{s} \in \widehat{S}^{(t)}$ connectet sind (und somit aber trotzdem den Wert von $\widehat{n}$ beeinflussen, oder noch schlimmer analog Provider $\widehat{s} \in \widehat{S}^{(t)}$, die als "angebunden" gelten, weil sie mit marginal wenigen Usern (im worst case mit einem einzigen) connectet sind. Solche Teilnehmer spielen eigentlich für den zahlenmäßigen WunderPass-Fortschritt keinerlei Rolle, beeinflussen jedoch unsere relevanten Messgrößen (KPI).

Von daher benötigen wir noch eine weitere Präzisierung in Form von

\begin{itemize}
  \item "80-20-Regel" heranziehen, indem man die Mengen $\widehat{U}^{(t)}$ und $\widehat{S}^{(t)}$ so modifiziert/verkleinert, dass $\Gamma(t)$ davon kaum einen Einfluss spürt (sich lediglich um einen marginalen Prozentsatz verringert).
  \item Formeln auf die davon modifizierten Größen $\widehat{\widehat{n}}$ und $\widehat{\widehat{m}}$ anpassen.
  \item $\Rightarrow$ Die Grenzen von [Theorem \ref{theremConnectionsCount}][Aussage (ii)] werden damit deutlich schärfer.
  \item $\Rightarrow$ kann sicherlich in Abschnitt \ref{sec:eco_zahlen_business_plan} für den Umgang mit dem Verhältnis $\frac{\widehat{n}}{\widehat{m}}$ genutzt werden.
  \item Wird vermutlich auch Relevanz bei den "individuellen" (Definition erfolgt noch) User- und Provider-Pools zum Einsatz kommen.
\end{itemize}

\todo{[ende TODO3]}
\vspace{1cm}

\todo{[TODO4.1]["Exklusive Connections"]}
\vspace{0.3cm}

\begin{itemize}
  \item Eine Connection zu einem Service-Provider ist exklusiv, wenn der zugehörige User zu keinem anderen Service-Provider connectet ist.
    \item Es gibt mindestens $n_{nexcl} \geq \Gamma(t) - \widehat{n}$ nicht exklusive Connections.
  
\end{itemize}

\todo{[ende TODO3.1]}
\vspace{1cm}











% !TEX root = paper.tex

\todo{[TODO6][deprecated Inhalt verarbeiten]}
\vspace{0.3cm}

Mit diesen geschaffenen Formalisierungs-Werkzeugen lässt sich nun auch die übergeordnete WunderPass-Vision formal erfassen - und zwar indem man den Zeitpunkt $t_{*} \in T$ ihrer Erreichung benennt:

\begin{Def}\label{defVision}

Wir betrachten die WunderPass-Vision zu einem Zeitpunkt $t_{*} \in T$ als erreicht, falls

\vspace{0.3cm}

\begin{equation}
\label{eq:1}
  \alpha^{(t_{*})}_{ij} = 1 \textrm{ für alle } i \in \{1,...,n\} \textrm{ und } j \in \{1,...,m\}
\end{equation}\\
erfüllt ist. Darüber hinaus ist es noch nicht ganz klar, welche Aussage für die Zeitpunkte $t > t_{*}$ hinsichtlich der Visions-Erreichung zu treffen sei. Grundsätzlich ist es ja durchaus denkbar, die obige Voraussetzung gelte für $t > t_{*}$ nicht mehr. Bleibt die Vision in diesem Fall trotzdem als 'erreicht' zu betrachten?

\end{Def}

\vspace{1cm}

Die gelungene Formalisierung unserer Vision mittels Definition \ref{defVision} mag einen Fortschritt hinsichtlich unserer "Business-Mathematics" darstellen, bleibt jedoch losgelöst zunächst einmal ziemlich wertlos. Zum einen ist das Erreichen der Vision im formellen Sinne der Definition \ref{defVision} weder praxistauglich noch akribisch erforderlich. Zudem bleibt zum anderen der resultierende (intrinsische) Business-Value der Visions-Erreichung bisher weiterhin nicht ohne Weiteres erkennbar.
Vielmehr sollten wir die Anforderung von Gleichung \eqref{eq:1} als eine Messlatte unseres Fortschritts heranziehen, und eher als (unerreichbare) 100\%-Zielerreichungs-Marke betrachten. Zudem müssen wir zeitnah - obgleich die vollständige oder nur fortschreitend partielle - Zielerreichung unserer Vision in klaren, quantifizierbaren Business-Value übersetzen.

Dazu definieren wir als erstes ein intuitives Maß der Zielerreichung:

\vspace{0.3cm}

\begin{Def}\label{defGamma2}

\begin{equation*}
  \Gamma : T \rightarrow \mathbb{N} 
\end{equation*}

\begin{equation*}
  \Gamma(t):= \sum_{i=1}^n \sum_{j=1}^m \alpha^{(t)}_{ij} 
\end{equation*}

\end{Def}

\vspace{1cm}

Damit liefert uns die definierte $\Gamma$-Funktion aber auch ein extrem greifbares und intuitiv nachvollziehbares Fortschrittsmaß unseres Vorhabens. Zudem fügt sich dieses perfekt in unsere mittels Definition \ref{defVision} quantifizierte Unternehmens-Vision und unterliegt einer fundamentalen (bezifferbaren) Obergrenze. Dies zeigt folgendes Lemma:

\vspace{0.3cm}

\begin{Lemma}

Es gelten folgende Aussagen:

\begin{align}
\Gamma(t) &\leq n^{(t)} * m^{(t)} \textrm{ für alle } t \in T \tag{i} \label{eq:l1_erste} \\ 
  \text{es gilt Gleichheit bei }  \eqref{eq:l1_erste} &\Leftrightarrow \text{ es gilt Gleichung } \eqref{eq:1} \text{ aus Def } \ref{defVision} \tag{ii} \label{eq:l1_zweite}
\end{align}

\end{Lemma}

\vspace{0.3cm}

Gleichung \eqref{eq:l1_zweite} ermöglicht uns die Definition \ref{defGamma} auf ein relatives Zielereichungs-Maß auszuweiten:

\vspace{0.3cm}

\begin{Def}\label{defKleinGamma}
\begin{equation*}
  \gamma : T \rightarrow [0; 1] 
\end{equation*}

\begin{equation*}
  \gamma(t):= \frac{\Gamma(t)}{n^{(t)} * m^{(t)}}
\end{equation*}

\end{Def}

\todo{[ende TODO6]}
\vspace{1.5cm}    % binde die Datei '[Economics][Quantifizierung][deprecated].tex' ein






\todo{Ab hier WIP}
\vspace{1cm}


\subsubsection{Business-Plan in Mathematics}
\label{sec:eco_zahlen_business_plan}

Diese letzten Werkzeuge lassen und Begriffe wie "Zielsetzung" bzw. "Milestone" einführen.

\vspace{0.3cm}

\begin{Def}\label{defZiel}

Seien $t \in T$ und zudem entsprechend $(n^{(t)}, m^{(t)}) = dP^{(t)}$ der angenommene Zustand der digitalen Welt zu einem beliebig gewählten Zeitpunkt. Wir definieren die - allein durch WunderPass zu bestimmende - Zielfunktion als

\end{Def}


\subsubsection{Quantifizierung des Status quo}
\label{sec:eco_zahlen_status_quo}



\paragraph{Vernetzung \& Netzwerk-Effekt}
\label{sec:zahlen_status_quo_netzwerk_effekt}

\textrm{ }
\vspace{0.3cm}

Die WunderPass-Vision steht in ihrer Formulierung ganz klar im Sinne einer gewissen "Vernetzung". Wir möchten, dass möglichst viele User sich mit möglichst vielen Service-Providern "connecten" (bzw. connectet sind/bleiben). Schränkt man seine Sichtweise alleinig auf diese Vision (ohne diese zunächst zu hinterfragen), liefern uns die zuletzt eingeführten Größen $\alpha^{(t)}_{ij}$, $\Gamma(t)$ und $\gamma(t)$ ziemlich gute Gradmesser, um zweifelsfreie Aussagen hinsichtlich der Vergleichbarkeit zweier Zeitpunkte $t_1, t_2 \in T$ treffen zu können. Es ist irgendwie klar, $\alpha^{(t)}_{ij} = 1$ sei im Sinne unserer Vision irgendwie besser als $\alpha^{(t)}_{ij} = 0$.

Aus diesem Blickwinkel (in dem die Vision zunächst ein Selbstzweck bleibt) erscheint die folgende Definition mehr als intuitiv einleuchtend, um die obige Formulierung "irgendwie besser" zu formalisieren und vor allem zu quantifizieren. 



















\vspace{0.3cm}

\todo{WIP:}
Hier stand vorher Definition \ref{defRelation}

\vspace{1cm}

Man beachte, dass es sich bei der definierten Ordnungsrelation gar um eine \href{https://de.wikipedia.org/wiki/Ordnungsrelation#Totalordnung}{Totalordnung} handelt!
Der Form halber ergänzen wir an der Stelle noch um zwei weitere - schematisch induzierte - Relationen auf unserem Zeitstrahl $T$:

\vspace{0.3cm}

\todo{WIP:}
Hier stand vorher Definition \ref{defRelationen}

\vspace{0.3cm}

Auch für die letzten beiden Relationen wollen wir fortan die vereinfachte Schreibweise $t_1 \prec t_2$ und $t_1 \simeq t_2$ nutzen.
























 

\vspace{1cm}

Diese Netzwerk-Bewertungs-Modell besitzt jedoch im aktuellen Zustand drei wesentliche Schwachstellen:

\begin{itemize}
  \item Es beschreibt uns misst weiterhin ausschließlich den intrinsischen Wert der Vernetzung innerhalb unserer kleinen "Visions-Welt", dem es noch an Bezug zur "Außenwelt" und dem Business-Case fehlt. Diesen Umstand wollen wir weiterhin zunächst einmal ignorieren.
  \item Es bewertet in der aktuellen Form ausschließlich "unsere Welt" bzw. unseren Fortschritt als Ganzes. Die definierte "besser"-Relation misst das "Besser" aus Sicht der Allgemeinheit. Der einzelne Teilnehmer bleibt individuell unberücksichtigt. Es ist schwer vorstellbar, ein Ökosystem zu designen, welches intrinsisch nach dem Wohl/Optimum Aller strebt (und damit eben einmal einen formalen Beweis für das Funktionieren des Kommunismus zu liefern.) 
  \item Es lässt den sogenannten \href{https://de.wikipedia.org/wiki/Netzwerkeffekt}{Netzwerkeffekt} außer Acht! Denn selbst wenn man eben einmal das Problem des Bullet 1 aus der Welt schafft, und ein Preisschild an den Mehrwert einer Connection zwischen User und Provider bekommt. Die Literatur zum besagten Netzwerkeffekt liefert gute Argumente für die Annahme, eine von uns anvisierte User-Provider-Connection kann nur sehr selten alleinstehend in ihrem Mehrwert bewertet werden. Vielmehr bemisst sich dieser etwaige Mehrwert in dem Zusammenspiel und den Synergien mit anderen User-Provider-Connections. Es lassen sich viele Beispiele finden, um diesen Umstand zu begründen. So kann es z. B. sein, dass ein Finance-Aggregator-Service für einen User um so wertvoller wird, je mehr Finance-Provider der User selbst mit seiner WunderIdentity connectet. Hierbei wird es kaum einen Unterschied für ihn machen, ob die genannten Finance-Provider mit 100 anderen WunderPass-Usern connectet seien oder mit 10 Mio. Im Case einer Splitwise-Connection (oder auch einer etwaigen EventsWithFriends-App) dagegen entsteht der Mehrwert erst dann, wenn auch ganz viele Freunde des Users diese Splitwise-Connection mit WunderPass besitzen. Andernfalls beläuft sich der Mehrwert seiner eigenen Connection so ziemlich gen Null.
\end{itemize}

\vspace{1cm}

Insbesondere der letzte Punkt wirft einige interessante Fragen auf, zu denen wir eine Antwort finden werden müssen. Oder zumindest Hypothesen und Annahmen treffen.
Was bedeutet eigentlich

\begin{equation*}
  \alpha^{(t)}_{kj} * \alpha^{(t)}_{lj} = 1 \textrm{ für zwei User } u^{(t)}_k, u^{(t)}_l \in U^{(t)} \textrm{ die beide mit Privider } s^{(t)}_j \in S^{(t)} \textrm{ connectet sind?}
\end{equation*}
Sind diese dann damit gleichbedeutend in irgendeiner Weise ebenfalls "\textit{miteinander connectet}"? Und was würde eine solche Implikation für unser bisheriges Modell bedeuten? Wie (un)abhängig ist eine solche "indirekte Connection" von ihrer "Brücke" - dem Service-Provider? All diese Fragen lassen sich zudem analog auf "indirekte Connections" zwischen Providern übertragen - die dann etwaige User als "Brücke" nützten. Zu guter Letzt ließe sich diese neue Komplexität beliebig potenzieren, indem man mittels Rekursion indirekte Connections "zweitens, drittens,... Grades" definiert.

Um der aufkommenden Komplexität Herr zu werden, wollen wir uns zunächst einmal dem zweiten der oben genannten Schwachstellen unseres bisherigen Modells zuwenden, und dieses idealerweise dahingehend erweitern, auch individuelle Bewertungen unserer Teilnehmer $u \in U^{(t)}$ und $s \in S^{(t)}$ zu erfassen.


\subsubsection{Individuelle Wertschöpfung der Teilneher}
\label{sec:eco_zahlen_teilnehmer}

Hallo