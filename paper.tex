\documentclass[11pt]{scrartcl}


%---------Konfiguration------


%-----Sprache und Zeichen----
\usepackage[utf8]{inputenc}
% \usepackage{ucs}
% \usepackage[T1]{fontenc}

% Zeilenumbrüche der deutschen Sprache
% \usepackage[ngerman]{babel}


%---------Farben-------------
\usepackage[RGB]{xcolor}
\definecolor{dunkelgruen}{RGB}{0 136 0}
\newcommand\todo[1]{\textcolor{red}{#1}}


%---------Links/Refs---------
\usepackage[colorlinks=true, urlcolor=blue]{hyperref}


%---------Grafiken-----------
\usepackage{graphicx}
\usepackage{subfig}



%---------Sonstiges----------
\usepackage{parcolumns}
\usepackage{enumitem}



%---------Mathe--------------
\usepackage{amsthm, amssymb, mathtools}
% \usepackage{amsmath, amssymb, amstext ,lipsum}



%---------Umgebungen---------
\usepackage[framemethod=tikz]{mdframed}

%---------Business---------
\mdtheorem[
  linecolor=dunkelgruen,
  frametitlefont=\sffamily\bfseries\color{white},
  frametitlebackgroundcolor=dunkelgruen,
]{Business-Def}{Definition}

\mdtheorem[
  linecolor=dunkelgruen,
  frametitlefont=\sffamily\bfseries\color{white},
  frametitlebackgroundcolor=dunkelgruen,
]{Hypothese}{Hypothese}

\mdtheorem[
  linecolor=gray,
  frametitlefont=\sffamily\bfseries\color{white},
  frametitlebackgroundcolor=gray,
]{Praemisse}{Prämisse}

\mdtheorem[
  linecolor=red,
  frametitlefont=\sffamily\bfseries\color{white},
  frametitlebackgroundcolor=red,
]{Abgrenzung}{Abgrenzung}

\mdtheorem[
  linecolor=violet,
  frametitlefont=\sffamily\bfseries\color{white},
  frametitlebackgroundcolor=violet,
]{Quelle}{Quellen}

\mdtheorem[
  linecolor=violet,
  frametitlefont=\sffamily\bfseries\color{white},
  frametitlebackgroundcolor=violet,
]{Zitat}{Zitat}

\mdtheorem[
  linecolor=blue,
  frametitlefont=\sffamily\bfseries\color{white},
  frametitlebackgroundcolor=blue,
]{Fazit}{Conclusion}

\mdtheorem[
  linecolor=red,
  frametitlefont=\sffamily\bfseries\color{white},
  frametitlebackgroundcolor=red,
]{Problem}{Problem}

\mdtheorem[
  linecolor=cyan,
  frametitlefont=\sffamily\bfseries\color{black},
  frametitlebackgroundcolor=cyan,
]{Solution}{Lösung}

\mdtheorem[
  linecolor=dunkelgruen,
  frametitlefont=\sffamily\bfseries\color{white},
  frametitlebackgroundcolor=dunkelgruen,
]{Konzept}{Konzept}


\mdtheorem[
  linecolor=dunkelgruen,
  frametitlefont=\sffamily\bfseries\color{white},
  frametitlebackgroundcolor=dunkelgruen,
]{NFT-Prop}{NFT-Property}



%---------Mathe------------
\mdtheorem[
  linecolor=gray,
  frametitlefont=\sffamily\bfseries\color{white},
  frametitlebackgroundcolor=gray,
]{Def}{Definition}

\mdtheorem[
  linecolor=dunkelgruen,
  frametitlefont=\sffamily\bfseries\color{white},
  frametitlebackgroundcolor=dunkelgruen,
]{Theorem}{Theorem}

\mdtheorem[
  linecolor=blue,
  frametitlefont=\sffamily\bfseries\color{white},
  frametitlebackgroundcolor=blue,
]{Lemma}{Lemma}

\mdtheorem[
  linecolor=red,
  frametitlefont=\sffamily\bfseries\color{white},
  frametitlebackgroundcolor=red,
]{Assumption}{Annahme}

\mdtheorem[
  linecolor=violet,
  frametitlefont=\sffamily\bfseries\color{white},
  frametitlebackgroundcolor=violet,
]{Example}{Beispiel}

\mdtheorem[
  linecolor=cyan,
  frametitlefont=\sffamily\bfseries\color{black},
  frametitlebackgroundcolor=cyan,
]{Algo}{Algorithmus}




%---------Backup-------------



% Zähler:

% Eigenen Zähler erzeugen
%\newcounter{counter}
%\newcounter{sub_counter}[counter]

%Zähler initialisieren
%\setcounter{counter}{0}
%\setcounter{sub_counter}{0}

% Equationumgebung auf den Zähler umdefinieren
% \arabic sorgt für die Nummerierung mit arabischen Zahlen. 
% Alternativ wäre auch \roman für kleine römische Zahlen oder 
% \Roman für große römische Zahlen denkbar. 
% Auch Buchstaben sind mit \alph und \Alph möglich.
% \renewcommand{\theequation}{\roman{mycount}}

%\newcommand{\uproman}[1]{\uppercase\expandafter{\romannumeral#1}}
%\newcommand{\lowroman}[1]{\romannumeral#1\relax}
%\renewcommand{\theequation}{\roman{equation}}
%\renewcommand{\proofname}{Beweis}




% \section
% \subsection
% \subsubsection 
% \paragraph{Einleitende Worte}
% \subparagraph





 
%---------Dokument-----------

%---------Titel--------------
\title{WunderPass-White-Paper}
\author{G. Fricke, S.Tschurilin}
\date{\today{}, Berlin}
 
%---------Inhalt-------------
\begin{document}

\maketitle
\tableofcontents{}

% !TEX root = paper.tex
\section{Abstract}
\label{sec:abstract}
\todo{TODO: Abstract}    % binde die Datei 'Abstract.tex' ein
% !TEX root = paper.tex

\section{Einleitung}
\label{sec:einleitung}

% !TEX root = C:/Users/Slava/White-Paper/[02][Einleitung]/[Einleitung].tex

\subsection{Identität}
\label{sec:einleitung_identitaet}

Zunächst einmal eine gänzlich kontextfreie Sicht auf den Begriff "Identität" quasi "ganz von Null". Es folgt die \href{https://de.wikipedia.org/wiki/Identit%C3%A4t}{Wikipedia-Definition}:

\vspace{0.3cm}

\begin{Business-Def}[Identität]\label{defIdentity}

\textbf{Identität} ist die Gesamtheit der Eigentümlichkeiten (\textbf{"Gesamtheit persönlicher Eigenheiten"}), die eine Entität, einen Gegenstand oder ein Objekt kennzeichnen und als \textbf{Individuum} von anderen unterscheiden. In ähnlichem Sinn wird der Begriff auch zur \textbf{Charakterisierung von Personen} verwendet. […] So folgt die rechtliche \href{https://de.wikipedia.org/wiki/Identit%C3%A4tsfeststellung}{Identitätsfeststellung} den für \textbf{Inklusion} und \textbf{Exklusion} relevanten Markern moderner bürgerlicher Gesellschaften.

\end{Business-Def}

\vspace{0.3cm}

Die nahezu philosophische Auseinandersetzung mit dem allgemeinen Verständnis der Identität wollen wir an dieser Stelle nicht weiter vertiefen und verweisen stattdessen u. a. folgende Quellen:

\vspace{0.3cm}

\begin{Quelle}

\begin{itemize}
  \item \href{https://blockchainopedia.atlassian.net/wiki/spaces/RESEARCH/pages/81264861/Digitale+Identit+t+-+eine+allgemeine+Sicht}{Confluence}
  \item \href{https://vsis-www.informatik.uni-hamburg.de/getDoc.php/publications/191/InfTage_CPK.pdf}{Diplomarbeit - Christian Philip Kunze}
\end{itemize}

\end{Quelle}

\vspace{0.3cm}

\todo{TODO: Oben zitierter Confluence-Artikel ist natürlich nicht öffentlich. Ggf. sollte man relevante Dinge daraus hier einarbeiten und den Link entfernen. Insbesondere der im Confluence thematisierte Begriff der \textbf{Identitäts-Feststellung} könnte für unsere Zwecke von Relevanz sein.}

\vspace{0.3cm}

Die Deutung des Begriffs der \textbf{Identität} ist also ungemein stark abhängig von der Perspektive, aus der die Deutung erfolgt. Die Schaffung eines übergeordneten Identitäts-Verständnis - insbesondere unter Einbeziehung der \textbf{"Identitäts-Digitalisierung} - ist eine riesengroße Herausforderung. Aber gleichzeitig eine eben so große Chance. Da die eben angesprochene Digitalisierung aktuell nach wie vor größtenteils in staatlicher Hand liegt, im Folgenden noch eine weitere wesentliche (Perspektive-abhängige und etwas überspitzt formulierte) Identitäts-Definition: 

\vspace{0.3cm}

\begin{Business-Def}[gesellschaftliches/staatliches Verständnis der Identität]\label{defStaatIdentity}

Ich bin genau der, von dem mein Ausweis behauptet, ich sei es.

\end{Business-Def}

\vspace{0.5cm}
    % binde die Datei '[Einleitung][Identität].tex' ein
% !TEX root = C:/Users/Slava/White-Paper/[02][Einleitung]/[Einleitung].tex

\subsection{Verständnis der digitalen Identität}
\label{sec:einleitung_digitale_identitaet}

Der Begriff der Identität ist unheimlich vielschichtig und komplex. Er kann aber auch - bei Weglassen philosophischer und subtiler Sichtweisen - intuitiv gänzlich trivial aufgefasst werden. Zumindest in der realen (analogen) Welt:

Ich bin ich! Ich trete stets mit derselben Identität auf - ob im Freundeskreis, bei der Arbeit oder beim Elternabend. Die Rollen und die relevanten Identitätsmerkmale mögen sich bei unterschiedlichen Anlässen unterscheiden, aber es bleibt dieselbe Person. Wenn man sich Geld von einem Kollegen auf Arbeit leiht, kann man es ihm auch dann zurückgeben, wenn man sich zufällig im Restaurant trifft. Weil kein Zweifel an den Identitäten der beiden Betroffenen besteht. \textbf{Dies ist in der digitalen Welt ganz anders.}

\vspace{0.2cm}

Die Definition von digitaler Identität erscheint auf den ersten Blick nahezu trivial:

\vspace{0.3cm}

\begin{Business-Def}[Digitale Identität]\label{defDigIdentity}

Die digitale Identität ist nichts anderes als ein eindeutiger (technischer) Identifier/Username/Kundennummer - ein Primary Key in einer Datenbanktabelle, wo das vermeintliche Individuum zu einer "Entität" wird.

\vspace{0.2cm}

Angereichert wird der zum technischen Identifier gehörende Entitäts-Datensatz mit zusätzlichen Properties ganz im Sinne der obigen allgemeinen Identitäts-Definition \ref{defIdentity}. 

\end{Business-Def}

\vspace{0.3cm}

Mit ein wenig technischem Verständnis erkennt man sofort das aus der eben formulierten Definition resultierende Problem: \textbf{Diese ist nämlich in unserer aktuellen digitalen Welt alles andere als eindeutig}. Und zwar deshalb nicht, weil sie auf Datenmodellierungs-Ebene zu interpretieren ist, die jeder digitale Service-Provider für sich allein vornimmt. Der so simplen und unmissverständlich klaren Definition der \textit{digitalen Identität} fehlt also eine winzige Kleinigkeit, deren Fehlen das Verständnis der \textit{digitalen Identität} plötzlich von \textit{trivial} zu \textit{höchst komplex} hievt: \textbf{Der Forderung \textit{global eindeutig} zu sein.}

Dieser Umstand verstärkt konsequenterweise sogar das im vorigen Abschnitt bereits aufgegriffene Problem hinsichtlich des komplexen \textit{Identitäts-Verständnis}: \textbf{Das Identitäts-Verständnis ist stark Perspektive-abhängig}. Um dies zu verdeutlichen transformieren wir die (bereits unbefriedigende) Definition \ref{defStaatIdentity} in die digitale Welt und bekommen ein sehr sprechendes Analogon:

\vspace{0.3cm}

\begin{Business-Def}[Online-Account = (eine) digitale Identität]\label{defAccount}

Ich bin genau der, als den mich ein jeder Online-Provider in seinem Datenmodell modelliert.

\end{Business-Def}

\vspace{0.3cm}

Damit hat die Digitalisierung - gleichwohl sie die Mittel besäße, Abhilfe für viele Probleme im Kontext der \textit{Identität} beizusteuern - das \textbf{Identitäts-Verständnis} sogar noch komplexer gemacht, als es vorher schon war.

Zusammengefasst:

\vspace{0.3cm}

\begin{Fazit}

\begin{itemize}
  \item Eine \textbf{digitale Identität} entspricht einer (User-)Entität innerhalb der Datenmodells eines beliebigen digitalen Service-Provider.
  \item Es existiert keinerlei Forderung/Spezifikation/Konsens nach Einheitlichkeit oder gar Eindeutigkeit der \textbf{digitalen Identität}. Auf Grund dessen kann auch keinesfalls die Rede von \textbf{der} digitale Identität sein. Stattdessen besitzt ein Individuum zig - wenn nicht gar hunderte - digitale Identitäten. 
  \item Es existieren keinerlei "Querverweise" zwischen der Vielzahl der digitale Identitäten eines Einzelnen, die es erlauben würden, die vielen Online-Account (= digitale Identitäten) zu einer \textbf{einzigen digitalen Identität} zu konsolidieren.
\end{itemize}

\end{Fazit}

\vspace{0.3cm}

\todo{ab hier WIP}

\vspace{0.3cm}

\textbf{\textit{Anologie in die analoge Welt:}}

\vspace{0.3cm}

Aufgrund der gängigen Praxis nahezu aller Web-Service-Anbieter/Apps/Online-Shops existieren Zig - wenn nicht gar Hunderte - von digitalen Kopien meines Ichs. 

Das eine Ich darf nur in dem einen Laden einkaufen, das andere nur in dem anderen. Die unterschiedlichen Ichs haben unterschiedliche Kreditkarten dabei (hinterlegte oder akzeptierte Zahlungsmittel bei unterschiedlichen Anbietern) - manche gar keine (Zahlungsmittel wird nicht akzeptiert). Einige Ichs haben ihren Ausweis dabei (Ident-Verfahren durchgeführt), andere wieder nicht. Gleiches gilt für den Führerschein (Anmeldung bei unterschiedlichen CarSharings). Und während das eine Ich bereits einen frischen Führerschein dabei hat, hat das andere noch den abgelaufenen (lange nicht benutzt und Führerschein abgelaufen). Die Ichs haben teils unterschiedliche Telefonnummern oder unterschiedliche Email-Adressen. Manche Ichs haben ihr Telefon komplett vergessen. Manche Ichs sind bereits längt tot oder kurz davor (Account verstaubt oder vergessen, überhaupt einen zu besitzen). Die Ichs sind gut vernetzt (Telefonnummern, WhatsApp, Facebook, LinkedIn, Xing), aber die einen Ichs kennen manche Leute nicht, die die anderen Ich kennen und umgekehrt. Und wenn sie irgendwie doch von der letzten Party erkennen, wissen sie plötzlich den Namen des Gegenüber nicht mehr oder auch nicht, worüber man bei genannter Party gesprochen hat.

Das alles ist eine metaphorisch polemische Darstellung des digitalen Status quo den vorherrschenden schier unendlichen Multi-Accountings in der Web2.0-Welt. Jeder Account ist das Abbild meiner Identität in die digitale Welt. Es bin immer ich, der hinter jeder dieser Identitäten steht. Jede dieser digitalen Identitäten ist fraglos eine Identität im Sinne der Definition. Sie kann gar ein detailliertes und durchaus sehr vertrauenswürdigen Abbild sein - um Fake-Identitäten soll es hierbei gar nicht gehen - aber sie ist stets eine weitere Kopie. Ich lasse also Zig und Hunderte Kopien meines Selbst in die digitale Welt raus, ohne dass sie als die Kopie derselben echten Identität erkennbar sind.

Dies kann natürlich an vielen Stellen sogar von Vorteil sein.

Einige meiner Ichs sind auf so weit voneinander entfernten Kontinenten unterwegs, dass sie sich niemals treffen oder von dem gegenseitigen Geschehen beeinflusst werden (Amazon vs. CarSharing). Andere Ichs sind wiederum so schüchtern, dass sie sehr gerne unerkannt bleiben (Datenschutz/Privatsphäre). 

Alle meine Ichs, die aber stets ihre Brieftasche mit sich führen, werden gewissen Interesse daran haben, das dem einen dieser Ichs nicht das Bargeld, dem anderen die Kreditkarte und dem dritten der Ausweis fehlt. Sie würden gerne eine gemeinsame Brieftasche haben, in der ihre gemeinsame Identität für alle Zwecke bereitliegt.

\vspace{0.3cm}

\todo{TODO: Ggf. noch den Sign-Up/-In als Identifizierung einer Online-Identity einbeziehen und erklären.}
\vspace{0.5cm}
    % binde die Datei '[Einleitung][digitale Identität].tex' ein
% !TEX root = paper.tex
\subsection{Missstände der digitalen Identität}
\label{sec:einleitung_probleme_digitaler_identitaet}

\vspace{0.3cm}

Das im letzte Kapitel beleuchtete Verständnis der \textit{digitalen Identität} lässt bereits erahnen, dieses sei alles andere als optimal. Nicht aus technischer Sicht, nicht aus gesetzlicher Sicht und schon gar nicht aus Sicht des Anwenders. Profitierende Akteure des Status quo in diesem Kontext, sind bestenfalls diejenigen, die sich aufgrund einer etwaigen Vormachtstellung an Ineffizienzen des Gesamtsystems bereichern können, weil sie eben weniger Nachteile durch besagte Ineffizienzen erfahren als der restliche Markt. Also Google, Apple, Amazon, Facebook etc. Nur darf der Umstand, die größten Player da draußen, haben gar kein eigenes Interesse daran, das aktuelle Verständnis der \textit{digitalen Identität} (öffentlich) zu hinterfragen, nicht darüber hinwegtäuschen, das dieses tatsächlich alles andere als optimal und sehr wohl zu hinterfragen sei.

Dies liegt in erster Linie daran, dass die Einsicht zur Notwendigkeit einer sauberen Spezifikation der digitalen Identität erst viel später reifte, als ihre praktische Notwendigkeit. Spätestens mit dem massentauglichen Vormarsch des Web 2.0, mussten von so gut wie jedem Online-Dienst Userdaten modelliert werden. Da wären Gedanken, wie wir diese hier anstellen, hellseherisch gewesen. Die heutigen Definitionen \ref{defDigIdentity} und \ref{defAccount} entstanden also aus damaliger Sicht "by doing" und nicht etwa aus (dummen) Überlegungen.

\vspace{0.1cm}

Denn für Anbieter von Online-Diensten ist es schier unabdingbar, Daten des Users - also zumindest einen Teil der \textit{Identität} - zu erfassen: Sei es 

\begin{itemize}
  \item im Falle eines Versandhandels: \textbf{die Lieferadresse}
  \item im Falle der Absicherung gegenüber Jugendlichen: \textbf{die Altersfreigabe}
  \item im Falle von Entgeltforderungen: \textbf{Konto- oder Kreditkartendaten}
\end{itemize}
Auch die für das Marketing Verantwortlichen eines solchen Anbieters sind vielmals an einem \textbf{registrierten und wiedererkennbaren Kunden} und an dessen Kaufverhalten interessiert. [Der letzte Absatz folgte vielen Formulierungen der \href{https://vsis-www.informatik.uni-hamburg.de/getDoc.php/thesis/47/DA_Gordian_Kaulbarsch.pdf}{Diplomarbeit "Identitäten und ihre Schnittstellen auf Basis von Ontologien in einer dezentralen Umgebung"}]. 

\vspace{0.3cm}

Aber die ebenso suboptimale Fortentwicklung der eher ungesteuert geborenen \textit{digitalen Identität} blieb fortan nicht nur dem "Ist-Eben-So-Gewachsen" geschuldet.

Im Gegensatz zum User war es für den Dienstanbieter meist interessanter, \textbf{Informationen über die Nutzer an zentraler Stelle vorzuhalten}, deren Kontrolle ihm selbst oblag. Denn besagte Datenerfassung - gegeben durch freiwilligen oder gar erzwungen durch verpflichtend eingeforderten Daten-Input seitens des Anwenders - ermöglichte dem Dienstanbieter die Wiedererkennung und Verfolgung des Users, bzw. das Speichern und Auslesen von identifizierenden Dateien – sogenannten Cookies – und die Vergabe von zusätzlich identifizierenden Session-IDs. Auf diese Weise ließen und lassen sich heute noch extrem große Mengen an Daten erfassen, verknüpfen und systematisch auswerten. [Der letzte Absatz folgte vielen Formulierungen der \href{https://vsis-www.informatik.uni-hamburg.de/getDoc.php/thesis/47/DA_Gordian_Kaulbarsch.pdf}{Diplomarbeit "Identitäten und ihre Schnittstellen auf Basis von Ontologien in einer dezentralen Umgebung"}].

\vspace{0.3cm}

Ungeachtet dessen, wem oder was die besagte suboptimale "Geburt" und Fortentwicklung der digitalen Identität geschuldet sei, wollen wir im folgenden die konkreten Probleme und Missstände dieser aufarbeiten.

\vspace{0.3cm}

\begin{Problem}[fehlende Eindeutigkeit]

Das Problem der fehlenden Eindeutigkeit der Identität in der digitalen Welt wird am besten deutlich an dem Vergleich des sprachlichen Unterschieds zwischen den beiden Begriffen \textit{"dasselbe"} und \textit{"das Gleiche"}. Während ich im REWE-Supermarkt und am EasyJet-Terminal am Flughafen dieselbe Person darstelle, bin ich beim (online) REWE-Lieferdienst und beim Buchen eines Flugtickets auf der EasyJet-Homepage - aus Sicht der beiden Dienstleister - nur der gleiche Online-Konsument. Bestenfalls ist dies überhaupt erkennbar...

Ich bin mit \textit{denselben} Personen befreundet, mit denen ich auch gleichzeitig auf WhatsApp, Facebook, LinkedIn etc. connectet, ohne dass die sichere - geschweige denn zweifellos logisch implizierte - Gewissheit besteht, dass es sich tatsächlich stets um dieselbe Person handelt. Es könnte theoretisch ja auch ein Fake-Account sein (Facebook) oder längst veraltete Telefonnummer (WhatsApp), die sich hinter der geglaubten Identität verbirgt.

Die Sicherstellung der Eindeutigkeit erfolgt stets analog: Z. B. aus einem  (plausiblen) Chat-Verlauf bei WhatsApp oder einem Foto auf Instagram, wo man selbst drauf ist, was die geglaubte Identität beweist.

\vspace{0.2cm}

Dass diese \textit{analoge Verifizierung} aber nichts taugt, zeigt spätestens das Beispiel, dass ich sowohl eine KFZ-Führerschein- als auch eine Motorboots-Führerschein-Identität habend, bei einem Alkohol-Vergehen - was gesetzlich beide Identitäten beträfe - nur an derjenigen Identität belangt werde, die im direkten Zusammenhang mit dem Vergehen stand. Weil es eben oft bürokratisch und schwierig ist zwei \textit{gleiche} Datensätze aus unterschiedlichen digitalen Systemen zu \textit{derselben} Person zusammenzuführen. Weil eben Gleichheit keine Eindeutigkeit garantiert.

\vspace{0.2cm}

Verkörpert wird das Problem der fehlenden Eindeutigkeit in der digitalen Welt durch den sogenannten "Sign-Up", wo ich mich mal mit meiner Email-Adresse, mal mit meiner Telefonnummer, mal mit einem frei wählbaren Nickname und mal mit Google oder Facebook registrieren kann.

\end{Problem}

\vspace{0.3cm}


\begin{Problem}[Redundanz und fehlerbehaftete Daten]

Kann heutzutage noch irgendeiner zählen, wie oft er schon sein Email-Adresse eingeben musste, um sich irgendwo zu registrieren? Und das trotz sämtlicher Browser-Autovervollständigung. Wie oft seine Adresse bei Versandhandeln? Seine Kreditkarten-Nummer oder zumindest -CVC? Ebenso werden die meisten die Konsequenzen von Umzügen in eine neue Wohnung, den Wechsel der Telefonnummer oder den Verlust oder Ablauf einer Kreditkarte im Hinblick auf die bürokratischen Konsequenzen bei etwaigen Online-Diensten einzuordnen wissen. \textbf{Fuckup pur}.

Und das alles nur, weil unsere Daten abermals und abermals redundant von jedem Online-Service separat gespeichert werden. Ich ziehe nur einmal um, muss diese Info aber zig Mal mit Anderen teilen. ich verliere nur einmal meine Kreditkarte - und bekomme eine neue - muss dies aber an zig Stellen manuell aktualisieren. Ich wechsele meine Telefonnummer und es wird von 100 Kontakten trotzdem 10 geben, die mich deswegen nicht mehr erreichen können werden. Es wird Stellen geben, wo sich Typos in meine persönlichen Daten, meine Email-Adresse oder meine Telefonnummer einschleichen, von denen ich nichts ahne und andere Stellen, von denen ich noch nicht einmal mehr weiß, sie besäßen noch Daten von mir, die zu aktualisieren sind.

Dies ist nicht nur ein Problem beim User (Aufwand) sondern ebenso großes Problem beim Dienstleister (falsche Daten).

\end{Problem}

\vspace{0.3cm}


\begin{Problem}[mangelhafte UX]

\todo{TODO: ausformulieren}
Man muss sich ständig neu registrieren, Accounts und Passwörter managen etc.

\end{Problem}

\vspace{0.3cm}



\begin{Problem}[Datenschutz]

\todo{TODO: ausformulieren}
\begin{itemize}
  \item meine Daten liegen an zig/hunderten Stellen gespeichert
  \item Hacks sind an zig/hunderten Stellen möglich
\end{itemize}

\end{Problem}

\vspace{0.3cm}



\begin{Problem}[Daten werden nicht dort erfasst, wo sie gebraucht werden]

\todo{TODO: ausformulieren}
\begin{itemize}
  \item Daten werden an anderer Stelle erhoben als sie gebraucht werden --> Beispiel mit der Supermarkt-Kassiererin bzw. Fluggesellschaften
\end{itemize}

\end{Problem}

\vspace{0.3cm}


\begin{Problem}[Datenmissbrauch/Bereicherung]

\todo{TODO: ausformulieren}
Big Tech nutzt meine Daten, um daran Geld zu verdienen. Und ich werde nicht an der Wertschöpfung beteiligt.

\end{Problem}

\vspace{0.3cm}


\begin{Problem}[Abhängigkeit von Big Tech]

\todo{TODO: ausformulieren}
Derzeit dominieren zentrale ID-Provider wie Google und Facebook die Verwaltung von Identitätsdaten sehr vieler IT-Dienste weltweit, was zu einer großen Abhängigkeit unserer Gesellschaft in Bezug auf den Fortgang der Digitalisierung führt.

\end{Problem}

\vspace{0.3cm}


\begin{Problem}[Ungenutzte Möglichkeiten]

\todo{TODO: ausformulieren}
Daten-Querverweise $\rightarrow$ Beispiel anführen 

(zB aus \href{https://norbert-pohlmann.com/glossar-cyber-sicherheit/self-sovereign-identity-ssi/}{Vorlesung zu SSI})

\end{Problem}

\vspace{0.5cm}
    % binde die Datei '[Einleitung][Probleme].tex' ein

\vspace{0.5cm}
    % binde die Datei 'Einleitung.tex' ein
% !TEX root = paper.tex
\section{Vision}
\label{sec:vision}

\todo{TODO: Einleitung formulieren}

\vspace{0.3cm}

\todo{TODO: Aufgreifend aus vorigen Kapitel (verlinken)}

\vspace{0.3cm}

\begin{Solution}[fehlende Eindeutigkeit]

\todo{TODO: ausformulieren}
Kryptografische Identität ist eindeutig $\rightarrow$ private key

\end{Solution}

\vspace{0.3cm}


\begin{Solution}[Redundanz und fehlerbehaftete Daten]

\todo{TODO: ausformulieren}
Eine einzige globale Tabelle als Identity-Management-Service in der Blockchain

\end{Solution}

\vspace{0.3cm}


\begin{Solution}[mangelhafte UX]

\todo{TODO: ausformulieren}
Einen private key muss man nicht mehrfach registrieren.

\end{Solution}

\vspace{0.3cm}


\begin{Solution}[Datenschutz]

\todo{TODO: ausformulieren}
\begin{itemize}
  \item meine Daten liegen an einer einzigen Stelle gespeichert
  \item Daten sind verschlüsselt und unhackbar
  \item Derart ließen sich auch Identitätsdaten auf automatisierte Weise kontrolliert weitergeben. 'Kontrolliert' in diesem Zusammenhang bedeutet die Möglichkeit für den Anwender, selbst zu entscheiden, an wen er welche Daten wann und zu welchen Bedingungen übermittelt.
\end{itemize}

\end{Solution}

\vspace{0.3cm}


\begin{Solution}[Datenmissbrauch/Bereicherung]

\todo{TODO: ausformulieren}
Ich werde an der Verwendung meiner Daten monetär beteiligt (Token-Economics)

\end{Solution}

\vspace{0.3cm}


\begin{Solution}[Abhängigkeit von Big Tech]

\todo{TODO: ausformulieren}
Bei Self-Sovereign Identity (SSI) oder selbstbestimmter Identität kontrollieren und besitzen Nutzer ihre digitalen Identitäten und weitere verifizierbare digitale Nachweise (Verifiable Credentials (VC)), ohne hierfür auf eine zentrale Stelle, wie etwa Facebook oder Google, angewiesen zu sein. Sie sind somit komplett unabhängig von Dritt-Instanzen und entscheiden vollkommen eigenständig, wer welche Identitätsdaten zur Verfügung gestellt bekommt, da alle Identitätsdaten ausschließlich bei ihnen gespeichert werden. Dadurch ist ein einfacher, flexibler, sicherer und vertrauenswürdiger Austausch von manipulationssicheren digitalen Nachweisen zwischen Nutzer und Anwendungen möglich.

\end{Solution}

\vspace{0.3cm}


\begin{Solution}[Ungenutzte Möglichkeiten]

\todo{TODO: ausformulieren}
Daten-Querverweise $\rightarrow$ Beispiel anführen 

(zB aus \href{https://norbert-pohlmann.com/glossar-cyber-sicherheit/self-sovereign-identity-ssi/}{Vorlesung zu SSI})

\end{Solution}

\vspace{0.5cm}    % binde die Datei 'Vision.tex' ein
% !TEX root = paper.tex

\section{Umsetzung}
\label{sec:ansatz}
\todo{TODO}

\vspace{0.5cm}    % binde die Datei 'Unser Ansatz.tex' ein
% !TEX root = paper.tex

\section{Economics} 
\label{sec:economics}

% !TEX root = C:/Users/Slava/White-Paper/[05][Economics]/Economics.tex

\subsection{Einleitung}
\label{sec:eco_einleitung}

Wir beginnen mit einer gewagten Behauptung:

\vspace{0.2cm}

\begin{Hypothese}[Daten haben einen Wert]
\textbf{Digitale Daten besitzen einen realen} (nicht leicht zu beziffernden) \textbf{Value} - zumindest für die von ihnen direkt oder indirekt adressierten digitalen Individuen (User und Service-Provider). Dieser Value existiert bereits in Isolation des einzelnen Datensatzes, wird aber mit Zunahme der verfügbaren Gesamtdaten innerhalb eines Netzwerks durch entstehende Synergien nicht nur in Summe sondern zudem ebenfalls pro einzelnem Datensatz stets größer (Netzwerkeffekt). Eine einzelne Information - ein Datensatz - besitzt also bereits einen isolierten Mehrwert - anfangs vielleicht nur für sehr wenige Teilnehmer des Netzwerks - und gewinnt zudem zunehmend weiter an Wert mit Wachstum des "Wissens" des Gesamtnetzwerks und verhilft auch anderen Dateninformationen zu deren Wertsteigerung.

Wir behaupten damit also, jeder digitale Datensatz habe sogar einen Value über die von ihm adressierten digitalen Individuen hinaus. Und zwar für die Gesamtheit des Netzwerks und all seiner Teilnehmer. Dies jedoch natürlich nicht im gleichen Maße für alle.

\vspace{0.1cm}

Damit ist \textbf{digitale Datenerfassung und -auswertung wertschöpfend} für die gesamte digitale Welt und wünschenswert. Lediglich die \textbf{Verteilung des geschöpften Values muss hinterfragt werden}. Wir möchten ein Okösystem definieren, der genau dies gerecht und transparent tut.
 
\end{Hypothese}

\vspace{0.3cm}
Rein formal mathematisch betrachtet, ist die formulierte Behauptung ziemlich einleuchtend - zumindest wenn man auf die Forderung, dieser "Value" (sei er mit $v_{data}$ bezeichnet) habe ein positives Vorzeichen, verzichtet. Als "Value" also zunächst lediglich abstrakt einen "Impact" annimmt (der auch einen "Schaden" mit $v_{data} < 0$ darstellen könnte, wenn die Daten in irgendeiner Weise missbraucht werden). Von 

\begin{equation*}
\vert v_{data} \vert > 0
\end{equation*}

können wir also ziemlich bedenkenlos ausgehen.

\vspace{0.2cm}

Auf der gesellschaftlich/sozialen Ebene wird man dagegen deutlich mehr Widerspruch zur getätigten Hypothese ernten (bzw. von abstrakt-denken-könnenden Menschen zumindest das negative Vorzeichen von $v_{data}$ vorgehalten bekommen). Denn leider ist die elektronische Datenverarbeitung - in ihrem wahren Sinne des Wortes - ziemlich in Verruf geraten. Man hört den Tadel von mit Datenschutz in Verbindung gebrachten Missständen deutlich lauter als die Anerkennung des Nutzens von Datenerfassung und ihrer Verarbeitung - auf die im Übrigen so gut wie niemand mehr verzichten können würde. Weil Menschen eben schnell vergessen und zudem in der Regel kein ausreichendes technisches Verständnis besitzen. 

Kein Mensch würde doch heute wieder bei Taxizentralen anrufen wollen und seine Abholadresse durchgeben, weil GPS-Lokalisierungen unterbunden werden sollen. Gleiches bei Food-Lieferanten. Und auch nur die Wenigsten auf die Intelligenz von Google, weil Google nur das für uns tun kann, was sie für uns tun, weil sie das über uns wissen, was sie eben über uns wissen. Der Zug in diesem Kontext ist bereits unumkehrbar abgefahren. Weil eben selbst so gut wie in jeder Lebenssituation von dem besagten Daten-Value $v_{data}$ profitieren. Und zwar mit unbestreitbarem positiven Vorzeichen.

Was die besagten Datenschutz-Skeptiker da beanstanden, ist tatsächlich etwas ganz anderes als sie glauben: Es ist nicht die Datenerfassung, -verarbeitung und -monetarisierung, sondern die teils unfaire Verteilung von $v_{data}$ an die Netzwerkteilnehmer (insbesondere die beteiligten). Das ist auch genau das Problem, was wir mit WunderPass zu lösen versuchen.

\vspace{0.2cm}

Aber zunächst einmal zurück zu unserer einleitenden Hypothese. Sie formal zu beweisen ist äußerst schwer - wenn nicht gar unmöglich. Sie ist aber - laut unserer festen Überzeugung - trotzdem wahr, was wir an folgendem vielschichtigem (zugegeben ziemlich konstruiertem) Beispiel - bestehend aus "Journeys" mehrerer User - veranschaulichen möchten.

\vspace{0.3cm}

\begin{Example}[Ökosystem von Datensätzen] 

\vspace{0.2cm}

\underline{\textbf{Setup:}}

\begin{itemize}
  \item User A plant im Zeitraum x eine Reise von Berlin nach London.
  \begin{itemize}
  	\item User A hat bereits seinen Flug bei einem Flug-Provider gebucht (z. B. EasyJet).
  	\item User A hat ebenso ein Hotel gebucht (z. B. über HRS).
  	\item User A besitzt einen Airbnb-Account und hat in dem letzten Jahr bereits häufig Wohnungen im Ausland angemietet.
  \end{itemize}
  \item User B plant in demselben (oder zumindest stark überlappenden) Zeitraum x eine Reise von London nach Berlin.
  \begin{itemize}
  	\item User B hat ebenso seinen Flug gebucht - und zwar bei demselben Flug-Provider  wie User A.
  	\item User B hat für den Zeitraum seiner Abwesenheit seine Wohnung in London bei Airbnb zur Vermietung eingestellt, jedoch bisher kein Angebot erhalten.
  	\item User B hat für seinen Aufenthalt in Berlin ein Auto bei einem 
  	\newline Autovermietung-Provider (z. B. Sixt) reserviert.
  	\item User B scheint keinen Account bei Providern privaten Car-Sharings (wie Drivy) zu besitzen.
  \end{itemize}
  \item User C wohnt in Berlin, besitzt ein Auto, welches er im Zeitraum x (oder einem überlappenden Zeitraum) nicht benötigt, und es deshalb bei einem Provider von privatem Car-Sharing (z. B. Drivy) zur Vermietung angeboten, ohne jedoch bisher ein Angebot erhalten zu haben.
\end{itemize}

\vspace{0.3cm}

\underline{\textbf{Informationsgehalt \& -value:}}

\vspace{0.2cm}

Wir wollen an dieser Stelle den gänzlich offensichtlichen (und tendenziell isolierten) Informationsgehalt/Datenverarbeitung - wie z. B. Reservierungsbestätigungen, Rechnungen oder schlichtweg zusammenfassende "Reminder" - der im obigen Setup-Kontext stehenden Daten ignorieren und diese stattdessen in einem deutlich stärker "rausgezoomten" und übergeordnetem Kontext betrachten und auf mögliche Synergien auswerten.

\vspace{0.1cm} 

Im Folgenden eine punktuelle Zusammenfassung der relevanten Informationen bzw. vorliegenden Datensätzen unseres Beispiel-Szenarios - teils samt erfolgter Interpretation:

\vspace{0.2cm}

\begin{tabular}[h]{|c|c|c|c|c}
\hline
\textbf{insight} & \textbf{Information} & \textbf{time} & \textbf{data owner} \\
\hline
\textbf{info 1} & [Berlin $\rightarrow$ London] zu Zeitraum x & $x$ & EasyJet und User A \\
\hline
\textbf{info 2} & [London $\rightarrow$ Berlin] zu Zeitraum x & $x$ & EasyJet und User B \\
\hline
\textbf{info 3} & User A benötigt Unterkunft in London & $x$ & \parbox{3.5cm}{1st: User A \\ 2nd: EasyJet \& HRS} \\
\hline
\textbf{info 4} & User A hat Unterkunft in London & $x$ & HRS und User A \\
\hline
\textbf{info 5} & \parbox{5.7cm}{User A hätte theoretisch Interesse \\ an Airbnb- Wohnung in London} & $x$ & Airbnb und User A \\
\hline
\textbf{info 6} & \parbox{5.5cm}{User B sucht einen Mieter \\ für Wohnung in London} & $\approx x$ & Airbnb und User B \\
\hline
\textbf{info 7} & User B benötigt ein Auto in Berlin & $\approx x$ & Sixt und User B \\
\hline
\textbf{info 8} & \parbox{5.5cm}{User C möchte sein Auto \\ in Berlin vermieten} & $\approx x$ & Drivy und User C \\
\hline
\end{tabular}\vspace*{0.3cm}\\*

\vspace{0.3cm}

Aus obiger Auflistung wird bereits ersichtlich, worauf wir hier eigentlich hinauswollen: Nämlich die offensichtliche Tatsache, die tatsächliche "Journey" weiche möglicherweise stark von der optimalen "Journey" (optimal im Sinne der Gesamtheit aller betroffenen Teilnehmer unseres Beispiel-Cases) ab, weil kein \textit{vollumfängliches Wissen aller beteiligten Teilnehmer über alle Gegebenheiten besteht}. Das Problem hierbei ist schlichtweg die Tatsache, dass oben aufgezählte \textit{Insights} nur einigen der Teilnehmer bekennt sind, jedoch auch andere Teilnehmer betreffen. Wir können hierbei von \textit{Informations-Vor- und nachteilen} bestimmter Teilnehmer sprechen.

\vspace{0.2cm}

Angenommen obige \textit{Insights} lägen allen Teilnehmern vor. Dann ergäben sich folgende zusätzliche \textit{Insights}:

\vspace{0.2cm}

\begin{tabular}[h]{|c|c|c|c}
\hline
\textbf{insight} & \textbf{Information} & \textbf{owner} \\
\hline
\textbf{info 9} & User A könnte Wohnung von User B in London mieten & "Gott" \\
\hline
\textbf{info 10} & \parbox{10cm}{\textbf{gegeben Info 9:} \\ (1) Stornierungsrisiko der HRS-Buchung seitens User A  \\ (2) HRS könnte u. U. das Zimmer von User A gewinnbringender weitervermieten (bei großer Nachfrage)} & "Gott" \\
\hline
\textbf{info 11} & User B könnte den Wagen von User C in Berlin anmieten & "Gott" \\
\hline
\textbf{info 12} & \parbox{10cm}{\textbf{gegeben Info 11:} \\ (1) Stornierungsrisiko der Sixt-Reservierung seitens User B  \\ (2) Sixt könnte u. U. den reservierten Wagen von User B gewinnbringender vermieten (bei großer Nachfrage)} & "Gott" \\
\hline
\end{tabular}\vspace*{0.3cm}\\*

\vspace{0.2cm}

Zusammenfassend stellen wir den bisher aufgearbeiteten Informationsgehalt pro betroffenen Teilnehmer auf.

\vspace{0.1cm}

Legende: 

User A sei abgekürzt mit \textcolor{red}{\textbf{A}}, User B mit \textcolor{green}{\textbf{B}} und User C mit \textcolor{blue}{\textbf{C}}.

EasyJet sei mit \textcolor{violet}{\textbf{EJ}}, HRS mit \textcolor{dunkelgruen}{\textbf{HRS}}, Airbnb mit \textcolor{cyan}{\textbf{ABN}}, Sixt mit \textcolor{gray}{\textbf{SX}} und Drivy mit \textcolor{orange}{\textbf{DRV}}.

\vspace{0.2cm}

\begin{tabular}[h]{|c|c|c|c|c|c|c|c|c|c|c}
\hline
\textbf{info} & \textbf{owner} & \textcolor{red}{\textbf{A}} & \textcolor{green}{\textbf{B}} & \textcolor{blue}{\textbf{C}} & \textcolor{violet}{\textbf{EJ}} & \textcolor{dunkelgruen}{\textbf{HRS}} & \textcolor{cyan}{\textbf{ABN}} & \textcolor{gray}{\textbf{SX}} & \textcolor{orange}{\textbf{DRV}} \\
\hline
\textbf{1} & \textcolor{red}{\textbf{A}} + \textcolor{violet}{\textbf{EJ}} & $\surd$ & \textcolor{dunkelgruen}{+} & (o) & $\surd$ & \textcolor{dunkelgruen}{+} & \textcolor{dunkelgruen}{+} & \textcolor{dunkelgruen}{+} & \textcolor{dunkelgruen}{+} \\
\hline
\textbf{2} & \textcolor{green}{\textbf{B}} + \textcolor{violet}{\textbf{EJ}} & (o) & $\surd$ & \textcolor{dunkelgruen}{+} & $\surd$ & \textcolor{dunkelgruen}{+} & \textcolor{dunkelgruen}{+} & \textcolor{dunkelgruen}{+} & \textcolor{dunkelgruen}{+} \\
\hline
\textbf{3} & \parbox{1.8cm}{\textcolor{red}{\textbf{A}} + evtl. \\ (\textcolor{violet}{\textbf{EJ}}+\textcolor{dunkelgruen}{\textbf{HRS}})} & $\surd$ & \textcolor{dunkelgruen}{++} & (o) & (o) & \textcolor{dunkelgruen}{++} & \textcolor{dunkelgruen}{++} & (o) & (o) \\
\hline
\textbf{4} & \textcolor{red}{\textbf{A}} + \textcolor{dunkelgruen}{\textbf{HRS}} & $\surd$ & ? & (o) & (o) & $\surd$ & ? & (o) & (o) \\
\hline
\textbf{5} & \textcolor{red}{\textbf{A}} + \textcolor{cyan}{\textbf{ABN}} & $\surd$ & \textcolor{dunkelgruen}{++} & (o) & (o) & ? & $\surd$ & (o) & (o) \\
\hline
\textbf{6} & \textcolor{green}{\textbf{B}} + \textcolor{cyan}{\textbf{ABN}} & \textcolor{dunkelgruen}{++} & $\surd$ & (o) & (o) & (o) & $\surd$ & (o) & (o) \\
\hline
\textbf{7} & \textcolor{green}{\textbf{B}} + \textcolor{gray}{\textbf{SX}} & (o) & $\surd$ & \textcolor{dunkelgruen}{++} & (o) & (o) & (o) & $\surd$ & \textcolor{dunkelgruen}{++} \\
\hline
\textbf{8} & \textcolor{blue}{\textbf{C}} + \textcolor{orange}{\textbf{DRV}} & (o) & \textcolor{dunkelgruen}{++} & $\surd$ & (o) & (o) & (o) & (o) & $\surd$ \\
\hline
\textbf{9} & ----- & \textcolor{dunkelgruen}{+++} & \textcolor{dunkelgruen}{+++} & (o) & (o) & \textcolor{red}{- -} & \textcolor{dunkelgruen}{++} & (o) & (o) \\
\hline
\textbf{10} & ----- & $\surd$ & $\surd$ & (o) & (o) & \textcolor{dunkelgruen}{+++} & (o) & (o) & (o) \\
\hline
\textbf{11} & ----- & (o) & \textcolor{dunkelgruen}{+++} & \textcolor{dunkelgruen}{+++} & (o) & (o) & (o) & \textcolor{red}{- -} & \textcolor{dunkelgruen}{++} \\
\hline
\textbf{12} & ----- & (o) & $\surd$ & $\surd$ & (o) & (o) & (o) & \textcolor{dunkelgruen}{+++} & (o) \\
\hline
\end{tabular}\vspace*{0.3cm}\\*

\vspace{0.2cm}

\todo{Interpretation Verlierer/Gewinner der zusätzlichen Insights}:

\begin{itemize}
  \item Info 9 ist zwar absolut nicht im Sinne von HRS, kann HRS jedoch das Bekanntwerden dieser Info nicht verhindern, bekommt Info 10 für sie an signifikanter Relevanz (Value).
  \item Gleiches gilt für die Infos 11 und 12 aus Sicht von Sixt. 
\end{itemize}

\vspace{0.2cm}
\todo{Gesamt-Value mit zusätzlichen Insights vs. ohne}

\end{Example}

\vspace{0.3cm}

\todo{WIP}

\vspace{0.3cm}

\begin{Fazit}[unser Ökosystem generiert Value]

\begin{itemize}
  \item Wir schöpfen Mehrwert, indem wir Datenerfassung ermöglichen (die ja einen nachgewiesenen Value besitzen. \todo{Beispiele für Value durch Querverweise}
  \item Besitzer der Daten werden entlohnt.
  \item Nutzer der Daten zahlen für Daten, generieren damit aber Value, der wiederum entlohnt wird.
  \item Am Ende haben alle Teilnehmer entweder Value generiert oder aber im Wert des values verkonsumiert
  \item Wir partizipieren am extrinsischen Wert des Tokens (Kurs-Entwicklung durch positive Wertschöpfung des gesamten Ökosystems).
  \item Incentives sind nötig, um das Henne-Ei-Problem zu lösen
  \item Incentives sollten nachträglich mit der dadurch geschaffenen Wertschöpfung verrechtet werden. 
\end{itemize}

\end{Fazit}

\vspace{0.3cm}

\todo{TODO}    % binde die Datei '[Economics][Einleitung].tex' ein
% !TEX root = C:/Users/Slava/White-Paper/[05][Economics]/Economics.tex

\subsection{Goals}
\label{sec:eco_goals}
\todo{TODO}    % binde die Datei '[Economics][Goals].tex' ein
% !TEX root = paper.tex
\subsection{Quantifizierung}
\label{sec:eco_zahlen}
\todo{Einleitung - Start}

Wir wollen den Mehrwert von User-Provider-Connections mittels Wunderpass einen bezifferbaren Mehrwert verleihen und diesen fundiert argumentieren. Dazu müssen wir diesen Value messen und beziffern können. Die Ergebnisse dieses Kapitels werden insbesondere für das im Kapitel \ref{sec:wpt_reward_pool} beleuchteten "Reward-Pools" von großer Bedeutung sein. Bzw. sogar im gesamten übergeordneten Kapitel \ref{sec:eco_wpt}.
\todo{Einleitung - Ende}

% !TEX root = paper.tex
\subsubsection{Grundlegende Definitionen}
\label{sec:eco_zahlen_def}

Sei $t_0$ der initiale Zeitpunkt all unserer Messungen und Betrachtungen (vermutlich der Zeitpunkt des MVP-Launches).

Darauf aufbauend betrachten wir das künftige Zeitintervall $T$, welches einzig an Relevanz für unser Vorhaben und alle in diesem Kapitel getätigten Ausführungen besitzt:

\begin{equation*}
  T = [t_0; \infty[
\end{equation*}
Der Zeitstrahl muss nicht zwingend unendlich sein. Er muss ebenfalls nicht zwingend infinitesimal fortlaufend sein und kann stattdessen je nach Kontext endlich und/oder diskret betrachtet werden. Also z. B. auch wahlweise als 

\begin{equation*}
  T = [t_0; t_{ende}]
\end{equation*}

\begin{equation*}
  T = [t_0; t_1;...; t_{ende}]
\end{equation*}
definiert sein. In letzteren beiden Fällen wird jedoch $t_{ende}$ in aller Regel eine kontextbezogene (unverzichtbare) Bedeutung haben, die eine solche Definition des Zeitstrahls unverzichtbar macht. So könnte $t_{ende}$ z. B. für eine mathematisch quantifizierbare Erreichung unserer Vision stehen. \\

Sei $\mathbf{t \in T}$ fortan stets ein beliebiger Zeitpunkt, zu welchem wir eine Aussage treffen möchten. \\


Wir definieren die Anzahl aller zum Zeitpunkt $t$ potenziellen User $U^{(t)}$ überhaupt und ihre (maximale) Anzahl $n^{(t)}$ als \\

\begin{Def}\label{defU}
\begin{equation*}
  U^{(t)} = \left\{ u^{(t)}_1; u^{(t)}_2;...; u^{(t)}_{n} \right\}
\end{equation*}
\end{Def} 

\vspace{0.3cm}


Und ganz analog dazu ebenfalls die potenziellen Service-Provider $S^{(t)}$ und ihre (maximale) Anzahl $m^{(t)}$ als \\

\begin{Def}\label{defS}
\begin{equation*}
  S^{(t)} = \left\{ s^{(t)}_1; s^{(t)}_2;...; s^{(t)}_{m}\right\}
\end{equation*}
\end{Def}

\vspace{0.3cm}

Man beachte, dass die definierten Mengen $U^{(t)}$ und $S^{(t)}$ bzw. ihre Größe gewissermaßen den Fortschritt der Digitalisierung insgesamt beschreiben (potenzielle User brauchen einen Zugang zum digitalen Ökosystem und potenzielle Provider sind unabhängige Service-Dienstleister, die eigenmächtig darüber entscheiden, zu solchen zu werden) und in keiner Weise im Einfluss Wunderpasses stehen. Viel mehr beschreiben sie die "Umstände der Welt", mit denen WunderPass (wie alle anderen) "arbeiten" müssen.  

\vspace{0.6cm}


Nun definieren den \textbf{\textit{Connection-Koeffizienten}} zwischen den eben definierten potenziellen Usern $\mathbf{U^{(t)}}$ und den Service-Providern $\mathbf{S^{(t)}}$ zum Zeitpunkt $t$ als boolesche Funktion $\mathbf{\alpha^{(t)}}$, die über über die Tatsache \textit{"is connected"} bzw. \textit{"is not connected"} entscheidet: \\

\begin{Def}\label{defKoeff}

\begin{equation*}
  \alpha^{(t)} : U^{(t)} \times S^{(t)} \rightarrow \{0; 1\} 
\end{equation*}

\[
\alpha^{(t)}(u, s):=\left\{%
\begin{array}{ll}
    1, & \hbox{falls User $u \in U^{(t)}$ mit mit Provider $s \in S^{(t)}$ connectet ist} \\
    0, & \hbox{andernfalls} \\
\end{array}%
\right.
\]

\vspace{1cm}

Bzw. wenn man die diskreten Auslegungen der Pools $U^{(t)} = \left\{ u^{(t)}_1; u^{(t)}_2;...; u^{(t)}_{n} \right\}$ und $S^{(t)} = \left\{ s^{(t)}_1; s^{(t)}_2;...; s^{(t)}_{m} \right\}$ heranzieht, alternativ als

\[
\alpha^{(t)}_{ij}:=\left\{%
\begin{array}{ll}
    1, & \hbox{falls User $u^{(t)}_i \in U^{(t)}$ mit mit Provider $s^{(t)}_j \in S^{(t)}$ connectet ist} \\
    0, & \hbox{andernfalls} \\
\end{array}%
\right.
\]

\end{Def}

\vspace{0.6cm}

Man beachte, dass wir bei den diskreten/Aufzählungs-basierten Definitionen oben, der Übersicht halber etwas "geschlampt" haben, indem wir - klar zeitbedingte - Indizes stillschweigend als $n$ und $m$ bezeichnet haben, gleichwohl diese korrekterweise $n^{(t)}$ und $m^{(t)}$ lauten müssten. Nur verwirrt eben ein Ausdruck wie $u^{(t)}_{n^{(t)}}$ mehr, als dieser in seiner pedantischen Korrektheit einen Mehrwert generiert. Wir werden genannte Ungenauigkeit zudem im weiteren Verlauf in gleicher Weise fortführen und gehen davon aus, der Leser wisse damit umzugehen. 

\vspace{0.3cm}


    % binde die Datei '[Economics][Quantifizierung][Definitionen].tex' ein

\subsubsection{Zustandsbeschreibung der digitalen Welt}
\label{sec:eco_zahlen_zustand_digitalisierung}

Mit diesen geschaffenen Formalisierungs-Werkzeugen lassen sich nun einige Dinge formal deutlich besser greifen. Und zwar zum einen im Folgenden die übergeordneten "Umstände der digitalen Welt" (auf die WunderPass bestenfalls sehr geringfügig Einfluss üben kann) aber zum anderen ebenfalls unser gesamtes Vorhaben inklusive der übergeordneten WunderPass-Vision, die in den darauf folgenden Kapitels beleuchtet wird. 

\vspace{0.3cm}

Aufgrund der bereits weiter oben erwähnten nicht möglichen Einflussnahme auf die Mengen $U^{(t)}$ und $S^{(t)}$ benötigen wir noch ein weiteres Hilfsmittel, dessen Existenz wir im Folgenden einfach voraussetzen möchten - und diese mit Möglichkeiten der Markt-Analyse rechtfertigen.

\vspace{0.3cm}

\begin{Assumption}[Digitalisierungs-Orakel]\label{assumptionOrakel}

Sei $t \in T$. Anstatt die (nicht wirklich berechtigte) Kenntnis der Mengen $U^{(t)}$ und $S^{(t)}$ vorzugeben, wollen wir lieber die (realistischere) Existenz einer "Schätzfunktion" $dP^{(t)}$ (digital progress) annehmen. Wir definieren $dP^{(t)}$ als

\begin{align*}
dP &: T \rightarrow \mathbb{N} \times \mathbb{N}  \\
dP^{(t)} &:= \left(n^{(t)}, m^{(t)}\right)
\end{align*}
wobei $n^{(t)} = |U^{(t)}|$ und $m^{(t)} = |S^{(t)}|$ darstellen sollen, ohne dafür zwingend die exakten Mengen $U^{(t)}$ und $S^{(t)}$ kennen zu müssen.

\end{Assumption}

\vspace{0.3cm}

Und auf der letzten Annahme aufbauend der Vollständigkeit halber die aus praktischer Sicht vollkommen alternativlose Annahme ergänzen, laut der Service-Provider stets eine große Anzahl an Users ansprechen/bedienen und damit zahlenmäßig den Usern stark unterlegen sind.

\vspace{0.3cm}

\todo{[TODO1][Annahme 2 ist noch buggy]}

\begin{Assumption}[Verhältnismäßigkeit der Teilnehmer]\label{assumptionRatio}
Für alle $t \in T$ mit $\left(n^{(t)}, m^{(t)}\right) = dP^{(t)}$ gilt:

\begin{equation*}
m^{(t)} << n^{(t)} \tag{i}
\end{equation*}

\vspace{0.3cm}

Diese Aussage mag zahlenmäßig noch etwas "griffiger" formuliert werden. Dafür möchten wir das Verhältnis der Größen $n^{(t)}$ und $m^{(t)}$ abschätzen: Für unseren Zeithorizont, an dessen Ende - einem ausreichend späten, aber auch nicht in unabsehbar fernen Zukunft liegenden Zeitpunkt $t_{end} \in T$ - wir von einer WunderWelt sprechen, sei die Annahme

\begin{align*}
n^{(t_{end})} &\thickapprox 10 Mrd. \textrm{ und } m^{(t_{end})} \thickapprox 10.000 \textrm{ bzw.} \\
dP^{(t_{end})} &\thickapprox (10^{10}, 10^{4}) = 10^{4} * (10^{6}, 1) \tag{ii}
\end{align*}
nicht ganz abwegig. Genauso wenig unvernünftig scheint die Annahme, WunderPass begänne seine Welteroberung mit einem MVP mit lediglich einem einzigen Service-Provider - z. B. dem Guard (siehe Kap. \ref{sec:guard}] - und einer überschaubaren Anzahl an angepeilten Usern, also

\begin{align*}
n^{(t_0)} &\thickapprox 1.000 \textrm{ und } m^{(t_0)} = 1 \textrm{ bzw.} \\
dP^{(t_0)} &\thickapprox (1.000, 1) \tag{iii}
\end{align*}

\vspace{0.3cm}

Mit den beiden zuletzt getroffenen (quantitativen) Annahmen (ii) und (iii) lässt sich auch die initiale (qualitative) Annahme (i) ebenfalls quantifizieren:

\vspace{0.3cm}

Für alle $t \in ]t_0; t_{end}[$ mit $\left(n^{(t)}, m^{(t)}\right) = dP^{(t)}$ gilt:

\begin{equation*}
1.000 = \frac{n^{(t_0)}}{m^{(t_0)}} < \frac{n^{(t)}}{m^{(t)}} < \frac{n^{(t_{end})}}{m^{(t_{end})}} = 1.000.000 \tag{iv}
\end{equation*}

\end{Assumption}

\vspace{0.3cm}

Wir fassen Annahme \ref{assumptionRatio} in einer abschließenden Definition zusammen:

\vspace{0.3cm}

\begin{Def}\label{defRatio}

Seien $t \in T$ und $dP^{(t)} = \left(n^{(t)}, m^{(t)}\right)$ wie in Annahme \ref{assumptionOrakel} beschrieben. Wie definieren die "Verhältnismäßigkeit der Teilnehmer" als

\begin{align*}
\sigma &: T \rightarrow \mathbb{Q} \\
\sigma^{(t)} &= \frac{n^{(t)}}{m^{(t)}}
\end{align*}

\vspace{0.3cm}

Zudem halten wir fest, Annahme \ref{assumptionRatio} lege nahe, man könne in der Praxis stets von

\begin{equation*}
1.000 < \sigma^{(t)} < 1.000.000
\end{equation*}
ausgehen.

\end{Def}

\todo{[ende TODO1]}

\vspace{0.6cm}



\subsubsection{Zustandsbeschreibung WunderPass - simple Betrachtung}
\label{sec:eco_zahlen_zustand_wp}

% !TEX root = paper.tex

\paragraph{Status quo} 
\label{sec:eco_zahlen_zustand_wp_now}
\textrm{ }

\vspace{0.3cm}

Aufbauend auf die bisher erzielten Ergebnisse, wollen wir nun auch dem Stand von WunderPass für einen beliebigen Zeitpunkt $t \in T$ einen formalisierten Charakter verleihen und definieren zunächst einmal mittels der in Def \ref{defKoeff} beschriebenen Koeffizienten $\alpha^{(t)}_{ij}$ die sogenannten "connected Pools" von Usern und Service-Providern zum Zeitpunkt $t \in T$:

\vspace{0.3cm}

\begin{Def}\label{defPools}

Wir definieren den "connected User-Pool" $\widehat{U}^{(t)} \subseteq U^{(t)}$ und den "connected Service-Provider-Pool" $\widehat{S}^{(t)} \subseteq S^{(t)}$ als

\begin{align}
\widehat{U}^{(t)}:&= \left\{u \in U^{(t)} \mid \exists s^{*} \in S^{(t)} \textrm{ mit } \alpha^{(t)}(u, s^{*}) = 1 \right\} \tag{i} \\ 
\widehat{S}^{(t)}:&= \left\{s \in S^{(t)} \mid \exists u^{*} \in U^{(t)} \textrm{ mit } \alpha^{(t)}(u^{*}, s) = 1 \right\} \tag{ii}
\end{align}

\vspace{0.3cm}

Für die diskrete/sortierte Variante ist dies wieder gleichbedeutend mit
\begin{align}
\widehat{U}^{(t)} &= \left\{ \widehat{u}^{(t)}_1; \widehat{u}^{(t)}_2;...; \widehat{u}^{(t)}_{\widehat{n}} \right\} \tag{iii} \\ 
\widehat{S}^{(t)} &= \left\{ \widehat{s}^{(t)}_1; \widehat{s}^{(t)}_2;...; \widehat{s}^{(t)}_{\widehat{m}}\right\} \tag{iv}
\end{align}

\vspace{0.6cm}

Der Wert $\widehat{n} \leq n$ beschreibt die Größe des connecteten User-Pools - also die Anzahl $\widehat{n}$ der tatsächlich mit WunderPass connecteten User unter den $n$ potenziellen Usern. Analog steht $\widehat{m} \leq m$ für die Anzahl der tatsächlich mit WunderPass connecteten Prividern. Der Vollständigkeit halber übertragen wir das aus Def \ref{defKoeff} stammende Verständnis der Connection-Koeffizienten auch auf die eben definierten "connected Pools"

\[
\widehat{\alpha}^{(t)}_{ij}:=\left\{%
\begin{array}{ll}
    1, & \hbox{falls User $\widehat{u}^{(t)}_i \in \widehat{U}^{(t)}$ mit mit Provider $\widehat{s}^{(t)}_j \in \widehat{S}^{(t)}$ connectet ist} \\
    0, & \hbox{andernfalls} \\
\end{array}%
\right. \tag{v}
\] 

\end{Def}

\vspace{0.6cm}

Man beachte bei den diskreten/sortierten Schreibweisen der definierten Mengen $U^{(t)}$, $\widehat{U}^{(t)}$, $S^{(t)}$ und $\widehat{S}^{(t)}$, dass in aller Regel $u^{(t)}_i \neq \widehat{u}^{(t)}_i$ und $s^{(t)}_j \neq \widehat{s}^{(t)}_j$ gelten. Die sich teils trivial aus den letzten Definitionen ergebenden Zusammenhänge fallen wir in Form eines Theorems zusammen:

\vspace{0.3cm}

\input{[Economics][Quantifizierung][theoremPools]}    % binde die Datei '[Economics][Quantifizierung][theoremPools].tex' ein


Es ist klar, dass WunderPass sich in gewisser Weise an den definierten numerischen Messgrößen ihrer angebundenen Teilnehmer $\widehat{n}$ und $\widehat{m}$ messen können wird. Zusätzlich dazu möchten wir ein - womöglich deutlich relevanteres - numerisches Maß formalisieren. Nämlich die intuitive und sehr simple KPI "Gesamtzahl bestehender User-to-Provider-Connections" zum Zeitpunkt $t \in T$.

\vspace{0.3cm}

\begin{Def}\label{defGamma}

\begin{equation*}
  \Gamma : T \rightarrow \mathbb{N} 
\end{equation*}

\begin{equation*}
  \Gamma(t):= \sum_{i=1}^n \sum_{j=1}^m \alpha^{(t)}_{ij} \textrm{ mit } (n, m) = \left(n^{(t)}, m^{(t)}\right) = dP^{(t)}
\end{equation*}

\end{Def}

\vspace{0.6cm}

Nun beweisen wir folgende sich ergebende Zusammenhänge:

\begin{Theorem}\label{theremConnectionsCount}

Sei $t \in T$ ein beliebiger Zeitpunkt, $(n, m) = dP^{(t)}$ und $\widehat{U}^{(t)} = \left\{ \widehat{u}^{(t)}_1; \widehat{u}^{(t)}_2;...; \widehat{u}^{(t)}_{\widehat{n}} \right\}$ und $\widehat{S}^{(t)} = \left\{ \widehat{s}^{(t)}_1; \widehat{s}^{(t)}_2;...; \widehat{s}^{(t)}_{\widehat{m}} \right\}$ die connecteten Teilnehmer-Pools mit $\widehat{n} < n$ sowie $\widehat{m} < m$.

Zudem soll in Anlehnung an Annahme \ref{assumptionRatio} $\widehat{m} << \widehat{n}$ gelten. Dann gelten zusätzlich auch folgende Aussagen:

\vspace{0.3cm}

\begin{align*}
\Gamma(t)&= \sum_{i=1}^{\widehat{n}} \sum_{j=1}^{\widehat{m}} \widehat{\alpha}^{(t)}_{ij} \tag{i} \label{theremConnectionsCount_1} \\ 
\widehat{n} &\leq \Gamma(t) \leq \widehat{n} * \widehat{m} \tag{ii} \\
\widehat{n} = \Gamma(t) &\Leftrightarrow \forall i \in \left\{1;...; \widehat{n} \right\}
\textrm{ gilt } \sum_{j=1}^{\widehat{m}} \widehat{\alpha}^{(t)}_{ij} = 1 \tag{iii} \\
\Gamma(t) = \widehat{n} * \widehat{m} &\Leftrightarrow \widehat{\alpha}^{(t)}_{ij} = 1 \textrm{  } \forall i \in \left\{1;...; \widehat{n} \right\} \textrm{ und } \forall j \in \left\{1;...; \widehat{m} \right\} \tag{iv}
\end{align*}

\vspace{0.3cm}

Man beachte, Aussage (ii) impliziert insbesondere

\begin{equation*}
\Gamma(t) = 0 \Leftrightarrow \widehat{n} = 0 \Leftrightarrow \widehat{U}^{(t)} = \emptyset  \Leftrightarrow \widehat{S}^{(t)} = \emptyset
\end{equation*}

\vspace{0.3cm}

Aussage (iv) beschreibt dagegen quasi eine "\textbf{\textit{Voll-Vernetzung}}" der aktuell connecteten Teilnehmer!

\end{Theorem}

\vspace{0.3cm}


\begin{proof}[Beweis] \textrm{ }

\vspace{0.3cm}

Die Aussage (i) ist intuitiv nahezu trivial. Das explizite Vorrechnen dagegen etwas aufwendig, erfolgt aber in Grunde sehr ähnlich wie der Beweis der Aussagen (v) und (vi) des Theorems \ref{theoremPools}.

\vspace{0.4cm}

zu (ii): 

$\Gamma(t) \leq \widehat{n} * \widehat{m}$ ergibt sich aus
\begin{equation*}
  \Gamma(t) = \sum_{i=1}^{\widehat{n}} \sum_{j=1}^{\widehat{m}} \widehat{\alpha}^{(t)}_{ij} \leq \sum_{i=1}^{\widehat{n}} \sum_{j=1}^{\widehat{m}} 1 = \widehat{n} * \widehat{m}
\end{equation*}

\vspace{0.3cm}

Nun zeigen wir $\widehat{n} \leq \Gamma(t)$. Für $n = 0$ ergibt sich die Aussage aus Punkt (iv) aus Theorem \ref{theoremPools}. Sei also $n > 0$. Dann ist

\begin{align*}
\Gamma(t) &\overset{\text{(i)}}{=} \sum_{i=1}^{\widehat{n}} \sum_{j=1}^{\widehat{m}} \widehat{\alpha}^{(t)}_{ij} \\
&\overset{(Def \ref{defPools})}{\geq} \sum_{i=1}^{\widehat{n}} 1 = \widehat{n}
\end{align*}

\vspace{0.4cm}

zu (iii): 
"$\Leftarrow$" ist trivial. 

\vspace{0.4cm}

Zu "$\Rightarrow$": Sei $\widehat{n} = \Gamma(t)$. Angenommen es gäbe ein $i^{*} \in \left\{1;...; \widehat{n} \right\}$ mit $\sum_{j=1}^{\widehat{m}} \widehat{\alpha}^{(t)}_{i^{*}j} > 1$. Dann müsste es aufgrund der Annahme aber auch ein $i^{**} \in \left\{1;...; \widehat{n} \right\}$ mit $\sum_{j=1}^{\widehat{m}} \widehat{\alpha}^{(t)}_{i^{**}j} < 1$ also $\sum_{j=1}^{\widehat{m}} \widehat{\alpha}^{(t)}_{i^{**}j} = 0$ geben. In diesem Fall wäre aber $\widehat{u}^{(t)}_{i^{**}} \notin \widehat{U}^{(t)}$ und somit auch $i^{**} \notin \left\{1;...; \widehat{n} \right\}$. Widerspruch!

\vspace{0.4cm}

zu (iv): 
"$\Leftarrow$" ist wieder trivial.

\vspace{0.4cm} 

Zu "$\Rightarrow$": Es gelte also $\Gamma(t) = \widehat{n} * \widehat{m}$. Angenommen es gäbe ein $i^{*} \in \{1,...,\widehat{n}\}$ und ein $j^{*} \in \{1,...,\widehat{m}\}$, sodass $\alpha^{(t)}_{i^{*}j^{*}} = 0$. Dann wäre unter Gültigkeit der Aussage (i)

\begin{align*}
\widehat{n} * \widehat{m} &= \Gamma(t) = \sum_{i=1}^{\widehat{n}} \sum_{j=1}^{\widehat{m}} \widehat{\alpha}^{(t)}_{ij} \\
&\leq \left(\sum_{i=1}^{\widehat{n}} \sum_{j=1}^{\widehat{m}} 1 \right) - 1 = \widehat{n} * \widehat{m} - 1 < \widehat{n} * \widehat{m} \\
\end{align*}
Widerspruch!
  
\end{proof}
\vspace{0.6cm}


 % binde die Datei '[Economics][Quantifizierung][WP][Status quo].tex' ein

% !TEX root = paper.tex

\paragraph{Messbarkeit Status quo} 
\label{sec:eco_zahlen_zustand_wp_nowVlaue}
\textrm{ }

\vspace{0.3cm}

Kommend von der intuitiven Annahme, die Größe der definierten "connected Pools" $\widehat{U}^{(t)}$ und $\widehat{S}^{(t)}$ sei irgendwie erstrebenswert in unserem Sinne, definierten wir im vorangehenden Abschnitt das - aus unserer Sicht fundierteres und geeigneteres - Maß $\Gamma(t)$, um dem Verständnis von "erstrebenswerter Zustand" besser gerecht zu werden.

In diesem Abschnitt wollen wir die - bisher eher wertfrei/objektiv formulierten -  Ergebnisse des vorigen Abschnitts in den Kontext der "Erstrebenswertigkeit" stellen. Also eine formale und quantifizierbare Vergleichbarkeit unserer - ohnehin beim Lesen des letzten Abschnitts mitschwingender - Intuition schaffen, die Werte 
\begin{itemize}
  \item $\widehat{n} = |\widehat{U}^{(t)}|$, 
  \item $\widehat{m} = |\widehat{S}^{(t)}|$ und vor allem 
  \item $\Gamma(t)$ 
\end{itemize}
seien umso besser je größer sie seien. Alle der eben genannten Größen, denen wir hier eine intuitiv spürbare "Erstrebenswertigkeit" beimessen, besitzen einen klaren Zeitbezug. Daher überrascht es kaum, wir strebten die genannte quantifizierbare Vergleichbarkeit für je zwei beliebige Zeitpunkte $t_1, t_2 \in T$ an. Formale Vergleichbarkeit schreit nur so nach der mathematisch verstandenen "Ordnungsrelation":

\vspace{0.3cm}

\begin{Def}\label{defRelation}

Wir bedienen uns der in Definition \ref{defGamma} beschriebenen Funktion $\Gamma(t)$, um damit eine \href{https://de.wikipedia.org/wiki/Ordnungsrelation}{Ordnungsrelation} 
auf unserem Zeitstrahl $T$ für je zwei beliebige Zeitpunkte $t_1, t_2 \in T$ zu erhalten: 

\vspace{0.3cm}

\begin{equation*}
  R_{\preceq} \subseteq T \times T \textrm{ mit}
\end{equation*}

\begin{equation*}
  R_{\preceq}:= \left\{ (t_1, t_2) \in T \times T \mid \Gamma(t_1) \leq \Gamma(t_2) \right\}
\end{equation*}
\vspace{1cm}
Mittels $R_{\preceq}$ erhalten wir eine Ordnung unseres Zeitstahls $T$ und erklären zudem insbesondere, was "erstrebenswert" bedeutet. Ein beliebiger Zeitpunkt $t_1 \in T$ ist nämlich verbal genau dann "nicht weniger erstrebenswert" in Sinne unserer Vision als ein beliebiger anderer Zeitpunkt $t_2 \in T$, falls $(t_1, t_2) \in R_{\preceq}$ gilt.

\vspace{0.3cm}

\begin{equation*}
  \textrm{Wir schreiben fortan statt } (t_1, t_2) \in R_{\preceq} \textrm{ lieber } t_1 \preceq t_2 
\end{equation*}

\end{Def}

\vspace{1cm}

Man beachte, dass es sich bei der definierten Ordnungsrelation gar um eine \href{https://de.wikipedia.org/wiki/Ordnungsrelation#Totalordnung}{Totalordnung} handelt!
Der Form halber ergänzen wir an der Stelle noch um zwei weitere - schematisch induzierte - Relationen auf unserem Zeitstrahl $T$:

\vspace{0.3cm}

\begin{Def}\label{defRelationen}

Um zusätzlich zur in Def \ref{defRelation} definierten Ordnungsrelation "$\preceq$", auch dem Verständnis von "echt besser" und "gleich gut" Rechnung zu tragen, definieren wir die beiden Relationen "$\prec$" und "$\simeq$"

\vspace{0.3cm}

\begin{equation*}
  R_{\prec}:= \left\{ (t_1, t_2) \in T \times T \mid \Gamma(t_1) < \Gamma(t_2) \right\}
\end{equation*}

\begin{equation*}
  R_{\simeq}:= \left\{ (t_1, t_2) \in T \times T \mid \Gamma(t_1) = \Gamma(t_2) \right\}
\end{equation*}

\vspace{1cm}

Bei $R_{\prec}$ handelt es sich im Übrigen wieder um eine Ordnungsrelation. Bei $R_{\simeq}$ dagegen nicht.

\end{Def}

\vspace{0.3cm}

Auch für die letzten beiden Relationen wollen wir fortan die vereinfachte Schreibweise $t_1 \prec t_2$ und $t_1 \simeq t_2$ nutzen.

\vspace{0.6cm}

 % binde die Datei '[Economics][Quantifizierung][WP][Messbarkeit].tex' ein

% !TEX root = [Economics][Quantifizierung].tex

\paragraph{Fazit} 
\label{sec:eco_zahlen_zustand_wp_fazit}
\textrm{ }

\vspace{0.3cm}

Ungeachtet des Werts der bisher erzielten erfolgreichen Ergebnisse hinsichtlich der quantitativen Einordnung des WunderPass-Fortschritts zu einem Zeitpunkt $t \in T$, besitzt der Zusatz "...simple Betrachtung" innerhalb der Überschrift des gegenwärtigen Kapitels durchaus seine Rechtfertigung.

Wir haben zwar die Größe $\Gamma(t)$ als sehr gut geeigneten Gradmesser für den Fortschritt WunderPasses herausgearbeitet und dieses ebenfalls in Abhängigkeit der intuitiven Erfolgsmesser $\widehat{n}$ und $\widehat{m}$ gesetzt sowie nach unten und oben abgeschätzt. Jedoch scheint unser Ökosystem zu komplex und unsere bisherige Betrachtungsweise zu global geprägt, als dass wir guten Gewissens den besagten Zusatz "...simple Betrachtung" in der Überschrift des gegenwärtigen Kapitels weglassen könnten. Den geäußerten Zweifel verdeutlicht folgendes 

\vspace{0.3cm}

\begin{Example*}
Wir nehmen den Zustand zum Zeitpunkt $t \in T$ mit $\widehat{m} = 5$ angebundenen Service-Providern und als durch $\Gamma(t) \approx 50.000$ beschrieben an und schauen uns drei Szenarien an, die allesamt die getroffene Annahme hergeben:

\vspace{0.3cm}

\begin{enumerate}
  \item Wir könnten von $\widehat{n} = 50.000$ angebundenen Usern ausgehen, von denen je 10.000 mit je einem einzigen der $\widehat{m} = 5$ Provider connectet wären und keinem anderem. 
  \item Genauso könnten dieselben $\widehat{n} = 50.000$ angebundene User so verteilt sein, dass 49.996 (quasi alle) mit demselben einzelnen Provider connectet sind, und die restlichen 4 (also quasi niemand) User mit je einem anderen der verbleibenden 4 Provider verbunden sind.
  \item Ein ganz anderes Szenario wäre der Fall von $\widehat{n} \approx 25.000$, von denen jeder mit denselben zwei unserer fünf Service-Providern connectet wäre (und keinem anderen) und zudem ein paar vereinzelte zusätzliche User mit je einem der verbleibenden drei unserer fünf Provider.
\end{enumerate}

\vspace{0.3cm}

Rein an den Größen $\widehat{n}$, $\widehat{m}$, $\Gamma(t)$ gemessen, scheint Fall (3) aufgrund von $\widehat{n} = 25.000$ der schlechteste zu sein. Rein intuitiv scheint genau dieser Fall aber der beste zu sein. Dies ist aber nur ein Gefühl. Es lassen sich ebenso gute Argumente finden, warum Fall (1) oder Fall (2) der beste sein könnten. Es kommt eben darauf an...Gleichwohl für alle der Fälle $\Gamma(t) = 50.000$ gilt, lässt sich zweifelsfrei entscheiden, welcher zwingend der beste sein soll.

Was sich jedoch objektiv beurteilen lässt, ist die Tatsache, dass in Fall (2) vier der fünf Service-Provider quasi "wertlos" sind. Und in Fall (3) immer noch drei von fünf!

\vspace{0.3cm}

Wir könnten also unsere Gegenüberstellung der drei angeführten Cases auch zur folgenden quantitativen Beurteilung stellen:

\vspace{0.3cm}

\begin{enumerate}
  \item $\Gamma_1(t) = 50.000$, $\widehat{n}_1 = 50.000$ und $\widehat{m}_1 = 5$
  \item $\Gamma_2(t) = 50.000$, $\widehat{n}_2 = 50.000$ und $\widehat{m}_2 = 1$
  \item $\Gamma_3(t) = 50.000$, $\widehat{n}_3 = 25.000$ und $\widehat{m}_3 = 2$
\end{enumerate}

\vspace{0.3cm}

Was ist also besser?

\end{Example*}

\vspace{0.6cm}

 % binde die Datei '[Economics][Quantifizierung][WP][Fazit].tex' ein




\subsubsection{Zustandsbeschreibung WunderPass - detaillierte Sicht}
\label{sec:eco_zahlen_zustand_wp_advanced}


\paragraph{Teilnehmer} 
\label{sec:eco_zahlen_zustand_wp_advanced_teilnehmer}
\textrm{ }

\vspace{0.3cm}

\todo{WIP}
\vspace{1cm}

\todo{[TODO4]["individuelle" User- und Provider-Pools]}
\vspace{0.3cm}

Nicht alle Teilnehmer innerhalb der WunderPass-Netzwerks sind gleichbedeutend. Dies ist zweifelsfrei klar hinsichtlich der Unterscheidung zwischen connecteten Usern $\widehat{u} \in \widehat{U}^{(t)}$ und Service-Providern $\widehat{s} \in \widehat{S}^{(t)}$. Jedoch bestehen ebenfalls signifikante Unterschiede jeweils innerhalb der beiden Teilnehmerklassen $\widehat{U}^{(t)}$ und $\widehat{S}^{(t)}$ (siehe \todo{[TODO4]["Sättigung"]}). Um dieser Unterscheidung unserer Teilnehmer gerecht zu werden, definieren wir "connected Pools" pro individuellen Teilnehmer als

\vspace{0.3cm}

\begin{Def}\label{defTeilnehmerPool}

Sei $t \in T$, $\widehat{U}^{(t)}$ und $\widehat{S}^{(t)}$ die übergeordneten "connected" User- und Provider-Pools und $\widehat{u}_{*} \in \widehat{U}^{(t)}$ und $\widehat{s}_{*} \in \widehat{S}^{(t)}$ die entsprechenden Teilnehmer, für deren individuelle "connected Pools" wir uns an dieser Stelle interessieren. Wir definieren die genannten "connected Pools" als

\begin{align*}
&accounts : \widehat{U}^{(t)} \rightarrow \mathcal{P}\left(\widehat{S}^{(t)}\right) \\
&accounts^{(t)}(\widehat{u}_{*}) = \left\{\widehat{s} \in \widehat{S}^{(t)} \textrm{ } | \textrm{ } \alpha^{(t)}(\widehat{u}_{*}, \widehat{s}) = 1 \right\}
\end{align*}
und

\begin{align*}
&userbase : \widehat{S}^{(t)} \rightarrow \mathcal{P}\left(\widehat{U}^{(t)}\right) \\
&userbase^{(t)}(\widehat{s}_{*}) = \left\{\widehat{u} \in \widehat{U}^{(t)} \textrm{ } | \textrm{ } \alpha^{(t)}(\widehat{u}, \widehat{s}_{*}) = 1 \right\}
\end{align*}

\end{Def}

\vspace{0.6cm}

\begin{Lemma}\label{lemmaPools}

\begin{align}
\bigcup_{\widehat{u} \in \widehat{U}^{(t)}} \left(accounts^{(t)} (\widehat{u})\right) = \widehat{S}^{(t)} \tag{i} \\
\bigcup_{\widehat{s} \in \widehat{S}^{(t)}} \left(userbase^{(t)} (\widehat{s})\right) = \widehat{U}^{(t)} \tag{ii}
\end{align}

\end{Lemma}

\vspace{0.3cm}

\begin{proof}[Beweis] \textrm{ }

\vspace{0.3cm}

zu (i): Es ist

\begin{align*}
\bigcup_{\widehat{u} \in \widehat{U}^{(t)}} \left(accounts^{(t)} (\widehat{u})\right) &\overset{\text{Def \ref{defTeilnehmerPool}}}{=} \bigcup_{\widehat{u} \in \widehat{U}^{(t)}} \left\{\widehat{s} \in \widehat{S}^{(t)} \textrm{ } | \textrm{ } \alpha^{(t)}\left(\widehat{u}, \widehat{s}\right) = 1 \right\} \\
&\overset{(*)}{=} \left\{\widehat{s} \in \widehat{S}^{(t)} \mid \exists \widehat{u} \in \widehat{U}^{(t)} \textrm{ mit } \alpha^{(t)}\left(\widehat{u}, \widehat{s}\right) = 1 \right\} \\
&\overset{\text{Th. \ref{theoremPools} (\ref{theoremPools_6})}}{=} \widehat{S}^{(t)}
\end{align*}
Die Gleichheit zu (*) ergibt aus der Tatsache, Mengen-Vereinigungen ignorierten Doppelzählungen! 

\vspace{0.3cm}

Aussage (ii) folgt ganz analog!

\end{proof}

\vspace{0.3cm}


\begin{Theorem}\label{theremPoolsCount}

\begin{equation*}
\sum_{\widehat{u} \in \widehat{U}^{(t)}} |accounts^{(t)} (\widehat{u})| = \sum_{\widehat{s} \in \widehat{S}^{(t)}} |userbase^{(t)} (\widehat{s})| = \Gamma(t)
\end{equation*}

\end{Theorem}

\vspace{0.3cm}

\begin{proof}[Beweis] \textrm{ }

\vspace{0.3cm}

Es ist

\begin{align*}
\Gamma(t)&\overset{\text{Th. \ref{theremConnectionsCount} (\ref{theremConnectionsCount_1})}}{=} \sum_{i=1}^{\widehat{n}} \sum_{j=1}^{\widehat{m}} \widehat{\alpha}^{(t)}_{ij} \\
&= \sum_{\widehat{u} \in \widehat{U}^{(t)}} \textrm{  } \sum_{\widehat{s} \in \widehat{S}^{(t)}} \alpha^{(t)}\left(\widehat{u}, \widehat{s}\right) \\
&= \sum_{\widehat{u} \in \widehat{U}^{(t)}} \textrm{  } \sum_{\widehat{s} \in \widehat{S}^{(t)} \textrm{ mit } \alpha^{(t)}\left(\widehat{u}, \widehat{s}\right) = 1} 1 \\
&\overset{\text{Def \ref{defTeilnehmerPool}}}{=} \sum_{\widehat{u} \in \widehat{U}^{(t)}} \textrm{  } \sum_{s \in accounts^{(t)} (\widehat{u})} 1 \\
&= \sum_{\widehat{u} \in \widehat{U}^{(t)}} |accounts^{(t)} (\widehat{u})|
\end{align*}

\vspace{0.3cm}

Die zweite Gleichheit folgt analog, falls man die Kommutativität der Def \ref{defGamma} beachtet: 

\begin{equation*}
  \Gamma(t)= \sum_{i=1}^n \sum_{j=1}^m \alpha^{(t)}_{ij} = \sum_{j=1}^m \sum_{i=1}^n \alpha^{(t)}_{ij}
\end{equation*}

\end{proof}


\vspace{0.6cm}
\todo{[ende TODO4]}
\vspace{1cm}\todo{[TODO4]["individuelle" User- und Provider-Pools]}
\vspace{0.3cm}

Nicht alle Teilnehmer innerhalb der WunderPass-Netzwerks sind gleichbedeutend. Dies ist zweifelsfrei klar hinsichtlich der Unterscheidung zwischen connecteten Usern $\widehat{u} \in \widehat{U}^{(t)}$ und Service-Providern $\widehat{s} \in \widehat{S}^{(t)}$. Jedoch bestehen ebenfalls signifikante Unterschiede jeweils innerhalb der beiden Teilnehmerklassen $\widehat{U}^{(t)}$ und $\widehat{S}^{(t)}$ (siehe \todo{[TODO4]["Sättigung"]}). Um dieser Unterscheidung unserer Teilnehmer gerecht zu werden, definieren wir "connected Pools" pro individuellen Teilnehmer als

\vspace{0.3cm}

\begin{Def}\label{defTeilnehmerPool}

Sei $t \in T$, $\widehat{U}^{(t)}$ und $\widehat{S}^{(t)}$ die übergeordneten "connected" User- und Provider-Pools und $\widehat{u}_{*} \in \widehat{U}^{(t)}$ und $\widehat{s}_{*} \in \widehat{S}^{(t)}$ die entsprechenden Teilnehmer, für deren individuelle "connected Pools" wir uns an dieser Stelle interessieren. Wir definieren die genannten "connected Pools" als

\begin{align*}
&accounts : \widehat{U}^{(t)} \rightarrow \mathcal{P}\left(\widehat{S}^{(t)}\right) \\
&accounts^{(t)}(\widehat{u}_{*}) = \left\{\widehat{s} \in \widehat{S}^{(t)} \textrm{ } | \textrm{ } \alpha^{(t)}(\widehat{u}_{*}, \widehat{s}) = 1 \right\}
\end{align*}
und

\begin{align*}
&userbase : \widehat{S}^{(t)} \rightarrow \mathcal{P}\left(\widehat{U}^{(t)}\right) \\
&userbase^{(t)}(\widehat{s}_{*}) = \left\{\widehat{u} \in \widehat{U}^{(t)} \textrm{ } | \textrm{ } \alpha^{(t)}(\widehat{u}, \widehat{s}_{*}) = 1 \right\}
\end{align*}

\end{Def}

\vspace{0.6cm}

\begin{Lemma}\label{lemmaPools}

\begin{align}
\bigcup_{\widehat{u} \in \widehat{U}^{(t)}} \left(accounts^{(t)} (\widehat{u})\right) = \widehat{S}^{(t)} \tag{i} \\
\bigcup_{\widehat{s} \in \widehat{S}^{(t)}} \left(userbase^{(t)} (\widehat{s})\right) = \widehat{U}^{(t)} \tag{ii}
\end{align}

\end{Lemma}

\vspace{0.3cm}

\begin{proof}[Beweis] \textrm{ }

\vspace{0.3cm}

zu (i): Es ist

\begin{align*}
\bigcup_{\widehat{u} \in \widehat{U}^{(t)}} \left(accounts^{(t)} (\widehat{u})\right) &\overset{\text{Def \ref{defTeilnehmerPool}}}{=} \bigcup_{\widehat{u} \in \widehat{U}^{(t)}} \left\{\widehat{s} \in \widehat{S}^{(t)} \textrm{ } | \textrm{ } \alpha^{(t)}\left(\widehat{u}, \widehat{s}\right) = 1 \right\} \\
&\overset{(*)}{=} \left\{\widehat{s} \in \widehat{S}^{(t)} \mid \exists \widehat{u} \in \widehat{U}^{(t)} \textrm{ mit } \alpha^{(t)}\left(\widehat{u}, \widehat{s}\right) = 1 \right\} \\
&\overset{\text{Th. \ref{theoremPools} (\ref{theoremPools_6})}}{=} \widehat{S}^{(t)}
\end{align*}
Die Gleichheit zu (*) ergibt aus der Tatsache, Mengen-Vereinigungen ignorierten Doppelzählungen! 

\vspace{0.3cm}

Aussage (ii) folgt ganz analog!

\end{proof}

\vspace{0.3cm}


\begin{Theorem}\label{theremPoolsCount}

\begin{equation*}
\sum_{\widehat{u} \in \widehat{U}^{(t)}} |accounts^{(t)} (\widehat{u})| = \sum_{\widehat{s} \in \widehat{S}^{(t)}} |userbase^{(t)} (\widehat{s})| = \Gamma(t)
\end{equation*}

\end{Theorem}

\vspace{0.3cm}

\begin{proof}[Beweis] \textrm{ }

\vspace{0.3cm}

Es ist

\begin{align*}
\Gamma(t)&\overset{\text{Th. \ref{theremConnectionsCount} (\ref{theremConnectionsCount_1})}}{=} \sum_{i=1}^{\widehat{n}} \sum_{j=1}^{\widehat{m}} \widehat{\alpha}^{(t)}_{ij} \\
&= \sum_{\widehat{u} \in \widehat{U}^{(t)}} \textrm{  } \sum_{\widehat{s} \in \widehat{S}^{(t)}} \alpha^{(t)}\left(\widehat{u}, \widehat{s}\right) \\
&= \sum_{\widehat{u} \in \widehat{U}^{(t)}} \textrm{  } \sum_{\widehat{s} \in \widehat{S}^{(t)} \textrm{ mit } \alpha^{(t)}\left(\widehat{u}, \widehat{s}\right) = 1} 1 \\
&\overset{\text{Def \ref{defTeilnehmerPool}}}{=} \sum_{\widehat{u} \in \widehat{U}^{(t)}} \textrm{  } \sum_{s \in accounts^{(t)} (\widehat{u})} 1 \\
&= \sum_{\widehat{u} \in \widehat{U}^{(t)}} |accounts^{(t)} (\widehat{u})|
\end{align*}

\vspace{0.3cm}

Die zweite Gleichheit folgt analog, falls man die Kommutativität der Def \ref{defGamma} beachtet: 

\begin{equation*}
  \Gamma(t)= \sum_{i=1}^n \sum_{j=1}^m \alpha^{(t)}_{ij} = \sum_{j=1}^m \sum_{i=1}^n \alpha^{(t)}_{ij}
\end{equation*}

\end{proof}


\vspace{0.6cm}
\todo{[ende TODO4]}
\vspace{1cm}





















\subsubsection{Other Stuff}
\label{sec:eco_zahlen_zustand_todo}

\vspace{2cm}
\todo{[TODO2][Abschätzung $\frac{\widehat{n}}{\widehat{m}}$]}
\vspace{0.3cm}

Aussagen aus Annahme \ref{assumptionRatio} und Theorem \ref{theremConnectionsCount} - Aussage (ii) - verwerten und Annahme \ref{assumptionRatio} deutlich verschärfen.

\todo{[ende TODO2]}
\vspace{1cm}


\todo{[TODO3]["Verdichtung"]}
\vspace{0.3cm}

Die Maße $\widehat{n}$, $\widehat{m}$ und $\Gamma(t)$ sind sehr objektiv und teils zielführend. Sie scheinen aber nicht zu reichen. So kann es z. B. User $\widehat{u} \in \widehat{U}^{(t)}$ geben, die im worst case ausschließlich zu einem einzigen Provider $\widehat{s} \in \widehat{S}^{(t)}$ connectet sind (und somit aber trotzdem den Wert von $\widehat{n}$ beeinflussen, oder noch schlimmer analog Provider $\widehat{s} \in \widehat{S}^{(t)}$, die als "angebunden" gelten, weil sie mit marginal wenigen Usern (im worst case mit einem einzigen) connectet sind. Solche Teilnehmer spielen eigentlich für den zahlenmäßigen WunderPass-Fortschritt keinerlei Rolle, beeinflussen jedoch unsere relevanten Messgrößen (KPI).

Von daher benötigen wir noch eine weitere Präzisierung in Form von

\begin{itemize}
  \item "80-20-Regel" heranziehen, indem man die Mengen $\widehat{U}^{(t)}$ und $\widehat{S}^{(t)}$ so modifiziert/verkleinert, dass $\Gamma(t)$ davon kaum einen Einfluss spürt (sich lediglich um einen marginalen Prozentsatz verringert).
  \item Formeln auf die davon modifizierten Größen $\widehat{\widehat{n}}$ und $\widehat{\widehat{m}}$ anpassen.
  \item $\Rightarrow$ Die Grenzen von [Theorem \ref{theremConnectionsCount}][Aussage (ii)] werden damit deutlich schärfer.
  \item $\Rightarrow$ kann sicherlich in Abschnitt \ref{sec:eco_zahlen_business_plan} für den Umgang mit dem Verhältnis $\frac{\widehat{n}}{\widehat{m}}$ genutzt werden.
  \item Wird vermutlich auch Relevanz bei den "individuellen" (Definition erfolgt noch) User- und Provider-Pools zum Einsatz kommen.
\end{itemize}

\todo{[ende TODO3]}
\vspace{1cm}

\todo{[TODO4.1]["Exklusive Connections"]}
\vspace{0.3cm}

\begin{itemize}
  \item Eine Connection zu einem Service-Provider ist exklusiv, wenn der zugehörige User zu keinem anderen Service-Provider connectet ist.
    \item Es gibt mindestens $n_{nexcl} \geq \Gamma(t) - \widehat{n}$ nicht exklusive Connections.
  
\end{itemize}

\todo{[ende TODO3.1]}
\vspace{1cm}











% !TEX root = paper.tex

\todo{[TODO6][deprecated Inhalt verarbeiten]}
\vspace{0.3cm}

Mit diesen geschaffenen Formalisierungs-Werkzeugen lässt sich nun auch die übergeordnete WunderPass-Vision formal erfassen - und zwar indem man den Zeitpunkt $t_{*} \in T$ ihrer Erreichung benennt:

\begin{Def}\label{defVision}

Wir betrachten die WunderPass-Vision zu einem Zeitpunkt $t_{*} \in T$ als erreicht, falls

\vspace{0.3cm}

\begin{equation}
\label{eq:1}
  \alpha^{(t_{*})}_{ij} = 1 \textrm{ für alle } i \in \{1,...,n\} \textrm{ und } j \in \{1,...,m\}
\end{equation}\\
erfüllt ist. Darüber hinaus ist es noch nicht ganz klar, welche Aussage für die Zeitpunkte $t > t_{*}$ hinsichtlich der Visions-Erreichung zu treffen sei. Grundsätzlich ist es ja durchaus denkbar, die obige Voraussetzung gelte für $t > t_{*}$ nicht mehr. Bleibt die Vision in diesem Fall trotzdem als 'erreicht' zu betrachten?

\end{Def}

\vspace{1cm}

Die gelungene Formalisierung unserer Vision mittels Definition \ref{defVision} mag einen Fortschritt hinsichtlich unserer "Business-Mathematics" darstellen, bleibt jedoch losgelöst zunächst einmal ziemlich wertlos. Zum einen ist das Erreichen der Vision im formellen Sinne der Definition \ref{defVision} weder praxistauglich noch akribisch erforderlich. Zudem bleibt zum anderen der resultierende (intrinsische) Business-Value der Visions-Erreichung bisher weiterhin nicht ohne Weiteres erkennbar.
Vielmehr sollten wir die Anforderung von Gleichung \eqref{eq:1} als eine Messlatte unseres Fortschritts heranziehen, und eher als (unerreichbare) 100\%-Zielerreichungs-Marke betrachten. Zudem müssen wir zeitnah - obgleich die vollständige oder nur fortschreitend partielle - Zielerreichung unserer Vision in klaren, quantifizierbaren Business-Value übersetzen.

Dazu definieren wir als erstes ein intuitives Maß der Zielerreichung:

\vspace{0.3cm}

\begin{Def}\label{defGamma2}

\begin{equation*}
  \Gamma : T \rightarrow \mathbb{N} 
\end{equation*}

\begin{equation*}
  \Gamma(t):= \sum_{i=1}^n \sum_{j=1}^m \alpha^{(t)}_{ij} 
\end{equation*}

\end{Def}

\vspace{1cm}

Damit liefert uns die definierte $\Gamma$-Funktion aber auch ein extrem greifbares und intuitiv nachvollziehbares Fortschrittsmaß unseres Vorhabens. Zudem fügt sich dieses perfekt in unsere mittels Definition \ref{defVision} quantifizierte Unternehmens-Vision und unterliegt einer fundamentalen (bezifferbaren) Obergrenze. Dies zeigt folgendes Lemma:

\vspace{0.3cm}

\begin{Lemma}

Es gelten folgende Aussagen:

\begin{align}
\Gamma(t) &\leq n^{(t)} * m^{(t)} \textrm{ für alle } t \in T \tag{i} \label{eq:l1_erste} \\ 
  \text{es gilt Gleichheit bei }  \eqref{eq:l1_erste} &\Leftrightarrow \text{ es gilt Gleichung } \eqref{eq:1} \text{ aus Def } \ref{defVision} \tag{ii} \label{eq:l1_zweite}
\end{align}

\end{Lemma}

\vspace{0.3cm}

Gleichung \eqref{eq:l1_zweite} ermöglicht uns die Definition \ref{defGamma} auf ein relatives Zielereichungs-Maß auszuweiten:

\vspace{0.3cm}

\begin{Def}\label{defKleinGamma}
\begin{equation*}
  \gamma : T \rightarrow [0; 1] 
\end{equation*}

\begin{equation*}
  \gamma(t):= \frac{\Gamma(t)}{n^{(t)} * m^{(t)}}
\end{equation*}

\end{Def}

\todo{[ende TODO6]}
\vspace{1.5cm}    % binde die Datei '[Economics][Quantifizierung][deprecated].tex' ein






\todo{Ab hier WIP}
\vspace{1cm}


\subsubsection{Business-Plan in Mathematics}
\label{sec:eco_zahlen_business_plan}

Diese letzten Werkzeuge lassen und Begriffe wie "Zielsetzung" bzw. "Milestone" einführen.

\vspace{0.3cm}

\begin{Def}\label{defZiel}

Seien $t \in T$ und zudem entsprechend $(n^{(t)}, m^{(t)}) = dP^{(t)}$ der angenommene Zustand der digitalen Welt zu einem beliebig gewählten Zeitpunkt. Wir definieren die - allein durch WunderPass zu bestimmende - Zielfunktion als

\end{Def}


\subsubsection{Quantifizierung des Status quo}
\label{sec:eco_zahlen_status_quo}



\paragraph{Vernetzung \& Netzwerk-Effekt}
\label{sec:zahlen_status_quo_netzwerk_effekt}

\textrm{ }
\vspace{0.3cm}

Die WunderPass-Vision steht in ihrer Formulierung ganz klar im Sinne einer gewissen "Vernetzung". Wir möchten, dass möglichst viele User sich mit möglichst vielen Service-Providern "connecten" (bzw. connectet sind/bleiben). Schränkt man seine Sichtweise alleinig auf diese Vision (ohne diese zunächst zu hinterfragen), liefern uns die zuletzt eingeführten Größen $\alpha^{(t)}_{ij}$, $\Gamma(t)$ und $\gamma(t)$ ziemlich gute Gradmesser, um zweifelsfreie Aussagen hinsichtlich der Vergleichbarkeit zweier Zeitpunkte $t_1, t_2 \in T$ treffen zu können. Es ist irgendwie klar, $\alpha^{(t)}_{ij} = 1$ sei im Sinne unserer Vision irgendwie besser als $\alpha^{(t)}_{ij} = 0$.

Aus diesem Blickwinkel (in dem die Vision zunächst ein Selbstzweck bleibt) erscheint die folgende Definition mehr als intuitiv einleuchtend, um die obige Formulierung "irgendwie besser" zu formalisieren und vor allem zu quantifizieren. 



















\vspace{0.3cm}

\todo{WIP:}
Hier stand vorher Definition \ref{defRelation}

\vspace{1cm}

Man beachte, dass es sich bei der definierten Ordnungsrelation gar um eine \href{https://de.wikipedia.org/wiki/Ordnungsrelation#Totalordnung}{Totalordnung} handelt!
Der Form halber ergänzen wir an der Stelle noch um zwei weitere - schematisch induzierte - Relationen auf unserem Zeitstrahl $T$:

\vspace{0.3cm}

\todo{WIP:}
Hier stand vorher Definition \ref{defRelationen}

\vspace{0.3cm}

Auch für die letzten beiden Relationen wollen wir fortan die vereinfachte Schreibweise $t_1 \prec t_2$ und $t_1 \simeq t_2$ nutzen.
























 

\vspace{1cm}

Diese Netzwerk-Bewertungs-Modell besitzt jedoch im aktuellen Zustand drei wesentliche Schwachstellen:

\begin{itemize}
  \item Es beschreibt uns misst weiterhin ausschließlich den intrinsischen Wert der Vernetzung innerhalb unserer kleinen "Visions-Welt", dem es noch an Bezug zur "Außenwelt" und dem Business-Case fehlt. Diesen Umstand wollen wir weiterhin zunächst einmal ignorieren.
  \item Es bewertet in der aktuellen Form ausschließlich "unsere Welt" bzw. unseren Fortschritt als Ganzes. Die definierte "besser"-Relation misst das "Besser" aus Sicht der Allgemeinheit. Der einzelne Teilnehmer bleibt individuell unberücksichtigt. Es ist schwer vorstellbar, ein Ökosystem zu designen, welches intrinsisch nach dem Wohl/Optimum Aller strebt (und damit eben einmal einen formalen Beweis für das Funktionieren des Kommunismus zu liefern.) 
  \item Es lässt den sogenannten \href{https://de.wikipedia.org/wiki/Netzwerkeffekt}{Netzwerkeffekt} außer Acht! Denn selbst wenn man eben einmal das Problem des Bullet 1 aus der Welt schafft, und ein Preisschild an den Mehrwert einer Connection zwischen User und Provider bekommt. Die Literatur zum besagten Netzwerkeffekt liefert gute Argumente für die Annahme, eine von uns anvisierte User-Provider-Connection kann nur sehr selten alleinstehend in ihrem Mehrwert bewertet werden. Vielmehr bemisst sich dieser etwaige Mehrwert in dem Zusammenspiel und den Synergien mit anderen User-Provider-Connections. Es lassen sich viele Beispiele finden, um diesen Umstand zu begründen. So kann es z. B. sein, dass ein Finance-Aggregator-Service für einen User um so wertvoller wird, je mehr Finance-Provider der User selbst mit seiner WunderIdentity connectet. Hierbei wird es kaum einen Unterschied für ihn machen, ob die genannten Finance-Provider mit 100 anderen WunderPass-Usern connectet seien oder mit 10 Mio. Im Case einer Splitwise-Connection (oder auch einer etwaigen EventsWithFriends-App) dagegen entsteht der Mehrwert erst dann, wenn auch ganz viele Freunde des Users diese Splitwise-Connection mit WunderPass besitzen. Andernfalls beläuft sich der Mehrwert seiner eigenen Connection so ziemlich gen Null.
\end{itemize}

\vspace{1cm}

Insbesondere der letzte Punkt wirft einige interessante Fragen auf, zu denen wir eine Antwort finden werden müssen. Oder zumindest Hypothesen und Annahmen treffen.
Was bedeutet eigentlich

\begin{equation*}
  \alpha^{(t)}_{kj} * \alpha^{(t)}_{lj} = 1 \textrm{ für zwei User } u^{(t)}_k, u^{(t)}_l \in U^{(t)} \textrm{ die beide mit Privider } s^{(t)}_j \in S^{(t)} \textrm{ connectet sind?}
\end{equation*}
Sind diese dann damit gleichbedeutend in irgendeiner Weise ebenfalls "\textit{miteinander connectet}"? Und was würde eine solche Implikation für unser bisheriges Modell bedeuten? Wie (un)abhängig ist eine solche "indirekte Connection" von ihrer "Brücke" - dem Service-Provider? All diese Fragen lassen sich zudem analog auf "indirekte Connections" zwischen Providern übertragen - die dann etwaige User als "Brücke" nützten. Zu guter Letzt ließe sich diese neue Komplexität beliebig potenzieren, indem man mittels Rekursion indirekte Connections "zweitens, drittens,... Grades" definiert.

Um der aufkommenden Komplexität Herr zu werden, wollen wir uns zunächst einmal dem zweiten der oben genannten Schwachstellen unseres bisherigen Modells zuwenden, und dieses idealerweise dahingehend erweitern, auch individuelle Bewertungen unserer Teilnehmer $u \in U^{(t)}$ und $s \in S^{(t)}$ zu erfassen.


\subsubsection{Individuelle Wertschöpfung der Teilneher}
\label{sec:eco_zahlen_teilnehmer}

Hallo    % binde die Datei '[Economics][Quantifizierung].tex' ein
% !TEX root = paper.tex

\subsection{Token-Economics (WPT)}
\label{sec:eco_wpt}


\subsubsection{Einleitung}
\label{sec:wpt_einleitung}
\todo{WIP}

\begin{Praemisse}[generelle Anforderungen an den Token]

\begin{itemize}
  \item Der Token soll ein \textbf{echter} Utility-Token sein. Er braucht also zwingend einen \textbf{intrinsischen Wert}.
\end{itemize}

\vspace{0.2cm}

Die Teilnehmer (User und Provider) müssen einen intrinsischen Vorteil am Besitz von Tokens innerhalb des Ökosystems erfahren. Sie müssen quasi "irgendwas mit dem Token machen können" - und zwar innerhalb des Ökosystems und nicht mittels "Verkaufs nach außen". Wenn man als Teilnehmer die Möglichkeit besitzt, Tokens für/durch irgendetwas zu erwerben, muss auch die Möglichkeit bestehen, diesen für irgendetwas (anderes; "nützliches") innerhalb des Ökosystems auszugeben. Idealerweise verhält sich unser Token zur Euro, wie sich der Euro zum nicht existenten "Weltall-Taler" verhält - also ohne Rechtfertigung zu besitzen, das Ökosystem verlassen zu müssen.

Falls die Schaffung einer solchen Ökonomie nicht gelingen sollte - weil z.B. die Service-Provider mehr Value generieren, als sie innerhalb des Ökosystems "konsumieren" können -  sollte diese Forderung zumindest für den Teilnehmer "User" sichergestellt werden. Denn der User partizipiert in seinem Dasein eher als Konsument innerhalb des Digital-Universums, als als Wertschöpfer, weshalb seinem intrinsischen Vorteil am Besitz von Tokens mit dem damit ermöglichten Konsum von digitalen Dienstleistungen Genüge getan sein sollte.

\vspace{0.3cm}

\begin{itemize}
  \item Der Token sollte natürlich auch einen \textbf{extrinsischen} Wert besitzen.
\end{itemize}

\vspace{0.2cm}

Nicht all zu laut (der Community ggü.) kommuniziert, wäre unsere ganze Unternehmung im Falle des Fehlen des extrinsischen Werts nichts anderes als ein kommunistischer Akt. Nur diese Beschaffenheit des Tokens liefert uns ein Monetarisierungs-Modell. Und auch deutet zudem vieles darauf hin, die Service-Provider-Teilnehmer kämen ohne einen extrinsischen Wert nicht aus.

\vspace{0.3cm}

\begin{itemize}
  \item Der Token soll \textbf{nicht inflationär} sein - also einen definierten Cap besitzen.
\end{itemize}

\vspace{0.2cm}

Mit voriger Forderung - laut der man "etwas mit dem Token innerhalb des Systems machen kann", verleiht die gegenständige Forderung das dieses "Etwas", was mittels des Tokens ermöglicht wird, einem gewissen Qualitätsanspruch genügen muss. Je größer die Qualität dieses besagten "Etwas" - also z. B. einer Dienstleistung, die mit ausschließlich mit dem Token bezahlt werden kann - ist, desto \textit{wertvoller} wird auch der Besitz des Tokens. Und damit auch sowohl sein intrinsischer als auch extrinsischer Wert. Schlichtweg deshalb, weil der Token und somit der mögliche Konsum besagter Dienstleistung gecappt ist.

\vspace{0.3cm}

\begin{itemize}
  \item \todo{Kreislauf}.
\end{itemize}

\vspace{0.2cm}

\todo{Kreislauf-Beschreibung}

\end{Praemisse}

\vspace{0.3cm}


\begin{Praemisse}[Daten haben einen Wert]

\vspace{0.2cm}

\todo{TODO: Evaluierung extrem schwierig. Folgende Aussagen/Antworten sind zu beweisen.}

\vspace{0.2cm}

\begin{itemize}
  \item Wer besitzt Daten/Informationen?
  \item Für wen sind diese Daten von "Wert" (Geld verdienen)?
  \item Wie kann der Wert der Daten maximiert werden? Wer profitiert im welchen Maße davon?
  \item Wer würde für diese Daten bezahlen und wie viel?
  \item Wie ist die (maximale) Wertschöpfung zu verteilen? Wer wird beteiligt? Wie wird die maximierende Rolle der Wertschöpfung belohnt?
  \item Wer trägt etwaige Risiken und in welchem Verhältnis?
  \item Wie ist das alles in die Token-Economics zu integrieren? 
\end{itemize}

\end{Praemisse}


\vspace{0.3cm}

\begin{Fazit}[unser Ökosystem generiert Value]

\begin{itemize}
  \item Wir schöpfen Mehrwert, indem wir Datenerfassung ermöglichen (die ja einen nachgewiesenen Value besitzen?
  \item Besitzer der Daten werden entlohnt
  \item Nutzer der Daten zahlen für Daten, generieren damit aber Value, der wiederum entlohnt wird.
  \item Am Ende haben alle Teilnehmer entweder Value generiert oder aber im Wert des values verkonsumiert
  \item Wir partizipieren am extrinsischen Wert des Tokens (Kurs-Entwicklung durch positive Wertschöpfung des gesamten Ökosystems).
  \item Incentives sind nötig, um das Henne-Ei-Problem zu lösen
  \item Incentives sollten nachträglich mit der dadurch geschaffenen Wertschöpfung verrechtet werden. 
\end{itemize}

\end{Fazit}


\newpage
\subsubsection{Lösungsideen}
\vspace{0.3cm}
% !TEX root = paper.tex

\begin{itemize}
  \item Staken von Sub-Projects. 
  \begin{itemize}
  	\item Teilprojekt wird als \textbf{Curation Market} implementiert und bekommt damit seinen eigenen Token.
  	\item Die Projekteinlage erfolge in WUNDER-Tokens (Staking).
  	\item Investoren von WUNDER hätten damit die Möglichkeit, die für sie besonders interessanten Projekte stärker zu unterstützen als lediglich das übergeordnete WunderPass-Projekt.
  	\item Der WUNDER bekäme damit einen intrinsischen Wert: Man braucht ihn, um sich an den Teilprojekten zu beteiligen.
  \end{itemize}
  \item Idee für die Investment-Pools. 
  \begin{itemize}
  	\item Das Pool-Projekt wird als \textbf{Curation Market} implementiert und bekommt seinen eigenen Token (IPT).
  	\item Der IPT wird mittels \textbf{(Augmented) Bonding-Curves} implementiert, ist also gegen eine Einlage für jeden und immer mintbar.
  	\item Die Einlage für den IPT ist in WUNDER zu erbringen \todo{(Es ist noch unklar, wie man an WUNDER kommt, wenn es vorher keinen Token-Sale gegeben hat. Ob der WUNDER ebenfalls mittels Bonding-Curves abzubilden wäre, sei hier erst einmal mehr als unklar.)}
  	\item Der erste und größere Investor für das Pool-Projekt wäre WunderPass selbst. Für die erfolgte Einlage in den Projekt-Pool bekäme WunderPass IPT, die es für Incentivierungen und Rewards für die Nutzung von Pools verwenden könnte. Dieses Invest könnte (im Gegensatz zu den Einlagen anderer Investoren) zB. auch einem Locking unterliegen, um eine gewisse Preisstabilität des IPS zu gewährleisten.
  	\item Der Pool-Initiator müsste bei der Pool-Eröffnung IPT staken, die er unter bestimmten Umständen verlieren könnte, wenn sein Pool zB. ungenutzt bleibt. So könnte man sicherstellen, dass ernste Absichten hinter den Pools stecken und diese auch genutzt werden. 
  	\item Der Initiator wäre damit gleichzeitig auch Investor in das gesamte Pool-Projekt (da er ja für die Poolerstellung IPT kaufen muss).
  	\item Gleichzeitig müsste der Initiator jedoch auch für sein Staken (ins Risiko gehen) belohnt werden, falls der Pool läuft und genutzt wird. Diese Belohnung würde in Form von zusätzlichen IPT (Stake-Rewards) erfolgen und z.B durch WunderPass und/oder den anderen Poolteilnehmern als eine Art Gebühr getragen werden und dabei folgenden Faktoren folgen:
  	\begin{itemize}
  		\item Pool-Lifetime
  		\item Anzahl Teilnehmer
  		\item Pool-Einsatz/-Umsatz
  		\item etwaiger Gewinn aus Invests
  		\item NFT-Pass-Status
  		\item Upvoting durch andere IPT-Holder (als ein Art \textit{Master Pool-Creator})
  	\end{itemize}
  	\item In jedem Fall sollte der Staker im Normalfall (falls er nicht irgendwie Scheiße baut) bei der Auflösung des Pools mindestens seinen Einsatz zurückerhalten (also keinerlei Gebühren für die Nutzung des Pool-Service zahlen). In aller Regel sollte er mit mehr als dem ursprünglich gestakten Betrag rausgehen.
  	\item Der nötige Staking-Betrag könnte fix pro Pool sein oder aber variabel und dabei von folgenden Kriterien abhängen:
  	\begin{itemize}
  		\item geplante Pool-Lifetime
  		\item Anzahl Teilnehmer (min/max)
  		\item Pool-Einsatz pro Teilnehmer (min/max)
  		\item etwaige abgegeben Garantien seitens Pool-Creator (\textit{der Pool muss mindestens x, y und z erfüllen...}, bei deren Verfehlungen der Staker bestraft und im Erfolgsfall besonders entlohnt wird)
  		\item NFT-Pass-Status
  		\item Reputation in der Community (als ein Art \textit{Master Pool-Creator}
  	\end{itemize}
  	Die entscheidende Frage beim zu entrichtenden Stake-Betrag, ist die Klärung, ob es im Interesse des Stakers (Pool-Creator) sei, besonders viel (um größere Staking-Rewards zu erhalten) oder besonders wenig (um kein Risiko zu tragen) zu staken / staken zu müssen.
  	\item Ein weiterer sehr essenzieller Faktor für die Größe des zu stakenden Betrags könnte der Kurs des IPTs sein. Denn laut der \textbf{Bonding-Curves}-Implementierung würde der IPT-Preis mit steigender Zirkulation steigen, was mit der Zunahme von existierende Pools geschähe. Damit wäre die Erstellung neuer Pools mit ihrer zahlenmäßigen Zunahme stets kapital-intensiver (aber nicht gleichbedeutend teurer). \textbf{Die Frage hierbei ist also, ob der zu erbringende Stake des Pool-Creators auf den \textit{Total-Supply des IPTs} normiert werden sollte oder nicht}, die gänzlich mit der obigen Fragestellung einhergeht, ob der Pool-Creator eigentlich staken möchte oder das nur tun muss.
  	\begin{itemize}
  		\item Gegen eine Normierung spricht die Annahme/Hoffnung, ein Pool-Creator sei gleichzeitig auch ein großer Supporter des gesamten Projekt und glaube daran. Wenn der IPT-Preis steigt, ist dies gleichbedeutend mit der Zunahme an genutzten Pools, an denen der Pool-Creator als Staker, Besitzer von IPT und damit Projekt-Investor auch selbst (finanziell) profitiert.
  		\item Für eine Normierung spricht dagegen die potenzielle Gefahr, neue oder bestehende User durch eine zu hohe finanzielle Sicherheitseinlage davon abzuschrecken neue Pools zu erstellen.
  	\end{itemize}
  	Die Antwort auf diese Fragestellung könnte auch darin liegen, ob wir uns besonders viele oder lieber weniger aber besonders Teilnehmer-starke Pools wünschen.	
  	\item Alle anderen Pool-Teilnehmer müssen eine Gebühr für ihre Teilnahme am Pool (und die Nutzung des Service) erbringen. Auch das hat in IPT zu erfolgen. \todo{Ob die Gebühr bei Pool-Beitritt gewissermaßen als \textit{prepaid} zu erbringen ist oder aber \textit{on demand} für eine anfallende Aktion bleibt zunächst unklar}. Die Gebühren könnten sich nach folgenden Faktoren bzw. Features richten und könnten sowohl voraussehbar sein (dann beim Pool-Beitritt zu entrichten) als auch \textit{on demand} anfallen:
  	\begin{itemize}
  		\item pro Zeiteinheit (Vorauszahlung für einen gewissen Zeitraum als prepaid; danach Zahlungsaufforderung um im Pool zu bleiben
  		\item abhängig vom Einsatz (multiplikativ zum ersten Punkt)
  		\item abhängig vom etwaigen Gewinn aus Invests (on demand)
  		\item abhängig vom Pool-Creator (Community-/Staking-Status als Invest-Guru; prepaid)
  		\item bei Aufstockung des Invests (on demand)
  		\item bei Vorzeitigem Ausbezahlen und Verlassen des Pools (on demand)
  		\item gekoppelt an Beteiligung am Staken (prozentual auf die vorigen Punkte anzurechnen)
  		\item abhängig vom NFT-Pass-Status (prozentual auf die vorigen Punkte anzurechnen)
  	\end{itemize} 
  	\item Die Teilnehmer können bei dieser Logik aber nicht wie nicht wie die Staker zusätzlich als Projekt-Investoren angesehen werden, weil sie IPTs kaufen, da die gekauften IPTs direkt als Gebühr entrichtet werden. Für die Pool-Teilnehmer stellt der IPT also eher einen Utility- bzw. Purpose-Token dar weshalb die Höhe der zu entrichtenden Gebühr zweifelsfrei auf Basis von \textit{Total-Supply des IPTs} normiert werden muss (die Gebühr darf keinesfalls mit Zunahme von Pools steigen).
  	\item Die entrichteten \textbf{Gebühren gehen zu einem relevanten Teil in die Treasury des gemeinschaftlichen Investing-Pools-Contracts} und stellen damit eine gemeinschaftliche Projekt-Erwirtschaftung dar. Der verbleibende Teil der Gebühren geht an den Pool-Staker. Die Aufschlüsslung der Verteilung auf Project-Treasury und Pool-Staker könnte halbwegs komplex werden und folgende Festlegungen folgen bzw. Gegebenheiten berücksichtigen:
  	\begin{itemize}
  		\item Verteilung nach einem simplen prozentualen Schlüssel (zB. 50-50)
  		\item Absolute Mindest- und Obergrenzen des Projekt-Pool-Anteils (mindestens Betrag x geht an den Projekt-Pool; wenn es nicht reicht, alles; ab der Mindestgrenze erfolgt eine prozentuale Verteilung bis zu einer Maximalgrenze y für den Projekt-Pool; alles darüber geht an den Staker)
  		\item progressive Verteilung abhängig des erbrachten Staking-Betrags (so könnte der Staker pro gestaktem IPT einen prozentualem Anteil $x \in [0; 1]$ pro IPT an Gebühren für sich beanspruchen, wobei das $x$ mit Größe des gestakten Betrags progressiv stiege)
  		 \item Begünstigung des Stakers in Abhängigkeit seines NFT-Pass-Status.
  	\end{itemize}
  	\item Folgende Auswahl sollte vermutlich bei der Pool-Erstellung angeboten werden:
  	\begin{itemize}
  		\item $[$Pool-Creator erbringt den Stake; alle anderen Teilnehmer bezahlen die Gebühren$]$
  		\item $[$Alle Pool-Teilnehmer teilen sich den zu erbringenden Stake und alle anfallenden Gebühren$]$
  	\end{itemize} 
  	\item Bedingt durch den Gebühren-Mechanismus erwirtschaften das Pool-Projekt Einnahme für die gemeinschaftliche Pool-Treasury, womit auch der IPT auf natürliche Weise im Wert steigt (ohne dass neue Investoren hinzukommen müssen, die einen höheren Tokenpreis bezahlen müssen). Jeder IPT-Holder - also auch insbesondere die Pool-Staker - profitiert also an der Erstellung neuer Pools (der Staker also indirekt an seinem eigenen Pool als auch an den fremden, was additiv zu seinen direkten Stake-Rewards hinzukommt). Das fördert also Word-of-Mouth, was sowohl auf die Preissteigerung des IPTs durch neue IPT-Holder einzahlt, als auch die Pool-Treasury durch neue Pools und die dafür anfallenden Gebühren füttert, was wiederum den IPT-Kurs befeuert.
  	\item
  \end{itemize}
\end{itemize}

\vspace{0.3cm}

\underline{\textbf{Einsatzmöglichkeit von Bonding-Curves innerhalb unseres Token-Economics:}}

\vspace{0.2cm}

Zunächst einmal eine Formalisierung einer Token-Distribution mittels Bonding-Curves:

\vspace{0.2cm}

\begin{Def}[Token-Distribution mittels Bonding-Curves]
\label{defBC}

Angenommen man möchte einen Projekt-Token \textbf{TKN} herausgeben und dieses im Markt distribuieren. Der Mechanismus der \textit{Bonding-Curves} stellt hierbei ein alternatives Modell zu gängigen Tokensales (z.B. ICO) dar und folgt dabei einigen wesentlich Merkmalen, die ihn teils grundlegend von herkömmlichen Tokensales abgrenzen.

\begin{itemize}
  \item TKN wird von einem Smart-Contract verwaltet, der wesentlich mehr Logik implementiert als ein herkömmlicher ERC20-Contract.
  \item TKN kann jederzeit und von jedem gemintet werden. Dies geschieht gegen eine Einlage/Bezahlung in einer dafür definierten Währung (z.B. ETH oder USDT). Der Token wird quasi von dem Verwalter-Contract verkauft.
  \item Es existiert damit keine initiale bevorzugte Token-Ausgabe beim Contact-Launch an etwaige bevorzugte Parteien (Contract-Owner, Herausgeber, Investoren etc.). Die Ausgabe erfolgt ausschließlich gegen Einlage und kennt keine Bevorzugung entgegen der Contract-Logik. 
  \item Die Einnahmen aus der Tokenausgabe kommen ausschließlich der Token-Cont\-ract-Treasury $\mathbf{\mathcal{T}}$ zugute anstatt wie bei herkömmlichen Tokensales bestimmten Begünstigten (wie z.B. der Herausgeber-Company oder deren Gründern).
  \item TKN unterliegt keinem maximalen Gesamt-Supply. Es können stets neue Tokens ausgegeben werden, solange Interessenten existieren, die die Einlage für die Token-Ausgabe erbringen.
  \item Tokens können jederzeit von ihren Besitzern gegen einen - einer bestimmten Contract-Logik folgenden - Rückkaufpreis an den Token-Contract zurückgegeben werden. Zurückgegebene Tokens werden dabei sofort von dem Verwalter-Contract geburnt und somit aus der Zirkulation genommen.
  \item TKN kann selbstverständlich auch am Sekundärmarkt gehandelt werden (falls dieser bessere Konditionen hergibt als die Aussage bzw. Rücknahme durch den Token-Contract selbst).
\end{itemize}

\vspace{0.2cm}

Sei $i \in \mathbb{N}$ der aktuelle Gesamt-Supply von TKR (wir nehmen hier mal an, TKR sei atomar) und \textbf{\$} die Tausch- bzw. Einlage-Währung (\textbf{\$} ist hier abstrakt und nicht als US-Dollar zu verstehen).

Dann werden die durch die Token-Contract vorgegebenen Ausgabepreis (Kaufpreis) $\mathcal{K}$, Rücknahmepreis (Verkaufspreis) $\mathcal{V}$ und Contract-Treasury-Inhalt $\mathbf{\mathcal{T}}$ für den zuletzt ausgegebenen Token $i$ - jeweils in der Einheit \textbf{\$} - durch die jeweils supply-abhängigen Funktionen beschrieben:

\begin{align*}
\mathcal{K}, \mathcal{V}, \mathcal{T} &: \mathbb{N} \rightarrow \mathbb{Q}^{+} \\
\mathcal{K} \left( i \right) &:= \textrm{ Letzter Token-Ausgabepreis in \$ bei einem Gesamt-Supply von i} \\
\mathcal{V} \left( i \right) &:= \textrm{ Aktueller Token-Rückkaufkurs in \$ bei einem Gesamt-Supply von i} \\
\mathcal{T} \left( i \right) &:= \sum_{j = 1}^{i} \mathcal{K} \left( j \right)
\end{align*}

\vspace{0.2cm}

Die definierende Logik unseres Tokens lässt sich damit also formal als 

\begin{equation*}
TKR = \left( \mathcal{K}, \mathcal{V}, \$ \right)
\end{equation*}

schreiben, wobei hierbei $\mathbf{\mathcal{T}}$ ausgespart bleibt, da es implizit durch $\mathbf{\mathcal{K}}$ gegeben ist.

\vspace{0.2cm}

Sicherlich können und werden bei einer konkreten Implementierung eines mittels der durch $\mathbf{\mathcal{K}}$ und $\mathbf{\mathcal{V}}$ gegebenen \textit{Bonding-Curves} beschriebenen Tokens noch andere zu formalisierende Faktoren und Mechanismen eine Rolle spielen. Für die simpelste abstrakte Definition reichen die genannten Größen 
jedoch für den Moment aus.

\end{Def}

\vspace{0.3cm}

Die eben - auf diskrete Weise - definierten Funktionen $\mathcal{K} \left( i \right)$ und $\mathcal{V} \left( i \right)$ erfordern insofern noch eine zusätzliche Bemerkung, als dass diese explizit nur eine Token-weise Auskunft über Ausgabe- und Rücknahmepreis geben. Möchte man mehrere Token minten oder zurückkgeben, muss der Preis für jeden der Tokens separat ausgerechnet und anschließend addiert werden. Möchte man bei einem aktuellen Supply von $i \in \mathbb{N}$ nicht einen sondern $k \in \mathbb{N}$ mit $k \leq i$ Tokens minten bzw. zurückgeben, beläuft sich der gesamte Kauf- bzw- Verkaufspreis auf

\begin{align*}
\mathcal{K}_{k} \left( i \right) &:= \sum_{j = i + 1}^{i + k} \mathcal{K} \left( j \right) \\
\mathcal{V}_{k} \left( i \right) &:= \sum_{j = i - k + 1}^{i} \mathcal{V} \left( j \right).
\end{align*}

\vspace{0.3cm}

Das besonders Hervorhebenswerte an diesem Token-Ausgabemodell ist zweifelsfrei die gemeinschaftliche aus der Token-Ausgabe gefütterte Contract-Treasury aus echten Geldeinlagen, die nicht etwa einer dritten (Ausgabe-)Partei zugute kommt, sondern de facto den Tokeninhabern gehört. Die Existenz dieser Rücklagen gibt den ausgegebenen Tokens theoretisch einen realen Wert und ermöglicht einzig und allein die Implementierung des Rückkauf-Mechanismus. Der Gebrauch von dieser Möglichkeit und die Verankerung der Rückkauffunktion $\mathbf{\mathcal{V}}$ in der Logik des Token-Contracts gibt den Tokens dann auch praktisch einen realen Wert. Denn wenn ich eine definitive - in Contract-Logik verankerte - Sicherheit habe, ein Asset jederzeit verkaufen zu können, besitzt dieses Asset auch einen echten, intrinsischen Value, der nicht zwingend den marktwirtschaftlichen Mechanismen unterliegt - und somit auch keinen etwaigen Hypes um einen gut vermarkteten Tokensale ohne dahinterliegende Substanz. Vielmehr folgt der Tokenpreis der gemeinschaftlichen Projekt-Treasury $\mathbf{\mathcal{T}}$, die ihrerseits substanziell mit dem Projekterfolg einhergeht. 

\vspace{0.2cm}

Die letzte Einsicht lässt uns zu zwei wesentlichen Gedanken gelangen, die die entscheidenden Argumente für das \textit{Bonding-Curves}-Modell liefern könnten:

\begin{itemize}
  \item Der Rückgabepreis $\mathcal{V} \left( i \right)$ sollte eine direkte Abhängigkeit vom Treasury-Inhalt $\mathcal{T} \left( i \right)$ aufweisen.
  \item Nach bisheriger Definition hängt der Treasury-Inhalt $\mathcal{T} \left( i \right)$ ausschließlich vom aktuellen Supply $i \in \mathbb{N}$ (und den damit einhergehenden Ausgabepreisen $\mathcal{K} \left( j \right)$ für $j \leq i$) ab. Gepaart mit der ersten Forderung bedeutete dies implizit nichts anderes, als dass der aktuelle Rücknahmepreis ausschließlich von den Ausgabepreisen der bisherigen Tokens abhinge. Dies ist schlecht und ein sehr großes Problem der gängigen \textit{Bonding-Curves}-Implementierungen, da Koppelung des Rücknahmepreises - also des intrinsischen Werts des Tokens - ausschließlich an den Kaufpreis voriger Tokens - und die Wertentwicklung damit an den Kaufpreis etwaiger zukünftig ausgegebener Tokens, würde mathematisch alternativlos eine monoton steigende Ausgabepreis-Funktion $\mathcal{K} \left( i \right)$ erfordern. Ohne eine sehr stark fundierte projekt-bezogene Argumentation für ein monoton steigendes $\mathcal{K} \left( i \right)$ schrie das gesamte Modell nur so nach \textit{Pump \& Dump} und \textit{Hot Potatoe}. Und tatsächlich ist es so, dass nahezu alle \textit{Bonding-Curves}-Implementierungen Gebrauch von einer (streng) monoton steigenden Ausgabepreis-Kurve $\mathcal{K} \left( i \right)$ machen. Sie argumentieren mit anderem generierten Projekt-Value, der nicht durch die Contract-Treasury $\mathcal{T} \left( i \right)$ gemessen werden kann und diese Argumentation muss nicht zwingend falsch oder ungenügend sein. Uns reicht dies aber nicht - zumal es mehr als gute Gründe für ein monoton steigendes $\mathcal{K} \left( i \right)$ gibt (dazu später mehr). Somit bleibt uns nichts anderes als die - zumindest teilweise - Abkoppelung von $\mathcal{T} \left( i \right)$ und $\mathcal{K} \left( i \right)$, womit wir auch unsere obige Definition von $\mathcal{T} \left( i \right) = \sum_{j = 1}^{i} \mathcal{K} \left( j \right)$ wieder teils verwerfen müssen.
\end{itemize} 

\vspace{0.2cm}

Wir fassen zusammen:

\vspace{0.3cm}

\begin{Fazit}[Contract-Treasury als wichtigster Baustein zum Erfolg]

Wie sind der Überzeugung, der Contract-Treasury-Inhalt - gemessen als $\mathcal{T} \left( i \right)$ - sei der entscheidende Baustein für einen soliden \textit{Bonding-Curves}-Token. $\mathcal{T} \left( i \right)$ beeinflusst direkt den Rücknahmepreis $\mathcal{V} \left( i \right)$, verleiht dem Token damit einen echten geldwerten Value, was wiederum als Kaufargument für neue Tokens gilt und damit implizit auch zur Beurteilung des Ausgabepreises $\mathcal{K} \left( i \right)$ seitens etwaiger neuer Investoren hinzugezogen wird.

\vspace{0.2cm}

Konkret auf den Rücknahmepreis $\mathcal{V} \left( i \right)$ bezogen sehen wir kaum sinnvollere Alternativen als der gängigen Definition 

\begin{equation*}
\mathcal{V} \left( i \right) = \frac{\mathcal{T} \left( i \right)}{i}
\end{equation*}

zu folgen, was gleichbedeutend damit ist, dass der Besitz am Contract-Treasury-Inhalt pro rata auf die sich in Zirkulation befindenden Token verteilt wird, was sich konsequenterweise im Rücknahmepreis $\mathcal{V} \left( i \right)$ widerspiegelt. Mit der in Definition \ref{defBC} beschriebenen Treasury-Funktion $\mathcal{T} \left( i \right)$ ergibt sich damit für den Rücknahmepreis $\mathcal{V} \left( i \right)$

\begin{equation*}
\mathcal{V} \left( i \right) = \frac{\mathcal{T} \left( i \right)}{i} = \frac{\sum_{j = 1}^{i} \mathcal{K} \left( j \right)}{i}.
\end{equation*} 

\vspace{0.3cm}

\end{Fazit}

\newpage
%\vspace{0.5cm} % binde die Datei '[Economics][Token-Economics][Ideen].tex' ein
\subsubsection{Einbindung der Investing-Pools}
\vspace{0.3cm}
% !TEX root = paper.tex

\begin{itemize}
	\item Das Pool-Projekt wird als \textbf{Curation Market} implementiert und bekommt seinen eigenen Token (IPT).
	\item Der IPT wird mittels \textbf{(Augmented) Bonding-Curves} implementiert, ist also gegen eine Einlage für jeden und immer mintbar.
	\item Die Einlage für den IPT ist in WUNDER zu erbringen \todo{(Es ist noch unklar, wie man an WUNDER kommt, wenn es vorher keinen Token-Sale gegeben hat. Ob der WUNDER ebenfalls mittels Bonding-Curves abzubilden wäre, sei hier erst einmal mehr als unklar.)}
	\item Der erste und größere Investor für das Pool-Projekt wäre WunderPass selbst. Für die erfolgte Einlage in den Projekt-Pool bekäme WunderPass IPT, die es für Incentivierungen und Rewards für die Nutzung von Pools verwenden könnte. Dieses Invest könnte (im Gegensatz zu den Einlagen anderer Investoren) zB. auch einem Locking unterliegen, um eine gewisse Preisstabilität des IPS zu gewährleisten.
	\item Der Pool-Initiator müsste bei der Pool-Eröffnung IPT staken, die er unter bestimmten Umständen verlieren könnte, wenn sein Pool zB. ungenutzt bleibt. So könnte man sicherstellen, dass ernste Absichten hinter den Pools stecken und diese auch genutzt werden. 
	\item Der Initiator wäre damit gleichzeitig auch Investor in das gesamte Pool-Projekt (da er ja für die Poolerstellung IPT kaufen muss).
	\item Gleichzeitig müsste der Initiator jedoch auch für sein Staken (ins Risiko gehen) belohnt werden, falls der Pool läuft und genutzt wird. Diese Belohnung würde in Form von zusätzlichen IPT (Stake-Rewards) erfolgen und z.B durch WunderPass und/oder den anderen Poolteilnehmern als eine Art Gebühr getragen werden und dabei folgenden Faktoren folgen:
	\begin{itemize}
		\item Pool-Lifetime
		\item Anzahl Teilnehmer
		\item Pool-Einsatz/-Umsatz
		\item etwaiger Gewinn aus Invests
		\item NFT-Pass-Status
		\item Upvoting durch andere IPT-Holder (als ein Art \textit{Master Pool-Creator})
	\end{itemize}
	\item In jedem Fall sollte der Staker im Normalfall (falls er nicht irgendwie Scheiße baut) bei der Auflösung des Pools mindestens seinen Einsatz zurückerhalten (also keinerlei Gebühren für die Nutzung des Pool-Service zahlen). In aller Regel sollte er mit mehr als dem ursprünglich gestakten Betrag rausgehen.
	\item Der nötige Staking-Betrag könnte fix pro Pool sein oder aber variabel und dabei von folgenden Kriterien abhängen:
	\begin{itemize}
		\item geplante Pool-Lifetime
		\item Anzahl Teilnehmer (min/max)
		\item Pool-Einsatz pro Teilnehmer (min/max)
		\item etwaige abgegeben Garantien seitens Pool-Creator (\textit{der Pool muss mindestens x, y und z erfüllen...}, bei deren Verfehlungen der Staker bestraft und im Erfolgsfall besonders entlohnt wird)
		\item NFT-Pass-Status
		\item Reputation in der Community (als ein Art \textit{Master Pool-Creator}
	\end{itemize}
	Die entscheidende Frage beim zu entrichtenden Stake-Betrag, ist die Klärung, ob es im Interesse des Stakers (Pool-Creator) sei, besonders viel (um größere Staking-Rewards zu erhalten) oder besonders wenig (um kein Risiko zu tragen) zu staken / staken zu müssen.
	\item Ein weiterer sehr essenzieller Faktor für die Größe des zu stakenden Betrags könnte der Kurs des IPTs sein. Denn laut der \textbf{Bonding-Curves}-Implementierung würde der IPT-Preis mit steigender Zirkulation steigen, was mit der Zunahme von existierende Pools geschähe. Damit wäre die Erstellung neuer Pools mit ihrer zahlenmäßigen Zunahme stets kapital-intensiver (aber nicht gleichbedeutend teurer). \textbf{Die Frage hierbei ist also, ob der zu erbringende Stake des Pool-Creators auf den \textit{Total-Supply des IPTs} normiert werden sollte oder nicht}, die gänzlich mit der obigen Fragestellung einhergeht, ob der Pool-Creator eigentlich staken möchte oder das nur tun muss.
	\begin{itemize}
		\item Gegen eine Normierung spricht die Annahme/Hoffnung, ein Pool-Creator sei gleichzeitig auch ein großer Supporter des gesamten Projekt und glaube daran. Wenn der IPT-Preis steigt, ist dies gleichbedeutend mit der Zunahme an genutzten Pools, an denen der Pool-Creator als Staker, Besitzer von IPT und damit Projekt-Investor auch selbst (finanziell) profitiert.
		\item Für eine Normierung spricht dagegen die potenzielle Gefahr, neue oder bestehende User durch eine zu hohe finanzielle Sicherheitseinlage davon abzuschrecken neue Pools zu erstellen.
	\end{itemize}
	Die Antwort auf diese Fragestellung könnte auch darin liegen, ob wir uns besonders viele oder lieber weniger aber besonders Teilnehmer-starke Pools wünschen.	
	\item Alle anderen Pool-Teilnehmer müssen eine Gebühr für ihre Teilnahme am Pool (und die Nutzung des Service) erbringen. Auch das hat in IPT zu erfolgen. \todo{Ob die Gebühr bei Pool-Beitritt gewissermaßen als \textit{prepaid} zu erbringen ist oder aber \textit{on demand} für eine anfallende Aktion bleibt zunächst unklar}. Die Gebühren könnten sich nach folgenden Faktoren bzw. Features richten und könnten sowohl voraussehbar sein (dann beim Pool-Beitritt zu entrichten) als auch \textit{on demand} anfallen:
	\begin{itemize}
		\item pro Zeiteinheit (Vorauszahlung für einen gewissen Zeitraum als prepaid; danach Zahlungsaufforderung um im Pool zu bleiben
		\item abhängig vom Einsatz (multiplikativ zum ersten Punkt)
		\item abhängig vom etwaigen Gewinn aus Invests (on demand)
		\item abhängig vom Pool-Creator (Community-/Staking-Status als Invest-Guru; prepaid)
		\item bei Aufstockung des Invests (on demand)
		\item bei Vorzeitigem Ausbezahlen und Verlassen des Pools (on demand)
		\item gekoppelt an Beteiligung am Staken (prozentual auf die vorigen Punkte anzurechnen)
		\item abhängig vom NFT-Pass-Status (prozentual auf die vorigen Punkte anzurechnen)
	\end{itemize} 
	\item Die Teilnehmer können bei dieser Logik aber nicht wie nicht wie die Staker zusätzlich als Projekt-Investoren angesehen werden, weil sie IPTs kaufen, da die gekauften IPTs direkt als Gebühr entrichtet werden. Für die Pool-Teilnehmer stellt der IPT also eher einen Utility- bzw. Purpose-Token dar weshalb die Höhe der zu entrichtenden Gebühr zweifelsfrei auf Basis von \textit{Total-Supply des IPTs} normiert werden muss (die Gebühr darf keinesfalls mit Zunahme von Pools steigen).
	\item Die entrichteten \textbf{Gebühren gehen zu einem relevanten Teil in die Treasury des gemeinschaftlichen Investing-Pools-Contracts} und stellen damit eine gemeinschaftliche Projekt-Erwirtschaftung dar. Der verbleibende Teil der Gebühren geht an den Pool-Staker. Die Aufschlüsslung der Verteilung auf Project-Treasury und Pool-Staker könnte halbwegs komplex werden und folgende Festlegungen folgen bzw. Gegebenheiten berücksichtigen:
	\begin{itemize}
		\item Verteilung nach einem simplen prozentualen Schlüssel (zB. 50-50)
		\item Absolute Mindest- und Obergrenzen des Projekt-Pool-Anteils (mindestens Betrag x geht an den Projekt-Pool; wenn es nicht reicht, alles; ab der Mindestgrenze erfolgt eine prozentuale Verteilung bis zu einer Maximalgrenze y für den Projekt-Pool; alles darüber geht an den Staker)
		\item progressive Verteilung abhängig des erbrachten Staking-Betrags (so könnte der Staker pro gestaktem IPT einen prozentualem Anteil $x \in [0; 1]$ pro IPT an Gebühren für sich beanspruchen, wobei das $x$ mit Größe des gestakten Betrags progressiv stiege)
		 \item Begünstigung des Stakers in Abhängigkeit seines NFT-Pass-Status.
	\end{itemize}
	\item Folgende Auswahl sollte vermutlich bei der Pool-Erstellung angeboten werden:
	\begin{itemize}
		\item $[$Pool-Creator erbringt den Stake; alle anderen Teilnehmer bezahlen die Gebühren$]$
		\item $[$Alle Pool-Teilnehmer teilen sich den zu erbringenden Stake und alle anfallenden Gebühren$]$
	\end{itemize} 
	\item Bedingt durch den Gebühren-Mechanismus erwirtschaften das Pool-Projekt Einnahme für die gemeinschaftliche Pool-Treasury, womit auch der IPT auf natürliche Weise im Wert steigt (ohne dass neue Investoren hinzukommen müssen, die einen höheren Tokenpreis bezahlen müssen). Jeder IPT-Holder - also auch insbesondere die Pool-Staker - profitiert also an der Erstellung neuer Pools (der Staker also indirekt an seinem eigenen Pool als auch an den fremden, was additiv zu seinen direkten Stake-Rewards hinzukommt). Das fördert also Word-of-Mouth, was sowohl auf die Preissteigerung des IPTs durch neue IPT-Holder einzahlt, als auch die Pool-Treasury durch neue Pools und die dafür anfallenden Gebühren füttert, was wiederum den IPT-Kurs befeuert.
	\item
\end{itemize}

\vspace{0.5cm}
 % binde die Datei '[Economics][Token-Economics][Investing-Pools].tex' ein
\subsubsection{Bonding-Curves}
\vspace{0.3cm}
% !TEX root = paper.tex

Zunächst einmal eine Formalisierung einer Token-Distribution mittels Bonding-Curves:

\vspace{0.2cm}

\begin{Def}[Token-Distribution mittels Bonding-Curves]
\label{defBC}

Angenommen man möchte einen Projekt-Token \textbf{TKN} herausgeben und dieses im Markt distribuieren. Der Mechanismus der \textit{Bonding-Curves} stellt hierbei ein alternatives Modell zu gängigen Tokensales (z.B. ICO) dar und folgt dabei einigen wesentlich Merkmalen, die ihn teils grundlegend von herkömmlichen Tokensales abgrenzen.

\begin{itemize}
  \item TKN wird von einem Smart-Contract verwaltet, der wesentlich mehr Logik implementiert als ein herkömmlicher ERC20-Contract.
  \item TKN kann jederzeit und von jedem gemintet werden. Dies geschieht gegen eine Einlage/Bezahlung in einer dafür definierten Währung (z.B. ETH oder USDT). Der Token wird quasi von dem Verwalter-Contract verkauft.
  \item Es existiert damit keine initiale bevorzugte Token-Ausgabe beim Contact-Launch an etwaige bevorzugte Parteien (Contract-Owner, Herausgeber, Investoren etc.). Die Ausgabe erfolgt ausschließlich gegen Einlage und kennt keine Bevorzugung entgegen der Contract-Logik. 
  \item Die Einnahmen aus der Tokenausgabe kommen ausschließlich der Token-Cont\-ract-Treasury $\mathbf{\mathcal{T}}$ zugute anstatt wie bei herkömmlichen Tokensales bestimmten Begünstigten (wie z.B. der Herausgeber-Company oder deren Gründern).
  \item TKN unterliegt keinem maximalen Gesamt-Supply. Es können stets neue Tokens ausgegeben werden, solange Interessenten existieren, die die Einlage für die Token-Ausgabe erbringen.
  \item Tokens können jederzeit von ihren Besitzern gegen einen - einer bestimmten Contract-Logik folgenden - Rückkaufpreis an den Token-Contract zurückgegeben werden. Zurückgegebene Tokens werden dabei sofort von dem Verwalter-Contract geburnt und somit aus der Zirkulation genommen.
  \item TKN kann selbstverständlich auch am Sekundärmarkt gehandelt werden (falls dieser bessere Konditionen hergibt als die Aussage bzw. Rücknahme durch den Token-Contract selbst).
\end{itemize}

\vspace{0.2cm}

Sei $i \in \mathbb{N}$ der aktuelle Gesamt-Supply von TKR (wir nehmen hier mal an, TKR sei atomar) und \textbf{\$} die Tausch- bzw. Einlage-Währung (\textbf{\$} ist hier abstrakt und nicht als US-Dollar zu verstehen).

Dann werden die durch die Token-Contract vorgegebenen Ausgabepreis (Kaufpreis) $\mathcal{K}$, Rücknahmepreis (Verkaufspreis) $\mathcal{V}$ und Contract-Treasury-Inhalt $\mathbf{\mathcal{T}}$ für den zuletzt ausgegebenen Token $i$ - jeweils in der Einheit \textbf{\$} - durch die jeweils supply-abhängigen Funktionen beschrieben:

\begin{align*}
\mathcal{K}, \mathcal{V}, \mathcal{T} &: \mathbb{N} \rightarrow \mathbb{Q}^{+} \\
\mathcal{K} \left( i \right) &:= \textrm{ Letzter Token-Ausgabepreis in \$ bei einem Gesamt-Supply von i} \\
\mathcal{V} \left( i \right) &:= \textrm{ Aktueller Token-Rückkaufkurs in \$ bei einem Gesamt-Supply von i} \\
\mathcal{T} \left( i \right) &:= \sum_{j = 1}^{i} \mathcal{K} \left( j \right)
\end{align*}

\vspace{0.2cm}

Die definierende Logik unseres Tokens lässt sich damit also formal als 

\begin{equation*}
TKR = \left( \mathcal{K}, \mathcal{V}, \$ \right)
\end{equation*}

schreiben, wobei hierbei $\mathbf{\mathcal{T}}$ ausgespart bleibt, da es implizit durch $\mathbf{\mathcal{K}}$ gegeben ist.

\vspace{0.2cm}

Sicherlich können und werden bei einer konkreten Implementierung eines mittels der durch $\mathbf{\mathcal{K}}$ und $\mathbf{\mathcal{V}}$ gegebenen \textit{Bonding-Curves} beschriebenen Tokens noch andere zu formalisierende Faktoren und Mechanismen eine Rolle spielen. Für die simpelste abstrakte Definition reichen die genannten Größen 
jedoch für den Moment aus.

\end{Def}

\vspace{0.3cm}

Die eben - auf diskrete Weise - definierten Funktionen $\mathcal{K} \left( i \right)$ und $\mathcal{V} \left( i \right)$ erfordern insofern noch eine zusätzliche Bemerkung, als dass diese explizit nur eine Token-weise Auskunft über Ausgabe- und Rücknahmepreis geben. Möchte man mehrere Token minten oder zurückkgeben, muss der Preis für jeden der Tokens separat ausgerechnet und anschließend addiert werden. Möchte man bei einem aktuellen Supply von $i \in \mathbb{N}$ nicht einen sondern $k \in \mathbb{N}$ mit $k \leq i$ Tokens minten bzw. zurückgeben, beläuft sich der gesamte Kauf- bzw- Verkaufspreis auf

\begin{align*}
\mathcal{K}_{k} \left( i \right) &:= \sum_{j = i + 1}^{i + k} \mathcal{K} \left( j \right) \\
\mathcal{V}_{k} \left( i \right) &:= \sum_{j = i - k + 1}^{i} \mathcal{V} \left( j \right).
\end{align*}

\vspace{0.3cm}

Das besonders Hervorhebenswerte an diesem Token-Ausgabemodell ist zweifelsfrei die gemeinschaftliche aus der Token-Ausgabe gefütterte Contract-Treasury aus echten Geldeinlagen, die nicht etwa einer dritten (Ausgabe-)Partei zugute kommt, sondern de facto den Tokeninhabern gehört. Die Existenz dieser Rücklagen gibt den ausgegebenen Tokens theoretisch einen realen Wert und ermöglicht einzig und allein die Implementierung des Rückkauf-Mechanismus. Der Gebrauch von dieser Möglichkeit und die Verankerung der Rückkauffunktion $\mathbf{\mathcal{V}}$ in der Logik des Token-Contracts gibt den Tokens dann auch praktisch einen realen Wert. Denn wenn ich eine definitive - in Contract-Logik verankerte - Sicherheit habe, ein Asset jederzeit verkaufen zu können, besitzt dieses Asset auch einen echten, intrinsischen Value, der nicht zwingend den marktwirtschaftlichen Mechanismen unterliegt - und somit auch keinen etwaigen Hypes um einen gut vermarkteten Tokensale ohne dahinterliegende Substanz. Vielmehr folgt der Tokenpreis der gemeinschaftlichen Projekt-Treasury $\mathbf{\mathcal{T}}$, die ihrerseits substanziell mit dem Projekterfolg einhergeht. 

\vspace{0.2cm}

Die letzte Einsicht lässt uns zu zwei wesentlichen Gedanken gelangen, die die entscheidenden Argumente für das \textit{Bonding-Curves}-Modell liefern könnten:

\begin{itemize}
  \item Der Rückgabepreis $\mathcal{V} \left( i \right)$ sollte eine direkte Abhängigkeit vom Treasury-Inhalt $\mathcal{T} \left( i \right)$ aufweisen.
  \item Nach bisheriger Definition hängt der Treasury-Inhalt $\mathcal{T} \left( i \right)$ ausschließlich vom aktuellen Supply $i \in \mathbb{N}$ (und den damit einhergehenden Ausgabepreisen $\mathcal{K} \left( j \right)$ für $j \leq i$) ab. Gepaart mit der ersten Forderung bedeutete dies implizit nichts anderes, als dass der aktuelle Rücknahmepreis ausschließlich von den Ausgabepreisen der bisherigen Tokens abhinge. Dies ist schlecht und ein sehr großes Problem der gängigen \textit{Bonding-Curves}-Implementierungen, da Koppelung des Rücknahmepreises - also des intrinsischen Werts des Tokens - ausschließlich an den Kaufpreis voriger Tokens - und die Wertentwicklung damit an den Kaufpreis etwaiger zukünftig ausgegebener Tokens, würde mathematisch alternativlos eine monoton steigende Ausgabepreis-Funktion $\mathcal{K} \left( i \right)$ erfordern. Ohne eine sehr stark fundierte projekt-bezogene Argumentation für ein monoton steigendes $\mathcal{K} \left( i \right)$ schrie das gesamte Modell nur so nach \textit{Pump \& Dump} und \textit{Hot Potatoe}. Und tatsächlich ist es so, dass nahezu alle \textit{Bonding-Curves}-Implementierungen Gebrauch von einer (streng) monoton steigenden Ausgabepreis-Kurve $\mathcal{K} \left( i \right)$ machen. Sie argumentieren mit anderem generierten Projekt-Value, der nicht durch die Contract-Treasury $\mathcal{T} \left( i \right)$ gemessen werden kann und diese Argumentation muss nicht zwingend falsch oder ungenügend sein. Uns reicht dies aber nicht - zumal es mehr als gute Gründe für ein monoton steigendes $\mathcal{K} \left( i \right)$ gibt (dazu später mehr). Somit bleibt uns nichts anderes als die - zumindest teilweise - Abkoppelung von $\mathcal{T} \left( i \right)$ und $\mathcal{K} \left( i \right)$, womit wir auch unsere obige Definition von $\mathcal{T} \left( i \right) = \sum_{j = 1}^{i} \mathcal{K} \left( j \right)$ wieder teils verwerfen müssen.
\end{itemize} 

\vspace{0.2cm}

Wir fassen zusammen:

\vspace{0.3cm}

\begin{Fazit}[Contract-Treasury als wichtigster Baustein zum Erfolg]

Wie sind der Überzeugung, der Contract-Treasury-Inhalt - gemessen als $\mathcal{T} \left( i \right)$ - sei der entscheidende Baustein für einen soliden \textit{Bonding-Curves}-Token. $\mathcal{T} \left( i \right)$ beeinflusst direkt den Rücknahmepreis $\mathcal{V} \left( i \right)$, verleiht dem Token damit einen echten geldwerten Value, was wiederum als Kaufargument für neue Tokens gilt und damit implizit auch zur Beurteilung des Ausgabepreises $\mathcal{K} \left( i \right)$ seitens etwaiger neuer Investoren hinzugezogen wird.

\vspace{0.2cm}

Konkret auf den Rücknahmepreis $\mathcal{V} \left( i \right)$ bezogen sehen wir kaum sinnvollere Alternativen als der gängigen Definition 

\begin{equation*}
\mathcal{V} \left( i \right) = \frac{\mathcal{T} \left( i \right)}{i}
\end{equation*}

zu folgen, was gleichbedeutend damit ist, dass der Besitz am Contract-Treasury-Inhalt pro rata auf die sich in Zirkulation befindenden Token verteilt wird, was sich konsequenterweise im Rücknahmepreis $\mathcal{V} \left( i \right)$ widerspiegelt. Mit der in Definition \ref{defBC} beschriebenen Treasury-Funktion $\mathcal{T} \left( i \right)$ ergibt sich damit für den Rücknahmepreis $\mathcal{V} \left( i \right)$

\begin{equation*}
\mathcal{V} \left( i \right) = \frac{\mathcal{T} \left( i \right)}{i} = \frac{\sum_{j = 1}^{i} \mathcal{K} \left( j \right)}{i}.
\end{equation*} 

\vspace{0.3cm}

\end{Fazit}

\vspace{0.5cm} % binde die Datei '[Economics][Token-Economics][Bonding Curves].tex' ein
\newpage



\subsubsection{Lösungsidee 2}
\todo{TODO}
\vspace{0.3cm}

\begin{Solution}[möglicher Token-Flow]

\begin{itemize}
  \item Ein User nutzt einen Service-Provider A, der WunderPass unterstützt, und ist auch mit seinem WunderPass bei Provider A eingeloggt.
  \begin{itemize}
  	\item Beispiel 1: Der Service-Provider A ist ein Identity-Data-Management-Service, der die persönlichen Daten des Users verwaltet und bei Bedarf Dritten zur Verfügung stellen kann.
  	\item Beispiel 2: Der Service-Provider A ist EasyJet.
  \end{itemize}
  \item Der User und der Service-Provider A erzielen - wie auch immer - eine Übereinkunft darüber, dass die von Provider A verwalteten - den User betreffenden Daten - theoretisch mittels des WunderPass-Lookups mit Dritten geteilt werden können sollen. 
  \begin{itemize}
  	\item Beispiel 1: Die Daseinsberechtigung des Identity-Data-Management-Service beschränkt sich eigentlich ausschließlich auf das Teilen von Daten mit Dritten. Hierbei ist die obige Anforderung also trivialerweise unabdingbar.
  	\item Beispiel 2: Beim Beispiel von EasyJet könnten die besagten Daten z. B. gebuchte Flugtickets sein, die man mit Drittdiensten teilt, um daran ausgerichtet gezielte Werbeangebote im zugehörigen Ausland zu ermöglichen.
  \end{itemize}
  User und Provider einigen sich auf einen Preis/Preisformel für das Teilen dieser Daten - und zwar auf den konkreten Preis von \textbf{x WPT} (WunderPass-Utility-Token).
  \item Service-Provider B (der ebenfalls WunderPass unterstützt) möchte Userdaten des Service-Provider A nutzen, falls solche vorliegen.
  \begin{itemize}
  	\item Beispiel 1: Hierbei könnte Provider B so ziemlich jeder denkbare Online-Dienst sein, der irgendwelche persönlichen Userdaten benötigt (z. B. Adresse, Email, Kreditkarte etc.).
  	\item Beispiel 2: Hierbei könnte es sich z. B. um (schlecht ausgelastete) Hotels handeln, die anhand der EasyJet-Flugdaten über die Destination des Users wissend, besondere Angebote an ihn ausspielen wollen.
  \end{itemize}
  \item Service-Provider B callt der WunderPass-Lookup-Service, um die Existenz etwaiger Daten und deren \textbf{Preis x WPT} in Erfahrung zu bringen.
  \item Liegen Lookup-Daten vor, kann Provider B entscheiden, ob er diese zum angegebenen Preis beziehen möchte. 
  \item Möchte Service-Provider B Gebrauch vom Lookup machen, muss er in Vorleistung gehen und den Betrag von \textbf{2 * x WPT} in den Lookup-Contract einzahlen.
  \item Die eingebrachten \textbf{2 * x WPT} werden - abzüglich einer kleinen WunderPass-Fee - im Lookup-Contract gelockt. Service-Provider B erhält im Gegenzug einen \textit{Berechtigungs-Token} für den Abruf von entsprechenden Daten von Provider A (hierbei ist eher ein technischer Security-Token und kein Crypto-Token gemeint).
  \item Die Zugriffsberechtigung für das Abrufen der Daten von Provider A soll dabei einer \textbf{zeitlichen Beschränkung z} unterliegen (z. B. "eine Woche"). \textbf{z} ist hierbei ebenso individuell (Teilnehmer- und Daten-abhängig) zu sehen wie \textbf{x}.
  \item Service-Provider B fragt unter Vorlage des Berechtigungs-Token die gewünschten Daten beim Service-Provider A an.
  \begin{itemize}
  	\item Provider A muss den Berechtigungs-Token validieren (beim Lookup-Service).
  	\item A muss unter Umständen die Freigabe beim User einfordern (ggf. sollte der User in irgendeiner Weise "bestraft" werden, falls er den Datenzugriff verwehrt).
  	\item Provider A und B müssen einen gewissen "Handshake" implementieren, der A bescheinigt, wie vereinbart die korrekten Daten an B ausgeliefert zu haben. 
  \end{itemize}
  \item Provider A liefert die Daten an Provider B aus und erhält im Gegenzug ein Bestätigungszertifikat von B.
  \item Mit dem Bestätigungszertifikat kann Provider A seine Vergütung beim Lookup-Contract einlösen. Dabei wird die Hälfte der gelockten Einlage von Provider B (also an dieser Stelle die Hälfte von \textbf{2 * x WPT} - also \textbf{x WPT}) ausgeschüttet. Und zwar zur Hälfte an Provider A und zur andern Hälfte an den User.
  \item \textbf{x WPT} des ursprünglich eingezahlten Deposits von B bleiben weiterhin im Lookup-Contract gelockt. 
  \item Jede künftige Anfrage von B an A (bezüglich desselben Datensatzes) innerhalb des definierten Zeitraums \textbf{z} releast immer wieder die Hälfte des verbliebenen gelockten Deposits.
  \item Nach Ablauf des definierten \textbf{Zeitraums z}
  \begin{itemize}
  	\item bekommt B den verbliebenen (nicht ausgeschütteten) Teil seines Deposits zurückerstattet.
  	\item wird der \textit{Berechtigungs-Token} ungültig.
  	\item hat A keinen Anspruch mehr, für die Datenauslieferung an B vergütet zu werden (auch dann, falls er Daten ausgeliefert, ohne vorher die abgelaufene Gültigkeit des Berechtigungs-Tokens zu validieren).
  \end{itemize}
  \item Es ist denkbar, die an der User ausgezahlten Rewards in irgendeiner Weise (zeitlich) zu locken und deren Release an bestimmte Bedingungen zu knüpfen ($\rightarrow$ um den User zu incentivieren irgendetwas zu tun).
\end{itemize}

\vspace{0.5cm}

\underline{\textbf{Cashflow}}:

\begin{itemize}
  \item Provider B zahlt für den Lookup. Aber auch nur dann, falls er den Lookup nutzt. Andernfalls erhält er seinen getätigten Deposit (abzüglich einer kleinen Fee an WunderPass) zurück. Er zahlt in gleichem Teil an Provider A und den User. Aus der Verwertung der bezogenen Daten kreiert er einen Value (im Sinne seiner Dienstleistung). Einen Value, der auch durchaus im Sinne des Users sein könnte. Es ist also gut denkbar, dass Provider B eine Rechtfertigung besitzt, den User an seinen Kosten zu beteiligen (z. B. mittels einer Fee für die erbrachte Dienstleitung, die den gekauften Datensatz erforderte; idealerweise ebenfalls in \textbf{WPT} vom User zu erbringen).
  \item Provider A ist der klare Nutznießer des Datenaustauschs. Der "Daten-Trade" hat - direkt betrachtet - erst einmal gar nichts mit seinem Kerngeschäfts zu tun (es sei denn, A sei wie in Beispiel 1 ein Identity-Data-Management-Service, dessen Kerngeschäft ausschließlich darin besteht, Daten zur Verfügung zu stellen). In der perfekten WunderWelt kann Provider A in einem anderen Case, analog als Provider B auftreten, um seine erhaltenen Token-Rewards für für ihn relevante Daten auszugeben.
  \item Der User scheint hierbei auch der Nutznießer von etwaigen "Daten-Deals" zu sein. Seine Stellung als solcher ist aber weniger klar als diejenige von Provider A, da er von dem stattgefundenen Datenaustausch indirekt ebenso profitieren könnte, indem er z. B. auf Basis der Datennutzung eine bessere Dienstleistung von Provider B erhält. Der Pitch "der User monetarisiert seine Daten" kling zwar sehr attraktiv, muss man hierbei jedoch sehr aufpassen, den Bogen nicht zu überspannen. Denn - während die Rolle von Provider A als Profiteur unbestreitbar ist - wird die Zahlungsbereitschaft von Provider B von Fall zu Fall ganz unterschiedlich und nur bedingt vorhanden sein. Denn schließlich ist es alles andere als selbsterklärend, ein Online-Shop solle für Adressdaten des Users bezahlen, um seine Bestellung zustellen zu können, während der User davon profitiert. In diesem Fall wäre es eher nachvollziehbar, Provider B und der User würden sich die an Provider A zu entrichtenden Fees für die Bereitstellung der Adressdaten teilen. Hierbei ist die \textbf{Verteilung der Fees leider extrem heterogen}.
\end{itemize} 

\end{Solution}


\vspace{0.3cm}

\todo{TODO}




\subsubsection{Kreislauf}
\label{sec:wpt_kreislauf}
\todo{TODO}
\vspace{0.3cm}

\subsubsection{Token-Design}
\label{sec:wpt_design}
\todo{TODO}
\vspace{0.3cm}

\subsubsection{Incentivierung}
\label{sec:wpt_incent}
\todo{TODO}
\vspace{0.3cm}

\subsubsection{Milestones-Reward-Pool}
\label{sec:wpt_reward_pool}
\todo{TODO}
\vspace{0.3cm}

\subsubsection{WPT in Zahlen}
\label{sec:wpt_zahlen}
\todo{TODO}
\vspace{0.3cm}

\subsubsection{Fazit}
\label{sec:wpt_fazit}
\todo{TODO}    % binde die Datei '[Economics][Token-Economics].tex' ein
% !TEX root = paper.tex

\subsection{Fazit}
\label{sec:eco_fazit}
\todo{TODO}    % binde die Datei '[Economics][Fazit].tex' ein    % binde die Datei 'Economics.tex' ein
% !TEX root = paper.tex

\section{NFT-Pass}
\label{sec:nft-pass}

%------------------------------------------------Einleitende Worte---------------------------------

Ein exzellentes Mittel, um \textit{WunderPass} als Geschäftsmodell, Unternehmung und Unternehmen ein symbolisches - gewissermaßen plastisches - Sinnbild einzuverleiben, ist die Repräsentation von \textit{WunderPass} als Service/Protokoll mittels eines - eigens dafür kreierten - NFTs: \textbf{"Des WunderPass"} (im Folgenden auch \textit{NFT-Pass} genannt)

\vspace{0.3cm}

\begin{Fazit}[\textit{WunderPass} deabstrahiert durch \textbf{"den WunderPass"} als NFT]

"Ich nutze \textit{WunderPass}" wird symbolisiert durch "Ich besitze \textbf{meinen WunderPass}"!

\end{Fazit}

\vspace{0.3cm}

%------------------------------------------------Konzept-Kapitel-----------------------------------

\subsection{Konzeption}
\vspace{0.3cm}

% -----------------Einleitende Worte-------------
% !TEX root = paper.tex

Unser Anspruch an den zu modellierenden \textit{NFT-Pass} ist grob der folgende:

\vspace{0.2cm}

\begin{itemize}
  \item Der \textit{NFT-Pass} muss sich ganz klar von dem Großteil der heutigen - in größter Regel als Sammlerstück verstandenen - den Markt überflutenden NFTs abgrenzen. Er braucht einen klar ersichtlichen \textbf{intrinsischen Wert}. Man muss also "etwas mit dem \textit{NFT-Pass} anfangen/machen können" und diesen nicht "lediglich besitzen", um ihn ausschließlich mit einer gewissen Wahrscheinlichkeit gewinnbringend weiterverkaufen zu können ("Hot Potato"). Der Token bedarf also gewisser Eigenschaften eines \textit{Governance-Tokens} (DAO) oder Ähnlichem.
  \item 
  \begin{sloppypar}  
  Der \textit{NFT-Pass} braucht ungeachtet des vorigen Bullet-Points jedoch trotzdem zusätzlich ebenso eine ähnliche Beschaffenheit - wie solche der aktuell üblichen marktbeherrschenden NFTs - als Sammlerstück - gleichwohl nicht erstrangig. 
  \end{sloppypar}
  \item Anders als die aktuell gängigen NFTs soll unser \textit{NFT-Pass} \textbf{nicht begrenzt} in der Anzahl seiner Stücke sein. Stattdessen sollen theoretisch beliebig viele \textit{NFT-Pässe} existieren können. Nichtsdestotrotz soll unser \textit{NFT-Pass} ebenso die Eigenschaft der "nicht inflationären Begehrtheit" einverleibt bekommen. Dies möchten wir mittels einer ausgeklügelten Minting-Logik abbilden, die ein \textbf{endliches Sub-Set} an raren und begehrten \textit{NFT-Pässen} innerhalb des \textbf{unendlichen Gesamt-Sets} der \textit{NFT-Pässe} sicherstellt. Soll heißen: Es werden einerseits \textit{NFT-Pässe} exis\-tieren, die den heutigen NFTs - im Sinne ihres Sammlerwertes - gleichkommen, während die restlichen andererseits mit ihrer steigenden Gesamtanzahl zunehmend entwerten, bis sie irgendwann (als NFT betrachtet) nahezu wertlos und lediglich "funktional" werden.
  \item Die Rarität und Begehrtheit unseres \textit{NFT-Pass} soll Gamification-Mechanismen folgen:
  \begin{itemize}
    \item Wir brauchen an etwaigen Stellen ein (wertbestimmendes) \textit{first-come-first-serve-Prinzip}.
    \item Wir brauchen an anderen Stellen ein (ebenso wertbestimmendes) Zufallsprinzip.
    \item Wir brauchen irgendwo ebenso ein (geringes) Maß an persönlicher Individua\-lisierung des \textit{NFT-Pass} - ausschließlich durch den User gesteuert.
    \item Abrundend könnte ein \textbf{gemeinnützig wertbestimmendes} (randomisiertes) Merkmal wirken. (Beispiel: Wenn die \textit{NFT-Pässe} irgendwann inflationär geworden sind, könnte der zehn-millionste plötzlich wieder richtig krass sein.)
  \end{itemize}
  \item Der \textit{NFT-Pass} muss gänzlich transparent und vor allem verständlich für den interessierten - gleichwohl vielleicht technisch nicht bewandertsten - User sein.
\end{itemize}

\vspace{0.3cm}

In den kommenden Abschnitten folgt ein initialer Abriss unserer Vorstellung des \textit{NFT-Pass}:

\vspace{0.3cm}    % binde die Datei '[NFT-Pass][Konzept][Einleitung].tex' ein


% -----------------NFT-Status--------------------
\subsubsection{Status-Property}
\vspace{0.2cm}
% !TEX root = paper.tex

Diese NFT-Property - die wir gleichzeitig als die Main-Property unseres \textit{NFT-Pass} ansehen - soll der oben formulierten Anforderung nach einem first-come-first-serve-Prinzip Rechnung tragen. Zeitlich früher ausgestellte NFT-Pässe sollen einen rareren und begehrteren \textit{Pass-Status} inne haben als die späteren. Und vor allem sollen die \textit{NFT-Pässe} eines bestimmten ausgestellten Status in ihrer Anzahl begrenzt sein und nach Erreichen einer zu definierenden Höchstgrenze fortan nie wieder ausgestellt (ge\-mintet) werden können.

\vspace{0.3cm}

\begin{NFT-Prop}[Pass-Status]

Wir definieren folgende \textit{NFT-Pass-Status} mit den dazugehörenden Eigenschaften:

\begin{itemize}
    \item Status A (\textbf{Diamond})
    \begin{itemize}
    	\item Anzahl Pässe: 200
    	\item Gemintet an Nummer: 1 bis 200
    \end{itemize}
    \item Status B (\textbf{Black})
    \begin{itemize}
    	\item Anzahl Pässe: 1.600
    	\item Gemintet an Nummer: 201 bis 1800
    \end{itemize}
    \item Status C (\textbf{Pearl})
    \begin{itemize}
    	\item Anzahl Pässe: 12.800
    	\item Gemintet an Nummer: 1801 bis 14.600
    \end{itemize}
    \item Status D (\textbf{Platin})
    \begin{itemize}
    	\item Anzahl Pässe: 102.400
    	\item Gemintet an Nummer: 14.601 bis 117.000
    \end{itemize}
    \item Status E (\textbf{Ruby})
    \begin{itemize}
    	\item Anzahl Pässe: 819.200
    	\item Gemintet an Nummer: 117.001 bis 936.200
    \end{itemize}
    \item Status F (\textbf{Gold})
    \begin{itemize}
    	\item Anzahl Pässe: 6.553.600
    	\item Gemintet an Nummer: 936.201 bis 7.489.800
    \end{itemize}
    \item Status G (\textbf{Silver})
    \begin{itemize}
    	\item Anzahl Pässe: 52.428.800
    	\item Gemintet an Nummer: 7.489.801 bis 59.918.600
    \end{itemize}
    \item Status H (\textbf{Bronze})
    \begin{itemize}
    	\item Anzahl Pässe: 419.430.400
    	\item Gemintet an Nummer: 59.918.601 bis 479.349.000
    \end{itemize}
    \item Status I (\textbf{White})
    \begin{itemize}
    	\item Anzahl Pässe: $\infty$
    	\item Gemintet an Nummer: 479.349.001 bis $\infty$
    \end{itemize}
\end{itemize}

\end{NFT-Prop}

\vspace{0.3cm}

Diese NFT-Property ist per Definition trivialerweise \textbf{deterministisch}: Es ist stets zweifellos klar, welchen Status ein an x-ter Stelle geminteter \textit{NFT-Pass} haben wird. Die hinzugezogene "Reverse-Halving-Logik" \textbf{belohnt die Early-Adopter} mit einem begehrten NFT, dessen Rarität per Protokoll mit der Zeit stets abnimmt.

\vspace{0.1cm}


    % binde die Datei '[NFT-Pass][Konzept][Status].tex' ein


% -----------------Überleitende Worte------------
Die Beschaffenheit dieser \textit{first-come-first-serve-Property} soll jedoch einzigartig bleiben. Die folgenden Properties werden nicht mehr deterministisch sein, um unserem \textit{NFT-Pass} ein \textbf{unvorherbestimmbares "Eigenleben"} einzuverleiben. 


% -----------------NFT-Hologramm-----------------
\subsubsection{Hologramm}
\label{sec:hologramm}
\vspace{0.3cm}
% !TEX root = paper.tex

Diese NFT-Property soll zwar einem ähnlichen abstufenden Raritätsprinzip zu Grunde liegen wie die Main-Property, dies jedoch nicht mehr einem first-come-first-serve- sondern stattdessen einem Zufallsprinzip folgend.

Ebenfalls abweichend von der Beschaffenheit der Main-Property soll bei dieser Pro\-perty die Rarität nicht mittels einer absoluten Obergrenze abgebildet werden, sondern mittels einer relativen. (Dies zahlt auf die oben formulierte Anforderung nach einem \textbf{gemeinnützig gewinnbringendem Value} unseres \textit{NFT-Pass} ein.

\vspace{0.3cm}

\begin{NFT-Prop}[Hologramm (Welt-Wunder)]

Wir definieren folgende \textit{NFT-Pass-Hologramme} mit den dazugehörenden Eigenschaften:

\begin{itemize}
    \item WW1
    \begin{itemize}
    	\item Mögliche Ausprägung: \textbf{Pyramids of Giza}
    	\item Anteil Pässe: 0,390625\% $\left( \frac{1}{256} \right)$
    \end{itemize}
    \item WW2
    \begin{itemize}
    	\item Mögliche Ausprägung: \textbf{Great Wall of China}
    	\item Anteil Pässe: 0,78125\% $\left( \frac{1}{128} \right)$
    \end{itemize}
    \item WW3
    \begin{itemize}
    	\item Mögliche Ausprägung: \textbf{Petra} 
    	\item Anteil Pässe: 1,5625\% $\left( \frac{1}{64} \right)$
    \end{itemize}
    \item WW4
    \begin{itemize}
    	\item Mögliche Ausprägung: \textbf{Colosseum} 
    	\item Anteil Pässe: 3,125\% $\left( \frac{1}{32} \right)$
    \end{itemize}
    \item WW5
    \begin{itemize}
    	\item Mögliche Ausprägung: \textbf{Chichén Itzá} 
    	\item Anteil Pässe: 6,25\% $\left( \frac{1}{16} \right)$
    \end{itemize}
    \item WW6
    \begin{itemize}
    	\item Mögliche Ausprägung: \textbf{Machu Picchu} 
    	\item Anteil Pässe: 12,5\% $\left( \frac{1}{8} \right)$
    \end{itemize}
    \item WW7
    \begin{itemize}
    	\item Mögliche Ausprägung: \textbf{Taj Mahal} 
    	\item Anteil Pässe: 25\% $\left( \frac{1}{4} \right)$
    \end{itemize}
    \item WW8
    \begin{itemize}
    	\item Mögliche Ausprägung: \textbf{Christ the Redeemer} 
    	\item Anteil Pässe: 50\% + x $\left( \frac{1}{2} + \frac{1}{256} \right)$
    \end{itemize}
\end{itemize}

\end{NFT-Prop}

\vspace{0.3cm}

Das Besondere an dieser Property spiegelt sich in der Tatsache wider, gewisse rar beschaffene Ausprägungen seien nur "zeitweise" ausgeschöpft, da sich ihre (rare) Anzahl lediglich \textbf{relativ} an der Gesamtzahl der aktuell \textit{ausgestellten NFT-Pässe} bemisst und nicht wie die Main-Property einer absoluten Obergrenze obliegt, deren Erreichung unumkehrbar ist. Soll heißen: Ist die prozentuale Obergrenze an Pässen mit einer bestimmten Ausprägung der gegenwärtigen Property zu einem be\-stimmten Zeitpunkt erreicht, kann zwar für einen gewissen Zeitraum kein Pass mit dieser Ausprägung mehr ausgestellt werden. Sobald jedoch die Gesamtanzahl der \textit{ausgestellten NFT-Pässe} wieder groß genug ist - sodass die Anzahl der vorhandenen \textit{NFT-Pässe} mit der betroffenen Ausprägung wieder die prozentuale Obergrenze unterschreitet - werden Pässe der besagten Ausprägung "wieder verfügbar".

\vspace{0.3cm}

\begin{Algo}[Verlosungs-Mechanismus für Hologramm-Property]

\begin{itemize}
    \item Zunächst bestimme man die Gesamtanzahl aller bisher geminteter Pässe $n$.
    \item Gleiches tue man nun für die Counts der geminteten Pässe pro Ausprägung der Hologramm-Property WW1 bis WW8 als entsprechende Größen $n_1, n_2,...,n_8$.
    \item Und damit anschließend die aktuelle prozentuale Verteilung der Ausprägung auf die aktuell geminteten Pässe als $\sigma_i:= \frac{n_i}{n}$ für $i \in \lbrace 1,...,8 \rbrace$ berechnen.
    \item Seien $\Theta_i$ für $i \in \lbrace 1,...,8 \rbrace$ die oben definierten \textbf{relativen} Obergrenzen der \newline Ausprägungen der Hologramm-Property WW1 bis WW8.
    \item Alle Ausprägungen mit $\sigma_i \geq \Theta_i$ können zum aktuellen Zeitpunkt nicht vergeben werden und damit auch nicht beim Minting eines neuen Pass berücksichtigt werden.
    \item Für die Ausprägungen mit $\sigma_i < \Theta_i$ berechnen wir den Normierungsfaktor
\end{itemize} 

\begin{equation*}
\omega := \sum_{\sigma_i < \Theta_i} \Theta_i \textrm{ } \leq 1
\end{equation*} 

\begin{itemize}
    \item Damit errechnen wir die aktuell vorliegenden Wahrscheinlichkeiten $\rho_i$ für unsere Hologramm-Ausprägungen als
\end{itemize} 

\[
\rho_i:=\left\{%
\begin{array}{ll}
    0, & \hbox{falls $\sigma_i \geq \Theta_i$} \\[0,3cm]
    \hbox{\LARGE $\frac{\Theta_i}{\omega}$,} & \hbox{falls $\sigma_i < \Theta_i$}. \\
\end{array}%
\right.
\] 

Man vergewissere sich an dieser Stelle gedanklich, auch für die neuen \newline Wahrscheinlichkeiten gelte \[\sum_{i = 1}^7 \rho_i \textrm{ } = 1.\]

\begin{itemize}
    \item Am Ende bestimme man mittels Randomisierung anhand der Wahrscheinlichkeiten $\rho_i$ für $i \in \lbrace 1,...7 \rbrace$ die zu vergebende Hologramm-Ausprägung. 
\end{itemize}

\end{Algo}

\vspace{0.3cm}

Was hier so kompliziert klingt, lässt sich aber super simpel veranschaulichen:

Die \textit{Verlosung} der Wunder erfolgt in einem periodischen 256er-Turnus ($256 = 2^{n}$ mit $n=8$ für die acht bereitgestellten Hologramme). Nach jedem 256. geminteten Pass schmeißt man 256 Lose in eine Lostrommel: Ein Los für die \textit{Pyramiden}, zwei für die \textit{Chinesische Mauer}, vier für \textit{Petra} etc. Die \textit{Jesus-Statue} kommt letztendlich mit 129 Losen in die Trommel.

Nun ziehen wir blind ein Los und vergeben das gezogenen Hologramm an den nächsten zu mintenden NFT-Pass. Wir tun dies solange, bis die Trommel leer ist. Anschließend fangen wir wieder von Vorne an und befüllen die Trommel erneut mit denselben 256 Losen.

\textbf{Achtung:} Wir befüllen die Trommel ausschließlich nachdem sie komplett leer geworden ist und nicht etwa zwischendurch mal.

\vspace{0.3cm}

    % binde die Datei '[NFT-Pass][Konzept][Wunder].tex' ein
%% !TEX root = paper.tex

Diese NFT-Property soll zwar einem ähnlichen abstufenden Raritätsprinzip zu Grunde liegen wie die Main-Property, dies jedoch nicht mehr einem first-come-first-serve- sondern stattdessen einem Zufallsprinzip folgend.

Ebenfalls abweichend von der Beschaffenheit der Main-Property soll bei dieser Pro\-perty die Rarität nicht mittels einer absoluten Obergrenze abgebildet werden, sondern mittels einer relativen. (Dies zahlt auf die oben formulierte Anforderung nach einem \textbf{gemeinnützig gewinnbringendem Value} unseres \textit{NFT-Pass} ein.

\vspace{0.3cm}

\begin{NFT-Prop}[Hologramm (Welt-Wunder)]

Wir definieren folgende \textit{NFT-Pass-Hologramme} mit den dazugehörenden Eigenschaften:

\begin{itemize}
    \item WW1
    \begin{itemize}
    	\item Mögliche Ausprägung: \textbf{Pyramids of Giza}
    	\item Anteil Pässe: 0,390625\% $\left( \frac{1}{256} \right)$
    \end{itemize}
    \item WW2
    \begin{itemize}
    	\item Mögliche Ausprägung: \textbf{Great Wall of China}
    	\item Anteil Pässe: 0,78125\% $\left( \frac{1}{128} \right)$
    \end{itemize}
    \item WW3
    \begin{itemize}
    	\item Mögliche Ausprägung: \textbf{Petra} 
    	\item Anteil Pässe: 1,5625\% $\left( \frac{1}{64} \right)$
    \end{itemize}
    \item WW4
    \begin{itemize}
    	\item Mögliche Ausprägung: \textbf{Colosseum} 
    	\item Anteil Pässe: 3,125\% $\left( \frac{1}{32} \right)$
    \end{itemize}
    \item WW5
    \begin{itemize}
    	\item Mögliche Ausprägung: \textbf{Chichén Itzá} 
    	\item Anteil Pässe: 6,25\% $\left( \frac{1}{16} \right)$
    \end{itemize}
    \item WW6
    \begin{itemize}
    	\item Mögliche Ausprägung: \textbf{Machu Picchu} 
    	\item Anteil Pässe: 12,5\% $\left( \frac{1}{8} \right)$
    \end{itemize}
    \item WW7
    \begin{itemize}
    	\item Mögliche Ausprägung: \textbf{Taj Mahal} 
    	\item Anteil Pässe: 25\% $\left( \frac{1}{4} \right)$
    \end{itemize}
    \item WW8
    \begin{itemize}
    	\item Mögliche Ausprägung: \textbf{Christ the Redeemer} 
    	\item Anteil Pässe: 50\% + x $\left( \frac{1}{2} + \frac{1}{256} \right)$
    \end{itemize}
\end{itemize}

\end{NFT-Prop}

\vspace{0.3cm}

Das Besondere an dieser Property spiegelt sich in der Tatsache wider, gewisse rar beschaffene Ausprägungen seien nur "zeitweise" ausgeschöpft, da sich ihre (rare) Anzahl lediglich \textbf{relativ} an der Gesamtzahl der aktuell \textit{ausgestellten NFT-Pässe} bemisst und nicht wie die Main-Property einer absoluten Obergrenze obliegt, deren Erreichung unumkehrbar ist. Soll heißen: Ist die prozentuale Obergrenze an Pässen mit einer bestimmten Ausprägung der gegenwärtigen Property zu einem be\-stimmten Zeitpunkt erreicht, kann zwar für einen gewissen Zeitraum kein Pass mit dieser Ausprägung mehr ausgestellt werden. Sobald jedoch die Gesamtanzahl der \textit{ausgestellten NFT-Pässe} wieder groß genug ist - sodass die Anzahl der vorhandenen \textit{NFT-Pässe} mit der betroffenen Ausprägung wieder die prozentuale Obergrenze unterschreitet - werden Pässe der besagten Ausprägung "wieder verfügbar".

\vspace{0.3cm}

\begin{Algo}[Verlosungs-Mechanismus für Hologramm-Property]

\begin{itemize}
    \item Zunächst bestimme man die Gesamtanzahl aller bisher geminteter Pässe $n$.
    \item Gleiches tue man nun für die Counts der geminteten Pässe pro Ausprägung der Hologramm-Property WW1 bis WW8 als entsprechende Größen $n_1, n_2,...,n_8$.
    \item Und damit anschließend die aktuelle prozentuale Verteilung der Ausprägung auf die aktuell geminteten Pässe als $\sigma_i:= \frac{n_i}{n}$ für $i \in \lbrace 1,...,8 \rbrace$ berechnen.
    \item Seien $\Theta_i$ für $i \in \lbrace 1,...,8 \rbrace$ die oben definierten \textbf{relativen} Obergrenzen der \newline Ausprägungen der Hologramm-Property WW1 bis WW8.
    \item Alle Ausprägungen mit $\sigma_i \geq \Theta_i$ können zum aktuellen Zeitpunkt nicht vergeben werden und damit auch nicht beim Minting eines neuen Pass berücksichtigt werden.
    \item Für die Ausprägungen mit $\sigma_i < \Theta_i$ berechnen wir den Normierungsfaktor
\end{itemize} 

\begin{equation*}
\omega := \sum_{\sigma_i < \Theta_i} \Theta_i \textrm{ } \leq 1
\end{equation*} 

\begin{itemize}
    \item Damit errechnen wir die aktuell vorliegenden Wahrscheinlichkeiten $\rho_i$ für unsere Hologramm-Ausprägungen als
\end{itemize} 

\[
\rho_i:=\left\{%
\begin{array}{ll}
    0, & \hbox{falls $\sigma_i \geq \Theta_i$} \\[0,3cm]
    \hbox{\LARGE $\frac{\Theta_i}{\omega}$,} & \hbox{falls $\sigma_i < \Theta_i$}. \\
\end{array}%
\right.
\] 

Man vergewissere sich an dieser Stelle gedanklich, auch für die neuen \newline Wahrscheinlichkeiten gelte \[\sum_{i = 1}^7 \rho_i \textrm{ } = 1.\]

\begin{itemize}
    \item Am Ende bestimme man mittels Randomisierung anhand der Wahrscheinlichkeiten $\rho_i$ für $i \in \lbrace 1,...7 \rbrace$ die zu vergebende Hologramm-Ausprägung. 
\end{itemize}

\end{Algo}

\vspace{0.3cm}

Was hier so kompliziert klingt, lässt sich aber super simpel veranschaulichen:

Die \textit{Verlosung} der Wunder erfolgt in einem periodischen 256er-Turnus ($256 = 2^{n}$ mit $n=8$ für die acht bereitgestellten Hologramme). Nach jedem 256. geminteten Pass schmeißt man 256 Lose in eine Lostrommel: Ein Los für die \textit{Pyramiden}, zwei für die \textit{Chinesische Mauer}, vier für \textit{Petra} etc. Die \textit{Jesus-Statue} kommt letztendlich mit 129 Losen in die Trommel.

Nun ziehen wir blind ein Los und vergeben das gezogenen Hologramm an den nächsten zu mintenden NFT-Pass. Wir tun dies solange, bis die Trommel leer ist. Anschließend fangen wir wieder von Vorne an und befüllen die Trommel erneut mit denselben 256 Losen.

\textbf{Achtung:} Wir befüllen die Trommel ausschließlich nachdem sie komplett leer geworden ist und nicht etwa zwischendurch mal.

\vspace{0.3cm}




% -----------------NFT-Pattern-------------------
\subsubsection{Pattern-Property}
\vspace{0.3cm}
% !TEX root = C:/Users/Slava/White-Paper/[06][NFT-Pass]/[NFT-Pass][Konzept].tex

\subsubsection{Pattern-Property}

\vspace{0.3cm}

\begin{sloppypar}
Diese NFT-Property soll ebenso wie die beiden vorigen einem abstufenden Raritätsprinzip zu Grunde liegen - und zwar ausschließlich dem Zufall folgend.
\end{sloppypar}

Im Gegensatz zu den beiden vorigen Properties obliegt die \textit{Pattern-Property} keiner absoluten Obergrenze - insbesondere auch dann nicht, falls einige Pattern zu einem Zeitpunkt verhältnismäßig unter- oder überrepräsentiert sind.

\vspace{0.3cm}

\begin{NFT-Prop}[Background (Pattern)]

Wir definieren folgende \textit{NFT-Pass-Background-Muster} mit den dazugehörenden Eigenschaften:

\begin{itemize}
    \item P1
    \begin{itemize}
    	\item Mögliche Ausprägung: \textbf{Safari Fun} 
    	\item Wahrscheinlichkeit: 0,1953125\% $\left( \frac{1}{512} \right)$
    \end{itemize}
    \item P2
    \begin{itemize}
    	\item Mögliche Ausprägung: \textbf{Triangular Bars} 
    	\item Wahrscheinlichkeit: 0,390625\% $\left( \frac{1}{256} \right)$
    \end{itemize}
    \item P3
    \begin{itemize}
    	\item Mögliche Ausprägung: \textbf{Pointillism} 
    	\item Wahrscheinlichkeit: 0,78125\% $\left( \frac{1}{128} \right)$
    \end{itemize}
    \item P4
    \begin{itemize}
    	\item Mögliche Ausprägung: \textbf{Wavy waves} 
    	\item Wahrscheinlichkeit: 1,5625\% $\left( \frac{1}{64} \right)$
    \end{itemize}
    \item P5
    \begin{itemize}
    	\item Mögliche Ausprägung: \textbf{Stony desert} 
    	\item Wahrscheinlichkeit: 3,125\% $\left( \frac{1}{32} \right)$
    \end{itemize}
    \item P6
    \begin{itemize}
    	\item Mögliche Ausprägung: \textbf{WunderPass} 
    	\item Wahrscheinlichkeit: 6,25\% $\left( \frac{1}{16} \right)$
    \end{itemize}
    \item P7
    \begin{itemize}
    	\item Mögliche Ausprägung: \textbf{Zigzag} 
    		\item Wahrscheinlichkeit: 12,5\% $\left( \frac{1}{8} \right)$
    \end{itemize}
    \item P8
    \begin{itemize}
    	\item Mögliche Ausprägung: \textbf{Linear}  
    	\item Wahrscheinlichkeit: 25\% $\left( \frac{1}{4} \right)$
    \end{itemize}
    \item P9
    \begin{itemize}
    	\item Mögliche Ausprägung: \textbf{Curves}
    	\item Wahrscheinlichkeit: 50,1953125\% $\left( \frac{257}{512} \right)$
    \end{itemize}
\end{itemize}

\end{NFT-Prop}

\vspace{0.3cm}

    % binde die Datei '[NFT-Pass][Konzept][Pattern].tex' ein
%% !TEX root = C:/Users/Slava/White-Paper/[06][NFT-Pass]/[NFT-Pass][Konzept].tex

\subsubsection{Pattern-Property}

\vspace{0.3cm}

\begin{sloppypar}
Diese NFT-Property soll ebenso wie die beiden vorigen einem abstufenden Raritätsprinzip zu Grunde liegen - und zwar ausschließlich dem Zufall folgend.
\end{sloppypar}

Im Gegensatz zu den beiden vorigen Properties obliegt die \textit{Pattern-Property} keiner absoluten Obergrenze - insbesondere auch dann nicht, falls einige Pattern zu einem Zeitpunkt verhältnismäßig unter- oder überrepräsentiert sind.

\vspace{0.3cm}

\begin{NFT-Prop}[Background (Pattern)]

Wir definieren folgende \textit{NFT-Pass-Background-Muster} mit den dazugehörenden Eigenschaften:

\begin{itemize}
    \item M1
    \begin{itemize}
    	\item Mögliche Ausprägung: \textbf{Safari} 
    	\item maximale Anzahl Pässe: 256
    	\item Wahrscheinlichkeit falls noch nicht aufgebraucht: 1,5625\% $\left( \frac{1}{64} \right)$
    \end{itemize}
    \item M2
    \begin{itemize}
    	\item Mögliche Ausprägung: \textbf{Bars} 
    	\item maximale Anzahl Pässe: 4.096
    	\item Wahrscheinlichkeit falls noch nicht aufgebraucht: 3,125\% $\left( \frac{1}{32} \right)$
    \end{itemize}
    \item M3
    \begin{itemize}
    	\item Mögliche Ausprägung: \textbf{Dots} 
    	\item maximale Anzahl Pässe: 65.536
    	\item Wahrscheinlichkeit falls noch nicht aufgebraucht: 6,25\% $\left( \frac{1}{16} \right)$
    \end{itemize}
    \item M4
    \begin{itemize}
    	\item Mögliche Ausprägung: \textbf{Muster festlegen} 
    	\item maximale Anzahl Pässe: 1.048.576
    	\item Wahrscheinlichkeit falls noch nicht aufgebraucht: 12,5\% $\left( \frac{1}{8} \right)$
    \end{itemize}
    \item M5
    \begin{itemize}
    	\item Mögliche Ausprägung: \textbf{Muster festlegen}  
    	\item maximale Anzahl Pässe: 16.777.216
    	\item Wahrscheinlichkeit falls noch nicht aufgebraucht: 25\% $\left( \frac{1}{4} \right)$
    \end{itemize}
    \item M6
    \begin{itemize}
    	\item Mögliche Ausprägung: \textbf{Muster festlegen}
    	\item maximale Anzahl Pässe: unbegrenzt  
    	\item Wahrscheinlichkeit: 50\% + x mit stets größer werdendem $x \in \left[ \frac{1}{64}; \frac{1}{2} \right]$
    \end{itemize}
\end{itemize}

\vspace{0.2cm}

Werden Pässe der Ausprägungen M1 bis M5 (aufgrund ihres Caps) im Laufe der Zeit aufgebraucht, geht deren Wahrscheinlichkeit auf die Property-Ausprägung M6 über. Für das x aus der Beschreibung der Ausprägung M6 gilt also:

\vspace{0.2cm}

\begin{equation*}
x = \frac{1}{64} + \sum_{aufgebrauchte \textrm{ } M \in \lbrace M1;...;M5 \rbrace} Wahrscheinlichkeit(M)
\end{equation*}

\end{NFT-Prop}

\vspace{0.3cm}

\begin{Algo}[Verlosungs-Mechanismus für Background-Property]

\begin{itemize}
    \item Zunächst bestimme man die Gesamtanzahl aller bisher geminteter Pässe $n$.
    \item Gleiches tue man nun für die Counts der geminteten Pässe pro Ausprägung der Background-Property M1 bis M6 als entsprechende Größen $n_1, n_2,...,n_6$.
    \item Seien $\Theta_i$ für $i \in \lbrace 1,...6 \rbrace$ die oben definierten \textbf{absoluten} Obergrenzen der \newline Ausprägungen der Background-Property M1 bis M6.
    \item Seien $\rho_i$ für $i \in \lbrace 1,...6 \rbrace$ die oben definierten Wahrscheinlichkeiten der \newline Ausprägungen der Background-Property M1 bis M6, deren Auswahl wir hier zusätzlich formalisieren wollen:
    
\[
\rho_i:=\left\{%
\begin{array}{ll}
    \hbox{\LARGE $\frac{1}{2^{7 - i}}$,} & \hbox{für $i = 1,...,5$} \\[0,3cm]
    \hbox{$\frac{1}{2} + \frac{1}{2^6}$,} & \hbox{für $i = 6$} \\
\end{array}%
\right.
\]    
    
    \item Alle Ausprägungen mit $n_i \geq \Theta_i$ sind aufgebraucht und können weder zum aktuellen Zeitpunkt noch in der Zukunft vergeben werden und damit fortan auch nicht beim Minting eines neuen Pass berücksichtigt werden.
    \item Wir unterteilen die Ausprägungen der Background-Property in \textit{"verbraucht"} und \textit{"verfügbar"}:
    
\begin{align*}
\overline{M} &:= \lbrace i \in \lbrace 1,...,6 \rbrace \textrm{ } | \textrm{ } n_i \geq \Theta_i \rbrace \textrm{     [verbraucht]} \\
M &:= \lbrace i \in \lbrace 1,...,6 \rbrace \textrm{ } | \textrm{ } n_i < \Theta_i \rbrace \textrm{     [verfügbar]}
\end{align*} 

Man vergewissere sich an dieser Stelle gedanklich, dass $M \cap \overline{M} = \emptyset$, $M \cup \overline{M} = \lbrace 1,...,6 \rbrace$ und $6 \in M$ gelten.
    
    \item Ist eine bestimmte Ausprägung $i \in \lbrace 1,...,5 \rbrace$ verbraucht, soll ihre Wahrscheinlichkeit $\rho_i$ auf die Ausprägung M6 übertragen werden (damit wir bei 100\% bleiben).

    \item Damit errechnen wir die aktuell vorliegenden neuen Wahrscheinlichkeiten $\widehat{\rho}_i$ für unsere Background-Ausprägungen als
\end{itemize} 

\[
\widehat{\rho}_i:=\left\{%
\begin{array}{ll}
	\hbox{0,} & \hbox{für $i \in \overline{M}$} \\[0,1cm] 
    \hbox{$\rho_i$,} & \hbox{für $i \in M$, $i \neq 6$} \\[0,3cm]
    \displaystyle \rho_1 + \sum_{i \in \overline{M}} \rho_i, & \hbox{für $i = 6$} \\
\end{array}%
\right.
\]

Man vergewissere sich an dieser Stelle gedanklich, dass auch für die neuen \newline Wahrscheinlichkeiten \[\sum_{i = 1}^6 \widehat{\rho}_i \textrm{ } = 1\] gilt.

\begin{itemize}
    \item Am Ende bestimme man mittels Randomisierung anhand der Wahrscheinlichkeiten $\widehat{\rho}_i$ für $i \in \lbrace 1,...6 \rbrace$ die zu vergebende Background-Ausprägung. 
\end{itemize}

\end{Algo}

\vspace{0.3cm}




% -----------------NFT-Edition-------------------
\subsubsection{Edition}
\vspace{0.3cm}
% !TEX root = paper.tex

Die Edition unseres WunderPasses soll als Property auf die anfangs geforderte Möglichkeit einer gewissen Individualisierung des WunderPasses durch seinen Besitzer einzahlen. Zu individuell darf eine solche NFT-Property aber auch nicht sein, da der NFT zwingend seinen Eigentümer wechseln können soll, da das ganze Unterfangen mit dem NFT-Pass andernfalls ad absurdum führte.

Um die Edition-Property noch etwas interessanter zu gestalten, sollen Exemplare jeder Edition nicht endlos verfügbar sein, sondern stattdessen irgendwann einmal \textit{aufgebraucht}. In solch einem Fall soll sich der User aber nicht einfach irgendeiner anderer Edition bedienen, sondern erhält die \textit{"Oberedition"} (Parent) seiner ursprünglich gewünschten Edition. Und sollte auch diese \textit{aufgebraucht} sein, dann wiederum die \textit{"Oberedition"} der \textit{"Oberedition"} usw. 

\vspace{0.2cm}

\begin{NFT-Prop}[Edition]

Als Ausprägung der WunderPass-NFT-\textbf{Edition} haben wir uns für \textbf{Städte} der Welt entschieden. Die \textbf{Parent-Edition} einer Stadt ist das dazugehörige \textbf{Land}, deren 
Parent-Edition wiederum der entsprechende \textbf{Kontinent} und als \textbf{oberste Editions-Ebene} dann die \textbf{Welt-Edition}. Letztere unterliegt folglich dann auch keiner stückweisen Obergrenze mehr.

\vspace{0.2cm}

\underline{\textbf{\textit{Beispiel einer Edition-Kette:}}}

\vspace{0.2cm}

\begin{equation*}
\textrm{Berlin } \rightarrow \textrm{ Germany } \rightarrow \textrm{ Europe } \rightarrow \textrm{ World }
\end{equation*} 

\vspace{0.2cm}

Es gilt das folgende grobe Regel-Set, was jedoch explizit auch nach Launch modifizierbar bleiben soll:

\begin{itemize}
    \item Die möglichen Editionen werden von uns bestimmt. Diese müssen nicht zwingend beim Launch des NFT vollständig benannt werden, sondern können stattdessen auch nachträglich eingepflegt werden. User-Wünsche (in welcher Form auch immer) sind dabei explizit erwünscht.
    \item Jede berücksichtigte \textit{Städte-Edition} ist genau \textbf{100} Mal verfügbar. Sind alle 100 Exemplare einer \textit{Städte-Edition} bereits gemintet (verbraucht), erhält die nächste Mint-Anfrage nach einem WunderPass derselben Edition automatisch die zu dieser Städte-Edition gehörende \textit{Landes-Edition}.
    \item Die \textit{Landes-Editionen} sind in einer maximalen Stückzahl von je \textbf{10.000} pro berücksichtigtem Land verfügbar. Sind auch diese aufgebraucht, wird die durch den User ausgewählte Stadt auf die ihrem Land übergeordnete \textit{Kontinent-Edition} gemappt.
    \item Die \textit{Kontinent-Editionen} sind in einer maximalen Stückzahl von je \textbf{1.000.000} für jeden Kontinent (außer der Antarktis) vorgesehen. Sollte auch diese Menge irgendwann erschöpfen, greifen wir zu der übergeordneten \textit{Welt-Edition}.
\end{itemize} 

\end{NFT-Prop}

\vspace{0.4cm}

\underline{\textbf{Quantitative Daten zu den Editionen:}}

\begin{itemize}
    \item Nach aktuellem Stand sind mindestens 693 \textit{Städte-Edition} vorgesehen.
    \item Die genannten \textit{Städte-Edition} verteilen sich dabei aktuell auf 179 \textit{Landes-Edition}.
    \item Die unterschiedlichen \textit{Kontinent-Edition} belaufen sich auf 6 (Nord- und Südamerika, Europa, Afrika, Asien und Australien).
    \item Die übergeordnete \textit{Welt-Edition} ist in ihrer Stückzahl unbegrenzt. 
    \item Die Auswahl der angebotenen \textit{Städte-Editionen} folgt (mit Augenmaß) in etwa folgender Logik:
    \begin{itemize}
    	\item Die Hauptstadt eines jeden mit einer \textit{Landes-Edition} versehenen Landes ist gleichzeitig auch eine verfügbare \textit{Städte-Edition}.
    	\item Mit Ausnahme der Hauptstädte erfordert die Größe einer Stadt (nach Einwohnern) ein Mindestmaß $m_1$, um als \textit{Städte-Edition} aufgenommen zu werden.
    	\item Sofern es das vorige Kriterium hergibt, sollen nach Möglichkeit für jedes Land mit einer eigenen \textit{Landes-Edition} mindestens seine 5 größten Städte mit einer eigenen \textit{Städte-Edition} versehen werden.
    	\item Überschreiten die $n$ größten Städte eines in die \textit{Landes-Editionen} aufgenommenen Landes eine bestimmte Mindestgröße $m_2$ (nach Einwohnern), werden alle $n$ Städte in die verfügbaren \textit{Städte-Editionen} aufgenommen. Dieses Kriterium wird aufgrund des vorigen ausschließlich für $n > 5$ relevant.
    	\item Städte der G7-Länder werden (ungeachtet etwaiger Mindestgröße) vermehrt in die \textit{Städte-Editionen} aufgenommen (bis zu 25 \textit{Städte-Editionen} pro G7-Land).
    \end{itemize} 
    \item Einzelne Städte können bei Bedarf auch bei Missachtung aller vorigen Kriterien aufgenommen werden.
\end{itemize}

\vspace{0.5cm}




    % binde die Datei '[NFT-Pass][Konzept][Edition].tex' ein

% -----------------Design------------------------
\subsubsection{Design}
\vspace{0.3cm}
% !TEX root = paper.tex

\subsubsection{Design}

\vspace{0.2cm}

\todo{TODO: Design}

\vspace{0.3cm}    % binde die Datei '[NFT-Pass][Konzept][Design].tex' ein
%% !TEX root = paper.tex

\subsubsection{Design}

\vspace{0.2cm}

\todo{TODO: Design}

\vspace{0.3cm}
\newpage


% -----------------Beispiel----------------------
\subsubsection{Beispielhafte Analyse der Collection}
\vspace{0.3cm}
% !TEX root = C:/Users/Slava/White-Paper/[06][NFT-Pass]/[NFT-Pass][Konzept].tex

\subsubsection{Beispiel}

\vspace{0.2cm}

\todo{TODO: Beispielrechnung für geminteten NFT-Pass mit der Nummer x}

Angenommen x sei 1.005.965.

\begin{itemize}
  \item vorrechnet, welche ersten 1.005.964 NFT-Pässe schon weggemintet sein könnten und Wahrscheinlichkeiten für den neu zu mintenden NFT-Pass erklären.
  \item neuen NFT-Pass unter Einbindung der Wahrscheinlichkeiten und vorgegaukelten Zufalls errechnet.
  \item geminteten neuen NFT-Pass als exakte Grafik in unserem Design hier abbilden.
\end{itemize}

\vspace{0.3cm}    % binde die Datei '[NFT-Pass][Konzept][Beispiel].tex' ein
%% !TEX root = C:/Users/Slava/White-Paper/[06][NFT-Pass]/[NFT-Pass][Konzept].tex

\subsubsection{Beispiel}

\vspace{0.2cm}

\todo{TODO: Beispielrechnung für geminteten NFT-Pass mit der Nummer x}

Angenommen x sei 1.005.965.

\begin{itemize}
  \item vorrechnet, welche ersten 1.005.964 NFT-Pässe schon weggemintet sein könnten und Wahrscheinlichkeiten für den neu zu mintenden NFT-Pass erklären.
  \item neuen NFT-Pass unter Einbindung der Wahrscheinlichkeiten und vorgegaukelten Zufalls errechnet.
  \item geminteten neuen NFT-Pass als exakte Grafik in unserem Design hier abbilden.
\end{itemize}

\vspace{0.3cm} 


% -----------------Intrinsischer Wert------------
\subsubsection{Intrinsischer Wert}
\vspace{0.3cm}
% !TEX root = paper.tex
\vspace{0.2cm}

\todo{TODO: intrinsischer Wert mittels Berechtigungen als Governance-Token}

\vspace{0.3cm}    % binde die Datei '[NFT-Pass][Konzept][Intrinsischer Wert].tex' ein
%% !TEX root = paper.tex
\vspace{0.2cm}

\todo{TODO: intrinsischer Wert mittels Berechtigungen als Governance-Token}

\vspace{0.3cm}


\vspace{0.5cm}






%------------------------------------------------Technische Implementierung------------------------

% !TEX root = paper.tex

\subsection{Technische Umsetzung}

\vspace{0.3cm}

\todo{TODO: technische Implementierung}

\vspace{0.3cm}

\begin{itemize}
  \item Abwandlung des ERC721-Standard, um unsere Metadaten-Logik zu bändigen.
  \item Die Metadaten werden wohl auch einem ähnlichen Konstrukt wie IPFS (off-chain) gespeichert werden und lediglich deren Hash als Datenfeld im Smart-Contract (on-chain), damit die Metadaten nicht nachträglich verändern werden können (dieses Vorgehen wird der absolute Standard sein).
  \item Unsere Metadaten sind jedoch so komplex, das deren Erzeugung (beim Minten) wohl einen zweiten Smart-Contract erfordern wird. Wir haben also quasi einen "Metadaten-Hybriden":
  \begin{itemize}
  	\item Erzeugung on-chain
  	\item Storing off-chain
  \end{itemize}
  \item Der Metadaten-Smart-Contract wird die oben skizzierte Logik implementieren
  \begin{itemize}
  	\item Wie viele Pässe gibts es bereits und welche (hinsichtlich Properties)?
  	\item Wie sind die aktuellen Verteilungen der Properties und deren Contstraints
  	\item Einbindung von Randomisierungs-Orakeln
  	\item Sicherstellung, dass die erzeugten Metadaten auch tatsächlich vom Caller (ERC721-Contract) verwendet wurden und keine nachträgliche Manipulation stattgefunden hat.
  \end{itemize}
  \item Es muss geklärt werden, ob hinsichtlich des Gedanken an den besagten "zweiten Smart-Contract" Standards/Best-Practices existieren, damit wir hier nicht das Rad neu erfinden.
  \item Es bleibt noch nicht ganz klar, wie die Metadaten nach ihrer Erzeugung nach IPFS gelangen, da dies laut meinem Verständnis ein Smart-Contract nicht selbst gewährleisten kann. Moritz Idee war grob die Folgende 
  \begin{itemize}
    \item Der Minting-Contract erzeugt den NFT, lässt seine Metadaten-Referenz jedoch zunächst ungesetzt (der NFT ist damit in gewisser Weise noch "unfertig"; kann in dem Zustand auch noch Constraints unterstellt sein).
    \item Der Minting-Contract callt den Metadaten-Contract mit dem Anliegen, Metadaten zu dem "unfertigen" NFT mit der zugehörigen ID zu erzeugen.
  	\item Der Metadaten-Contract erzeugt die Metadaten, hasht diese und gibt den Hash zurück an den Minting-Contract. Gleichzeitig publisht er ein Create-Event mit der Token-ID und den zugehörigen erzeugten Metadaten.
  	\item Der Minting-Contract speichert den erhaltenen Metadaten-Hash und wartet auf "approvement".
  	\item Das forcierte Event wird von einem dafür bestimmten (off-chain) Web3-Service vernommen und weiterverarbeitet: Die Metadaten werden geparst und nach IPFS gepusht. Als Ergebnis bekommen wir eine entsprechende IPFS-URI.
  	\item Unser Web3-Service stößt anschließend eine "Set-URI"-Transaktion mit den entsprechenden Input-Daten (Token-ID; IPFS-URI) beim Minting-Contract an, um den gesamten Minting-Prozess für den neuen Token abzuschließen.
  	\item Der Minting-Contract verifiziert die Metadaten mittels des gespeicherten Meta-Daten-Hashs (\todo{Hier ist nicht nicht ganz klar, wie. Ich weiß nicht, ob der Contract einfach die Daten von IPFS laden kann, um den Hash abzugleichen oder ob er vorher die URI implizit vorgeben muss, die irgendwie im Hash berücksichtigt werden muss, oder wie auch immer hier die Best-Practise aussieht}) und updatet die NFT-URI auf den Wert der übergebenen IPFS-URI. 
  	\item Hiermit ist der Minting-Prozess abgeschlossen, der NFT "fertig" gemintet und kann von etwaigen "Temporary-Locked-Constraint" entbunden werden und vom neuen Besitzer frei verfügt werden.
  \end{itemize}
  \item \textbf{Ein etwaiger Crypto-Freelancer muss auf die skizzierten Herausforderungen gechallenget werden.}
\end{itemize}    % binde die Datei '[NFT-Pass][Tech].tex' ein

\vspace{0.5cm}    % binde die Datei 'NFT-Pass.tex' ein
% !TEX root = paper.tex

\section{Abgrenzung zu SSI}
\label{sec:ssi}

\vspace{0.3cm}

\todo{TODO}

\vspace{0.3cm}

\todo{TODO: DID scheint für uns eine zentralere Rolle zu spielen als SSI. DID sollten wir also eher in WunderPass einbinden, als uns davon distanzieren zu versuchen.}

\vspace{0.3cm}    % binde die Datei 'SSI.tex' ein
% !TEX root = paper.tex
\section{Project 'Guard'}
\label{sec:guard}
\todo{TODO}    % binde die Datei 'Project Guard.tex' ein
% !TEX root = paper.tex

\section{Project 'Pools'}
\label{sec:pools}

\subsection{Einleitung}
\label{sec:pools-einleitung}
\vspace{0.3cm}
% !TEX root = paper.tex

Die Idee hinter den sogenannten \textit{Wunder-Pools} ist das Bündeln von Liquidität mehrerer User/Teilnehmer bzw. eine Art 'Treuehandverwahrung' in einem gemeinsamen Pool. Die Anwendungsfälle solche Pools können sehr zahlreich sein. Um im Folgenden nur einige Beispiele zu nennen:  

\begin{itemize}
  \item Gemeinsame Invests in (Crypto-)Assets.
  \item Pool für ein gemeinsames (Geburtstags-)Geschenk.
  \item Kicktipp-Pool (der über die gesamte Saison verwahrt werden muss).
  \item Wetten unter Freunden (z. B. Sportereinisse wie ein WM-Finale).
  \item Ausgleichspool für Auslagen von Geld an Freunde (Splitwise).
\end{itemize}

\vspace{0.2cm}

Das besondere an dem in den folgenden Abschnitten genauer zu beschreibenden Mo\-dell ist sein sehr allgemein gehaltener Ansatz, mit dem sich gleichzeitig Cases umsetzen lassen, die auf den ersten Blick sehr verschieden zu sein scheinen. Genauer genommen lassen sich solche Pools mit speziellen \textit{DAO-Strukturen} beschreiben.

Abgesehen von der den Pools zugrundeliegenden Geschäftslogik besteht der zentrale Ansatz unserer \textit{Wunder-Pools} darin, dem User ein rundes Produkt anzubieten - und zwar gänzlich unabhängig davon, welcher der oben genannten Cases nun tatsächlich umgesetzt werden soll. An dieser Stelle möchten wir uns daher ganz explizit von dem Status quo der heute gängigen UX in der Web3-Welt abgrenzen.

\vspace{0.4cm}

Ganz grob beschrieben, streben wir in etwa folgende Geschäftslogik an:

\begin{itemize}
  \item Ein User erstellt einen Pool (in unserer eigens designten Wunder-Pool-UI).
  \item Derselbe User wählt die gewünschte \textit{Pool-Art}, ein etwaiges dazugehöriges Regelwerk und fordert andere User auf, dem Pool beizutreten. Idealerweise erfolgt die Einladung mittels Suche nach der Wunder-ID bzw. eines sprechenden Namens des einzuladenden Teilnehmers (und nicht etwa anhand seiner Ethereum-Adresse oder sonstigem).
  \item Die eingeladenen Teilnehmer erhalten die Einladung (in der WunderPass-App oder der Wunder-Pool-Applikation) und können entscheiden, ob sie dem Pool beitreten möchten oder nicht. 
  \item In der Regel ist der definierte Einsatz sofort beim Beitritt des Pools zu entrichten und geht direkt in die Pool-Treasury. In einigen Cases kann der Einsatz evtl. auch zu einem späteren Zeitpunkt erfolgen oder gar ganz entfallen (z.B. beim Case \textit{Splitwise}).
  \item Der eingerichtete Geldpool kann nun als Gemeinschaftsvermögen/ -konto zu u. a. folgenden Zwecken verwendet werden:
  \begin{itemize}
  	\item zum gemeinschaftlichen Investieren in (Crypto-)Assets,
  	\item zum Verwahren \textit{"in Treuhand"} bei einer oder mehreren abgeschlossenen Wetten (oder auch z. B. Kicktipp)
  	\item etc.
  \end{itemize}
  \item Der Pool wird liquidiert und das gemeinschaftliche Geld (nach einem aus dem vorher gemeinsam festgelegten Regelwerk folgenden Verteilungsschlüssel) auf alle Pool-Mitglieder verteilt. Die Liquidierung selbst kann entweder ebenfalls durch das Regelwerk auf einen bestimmten Zeitpunkt und/oder Ereignis terminiert sein (z.B. Ende einer BuLi-Saison beim Case \textit{Kicktipp}) oder aber durch die Teilnehmer beschlossen werden (mittels einer DAO-Abstimmung). Die Errechnung des genann\-ten Verteilungsschlüssels möchten wir möglichst allgemein halten und übertragen diese Verantwortlichkeit einem \textit{abstrakten Oracle}, welches es stets Case-spezifisch zu definieren (und zu implementieren) gilt.
\end{itemize}

\vspace{0.2cm}

\underline{\textbf{Product-Sicht}}

\vspace{0.2cm}

Abschließend sei noch einmal betont, dass wir das/die aus den Wunder-Pools hervorgehende(n) Product(s) (mittelfristig) alternativlos user-friendly sehen. Ohne notwendigen Bezug zur Crypto-Szene, ohne MetaMask und ohne kryptische hexadezimale Wallet-Adressen. Stattdessen clean und simpel.

\vspace{0.5cm}    % binde die Datei '[Pools][Einleitung].tex' ein


\subsection{Formalisierungen}
\vspace{0.3cm}
% !TEX root = paper.tex

Zunächst einmal benötigen wir einige formale Werkzeuge und bedienen uns dafür folgender Definition:

\vspace{0.2cm}

\begin{Def}\label{defPoolTeilnehmer}

Im folgenden setzen wir Voraus, die Nutzung der Pools seitens der User erfordert zwingend den Besitz eines WunderPass (bzw. Wunder-ID) und betrachten von daher auch nur solche User.

\begin{equation*}
  U := \left\{ u_1; u_2;...; u_{n} \text{ } | \text{ } u_i \text{ besitzt eine Wunder-ID} \right\}
\end{equation*}

\vspace{0.2cm}

Wir stellen zusätzlich, dass der vorausgesetzte Besitz einer Wunder-ID mit dem Besitz von unterschiedlichen Wallets bzw. anderen durch die Wunder-ID implizierten Dingen einhergeht. So hat jeder User $u_i$ z.B. eine Telefonnummer mit seiner Wunder-ID verknüpft (anhand derer er mittels Kontakte-Scan auf dem Smartphone als Inhaber einer Wunder-ID und damit potenzieller Pool-Teilnehmer erkannt werden kann und soll). Des weiteren kann $u_i$ einen NFT-Pass (siehe Kapitel \ref{sec:nft-pass}) besitzen und/oder unser ERC20-Utility-Token (im Folgenden als \textit{WPT} bezeichnet; siehe Kapitel \todo{TODO: verlinken}). 

\vspace{0.2cm}

Wir formalisieren den in Kapitel \ref{sec:nft-pass} definierten NFT-Pass als die (geordnete) Menge aller bisher geminteter NFT-Pässe:

\begin{align*}
  WPN &:= \left\{ wpn_1; wpn_2;... \right\} \text{ mit} \\
  wpn_i &:= (s_i, w_i, m_i)
\end{align*}

\vspace{0.2cm}

Dabei repräsentiert $s_i$ den Status des NFT-Passes, $m_i$ sein Muster und $w_i$ das sich auf ihm abgebildete Weltwunder.

\vspace{0.2cm}

Den Besitz eines Pass-NFTs beschreiben wir durch die Funktion 

\vspace{0.2cm}

\begin{equation*}
  \omega : U \rightarrow \mathcal{P} \left( WPN \right)  
\end{equation*}

\begin{equation*}
  \omega(u):= \left\{ wpn \in WPN \text{ } | \text{ User u besitzt den Pass-NFT } wpn \right\}. 
\end{equation*}

\vspace{0.2cm}

Analog dazu definieren wir auch den Besitz am \textit{WPT} eines Users - mit dem Unterschied, dass der Funktionsbereich dieser Funktion aufgrund der Fungibilität von einer Potenzmenge auf einen simplen numerischen Wert zusammenfällt:

\vspace{0.2cm}

\begin{equation*}
  \varphi : U \rightarrow \mathbb{Q}  
\end{equation*}

\begin{equation*}
  \varphi(u):= \text{ Balance des Users u am ERC20-Token WPT}. 
\end{equation*}

\end{Def}

\vspace{0.5cm}    % binde die Datei '[Pools][Formalisierung].tex' ein


\subsection{Pool-Erzeugung}
\vspace{0.3cm}
% !TEX root = paper.tex

Die Erzeugung eines Pools findet in zwei Phasen statt: Der \textit{Initialisierungs-Phase} und der \textit{Joining-Phase}.

\vspace{0.5cm}

\underline{\textbf{Initialisierungs-Phase}}

\vspace{0.2cm}

Die Initialisierungs-Phase läuft in etwa in folgenden Schritten ab:

\begin{itemize}
	\item Ein Initiator (Admin) $u_A \in U$ erzeugt den Pool in einer dafür vorgesehenen \textit{Pool-Applikation} (vergleichbar mit z. B. der Erstellung einer WhatsApp-Gruppe). Der Initiator $u_A$ ist dabei selbst ein Teilnehmer des Pools. Unsere klare Absicht hierbei ist jedoch keine "gesonderten" Pool-Teilnehmer zu haben bzw. mit besonderen Rechten auszustatten. Die Unterscheidung zwischen dem Admin $u_A$ und anderen Pool-Teilnehmern $u \in U$ ist idealerweise - sofern es denn der spezielle Case zulässt - nur für die Initialisierungs-Phase von Nöten und kann anschließend entfallen.
	\item Der Admin definiert das Regelwerk für den zu erstellenden Pool:
	\begin{itemize}
		\item Art des Pools (Invest-Pool, Wette, Spende, Kicktipp, Splitwise etc.)
		\item privater oder öffentlich zugänglicher Pool
		\item etwaige Obergrenze an Teilnehmern
		\item Einsatz (minimaler, maximaler oder exakter Einsatz pro Teilnehmer und Währung des Einsatzes)
		\item Auszahlungslogik (per Abstimmung oder Adresse eines Oracle-Smart-Con\-tracts, der abhängig seiner Contract-Logik einen Auszahlungsschlüssel bereit\-stellt)
	\end{itemize}
\end{itemize}

\vspace{0.3cm}

\underline{\textbf{Joining-Phase}}

\vspace{0.2cm}

Die Teilnahme-Phase besteht grob aus folgenden Schritten:

\begin{itemize}
	\item Der Admin verliert seine Sonderstellung und wird stattdessen zum ersten Teilnehmer seines eigens initiierten Pools. 
	\item Der (ursprüngliche) Admin lädt Teilnehmer ein, sich am kreierten Pool zu beteiligen. Die Beteiligung erfordert dabei einen WunderPass (= Wunder-ID) seitens des Teilnehmers. Idealerweise sind die Wunder-IDs mit Telefonnummern verknüpft, mittels welcher sich die einzuladenden User in den Kontakten des Admins erkennbar als potenzielle Teilnehmer wiederfinden.
	\item Alternativ kann der (ursprüngliche) Admin - wie auch jeder andere bereits beigetretene Teilnehmer - einen Teilnahme-Link an (weitere) potenzielle Teilnehmer verschicken.
	\item Jeder adressierte User erhält die Einladung inklusive aller relevanten Informationen zum beizutretenden Pool (insbesondere des benötigten Einsatz) in seiner Wunder-Pool-Applikation, und muss diese lediglich entweder bestätigen oder ablehnen (\textit{Pull-Prinzip}). Insbesondere braucht der User für den Beitritt zum Pool kein MetaMask oder sonstiges Hilfsmittel (\textit{Push-Prinzip}; wie aktuell bei DAOs üblich). 
	\item Auch der Einsatz des neuen Teilnehmers muss nicht aktiv entrichtet (manuell an eine Wallet gesendet) werden, sondern wird stattdessen im Zuge des vorigen Schritts nach Bestätigung der Teilnahme am Pool automatisch eingezogen.
\end{itemize}

\vspace{0.2cm}

Wir fassen die bisher erzielten Ergebnisse etwas formaler zusammen:

\vspace{0.2cm}

\begin{Def}\label{defPool}

Ein (jungfräulicher) Pool im Sinne der oben aufgezählten Eigenschaften und Anforderungen lässt sich formal schreiben als

\begin{equation*}
  Pool := \left( \mathcal{U}, \mathcal{R}, \mathcal{T}, \mathcal{G} \right) \text{ mit}
\end{equation*}

\begin{align*}
  & \mathcal{U} = \left\{ u_1; u_2;...;u_n \right\} \subseteq U \text{ die Menge der n Pool-Teilnehmer, } \\
  & \mathcal{R} \text{ das Regelset des Pools, was es gesondert zu formalisieren gilt, } \\
  & \mathcal{T} = \left\{ s_1...;s_n \right\} \text{ mit } s_i \in \mathbb{Q} \text{ die Treasury des Pools und} \\
  & \mathcal{G} = \left\{ g_1...;g_n \right\} \text{ mit } g_i \in \mathbb{N} \text{ die Governance des Pools.}
\end{align*}


\noindent\hrulefill
\vspace{0.2cm}

Dabei beschreibt jedes $s_i$ den Einsatz des Teilnehmers $u_i \in \mathcal{U}$ ($s$ für Stake). Dieser Einsatz liegt dabei in einem vom Regelset $\mathcal{R}$ definierten Intervall $\mathcal{I} \subseteq \mathbb{Q}$. 

Damit haben wir bereits an dieser Stelle einen kleinen Teil der noch fehlenden Formalisierung von $\mathcal{R}$ identifiziert. Bei genauer Betrachtung fehlt uns noch die Einheit der Einsätze $s_i$. Diese wird sehr wahrscheinlich \textit{USDT} sein oder ein anderer Stable-Coin.

Zudem beachte man bereits an dieser Stelle, die Definition von $\mathcal{T}$ werde im Verlaufe der Lifetime eines Pools nicht so simpel bleiben können, als lediglich aus den eingebrachten Einsätzen der Teilnehmer zu bestehen. Die Pool-Treasury bedeutet nämlich mehr als nur die Menge der initialen Stakes. Etwaige Invests aus der Treasury heraus würden nämlich ebenfalls in der Treasury landen.

\vspace{0.2cm} 

Die $g_i$ dagegen beschreiben ganz simpel die Anzahl der Governance-Tokens pro User $u_i \in \mathcal{U}$. Man kann diese auch als Gesellschaftsanteile einer GbR betrachten. Das Stammkapital dieser Gesellschaft würde sich in diesem Vergleich auf

\begin{equation*}
  \kappa := \sum_{i=1}^{n} g_i 
\end{equation*}

belaufen. Der prozentuale Stimmrecht-Anteil eines Users $u_i \in \mathcal{U}$ ergäbe sich hieraus als 

\begin{equation*}
  \rho_i = \frac{g_i}{\kappa}, \text{   } \forall i = 1, 2, ...,n. 
\end{equation*}

\end{Def}

\vspace{0.5cm}

Die zusammengetragenen Anforderungen für die Initialisierung eines WunderPools lassen sofort deutlich erkennen, \textbf{diese Pools könnten mittels DAO-ähnlicher Strukturen implementiert werden}. Dies erscheint insofern umso logischer und konsequent, als dass wir bereits erkannt haben, die Pools stellten gesellschaftsrechtlich GbRs dar - also Gesellschaften und/oder Organisationen. Diese Erkenntnis wollen wir noch einmal als eine formale Annahme formulieren: 

\vspace{0.3cm}

\begin{Assumption}[Ein WunderPool bildet de facto eine GbR ab]
\label{assumptionGbR} 

Sei $\mathcal{P} := \left( \mathcal{U}, \mathcal{R}, \mathcal{T}, \mathcal{G} \right)$ ein WunderPool wie in Definition \ref{defPool} beschrieben. Wir ziehen die Analogie zu einer \textbf{Gesellschaft} bürgerlichen Rechts:

\begin{itemize}
	\item Die Menge $\mathcal{U}$ der Pool-Teilnehmer repräsentiert den \textbf{Gesellschafterkreis der Gesellschaft}.
	\item $\mathcal{G}$ beschreibt den \textbf{Cap-Table der Gesellschaft}.
	\item Das Pool-Regelwerk $\mathcal{R}$ ist nichts anderes als der \textbf{Gesellschaftervertrag zur Gesellschaft}.
	\item Die Pool-Treasury $\mathcal{T}$ modelliert zuletzt das \textbf{Gesellschaftskonto und/oder -depot der Gesellschaft}.
\end{itemize}

\end{Assumption}
 
\vspace{0.5cm}    % binde die Datei '[Pools][Erzeugung].tex' ein


\subsection{Pool-Lifetime}
\vspace{0.3cm}
% !TEX root = paper.tex

Eine (allgemeine) funktionale Beschreibung derjenigen WunderPool-Funktionalität, die der Überschrift der gegenständigen Sektion gerecht wird, ist insofern sehr schwierig, als dass sich diese deutlich schwerer auf unterschiedliche Pool-Cases verallgemeinern lässt. Wie anfangs in dem Einführungskapitel \ref{sec:pools-einleitung} ist die möglichste Verallgemeinerung aller Cases oberste Prämisse gewesen. Hier müssen wir versuchen zu verallgemeinern, was nur geht, und den Rest eben Case-spezifisch lösen. 

\vspace{0.1cm}

Wir schauen auf die Anfangs in Kapitel \ref{sec:pools-einleitung} hervorgehobenen Anwendungsfälle für die WunderPools an - nun mit kurzer Skizzierung ihrer Lifetime:

\begin{itemize}
  \item \textbf{\textit{Social Investing:}} Das ist mit der klarste Case für eine relevante Lifetime eines Pools. Während der Lifetime werden mögliche Invests vorgeschlagen, zur Abstimmung gestellt und im Erfolgsfall abgewickelt. Die Möglichkeiten zur Erweiterung von Investmöglichkeiten (Staking, Lending, Liquidity-Providing, Yield Farming, Aktien, ETFs etc.) scheinen schier unendlich. In diesem Case unterliegt \textbf{die Dauer der Lifetime auch keinerlei natürlicher Grenzen} - diese Art von Pool kann theoretisch ewig existieren.
  \item \textbf{\textit{Geschenk-Pool:}} In diesem Case besteht die Daseinsberechtigung des Pools eigentlich lediglich darin, bequem und einfach Geld einzusammeln und evtl. bis zum Kauf des Geschenks "in Treuhand" zu verwahren. Sind alle gewünschten Teilnehmer beigetreten (und somit ihren Beitrag zum Geschenk entrichtet), hat der Pool eigentlich bereits seinen Zweck erfüllt. Man kann zwar argumentieren, man könne die Auswahl des Geschenks mit DAO-Mitteln zur Abstimmung stellen, dies bleibt jedoch an den Haaren herbeigezogen, solange das Geschenk kein auf der Blockchain erwerbbares Asset ist. \textbf{Die Dauer der Lifetime der Pools in diesem Case sind also klar begrenzt}: Spätestens bis zu dem Moment des Kaufs des Geschenks.
  \item \textbf{\textit{Kicktipp-Pool:}} Das ist der Bilderbuch-Case für den Pool im Sinne der Treuhand-Verwahrung (eines Spieleinsatzes) über einen längeren Zeitraum. Hier wird eingezahlt, über einen Zeitraum (außerhalb des Pools) gespielt und am Ende - je nach Ergebnis - wieder ausgezahlt. Das Geld wird vom Pool also lediglich verwahrt und umverteilt. In der sogenannten \textit{Lifetime} des Pools passiert faktisch gar nichts. Man könnte sich sicherlich kreative Möglichkeiten zur Interaktion mit dem Pool überlegen (wie z.B. Abstimmungen über etwaige Regeländerungen oder über das Nachtragen von verspätet abgegebenen Tipps), dies beträfe aber nie die relevante Kernfunktionalität des Pools innerhalb dieses Cases. Die defacto \textit{'leere Lifetime'} des Pools endet in diesem Case mit Ablauf der Spielzeit, für die die Kicktipp-Runde eingerichtet wurde. Ihre \textbf{Dauer ist also begrenzt}.
  \item \textbf{\textit{Wetten:}} Dieser Case verhält sich sehr analog zum \textit{Kicktipp-Case}. Dazu muss jedoch klargestellt sein, dass wir den Case als eine einzige Wette (zwischen zwei oder mehr Leuten) verstehen, bei der der Pool der Treuhand-Verwahrung dient, und nicht etwa eine "Wett-Gruppe", wo immer mal wieder neue Wetten vorgeschlagen und umgesetzt werden. Der Pool dieses Cases bildet also eine einzige Wette ab und seine \textbf{Lifetime endet in dem Moment, wo das Ergebnis der Wette feststeht}.
  \item \textbf{\textit{Splitwise:}} Dies ist der außergewöhnlichste aller Cases. Hier existieren de facto weder eine echte Treasury noch eine Lifetime. Für Splitwise wird erst die Umverteilung interessant, wobei hier genau genommen der Betrag von 0 auf die Teilnehmer umverteilt wird. Da hier aber - im Gegensatz zu allen obigen Cases - auch negative Withdraws zulässig sind (also genau genommen eine Einzahlung von denjenigen Teilnehmern, die anderen Teilnehmern etwas schulden), klingt die Umverteilung des Betrags 0 plötzlich doch nicht mehr so abwegig. Die 0 signalisiert nur die Forderung, die verteilten Beträge (Schulden und Auslagen mit entsprechendem Vorzeichen) müssen sich auf 0 summieren. Da der Pool in diesem Case faktisch gar keine Lifetime besitzt, ist \textbf{die Dauer der Lifetime konsequenterweise begrenzt}.
\end{itemize}

\vspace{0.3cm}

Zusammenfassend halten wir fest, die Dauer der Pool-Lifetime ist nur für den \textit{Social-Investing-Case} theoretisch unbegrenzt. Bei allen anderen Cases wird der Pool nach einer bestimmten Zeit oder bei Eintreten eines bestimmten Ereignisses obsolet und muss/sollte anschließend aufgelöst werden. Und auch hinsichtlich relevanter Funktionalität während der \textit{Lifetime} scheint der \textit{Social-Investing-Case} ebenfalls der einzig interessante zu sein. 

\vspace{0.1cm}

Eine Verallgemeinerung erscheint also - zumindest für die zuletzt genannten vier Cases - evtl. doch im Rahmen des Möglichen. 

\vspace{0.3cm}

\todo{Etwaiges Austreten bestehender Teilnehmer oder Eintreten neuer Teilnehmer würde sich während der 'Lifetime' abspielen.}

\vspace{0.5cm}    % binde die Datei '[Pools][Lifetime].tex' ein


\subsection{Pool-Liquidierung}
\vspace{0.3cm}
% !TEX root = paper.tex

Für eine etwaige Pool-Liquidierung stellen sich exakt zwei Fragen: "\textbf{Wann} wird liquidiert?" und "\textbf{Wie} wird liquidiert?" Das \textbf{Wann} ist hierbei schnell geklärt. Es gibt grob folgende drei Möglichkeiten, von denen eine durch das in Definition \ref{defPool} definierte Regelset $\mathcal{R}$ zu spezifizieren ist:

\begin{itemize}
  \item $\mathcal{R}$ legt einen exakten Zeitpunkt fest, zu dem der Pool liquidiert werden soll.
  \item $\mathcal{R}$ definiert ein bestimmtes Ereignis, bei deren Eintreten der Pool liquidiert werden soll.
  \item $\mathcal{R}$ regelt, dass die Pool-Liquidierung per (DAO-)Abstimmung beschlossen werden muss.
\end{itemize}

\vspace{0.3cm}

Bullet 2 klingt hier leider noch nicht ausreichend abstrakt. Daher abstrahieren wir die genannten Forderungen in einer einzigen:

\vspace{0.2cm}

\begin{Fazit}[Liquidierungsentscheidung-Oracle]

Das in Definition \ref{defPool} definierte Regelset $\mathcal{R}$ definiert ein Oracle, welches zu jedem Zeitpunkt die Frage beantworten kann, ob der Pool zum jetzigen Zeitpunkt liquidiert werden soll oder nicht. 

Dieses Oracle kann beliebig einfach gestrickt sein (z.B. im Falle des obigen Bullet 1 einfach anhand $"SYSDATE <= T_{END}"$ über das Fortbestehen des Pools entscheidet) oder aber auch beliebig komplex. Dies braucht uns aber an dieser Stelle nicht weiter weiter interessieren. 

\end{Fazit}

\vspace{0.3cm}

Und da die Abstraktion mittels Oracle so bequem scheint, tun wir das Gleiche ebenfalls für das oben genannte \textbf{Wie}:

\vspace{0.2cm}

\begin{Fazit}[Auszahlungsschlüssel-Oracle]

Seien $\mathcal{P} = \left( \mathcal{U}, \mathcal{R}, \mathcal{T}, \mathcal{G} \right)$ 
der Pool und $\mathcal{U} = \left\{ u_1; u_2;...;u_n \right\}$ die Menge seiner $n$ Teilnehmer wie in Definition \ref{defPool} beschrieben und $v_{\mathcal{T}}$ der sich zum Liquidierungszeitpunkt in der Pool-Treasury $\mathcal{T}$ befindende Value. 

Falls der Pool lediglich als Treuhand-Verwahrung diente (also über die Zeit keine Veränderung der Treasury stattfand) ergibt sich $v_{\mathcal{T}}$ als  

\vspace{0.1cm}

\begin{equation*}
  v_{\mathcal{T}} = \sum_{i=1}^{n} s_i \text{ mit } s_i \text{ wie in Definition \ref{defPool}}
\end{equation*}

\vspace{0.2cm}

Wir definieren einen Auszahlungsvektor als

\begin{equation*}
  \varphi_{\mathcal{P}} = [\varphi_1, \varphi_1, ..., \varphi_n] \text{ mit } \sum_{i=1}^{n} \varphi_i = v_{\mathcal{T}} 
\end{equation*}

\vspace{0.2cm}

\todo{Ein mögliches $\varphi_{\mathcal{P}}$ ist tatsächlich $\mathcal{G}$}

\vspace{0.2cm}

\todo{$\mathcal{R}$ definiert das Oracle und sagt, ob die $\varphi_i$ negativ sein dürfen}

\vspace{0.2cm}

\todo{Oracle verteilt anhand von $\mathcal{T}$ und $\mathcal{G}$}

\end{Fazit}

\vspace{0.3cm}

\todo{TODO}

\vspace{0.5cm}    % binde die Datei '[Pools][Liquidierung].tex' ein


\subsection{Pool-Economics}
\vspace{0.3cm}
% !TEX root = paper.tex

\subsubsection{Einleitung}

\vspace{0.2cm}

Im gegenständigen Abschnitt erfolgt eine grundlegende finanzielle Beleuchtung des in den letzten Abschnitten beschriebenen Pools-Projekt - und zwar aus Sicht aller beteiligten Parteien - also aus Sicht von WunderPass, aus Sicht der User und aus Sicht etwaiger Projekt-Investoren und anderer -Stakeholder.

\vspace{0.1cm}

Dabei sollen gleichermaßen ein Monetarisierungmodell, ein zugehöriger Business-Plan sowie eine mögliche Utility-Token-Ökonomie, die diese Komponenten mittels \href{https://de.wikipedia.org/wiki/Mechanismus-Design-Theorie}{Mechanismus-Design} in Einklang zueinander bringt und in einem übergeordneten Ökonomie-Kreislauf verankert, gleichzeitig erarbeitet und miteinander verknüpft werden.

\vspace{0.2cm}

Am Ende soll idealerweise jede solcher Fragen wie,

\begin{itemize}
	\item \textit{Wer bezahlt den Pool-Service und wie viel?}
	\item \textit{Wer verdient am Pool-Service und wie viel?}
	\item \textit{Wie wird das Pool-Projekt finanziert und wie werden etwaige Investoren incentiviert und entlohnt?}
	\item \textit{Wie sieht der konkrete Business-Plan aus?}
	\item \textit{Wie wird der zugehörige Pool-Project-Token modelliert und in das übergeordnete Pool-Ökosystem integriert?}
	\item \textit{Wie sind Risiko und ROI von etwaigen Projekt-Invests zu beziffern?}
\end{itemize}

beantwortet sein.

\vspace{0.5cm}

Da der zentrale Bestandteil der eigentlichen Dienstleistung der Pools für seine Nutzer bereits in sehr starkem finanziellen Kontext - nämlich des \textit{Social-Investings} - steht, und wir uns im Folgenden mit dem finanziellen Gerüst des übergeordneten Pools-Projects beschäftigen möchten - das aber so gar nichts mit der Dienstleistung des \textit{Social-Investings} an sich gemein hat, müssen wir gleich zu Beginn eine essenzielle Abgrenzung ziehen, ohne deren unmissverständliches Bewusstsein beim Leser die folgenden Kapitel nur missverstanden werden können und werden.

\vspace{0.2cm}

\textbf{Man lese und verinnerliche also folgendes lieber gleich zehnmal:}

\vspace{0.2cm}

\begin{Abgrenzung}[Pools-Project-Economics haben nichts mit Invest/Economics eines einzelnen Pools (als Dienstleistung des Pools-Projects) zu tun.]

\vspace{0.2cm}

Die Dienstleistung unseres Pools-Projects hat im Sinne des \textit{Social-Investings} unausweichlich mit Geld zu tun. Die \textbf{Pools-Project-Economic} haben dies konsequenterweise ebenfalls.

\vspace{0.1cm}

\textbf{Dabei steht ausschließlich zweites im Fokus des gegenständigen Kapitels. Erstes dagegen bestenfalls beiläufig als Referenzgrundlage bis gar nicht.} Die User der Pools hantieren mit Geld, indem sie den Service nutzen. Projekt-Stakeholder verdienen idealerweise an der angebotenen Dienstleistung - wie sie es auch täten, falls die Dienstleistung keinerlei finanziellen Bezug hätte.

\vspace{0.75cm}

Wir wollen hier einige \textit{Fallstricke} für offensichtliche Missverständnisse und Verwechselungsgefahren ganz konkret beim Namen nennen:

\begin{itemize}
	\item Die Pools (als genutzte Dienstleistung) verfügen über Funds und Assets. Beides werden in aller Regel Tokens sein. Die Funds - als \textit{Fiat-Äquivalent} - vermutlich (aber auch nicht zwingend) mittels eines \textit{Stable-Coins} repräsentiert. Die Assets erst einmal nicht weiter spezifiziert. 
	
	\textbf{Diese finanziellen Mittel eines Pools stellen bestenfalls eine Referenzgrundlage zu anfallenden Service-Fees dar, sind kein direkter Bestandteil der Pools-Economics und verwenden ganz besonders NICHT den Pool-Project-Token als Basis-/Funding-Währung.}
	\item Die Monetarisierung des Pools-Service wird anhand von (prozentualen) Service-Fees erfolgen, die als Berechnungsgrundlage durchaus das Kapital des je\-weiligen Pools heranziehen kann und wird. 
	
	Konkret werden diese Fees in einer dafür definierten Währung anfallen, die ein \textit{Stable-Coin} UND/ODER der Pool-Project-Token sein kann. Die \textit{Monetarisier\-ungs-Währung} ist dabei zentraler Bestandteil der \textit{Pool-Economics}, die Währung der Pool-Funds eines Pools ist es dagegen absolut nicht und daher auch nicht maßgebend für die Fees-Abrechnung. Bei etwaigen Währungs-Diskrepanzen muss unter Umständen ein Umrechnungs- und Ad-Hoc-Umtausch-Mechanismus implementiert werden
	\item Ein in den folgenden Kapiteln definierter \textit{Token-\textbf{Staking}-Mechanismus} wird den Pool-Project-Token als Währung vorsehen und \textbf{hat dabei absolut nichts mit dem/den Pool-Kapital/-Funding/-Assets zu tun.}
	\item Jeder Pool wird eine \textbf{\textit{Pool-Treasury}} besitzen, die die Pool-Funds und die Pool-Assets verwaltet. Unser Pool-Project-Token wird gleichzeitig einem Mo\-dell folgen, bei dem eine sogenannte \textbf{\textit{Token-Contract-Treasury}} von großer Bedeutung sein wird, die wir künftig wahlweise auch als \textbf{\textit{Pools-Project-Treasury}}, \textbf{\textit{Pools-Token-Treasury}} oder als \textbf{\textit{Project-Token-Treasury}} be\-zeichnen. Ungeachtet der - nicht immer konsistenten Bezeichnung - ist diese dringend von der erstgenannten Treasury eines einzelnen Pools zu unterscheiden.
\end{itemize}

\end{Abgrenzung}

\vspace{0.5cm}



\subsubsection{WPT - Die grundlegende Idee eines Pools-Project-Utility-Tokens}
\vspace{0.2cm}

\paragraph{Monetarisierung \& Tokenisierung}
\textbf{ }
\vspace{0.3cm}

Der abstrakt gehaltenen Einleitung zum finanziellen Grundgerüst unseres Pool-Projekts wollen wir in diesem Abschnitt nun den konzeptuell gedanklichen Grundstein zur dessen tatsächlichen Economics-Realisierung legen, auf dem dann im Anschluss die folgenden Kapitel aufbauen.

\vspace{0.2cm}

Dazu folgen zunächst einige - mehr oder minder erklärungsbedürftige - rohe Aussagen: 

\vspace{0.2cm} 

\begin{Praemisse}[Monetarisierung]
\label{monetarisierung}
\vspace{0.2cm}

Die Monetarisierung unseres Pool-Service soll auf Basis (prozentualer) Fees (siehe \nameref{sec:fees}) - gemessen am (finanziellen) Volumen der erbrachten Dienst\-leistung - erfolgen. Für den Moment sehr plakativ betrachtet, ist dies gleichbedeutend mit: 

\vspace{0.2cm} 

\textbf{\textit{Mit je mehr Kohle die Pools hantieren, desto größer sollen die anfallenden Fees sein!}}

\end{Praemisse}

\vspace{0.5cm}

\begin{Praemisse}[Utility-Token als Monetarisierungs-Tool für alle Stakeholder]
\label{fees-for-token}
\vspace{0.2cm}

\textbf{Die Fees sollen mittels eines dafür geschaffenen Utility-Tokens abgerechnet, erhoben und erbracht werden!}

\vspace{0.5cm} 

Für den - unbestreitbar verkomplizierenden und technisch teils nicht unerheblich umständlichen - Umweg der Monetarisierung über einen Token sehen wir folgende schlagende Argumente, die auch in den anschließend folgenden Kapiteln immer mal wieder argumentativ zum Vorschein kommen werden:

\begin{itemize}
	\item Die Nutzung des Pools-Service kann als ein echtes \textit{\textbf{Gut}} - eine \textit{Utility} - angesehen werden, was unter Umständen nicht endlos verfügbar sei (begrenzte Skalierung auf der Blockchain), besonders begehrt (bei exzellenter Service-Qualität) oder im Übermaß vorhanden (bei anfänglicher Unbekanntheit des Service) sei. 
	
	Durch die Tokenisierung der Dienstleistung einverleibt man dieser den Stellenwert einer \textit{Ressource}, mit zugehörigen Eigenschaften wie \textbf{Verfügbarkeit}, \textbf{Qualität} und \textbf{Nachhaltigkeitsgedanken}, was bei digitalen Dienstleistungen oft unberücksichtigt bleibt. 
	
	Mit diesem Ansatz kommt das \textit{Marktwirtschaftsprinzip von Angebot \& Nachfrage} auch bei digitalen Services zum Tragen, was in der digitalen Welt heutzutage ausschließlich auf \textit{Nachfrage} reduziert wurde, da das \textit{Angebot} de facto als unendlich betrachtet wird.
	\item Die Tokenisierung eines Business-Modells eröffnet einem das sehr mächtige spieltheoretische Werkzeug des \href{https://de.wikipedia.org/wiki/Mechanismus-Design-Theorie}{Mechanismus-Design}, um sämtliche Projekt-Beteiligte bzw. -Stakeholder in ihrem Verhalten hinsichtlich des übergeordneten Projekterfolgs zu beeinflussen/incentivieren. Oder simple ausgedrückt: Das zu tun, was wir aus strategischen Überlegungen möchten, dass er/sie tut.
	\item \textbf{Direkte \& unbürokratische Projekt-Finanzierung}.
	
	Durch die Tokenisierung der Dienstleistung muss ein potenzieller Investor beim Kauf von Utility-Tokens lediglich vom Erfolg der Dientleistung=Utility selbst überzeugt sein (da eine Nachfrage nach der Dienstleistung direkt an die Nachfrage nach dem zugehörigen Utility-Token gekoppelt ist), anstatt bei seiner ROI-Evaluierung herkömmliche bürokratisch geregelte Venture-Capital-Aspekte wie etwaige Shareholders-Agreements und Exit-Szenarien hinzuziehen zu müssen.
	\item Technische und konzeptuelle Vereinfachung, Flexibilität und Direktheit bei \textit{Customer-Akquise} und \textit{CRM} mittels des Utility-Tokens, da
	\begin{itemize}
		\item die \textit{Marketing-Währung} in Form von Tokens die \textbf{Utility} selbst statt \textit{Fiat} in den Vordergrund rückt.
		\item Der \textit{Project-Owner} (in dem Fall also WunderPass) in aller Regel selbst ein großer Token-Holder sein wird und somit über die Mittel verfügt, das Marketing-Volumen zu erbringen (ohne dabei zusätzlich finanziell belastet zu werden).
	\end{itemize}
	\item Uneingeschränkte Transparenz für alle Projekt-Teilnehmer über Stake, Cash-Flows, Handlungen, Strategien etc. aller anderen Projekt-Teilnehmer und damit ihrer Position und Interessen innerhalb des Projekts mittels jederzeit offen einsehbarer dezentraler Smart-Contract-Logik.
	\item Uneingeschränkte Transparenz und Eliminierung von Interpretationsspielraum hinsichtlich des Business-Plans.
	\item Zu guter Letzt sei noch das - weniger auf harten Fakten als auf dem \textit{Opportunitiy-Gedanken} begründete - Argument des vermeintlichen \textit{Tokenisierungs-Trends} zu nennen, welches ein rein selbstzweck-getriebenes Interesse bei potenziellen Token-Investoren wecken könnte.
\end{itemize}

\end{Praemisse}

\vspace{0.5cm}

\paragraph{Die entscheidende Idee}
\textbf{ }
\vspace{0.3cm}

Allen relevanten Erklärungen vorweggreifend folgt unser fundamentale \\
\textit{Token-Economics}-Ansatz für die Pools-Project-Token:

\vspace{0.2cm}

\begin{Konzept}[Dividende auf den Pools-Project-Token]
\label{token-usp}
\vspace{0.2cm}

Zusätzlich zur \textit{Utility}-Beschaffenheit unseres Pools-Project-Tokens möchten \\
wir diesem noch eine gewisse \textit{Equity}-Eigenschaft einverleiben:

\vspace{0.2cm}

\textbf{Ein Token-Besitzer soll mittels des Tokens nicht nur den Pools-Service nutzen können oder an der steigenden Nachfrage nach diesem - durch eine positive Kursentwicklung - profitieren, sondern zusätzlich DIREKT an den generierten Erträgen des gesamten Pools-Projects beteiligt werden.}

\vspace{0.2cm}

Er soll demnach de facto als Anteilseigner des Pools-Projects gelten und an etwaigen Gewinnen des Projekts - in Form einer gewissen \textit{Dividende} - pro rata seines Token-Volumens partizipieren.

\vspace{0.2cm}

Die Implementierung dieses \textit{Equity}-Mechanismus soll selbstverständlich mittels eines Smart-Contracts sichergestellt sein, was unseren Token stark von anderen \\ \textit{Equity}-Tokens abhebt.

\vspace{0.2cm}

Durch diesen zusätzlichen Kniff, schaffen wir eine sich selbst verstärkende Synergie zwischen den \textit{Utility}- und \textit{Equity}-Eigenschaften unseres Pools-Project-Tokens, indem wir einen potenziellen User des Pools-Service (besitzt \textit{Utility} in Form des Tokens) gleichzeitig zu einem Projekt-Investor machen (besitzt \textit{Equity} in Form desselben Tokens). Dieser doppelte Synergieeffekt weitet sich auch unmittelbar auf die Kursentwicklung aus. DENN: Wachsende Nutzung des Pools-Services impliziert zwangsläufig eine steigende Token-Zirkulation (im Sinne der \textit{Utility}-Beschaffenheit) und steigenden Bedarf und somit Nachfrage nach dem Token UND generiert gleich\-zeitig zunehmenden Ertrag durch Service-Fees, was wiederum eine Wertsteigerung des Tokens aus seiner \textit{Equity}-Beschaffenheit nach sich zieht.

\end{Konzept}

\vspace{0.3cm}

Wie genau wir uns das eben formulierte Vorhaben in der Umsetzung planen, wird etwas weiter unten vertieft. Zunächst bleiben wir beim ökonomischen Teil des Token-Designs und erarbeiten einige relevante Mechanismen.

\vspace{0.5cm}


\paragraph{Token-Design}
\textbf{ }
\vspace{0.3cm}

Beim Design unseres Pools-Project-Tokens wollen wir uns stark an den Gedanken des spieltheoretischen Gebiets des \href{https://de.wikipedia.org/wiki/Mechanismus-Design-Theorie}{Mechanismus-Design} orientieren.

Dieses Wissenschaftsgebiet befasst sich im Wesentlichen damit als \textit{höhere Instanz eines Spiels} - also in dem Fall wir als Project-Owner - mittels Regelgestaltung und Incentivierungs-Mechanismen - also in unserem Fall mittels Token-Design - Einfluss auf das Verhalten der Spieler - also in dem Fall Nutzer des Pool-Service und Investoren - im Sinne des Spiels nehmen kann.

\vspace{0.1cm}

Entscheidend hinsichtlich letzter Formulierung ist dabei das \textit{"... im Sinne des Spiels..."} genaust möglich zu präzisieren und idealerweise zu quantifizieren und formalisieren.

\vspace{0.5cm}

\textbf{Was möchten wir also genau wie, wann und womit erreichen für unser Pools-Projekt?}

\vspace{0.5cm}

Dabei bewegen sich die \href{https://de.wikipedia.org/wiki/Mechanismus-Design-Theorie}{Mechanismus-Design}-Werkzeuge tendenziell auf einer granularen Ebene, weshalb die Antwort \textit{"Pools-Project to the moon!"} auf obige Frage nicht in deren Sinne stünde. Viel mehr ist obige Frage daher als

\begin{itemize}
	\item Welche Etappenziele möchten wir erreichen (Projekt-Funding, Wachstum, Exit etc.)?
	\item Welche Projekt-Stakeholder (Gründer, Project-Owner, Investoren, User etc.) werden gebraucht und wie können diese gewonnen und deren Interessen gewahrt werden?
	\item Welche Hebel und designte Einflussmöglichkeiten möchten wir mittels von Token-Mechanismen besonders stark in eigener Hand behalten, anstatt sie dem Zufall oder Markt-Gesetzen zu überlassen?
	\item Welche Synergien möchten wir schaffen/verstärken bzw. verhindern/bremsen?
	\item Letzeres ist nicht nur aus Sicht des Pools-Projekts für sich alleinstehend zu betrachten sondern insbesondere auch im Hinblick auf ein etwaiges künftiges Wunder-Ökosystem. 
	\item Wie können wir als Gründer/Project-Owner (finanziell) profitieren?
\end{itemize}

zu verstehen. Um das ganze nicht ausufern zu lassen, wollen wir diese Fragestellungen stark auf das Pools-Projekt, seinen Projekt-Token und insbesondere dessen erhofften Effekte fokussieren:

\vspace{0.3cm}

\begin{Assumption}[Erwünschte Effekte des Pools-Project-Tokens]
\label{token-anforderungen}
\vspace{0.2cm}

Folgende Anforderungen, Erwartungen und Absichten verfolgen wir mit dem zu designenden Projekt-Token und/oder beabsichtigen zu erfüllen:

\begin{itemize}
	\item Selbstverständlich stellt ein gewisses initiales Projekt-Funding mittels Token-Sale eine der ausschlaggebendsten Motivationen für den Token dar, um z. B. auch Entwicklungskosten zu decken. 
	\item Gleichzeitig müssen aber eben die initialen Kapitalgeber angemessen für ihr Risiko entlohnt werden und signifikant stärker an ihrem Token-Invest profitieren als spätere Token-Käufer.
	\item Nicht verkehrt wäre gleiches für die Gründer ;)
	\item Nicht nur für die zuletzt genannten early Investors sondern generell für alle Token-Investoren möchten wir einen transparenten, berechenbaren und vertrauenswürdigen Token schaffen, 
	\begin{itemize}
		\item dessen Kursentwicklung keiner künstlichen PR-getriebenen Hysterie mit anschließendem Crash unterliegt (\textit{Pump \& Dump}),
		\item dessen Value transparenten und idealerweise durch Smart-Contracts ge\-steuerten Mechanismen und Projekt-Entwicklungen folgt,
		\item dessen Value einen \textit{Utility-}Bezug hat und
		\item der idealerweise mittels eines AMMs (\textit{Automated Market Maker}) jederzeit handelbar sein soll.
	\end{itemize}
	\item Nicht ganz so essenziell wie das initiale Projekt-Funding jedoch ebenfalls nicht zu vernachlässigen ist die fortlaufende (operative) Projekt-Finanzierung, die gänzlich oder zumindest teilweise durch den Projekt-Token mitfinanziert werden könnte.
	\item Gleichwohl der oben skizzierte USP unseres Tokens (siehe \ref{token-usp}) \textit{Equity}-techni\-scher Natur ist, ist und bleibt unserer Pools-Project-Token substanziell ein \textbf{\textit{Utility-Token}}.
	\begin{itemize}
		\item Grundsätzlich wird die Zirkulation eines \textit{Utility-Tokens} stets stark korreliert mit der Nutzung/Nachfrage der Utility - also in unserem Fall dem Pools-Service - sein. Wie solch eine Korrelation konkret aussieht, haben wir mittels des Token-Designs maßgeblich in eigener Hand. So kann man mit Mitteln wie z. B. \textit{Staking} oder \textit{Locking} die Zirkulation künstlich verlangsamen bzw. eine künstliche Verknappung an zirkulierenden Tokens induzieren.
		\item In gewisser Überzeugung, ein echter \textit{Utility-Token} repräsentiere eine nur endlich verfügbare Ressource, streben wir einen deflationären Token an. Oder zumindest einen \textit{pseudo-deflationären} (also einen, der zwar theoretisch unendlich lange weitergemintet werden kann, dies jedoch ab einem bestimmten Moment absolut unwirtschaftlich wird).
	\end{itemize}
\end{itemize}

\end{Assumption}

\vspace{0.5cm}

An der Abarbeitung dieser Liste werden wir uns - nicht zwingend die Reihenfolge wahrend - durch das restliche Kapitel hangeln. Bevor wir uns gleich im Anschluss etwas detaillierter dem letzten Punkt der obigen Liste - nämlich der Einflussnahme auf die Token-Zirkulation - widmen, zunächst ein sich sofort ersichtlicher \textit{Quick-Win} hinsichtlich Bullet 4 der obigen Liste:

\vspace{0.3cm}

\begin{Konzept}[Das \textit{Bonding-Curves-Modell} als vielversprechendes Mittel für unseren Pool-Project-Token]
\label{bcm}
\vspace{0.2cm}

Der Wunsch nach einem \textbf{transparenten, berechenbaren und vertrauenswürdig\-en Token} aus der Anforderungsliste \ref{token-anforderungen} suggeriert, das \textit{Bonding-Curves-Modell} als Grundlage zur Modellierung unseres Pool-Project-Token in Betracht zu ziehen, da der \textit{Bonding-Curves-Ansatz}

\begin{itemize}
	\item mittels Einsatzes eines Smart-Contract-AMMs, \textbf{Transparenz und Berechenbarkeit} des Tokens garantiert,
	\item durch im Token-Contract vorgehaltene \textbf{Kapital-Deckung pro ausgegebenem Token} das Investrisiko deckelt und damit die gewünschte \textbf{Vertrauenswürdig\-keit} abbildet und
	\item letztendlich durch seinen Basis-Mechanismus zwingend einen in seiner Logik verankerten AMM mitliefert.
\end{itemize}

\vspace{0.3cm}

\todo{1-2 allgemeine fortführende Links zu \textit{Bonding-Curves} referenzieren.}

\vspace{0.5cm}

Tatsächlich werden wir den \textit{Bonding-Curves-Ansatz} für unseren Pool-Project-Token später wieder aufgreifen und uns seiner Anwendung - den Gedanken aus dem Anhang zu \nameref{sec:bonding-curves} folgend - bemühen.

\vspace{0.3cm}

Der Vorgriff darauf erfolgte an dieser Stelle lediglich aufgrund des direkten Kontext-Bezugs zu Bullet 4 aus Anforderungsliste \ref{token-anforderungen}.

\end{Konzept}

%\newpage
\vspace{0.5cm}


Nun kommen wir - wie bereits angekündigt zu Mechanismen der \textbf{Token-Zirkulation}:

\vspace{0.3cm}


\begin{Konzept}[Token-Zirkulation-Mechanismen]
\label{circulation}
\vspace{0.2cm}

Eines sofort vorweg:

\vspace{0.2cm}
\todo{\noindent\hrulefill}

\todo{Die gleich vorgestellten Gedanken und Konzepte sind als noch nicht sehr ausgereifte initiale Ideen und Entwurfsmuster zu verstehen, die es noch zu erforschen und besser zu verstehen gilt. Mögen diese vielleicht in ihren grundlegenden Ansätzen noch so fundiert und durchdacht sein, wäre ein Anspruch ihrer perfekten Ausformulierung in einem - nicht auf fundierten praktischen Produkt-Erfahrung aufbauenden - White-Paper - wie es dieses aktuell ist - nur anmaßend und eine Vortäuschung einer pseudo-fundierten Theorie, die es aber ohne praktische Erprobung nicht ist.}

\vspace{0.2cm}

\todo{Vielmehr gilt es, die folgenden Ideen und Ansätze in ihrem Grundsatz zu verinnerlichen, und dabei gleichzeitig, die etwaigen Konkretisierungen mit Augenmaß \textit{weich} zu deuten, um diese mit zunehmender praktischer Anwendung zu validieren, zu justieren oder zu verwerfen.}

\vspace{0.2cm}

\todo{Dieser Teil des White-Papers ist also mit voller Absicht bis auf weiteres als \textbf{WIP} anzusehen und soll hier als solches markiert sein.}

\todo{\noindent\hrulefill}
\vspace{0.5cm}


\textbf{Die folgenden Ausführungen betrachten den anvisierten Pool-Project-Token in seinem Dasein als \textit{Utility-Token}.}

\vspace{0.2cm}

Es bedarf wahrscheinlich keiner weiteren Erklärung, wir verfolgten im Großen und Ganzen einen sich \textbf{positiv entwickelnden Token-Kurs} und richteten unsere \textit{Mechanism-Design}-Überlegungen genau diesem Ziel folgend aus.

\vspace{0.2cm}

Den Markt-Gesetzen folgend geht ein steigender Kurs mit \textbf{steigender Nachfrage und/oder knapper werdendem Angebot} der durch den Token repräsentierten \textit{Utility} einher.

\vspace{0.2cm}

Da wir die \textit{Utility} unseres Pool-Project-Tokens als Zahlungsmittel für die anfallenden Service-Fees des Pools-Service definiert haben, stellt uns die Gegenüberstellung der gewünschten \textbf{Kurssteigerung des Tokens} vs. des \textbf{Angebot-Nachfrage-Prinzips} vor ein nicht unerhebliches Problem:

\vspace{0.4cm}

\textbf{Die Nutzung des Pools-Service erfolgt über einen gewissen (längeren) Zeitraum. Die Entrichtung der Fees geschieht dagegen in einem einzigen Moment, was die Nachfrage nach dem Pools-Service von der Nachfrage nach dem zugehörigen \textit{Utility-Token} nahezu gänzlich voneinander ent\-koppelt - wenn nicht gar das gesamte Verständnis von einer Nachfrage nach dem \textit{Utility-Token} in sich zusammenfallen lässt.} 

\vspace{0.4cm}

Um genau diesem Problem entgegenzuwirken und die \textit{Utility} - die wir unverändert bei der Service-Fee-Abrechnung belassen wollen - zeitlich auf die übergeordnete Pools-Nutzung-Dienstleistung auszudehnen und dabei gleichzeitig eine \textbf{künstliche Verknappung} der zirkulierenden Pool-Project-Tokens zu induzieren, bedienen wir uns zweier entscheidender Design-Mechanismen:

\vspace{0.5cm}

\underline{\textbf{Pending-Fees}}

\vspace{0.3cm}
\todo{WIP}

\vspace{0.3cm}

\todo{Gebühren-Zahler:} künstlichen Token-Bedarf über längeren Zeitraum mittels \textit{Pending-Fees} forcieren


\vspace{0.75cm}

\underline{\textbf{Staking}}

\vspace{0.3cm}

\vspace{0.3cm}
\todo{WIP}

\vspace{0.3cm}


\begin{itemize}
	\item Der Pool-Initiator müsste bei der Pool-Eröffnung WPT staken, die er unter bestimmten Umständen verlieren könnte, wenn sein Pool zB. ungenutzt bleibt. So könnte man sicherstellen, dass ernste Absichten hinter den Pools stecken und diese auch genutzt werden. 	
	\item Dazu müssen Tokens gekauft und über einen längeren Zeitraum gehalten werden.
	\item Dies reduziert den Token-Umlauf und erhöht damit den Kurs.
	\item Der Staker erhält einen Anteil an den veranschlagten Fees.
	\item Der Staker ist damit nicht nur User des \textit{Pool-Service} sondern gleichzeitig auch ein Investor (Token-Holder) in das übergeordnete Pools-Project (da er gezwungen ist, die Tokens über einen längeren Zeitraum zu halten)
	\item Der Staker ist damit mehrfach incentiviert durch 
	\begin{itemize}
		\item Nutzung der Dienstleistung als solches
		\item Erstellung des/mehreren eigenen Pools aufgrund der Fees-Gewinn-Beteiligung als Staker
		\item Erstellung vieler Pools allgemein aufgrund der Fees-Gewinn-Beteiligung als Token-Holder \textit{(word-of-mouth)}
	\item In jedem Fall sollte der Staker im Normalfall (falls er nicht irgendwie Scheiße baut) bei der Auflösung des Pools mindestens seinen Einsatz zurückerhalten (also keinerlei Gebühren für die Nutzung des Pool-Service zahlen). In aller Regel sollte er mit mehr als dem ursprünglich gestakten Betrag rausgehen.
	\end{itemize}
\end{itemize}


\end{Konzept}
\vspace{0.5cm}




\paragraph{Umsetzung}
\textbf{ }
\vspace{0.3cm}

Gleichwohl noch nicht richtig quantifizierbar, jedoch konzeptuell bereits solide Formen annehmend, wollen wir an dieser Stelle endlich unseren angestrebten \textit{Pool-Project-Token} einführen und fortan an einem konkreten anstatt wie bisher abstrakt gehaltenem Gebilde weiterarbeiten:

\vspace{0.3cm}

\begin{Solution}[WunderPool-Token (\textit{WPT})]
\label{wpt}
\vspace{0.2cm}

Folgenden bisher erarbeiteten wesentlichen Ergebnissen folgend definieren wir den WunderPool-Token (\textbf{WPT}) als den anvisierten \textit{Pool-Project-Token}:

\begin{itemize}
	\item Die Monetarisierung des Pools-Projekt erfolgt durch Service-Fees, die Mittels des \textit{Utility-Tokens} \textbf{WPT} veranschlagt und abgerechnet werden (Prämissen \ref{monetarisierung} und \ref{fees-for-token}).
	\item Die Venture-Invests der \textbf{WPT}-Käufer werden durch echte Kapital-Rücklagen innerhalb des \textbf{WPT}-Token-Contracts gedeckt (Design-Merkmal \ref{bcm}).
	\item Alle \textbf{WPT}-Holder werden finanziell am etwaigen Projekt-Erfolg beteiligt \\ (Design-Merkmal \ref{token-usp}).
	\item Ein AMM zur \textbf{WPT}-Distribution (und initialem Token-Sale) wird bereitgestellt (Design-Merkmal \ref{bcm}).
	\item Der \textbf{WPT}-Supply und -Kurs wird zwecks Vermeidung von Inflation des \textit{Utility-Tokens} in gewissem Rahmen kontrolliert (Design-Merkmal \ref{bcm}).
	\item Bei gegebenem Demand nach dem Pools-Service wird die aktuelle Zirkulation des \textbf{WPT-Utility-Tokens} künstlich aufrecht erhalten und der Anteil der verfügbaren an sich in Zirkulation befindenden \textbf{WPT} künstlich verknappt (Design-Merkmal \ref{circulation}).
\end{itemize}

\vspace{0.5cm}

Zur Motivation des Einsatzes von \textbf{Bonding-Curves} beim hier besonders prägenden Design-Merkmal \ref{bcm} sei zur allgemeinen Einführung auf den Artikel \todo{Link zu einem guten Artikel}, zur Inspiration auf den Artikel \href{https://medium.com/atchai/can-we-save-the-utility-token-55ef639370cf}{Utility-Token als Bunding-Curves-Modell} und hinsichtlich Umsetzung auf unseren eigenen Content aus dem Anhang zu \nameref{sec:bonding-curves} verwiesen.

\vspace{0.3cm}
\todo{TODO: Welcher der obigen Punkte wird in welchen der folgenden Kapitel en Detail aufgegriffen und weiter vertiegt, um die noch fehlende Quantifizierung darzustellen?}

\end{Solution}

\vspace{0.5cm}

\todo{WIP}

\begin{itemize}
	\item Bonding-Curves
	\begin{itemize}
		\item Warum?
		\item Modellierung mit den Denkansätzen aus dem Anhang zu \nameref{sec:bonding-curves} unter Einbeziehung der obigen Gewinnbeteiligung an Fees
	\end{itemize}	
\end{itemize}

\vspace{0.5cm}



\paragraph{Ausblick}
\textbf{ }
\vspace{0.3cm}

\todo{WIP}

\begin{itemize}
	\item Der erste und größere Investor für das Pool-Projekt wäre WunderPass selbst. Für die erfolgte Einlage in den Projekt-Pool bekäme WunderPass WPT, die es für Incentivierungen und Rewards für die Nutzung von Pools verwenden könnte. Dieses Invest könnte (im Gegensatz zu den Einlagen anderer Investoren) zB. auch einem Locking unterliegen, um eine gewisse Preisstabilität des WPT zu gewährleisten.	
	\item Erste Andeutung, dass die Token-Contract-Treasury zwar im ersten Schritt in \textit{USDT} modelliert, jedoch in \textit{WUNDER} geplant ist.
	\item Ausblick auf die anschließenden Kapitel, die das präsentierte Konzept umsetzen.
	\item Erstmals auf \textit{Economics-Excel} verweisen.
\end{itemize}


\vspace{0.5cm}



\subsubsection{Gebühren-Ordnung}
\label{sec:fees}
\vspace{0.2cm}

\paragraph{Gebühren-Modell}
\textbf{ }
\vspace{0.2cm}

\begin{Assumption}[Gebühren]\label{fees}

Es sollen in etwa folgende \textit{Basic-Fees} anfallen:

\begin{itemize}
	\item Grundgebühr von 1.9 \% auf den Deposit (für jeden Pool-Teilnehmer außer des Pool-Creators).
	\item Tradinggebühr von 0.1 \% auf jede Kauf- oder Verkaufsorder.
	\item Gewinnprovision von 9.9 \% auf einen durch den Pool erwirtschafteten \textbf{positiven} EBIT (bei Liquidierung des Pools).
\end{itemize}

\vspace{0.2cm}

Ergänzt werde diese durch etwaige \textit{Service-Fees}: 

\begin{itemize}
	\item Erweiterte Grundgebühr von zusätzlichen 1.5 \% auf den Deposit bei einem späteren Pool-Beitritt (additiv zu der obigen Basis-Grundgebühr).
	\item \textit{Leaving-Gebühr} von 6.9 \% auf den Cashout-Betrag bei vorzeitigem Verlassen des Pools und Cashout seitens eines Pool-Teilnehmers, falls der Cashout über den Pool-Contract erfolgt (und nicht z.B. mittels Verkaufs der Shares an einen anderen Pool-Teilnehmer oder am Sekundär-Markt).
\end{itemize}

\vspace{0.2cm}

Zudem sind folgende \textit{Benefits} hinsichtlich der Gebührenordnung für Inhaber eines Pass-NFTs \todo{(Kapitel verlinken)} vorgesehen:

\begin{itemize}
	\item Wegfall der Deposit-Grundgebühr für Inhaber eines PassNFTs des Status \textit{Diamond} und \textit{Black}.
	\item Reduzierung sämtlicher Gebühren, die auf User- und nicht Pool-Basis anfallen um
	\begin{itemize}
		\item 50 \% für Teilnehmer mit PassNFT-Status \textit{Diamond},
		\item 30 \% für Teilnehmer mit PassNFT-Status \textit{Black},
		\item 20 \% für Teilnehmer mit PassNFT-Status \textit{Pearl},
		\item 10 \% für Teilnehmer mit PassNFT-Status \textit{Platin}.
	\end{itemize}
\end{itemize}

\vspace{0.5cm}	

Aktuell nicht berücksichtigt jedoch grundsätzlich spannend sind die folgenden Gebühren-Aspekte und -Varianten:

\begin{itemize}
	\item Eine mögliche \textit{Trial-vs-Pro-Gebührenordnung}, bei der (stark) limitierte Pools (sowohl finanziell als auch feature-technisch) gänzlich kostenlos bleiben könnten, während eine unlimitierte Nutzung mit höheren Gebühren als den obigen einhergehen würde.
	\item \textit{Managed-Pools}: Pools, die von einem erprobten und erfolgreichen Pool-Creator hinsichtlich der Invests gesteuert, könnten eine höhere Teilnahme-Gebühr erfordern, an der auch der Creator maßgeblich beteiligt wird. 
\end{itemize}	

\end{Assumption}

\vspace{0.5cm}



\paragraph{Gebühren-Abrechnung}
\textbf{ }
\vspace{0.2cm}




\todo{WIP}

\begin{itemize}
	\item Klarstellung und Erklärung, dass die Gebühren zunächst in Fiat berechnet, jedoch am Ende in WPT veranschlagt werden.
	\item Erste Andeutung, dass die Token-Contract-Treasury zwar im ersten Schritt in \textit{USDT} modelliert, jedoch in \textit{WUNDER} geplant ist.
	\item Cash-Flows hinsichtlich der Fees:
	\begin{itemize}
		\item Wann fallen die Gebühren an? 
		\item Wann und wie werden diese in WPT transferiert? 
		\item Wann werden diese ausgezahlt? $\rightarrow$ \textit{Pending-Fees}
	\end{itemize}
\end{itemize}

\vspace{0.5cm}

Abgerechnet werden die auf den Pool anfallenden Fees (selbst die ausschließlich User-basierten) aufgrund von \textit{Mechanism-Design}-Überlegungen erst bei seiner Liquidierung.

\vspace{0.3cm}

\begin{Praemisse}[Abrechnung]

Sämtliche für einen Pool angefallenen Fees werden (ungeachtet ihres Fälligkeitszeit\-punkts) fließen erst bei seiner Liquidierung und werden zwischen Fälligkeit und Entrichtung in einem gesonderten Teil der \textit{Pool-Treasury} vorgehalten (ähnlich dessen, wo der gestakte Betrag des Pool-Creators verwahrt wird).

\vspace{0.2cm}

Wir werden diese finanziellen Mittel im weiteren Verlauf auch als \textbf{\textit{Pending-Fees}} bezeichnen.

\vspace{0.2cm}

In Analogie dazu werden wir an geeigneter Stelle folgend auch von \textbf{\textit{Staked-Fees}} sprechen - gleichwohl es sich dabei eher um eine Sicherheit als um tatsächliche Fees handelt.

\end{Praemisse}

\vspace{0.5cm}

\todo{Ende WIP}

\vspace{0.5cm}


\subsubsection{Business-Plan}
\vspace{0.2cm}

\todo{WIP}

\begin{itemize} 
	\item Business-Case (aus Investoren-Sicht) vorrechnen
	\begin{itemize}
		\item Wirtschaftlichkeit und Preisentwicklung
		\item (praktische) Obergrenze des eingebrachten Gesamtkapitals annehmen, mit der eine plausible und attraktive Rendite argumentiert werden kann.
	\end{itemize}
	\item Excel verlinken
\end{itemize}

\vspace{0.6cm}

Der \nameref{sec:fees} folgen einige Annahmen hinsichtlich des \textbf{Business-Plans} für eine Größenordnung von zwölf Monaten.

\vspace{0.3cm}

\begin{Assumption}[Business-Plan]\label{bp}

\vspace{0.75cm}

\todo{TODO: Zahlen an Excel anpassen}

\vspace{0.75cm}

Zunächst schätzen wir einige KPI ab, die es natürlich zu validieren gilt:

\begin{itemize}
	\item Wir gehen im Mittel von ca. 4-5 Teilnehmern je Pool aus.
	\item Wir gehen von einem durchschnittlichen Deposit von 200\$ je Teilnehmer und Pool aus - also einem durchschnittlichen initialen Pool-Kapital von 800-1000\$.
	\item Wir gehen des Weiteren von einer durchschnittlichen \textit{Pool-Lifetime} von ca. 6 Monaten aus,
	\item schätzen die durchschnittliche Anzahl an Tradings während der Pool-Lifetime auf 10-15,
	\item deren Trading-Volumen auf etwa $\frac{1}{3}$ des initialen Pool-Kapitals und schließlich 
	\item und einen daraus resultierenden konservativen mittleren Profit von 2.5 \% (auf die Pool-Lifetime von 6 Monaten also 5-6 \% p.a.).
	\item Zuletzt schätzen wir, jeder User betreibe im Mittel 2-3 Pools gleichzeitig.
\end{itemize}

\vspace{0.5cm}

Diesen geschätzten KPI zugrundeliegend setzen wir uns folgende Ziele hinsichtlich initiierter (gebührenpflichtiger) Pools - ungeachtet dessen, ob diese zu dem gegebenen Zeitpunkt noch existieren oder bereits liquidiert wurden:

\begin{itemize}
	\item 50 initiierte Pools nach 3 Monaten
	\item 150 initiierte Pools nach 6 Monaten
	\item 500 initiierte Pools nach 12 Monaten
\end{itemize}

\end{Assumption}

\vspace{0.5cm}

Damit ergeben sich folgende Business-Key-KPI:

\vspace{0.3cm}

\begin{Fazit}[Umsätze \& Forecast]

\vspace{0.75cm}

\todo{TODO: Zahlen an Excel anpassen}

\vspace{0.75cm}

Für einen durchschnittlichen Pool $\mathcal{P}$ mit dem initialen Pool-Kapital

\begin{equation*}
  vol^{\mathcal{P}} = 4.5 \cdot 200\$ = 900\$ 
\end{equation*}

approximieren wir die anfallenden Fees als Summe der Fee-Bestandteile

\begin{itemize}
	\item Grundgebühren: $fees_{G}^{\mathcal{P}} = \rho(nft) \cdot 0.019 \cdot (4.5 - 1) \cdot 200\$ $
	\item Trading-Gebühren: $fees_{T}^{\mathcal{P}} = \phi(nft) \cdot 0.001 \cdot 12.5 \cdot \frac{1}{3} \cdot vol^{\mathcal{P}} $
	\item Profit-Beteiligung: $fees_{P}^{\mathcal{P}} = \phi(nft) \cdot 0.099 \cdot 0.025 \cdot vol^{\mathcal{P}} $
\end{itemize}

wobei $\rho(nft)$ und $\phi(nft)$ Normierungsfaktoren darstellen, die die in Annahme \ref{fees} beschriebenen \textit{Benefits für PassNFT-Besitzer} berücksichtigen sollen, und von uns als 

\begin{itemize}
	\item $\rho(nft) \approx \frac{9}{10}$ und 
	\item $\phi(nft) \approx \frac{7}{8}$
\end{itemize}	

geschätzt werden sollen \todo{(für die Anfangsphase sind diese eher zu klein, im einge\-schwungenen Zustand viel zu groß)}.

\vspace{0.2cm}

Damit belaufen sich die einzelnen Fees-Bestandteile auf 

\begin{itemize}
	\item Grundgebühren: $fees_{G}^{\mathcal{P}} \approx 11.97\$ $
	\item Trading-Gebühren: $fees_{T}^{\mathcal{P}} \approx 3.28 \$ $
	\item Profit-Beteiligung: $fees_{P}^{\mathcal{P}} \approx 1.95 \$ $
\end{itemize}

und damit die im Mittel erwarteten Gesamt-Fees pro Pool auf

\begin{equation*}
  fees^{\mathcal{P}} = fees_{G}^{\mathcal{P}} + fees_{T}^{\mathcal{P}} + fees_{P}^{\mathcal{P}} \approx 17.20 \$. 
\end{equation*}

\vspace{0.67cm}

Bei einer Staking-Anforderung von 200 \% der geschätzten Pool-Fees \todo{(auf Staking verlinken)} und den in \ref{bp} getroffenen Business-Plan-Annahmen ergeben sich folgende näherungsweisen Forecasts:

\begin{itemize}
	\item Nach 3 Monaten: ca. 40 noch aktive und bereits ca. 10 liquidierte Pools.
	\begin{itemize}
		\item bereits \textit{umgesetzte Fees}: 266 \$ 
		\item \textit{Pending-Fees}: 1.064 \$ 
		\item \textit{Staked-Fees}: 2.129 \$ 
	\end{itemize}
	\item Nach 6 Monaten: ca. 100 noch aktive und bereits ca. 50 liquidierte Pools.
	\begin{itemize}
		\item bereits \textit{umgesetzte Fees}: 1.330 \$ 
		\item \textit{Pending-Fees}: 2.661 \$ 
		\item \textit{Staked-Fees}: 5.322 \$ 
	\end{itemize}
	\item Nach 12 Monaten: ca. 300 noch aktive und bereits ca. 200 liquidierte Pools.
	\begin{itemize}
		\item bereits \textit{umgesetzte Fees}: 5.322 \$ 
		\item \textit{Pending-Fees}: 7.983 \$ 
		\item \textit{Staked-Fees}: 15.966 \$ 
	\end{itemize}	 
\end{itemize}

\vspace{0.5cm}

Zu guter Letzt noch eine sehr bullishe Prognose:

\begin{itemize}
	\item Nach 5 Jahren: 450.000 noch aktive und bereits 550.000 liquidierte Pools.
	\begin{itemize}
		\item bereits \textit{umgesetzte Fees}: $\approx$ 15 Mio. \$ 
		\item \textit{Pending-Fees}: $\approx$ 12 Mio. \$ 
		\item \textit{Staked-Fees}: $\approx$ 24 Mio. \$ 
	\end{itemize}	 
\end{itemize}

\end{Fazit}

\vspace{0.5cm}


\todo{Ende WIP}


\vspace{0.5cm}

\subsubsection{Herleitung Bonding-Curves}
\vspace{0.2cm}

\todo{Andere Kapitel müssen vorgezogen werden. Im Ergebnis leiten sich aber folgende Kurven für den Token ab:}

\vspace{0.3cm}

\begin{Solution}[Token-Curves]

Sei $s \in \mathbb{N}$ der Token-Supply und 

\begin{equation*}
p \approx 1.06
\end{equation*}

der \textit{Profit-Koeffizient} \todo{(erklären weshalb, wofür, warum)}.
Dann leiten sich Verkaufs- und Kaufpreis-Kurve wie folgt ab:

\vspace{0.2cm}

Verkaufspreis-Kurve:

\begin{equation*}
V(s) = V_{0} \cdot p^{ln\left(2 \cdot \frac{s}{s_{o}}\right) \cdot ln\left(\frac{s}{s_{o}}\right)}
\end{equation*}

\vspace{0.2cm}

Kaufpreis-Kurve:

\begin{align*}
K(s) &= K_{0} \cdot p^{ln\left(2 \cdot \frac{s}{s_{o}}\right) \cdot ln\left(\frac{s}{s_{o}}\right)} \cdot \left( ln(p) \cdot \left( 2 \cdot ln\left( \frac{s}{s_{o}} \right) + ln(2) \right) + 1 \right) \\
 &= V(s) \cdot \left( ln(p) \cdot \left( 2 \cdot ln\left( \frac{s}{s_{o}} \right) + ln(2) \right) + 1 \right)
\end{align*}

\vspace{0.4cm}

wobei $s_{0}$ für einen sehr kleinen (initialen) Supply und 

\begin{equation*}
K_{0} = K(s_{0}) = V(s_{0}) = V_{0}
\end{equation*}

für seinen initialen Kauf- und Verkaufskurs stehen und dabei übrigens ganz nebenbei 

\begin{equation*}
K(s) = \left( s \cdot V(s) \right)^{\prime}
\end{equation*}

gilt.

\end{Solution}

\vspace{0.5cm}


\subsubsection{Beispielrechnung}
\vspace{0.2cm}

\todo{WIP}

\begin{itemize}
	\item Daten, Zahlen, Fakten
	\item Profiterwartung aus Sicht eines Token-Holders/-Investors
\end{itemize}

\vspace{0.5cm}

Wir rechnen ein bisschen rum, um ein Gefühl für den nötigen Token-Supply zu bekommen:

\vspace{0.3cm}

\begin{Example}[Rechnerei zum Token-Supply]

\vspace{0.75cm}

\todo{TODO: Zahlen an Excel anpassen}

\vspace{0.75cm}

Wir peilen den Token-Contract so zu stricken, dass wir im eingeschwungenen Zustand einen Tokenwert des \textit{W-PLT} von $\approx$ 1 Cent anpeilen, aber gleichzeitig auch die Grenzen $[$0.5 Cent; 2 Cent$]$ im Auge behalten.

\vspace{0.5cm}

Wir forcieren beim Projekt-Fortschritt über die Zeit hinsichtlich des Tokens

\begin{itemize}
	\item Einen günstigen \textit{W-PLT}-Preis für die Gründer/Company ($\approx$ 0.25 Cent pro Token)
	\item Einen guten \textit{W-PLT}-Preis für ganz frühe Investoren ($<<$ 0.1 Cent)
	\item Einen \textit{W-PLT}-Preis von $<$ 1 Cent für die Early-Pool-User bis zum eingeschwungenen Zustand.
	\item Einen kontrollierten \textit{W-PLT}-Preis $<$ 2 Cent für die Pool-User im eingeschwungenen Zustand.
	\item Ein zunehmendes Ziel-Projekt-Invest mit Fortschritt des Projekts.
	\item Einen zunehmenden (aber kontrollierten) Ziel-Supply von \textit{W-PLT} mit Fortschritt des Projekts.
	\item Einen zunehmenden (aber kontrollierten) \textit{W-PLT}-Kurs mit Fortschritt des Projekts.
	\item Einen zunehmenden \textit{Utility-Koeffizient} (als Verhältnis zwischen mindest und Ziel-Supply) mit Fortschritt des Projekts bis zu Zielzustand des Koeffizienten von 50 \%.
\end{itemize}

\vspace{1.0cm}

Im Folgenden wieder die obige Forecast-Aufstellung - nun aus Token-Sicht:


\begin{itemize}
	\item Nach 3 Monaten: ca. 40 noch aktive und bereits ca. 10 liquidierte Pools.
	\begin{itemize}
		\item bereits \textit{umgesetzte Fees}: 25.000 \textit{W-PLT} 
		\item mindestens bereits geburnte Tokens: 12.500 \textit{W-PLT} 
		\item \textit{Pending-Fees}: 100.000 \textit{W-PLT}  
		\item \textit{Staked-Fees}: 200.000 \textit{W-PLT}
		\item min Supply: 300.000 \textit{W-PLT}
		\item Ziel-Projekt-Invest: 60.000 \$ \todo{(davon 30-40k durch Gründer/Company)}
		\item Projekt-Treasury: 60.000 \$ + 1.000 \$ Fees-Cash $\approx$ 61.000 \$
		\item Ziel-Supply: 12.0 Mio. \textit{W-PLT}
		\item \textit{Utility-Koeffizient}: $\frac{300.000}{12.000.000} = 2.5 \%$
		\item $\varnothing$ Kaufpreis pro \textit{W-PLT}: 0.5 Cent
		\item Mindest-Value pro \textit{W-PLT}: $\approx$ 0.51 Cent
	\end{itemize}
	\item Nach 6 Monaten: ca. 100 noch aktive und bereits ca. 50 liquidierte Pools.
	\begin{itemize}
		\item bereits \textit{umgesetzte Fees}: 130.000 \textit{W-PLT}
		\item mindestens bereits geburnte Tokens: 65.000 \textit{W-PLT}  
		\item \textit{Pending-Fees}: 250.000 \textit{W-PLT}
		\item \textit{Staked-Fees}: 500.000 \textit{W-PLT} 
		\item min Supply: 750.000 \textit{W-PLT}
		\item Ziel-Projekt-Invest: 120.000 \$
		\item Projekt-Treasury: 120.000 \$ + 2.500 \$ Fees-Cash $\approx$ 122.500 \$
		\item Ziel-Supply: 16.0 Mio. \textit{W-PLT}
		\item \textit{Utility-Koeffizient}: $\frac{750.000}{16.000.000} = 4.6875 \%$
		\item $\varnothing$ Kaufpreis pro \textit{W-PLT}: 0.75 Cent
		\item Mindest-Value pro \textit{W-PLT}: $\approx$ 0.77 Cent
	\end{itemize}
	\item Nach 12 Monaten: ca. 300 noch aktive und bereits ca. 200 liquidierte Pools.
	\begin{itemize}
		\item bereits \textit{umgesetzte Fees}: 500.000 \textit{W-PLT}  
		\item mindestens bereits geburnte Tokens: 250.000 \textit{W-PLT}
		\item \textit{Pending-Fees}: 800.000 \textit{W-PLT}  
		\item \textit{Staked-Fees}: 1.6 Mio. \textit{W-PLT} 
		\item min Supply: 2.4 Mio \textit{W-PLT} 
		\item Ziel-Projekt-Invest: 200.000 \$
		\item Projekt-Treasury: 200.000 \$ + 10.000 \$ Fees-Cash $\approx$ 210.000 \$
		\item Ziel-Supply: 20.0 Mio. \textit{W-PLT}
		\item \textit{Utility-Koeffizient}: $\frac{2.400.000}{20.000.000} = 12.0 \%$
		\item $\varnothing$ Kaufpreis pro \textit{W-PLT}: 1 Cent
		\item Mindest-Value pro \textit{W-PLT}: $\approx$ 1.05 Cent
	\end{itemize}
	\item Nach 3 Jahren: ca. 20.000 noch aktive und bereits ca. 20.000 liquidierte Pools.
	\begin{itemize}
		\item bereits \textit{umgesetzte Fees}: 50 Mio. \textit{W-PLT}  
		\item mindestens bereits geburnte Tokens: 25 Mio. \textit{W-PLT}
		\item \textit{Pending-Fees}: 50 Mio. \textit{W-PLT}  
		\item \textit{Staked-Fees}: 100 Mio. \textit{W-PLT} 
		\item min Supply: 150 Mio \textit{W-PLT} 
		\item Ziel-Projekt-Invest: 20.0 Mio. \$
		\item Projekt-Treasury: 20.0 Mio. \$ + 0.5 Mio \$ Fees-Cash $\approx$ 20.5 Mio. \$
		\item Ziel-Supply: 1.4 Mrd. \textit{W-PLT}
		\item \textit{Utility-Koeffizient}: $\frac{150.000.000}{1.400.000.000} = 12.0 \%$
		\item $\varnothing$ Kaufpreis pro \textit{W-PLT}: $\approx$ 1.43 Cent
		\item Mindest-Value pro \textit{W-PLT}: $\approx$ xxx Cent
	\end{itemize}
	\item Nach 5 Jahren: 450.000 noch aktive und bereits 550.000 liquidierte Pools.
	\begin{itemize}
		\item bereits \textit{umgesetzte Fees}: 1.5 Mrd. \textit{W-PLT} 
		\item mindestens bereits geburnte Tokens: 750 Mio. \textit{W-PLT}
		\item \textit{Pending-Fees}: 1.2 Mrd. \textit{W-PLT}
		\item \textit{Staked-Fees}: 2.4 Mrd. \textit{W-PLT} 
		\item min Supply: 3.6 Mrd. \textit{W-PLT} 
		\item Ziel-Projekt-Invest: 144 Mio. \$
		\item Projekt-Treasury: 144 Mio. \$ + 16 Mio. \$ Fees-Cash $\approx$ 160 Mio. \$
		\item Ziel-Supply: 7.2 Mrd. \textit{W-PLT}
		\item \textit{Utility-Koeffizient}: $\frac{3.6}{7.2} = 50.0 \%$
		\item $\varnothing$ Kaufpreis pro \textit{W-PLT}: 2 Cent
		\item Mindest-Value pro \textit{W-PLT}: $\approx$ 2.22 Cent		
	\end{itemize}	 
\end{itemize}

\end{Example}

\vspace{0.5cm}

\todo{Ende WIP}

\vspace{0.5cm}



\subsubsection{Recap \& Ausblick}
\vspace{0.2cm}

\todo{Einbettung in das Wunder-Ökosystem und Link zwischen WPT zu WUNDER}


\vspace{0.5cm}



\subsubsection{Unberücksichtigte Inhalte \& Ideen}

\vspace{0.3cm}
\todo{WIP}
\vspace{0.5cm}

\begin{Praemisse}[Cash-Flow]

\begin{itemize}
	\item Deposit/Invest erfolgt in einem Stable-Coin (z.B. \textit{USDT}). \todo{Es sind aber auch mehrere zulässige Währungen denkbar.}
	\item Cashout erfolgt in derselben Währung wie der Deposit \todo{(oder zumindest in einer der zulässigen Währungen)}. 
	\item Fees werden in aller Regel prozentual am Volumen (also in \textit{USDT}) berechnet, jedoch in \textbf{\textit{W-PLT}} veranschlagt, bei dem man von Kursschwankungen ausgehen muss und ausgehen will. \todo{(Das muss sowohl bei der Token-Modellierung als evtl. auch bei der Gebührenordnung berücksichtigt werden.)}
	\item Ein Teil der Fees soll direkt an den (Bonding-Curves-basierten) \textit{W-PLT}-Contract gehen und damit die \textit{W-PLT}-Investoren/-Hodler belohnen.
	\item Die Fees sollen \todo{(aufgrund des genauer zu erklärenden Stakings)} erst bei der Liquidierung des Pools entrichtet werden.
	\item Die Fees sollen von allen Pool-Teilnehmers außer des Pool-Creators getragen werden.
	\item Für den Pool-Creator soll folgendes gelten:
	\begin{itemize}
		\item \todo{(Muss einen PassNFT besitzen.)}
		\item Soll einen prozentual an den geschätzten gesamten Pool-Fees gemessenen Betrag $x$ als Sicherheit staken ($x \in [50 \%; 200 \%]$). \todo{Kann theoretisch auch einer absoluten oder relativen Obergrenze unterliegen.}
		\item Soll selbst keine Fees bezahlen.
		\item Soll für das Staken mit einem Teil der erwirtschafteten Gebühren entlohnt werden.
	\end{itemize}
\end{itemize}

\end{Praemisse}

\vspace{0.5cm}



\begin{Algo}[\textit{W-PLT}-Design]

\begin{itemize}
	\item Wir haben Szenarien vorgerechnet und dabei den zugehörigen \textit{Ziel-Supply} angegeben. Daran orientieren wir uns.
	\item Abhängig des Szenarios (gemappt auf den \textit{Ziel-Supply}) können wir einen aus Fees-Einnahmen resultierenden erwarteten Profit ableiten - und zwar pro Zeiteinheit, die genau der durchschnittlichen Pool-Lifetime entspricht.
	\item Wir legen einen Zeitraum als Vielfaches der durchschnittlichen Pool-Lifetime fest, die wir einem Token-Investor zumuten, bis sein Invest profital wird. Z. B. 4 $\cdot$ Pool-Lifetime.
	\item Aus den obigen Daten können wir für diesen Zeitraum (ab Invest beim entsprechenden Supply) den erwarteten relativen Profit pro Token $x$ approximieren.
	\item Diesen erwarteten relativen Profit pro Token müssen wir bei den obigen Rechnungen noch als $x(supply) = profit(zeitraum; supply)$ ergänzen.
	\item $kaufpreis(supply) := x(supply) \cdot \frac{treasury}{supply}$
	\item Den Early-Investoren darf durchaus ein größerer Faktor zugemutet werden.
	\item Wir errechnen für einige supply-Milestones den Kaufpreis auf der Grundlage des letzten Bullets und approximieren dann dazwischen.
\end{itemize}

\end{Algo}



\vspace{1.0cm}

\todo{WIP}

\vspace{0.3cm}

\begin{itemize}
	\item Wann bezahlen die User die Fees?
	\item In welcher Form/Währung dürfen die Fees von den Usern erbracht/verrechnet werden und wird dann alles im Hintergrund sofort in \textit{W-PLT} umgewandelt?
	\item Ist es denkbar die Fees aus dem Stake-Pool des Creators zu verwenden und diesem seinen Stake in einer anderen Währung zurückzuerstatten?
	\item Split der Gebühren auf Staker und Projekt-Treasury $(\sigma_{S}; \sigma_{T})$ mit $\sigma_{S} + \sigma_{T} = 1$ definieren. Dazu gibt es einige denkbare Varianten:
	\begin{itemize}
		\item fester, statischer Split
		\item fester, statischer Split mit eingebauten Unter- und Obergrenzen für den Gesamtertrag des Stakers $\sigma_{S} \cdot fees^{\mathcal{P}}$
		\item $fees^{\mathcal{P}}$-abhängiger (progressiver) Split, bei dem der Anteil des Stakers $\sigma_{S}$ mit zunehmendem $fees^{\mathcal{P}}$ stets kleiner wird. Dies unter Umständen ebenfalls unter Berücksichtigung eingebauter Unter- und Obergrenzen für den Staker.
		\item Begünstigung des Stakers in Abhängigkeit seines NFT-Pass-Status.
	\end{itemize}	
	\item Anfängliche Air-Drops von \textit{W-PLT} an erste Pool-User müssen einer Locking-Periode unterliegen.
	\item Problem: Wenn Token-Holder aussteigen, nehmen diese nicht nur ihren Anteil an den bisher erwirtschafteten Fees-Einnahmen mit, sondern drücken zudem auch noch den Token-Kurs nach unten. Das stellt einen sich selbst verstärkenden Effekt dar, der irgendwie in den Griff zu bekommen ist (evtl. Locking-Periode oder berücksichtigende Verkaufs-Kurve).
	\item Es macht eine Locking-Periode von der Dauer der durchschnittlichen Pool-Lifetime viel Sinn, da die beim Verkauf der Tokens mitgenommenen Gewinne aus Fees-Einnahmen durch neu generierte Fees-Einnahmen egalisiert werden und der Token-Value somit stabil bleibt.
	\item Die Einlage für den \textit{W-PLT} ist idealerweise in WUNDER zu erbringen \todo{(Es ist noch unklar, wie man an WUNDER kommt, wenn es vorher keinen Token-Sale gegeben hat. Ob der WUNDER ebenfalls mittels Bonding-Curves abzubilden wäre, sei hier erst einmal mehr als unklar.)}
	
\end{itemize}

\vspace{1.0cm}

\begin{Problem}[\textit{USDT} vs. \textit{W-PLT} als Berechnungsgrundlage für Fees, Staking etc.]
\vspace{0.2cm}

\todo{Was nehmen wir hier?}

\vspace{0.5cm}

\todo{Folgend übernommene alte Test-Passagen zu dem Thema:}

\vspace{0.5cm}

Ein weiterer sehr essenzieller Faktor für die Größe des zu stakenden Betrags könnte der Kurs des IPTs sein. Denn laut der \textbf{Bonding-Curves}-Implementierung würde der \textit{W-PLT}-Preis mit steigender Zirkulation steigen, was mit der Zunahme von existierende Pools geschähe. Damit wäre die Erstellung neuer Pools mit ihrer zahlen\-mäßigen Zunahme stets kapital-intensiver (aber nicht gleichbedeutend teurer). \textbf{Die Frage hierbei ist also, ob der zu erbringende Stake des Pool-Creators auf den \textit{Total-Supply des W-PLT} normiert werden sollte oder nicht}, die gänzlich mit der obigen Fragestellung einhergeht, ob der Pool-Creator eigentlich staken möchte oder das nur tun muss.
	
\begin{itemize}
	\item Gegen eine Normierung spricht die Annahme/Hoffnung, ein Pool-Creator sei gleichzeitig auch ein großer Supporter des gesamten Projekt und glaube daran. Wenn der \textit{W-PLT}-Preis steigt, ist dies gleichbedeutend mit der Zunahme an genutzten Pools, an denen der Pool-Creator als Staker, Besitzer von \textit{W-PLT} und damit Projekt-Investor auch selbst (finanziell) profitiert.
	\item Für eine Normierung spricht dagegen die potenzielle Gefahr, neue oder bestehende User durch eine zu hohe finanzielle Sicherheitseinlage davon abzuschrecken neue Pools zu erstellen.
\end{itemize}

\vspace{0.2cm}
	
Die Antwort auf diese Fragestellung könnte auch darin liegen, ob wir uns besonders viele oder lieber weniger aber besonders Teilnehmer-starke Pools wünschen.

\vspace{0.5cm}
	
Die \textit{Pool-Teilnehmer} (außer des Creators) können bei dieser Logik aber nicht wie nicht wie die Staker zusätzlich als Projekt-Investoren angesehen werden, weil sie \textit{W-PLT} kaufen, da die gekauften \textit{W-PLT} direkt als Gebühr entrichtet werden. Für die Pool-Teilnehmer stellt der \textit{W-PLT} also eher einen Utility- bzw. Purpose-Token dar weshalb die Höhe der zu entrichtenden Gebühr zweifelsfrei auf Basis von \textit{Total-Supply des W-PLT} normiert werden muss (die Gebühr darf keinesfalls mit Zunahme von Pools steigen).


\end{Problem}


\newpage

    % binde die Datei '[Pools][Economic].tex' ein


\subsection{Pool-Vertrag}
\vspace{0.3cm}
% !TEX root = paper.tex

\todo{Herausgearbeitete Dinge zu $\mathcal{R}$ zusammentragen}

\begin{itemize}
  \item Vorgabe zur Teilnehmer-Menge $\mathcal{U}$
  \item Vorgabe zur Pool-Treasury $\mathcal{S}$:
  \begin{itemize}
  	\item Währung (zB \textit{USDT})
  	\item Intervall $\mathcal{I}$ für $s_i \in \mathcal{I}$
  \end{itemize}
  \item Definition der \textit{Liquidierungsentscheidung-Oracle}
  \item Definition der \textit{Auszahlungsschlüssel-Oracle}
  \item Optionale Forderung $\varphi_i \geq 0$
  \item etc.
\end{itemize}

\vspace{0.5cm}

    % binde die Datei '[Pools][Vertrag].tex' ein


    % binde die Datei 'Pools.tex' ein
% !TEX root = paper.tex

\section{Community}
\label{sec:community}
\todo{TODO}    % binde die Datei 'Community.tex' ein
% !TEX root = paper.tex
\section{Zusammenfassung}
\label{sec:fazit}
\todo{TODO}    % binde die Datei 'Zusammenfassung.tex' ein
% !TEX root = paper.tex

\section{Anhang}
\label{sec:anhang}

\vspace{0.3cm}

\input{[12][Anhang]/[Anhang][Textueller Input]}    % binde die Datei '[Anhang][Textueller Input].tex' ein
\newpage


% !TEX root = paper.tex

\subsection{Bonding-Curves}
\label{sec:bonding-curves}

\vspace{0.3cm}

\begin{itemize}
  \item \href{https://medium.com/@simondlr/tokens-2-0-curved-token-bonding-in-curation-markets-1764a2e0bee5}{Gute Einführung}
  \item \href{https://www.newsbtc.com/sponsored/zap-fun-profit/}{Anwendungsbeispiel}
  \item \href{https://docs.google.com/document/d/1VNkBjjGhcZUV9CyC0ccWYbqeOoVKT2maqX0rK3yXB20/edit}{White-Paper}
  \item \href{https://github.com/ConsenSysMesh/curationmarkets/blob/master/CurationMarkets.sol}{Smart-Contract}
\end{itemize}  



\vspace{0.5cm}

Zunächst einmal eine Formalisierung einer Token-Distribution mittels Bonding-Curves:

\vspace{0.2cm}

\begin{Def}[Token-Distribution mittels Bonding-Curves]
\label{defBC}

Angenommen man möchte einen Projekt-Token \textbf{TKN} herausgeben und dieses im Markt distribuieren. Der Mechanismus der \textit{Bonding-Curves} stellt hierbei ein alternatives Modell zu gängigen Tokensales (z.B. ICO) dar und folgt dabei einigen wesentlich Merkmalen, die ihn teils grundlegend von herkömmlichen Tokensales abgrenzen.

\begin{itemize}
  \item TKN wird von einem Smart-Contract verwaltet, der wesentlich mehr Logik implementiert als ein herkömmlicher ERC20-Contract.
  \item TKN kann jederzeit und von jedem gemintet werden. Dies geschieht gegen eine Einlage/Bezahlung in einer dafür definierten Währung (z.B. ETH oder USDT). Der Token wird quasi von dem Verwalter-Contract verkauft.
  \item Es existiert damit keine initiale bevorzugte Token-Ausgabe beim Contact-Launch an etwaige bevorzugte Parteien (Contract-Owner, Herausgeber, Investoren etc.). Die Ausgabe erfolgt ausschließlich gegen Einlage und kennt keine Bevorzugung entgegen der Contract-Logik. 
  \item Die Einnahmen aus der Tokenausgabe kommen ausschließlich der Token-Cont\-ract-Treasury $\mathbf{\mathcal{T}}$ zugute anstatt wie bei herkömmlichen Tokensales bestimmten Begünstigten (wie z.B. der Herausgeber-Company oder deren Gründern).
  \item TKN unterliegt keinem maximalen Gesamt-Supply. Es können stets neue Tokens ausgegeben werden, solange Interessenten existieren, die die Einlage für die Token-Ausgabe erbringen.
  \item Tokens können jederzeit von ihren Besitzern gegen einen - einer bestimmten Contract-Logik folgenden - Rückkaufpreis an den Token-Contract zurückgegeben werden. Zurückgegebene Tokens werden dabei sofort von dem Verwalter-Contract geburnt und somit aus der Zirkulation genommen.
  \item TKN kann selbstverständlich auch am Sekundärmarkt gehandelt werden (falls dieser bessere Konditionen hergibt als die Aussage bzw. Rücknahme durch den Token-Contract selbst).
\end{itemize}

\vspace{0.2cm}

Sei $i \in \mathbb{N}$ der aktuelle Gesamt-Supply von TKR (wir nehmen hier mal an, TKR sei atomar) und \textbf{\$} die Tausch- bzw. Einlage-Währung (\textbf{\$} ist hier abstrakt und nicht als US-Dollar zu verstehen).

Dann werden die durch die Token-Contract vorgegebenen Ausgabepreis (Kaufpreis) $\mathcal{K}$, Rücknahmepreis (Verkaufspreis) $\mathcal{V}$ und Contract-Treasury-Inhalt $\mathbf{\mathcal{T}}$ für den zuletzt ausgegebenen Token $i$ - jeweils in der Einheit \textbf{\$} - durch die jeweils supply-abhängigen Funktionen beschrieben:

\begin{align*}
\mathcal{K}, \mathcal{V}, \mathcal{T} &: \mathbb{N} \rightarrow \mathbb{Q}^{+} \\
\mathcal{K} \left( i \right) &:= \textrm{ Letzter Token-Ausgabepreis in \$ bei einem Gesamt-Supply von i} \\
\mathcal{V} \left( i \right) &:= \textrm{ Aktueller Token-Rückkaufkurs in \$ bei einem Gesamt-Supply von i} \\
\mathcal{T} \left( i \right) &:= \sum_{j = 1}^{i} \mathcal{K} \left( j \right)
\end{align*}

\vspace{0.2cm}

Die definierende Logik unseres Tokens lässt sich damit also formal als 

\begin{equation*}
TKR = \left( \mathcal{K}, \mathcal{V}, \$ \right)
\end{equation*}

schreiben, wobei hierbei $\mathbf{\mathcal{T}}$ ausgespart bleibt, da es implizit durch $\mathbf{\mathcal{K}}$ gegeben ist.

\vspace{0.2cm}

Sicherlich können und werden bei einer konkreten Implementierung eines mittels der durch $\mathbf{\mathcal{K}}$ und $\mathbf{\mathcal{V}}$ gegebenen \textit{Bonding-Curves} beschriebenen Tokens noch andere zu formalisierende Faktoren und Mechanismen eine Rolle spielen. Für die simpelste abstrakte Definition reichen die genannten Größen 
jedoch für den Moment aus.

\end{Def}

\vspace{0.3cm}

Die eben - auf diskrete Weise - definierten Funktionen $\mathcal{K} \left( i \right)$ und $\mathcal{V} \left( i \right)$ erfordern insofern noch eine zusätzliche Bemerkung, als dass diese explizit nur eine Token-weise Auskunft über Ausgabe- und Rücknahmepreis geben. Möchte man mehrere Token minten oder zurückkgeben, muss der Preis für jeden der Tokens separat ausgerechnet und anschließend addiert werden. Möchte man bei einem aktuellen Supply von $i \in \mathbb{N}$ nicht einen sondern $k \in \mathbb{N}$ mit $k \leq i$ Tokens minten bzw. zurückgeben, beläuft sich der gesamte Kauf- bzw- Verkaufspreis auf

\begin{align*}
\mathcal{K}_{k} \left( i \right) &:= \sum_{j = i + 1}^{i + k} \mathcal{K} \left( j \right) \\
\mathcal{V}_{k} \left( i \right) &:= \sum_{j = i - k + 1}^{i} \mathcal{V} \left( j \right).
\end{align*}

\vspace{0.3cm}

Das besonders Hervorhebenswerte an diesem Token-Ausgabemodell ist zweifelsfrei die gemeinschaftliche aus der Token-Ausgabe gefütterte Contract-Treasury aus echten Geldeinlagen, die nicht etwa einer dritten (Ausgabe-)Partei zugute kommt, sondern de facto den Tokeninhabern gehört. Die Existenz dieser Rücklagen gibt den ausgegebenen Tokens theoretisch einen realen Wert und ermöglicht einzig und allein die Implementierung des Rückkauf-Mechanismus. Der Gebrauch von dieser Möglichkeit und die Verankerung der Rückkauffunktion $\mathbf{\mathcal{V}}$ in der Logik des Token-Contracts realisiert den besagten realen Wert dann auch in der Praxis und nennt sich \textbf{\textit{"market maker of last resort"}}. Denn wenn ich eine definitive - in Contract-Logik verankerte - Sicherheit habe, ein Asset jederzeit verkaufen zu können, besitzt dieses Asset auch einen echten, intrinsischen Value, der nicht zwingend den marktwirtschaftlichen Mechanismen unterliegt - und somit auch keinen etwaigen Hypes um einen gut vermarkteten Tokensale ohne dahinterliegende Substanz. Vielmehr folgt der Tokenpreis der gemeinschaftlichen Projekt-Treasury $\mathbf{\mathcal{T}}$, die ihrerseits substanziell mit dem Projekterfolg einhergeht. 

\vspace{0.2cm}

Die letzte Einsicht lässt uns zu zwei wesentlichen Gedanken gelangen, die die entscheidenden Argumente für das \textit{Bonding-Curves}-Modell liefern könnten:

\begin{itemize}
  \item Der Rückgabepreis $\mathcal{V} \left( i \right)$ sollte eine direkte Abhängigkeit vom Treasury-Inhalt $\mathcal{T} \left( i \right)$ aufweisen.
  \item Nach bisheriger Definition hängt der Treasury-Inhalt $\mathcal{T} \left( i \right)$ ausschließlich vom aktuellen Supply $i \in \mathbb{N}$ (und den damit einhergehenden Ausgabepreisen $\mathcal{K} \left( j \right)$ für $j \leq i$) ab. Gepaart mit der ersten Forderung bedeutete dies implizit nichts anderes, als dass der aktuelle Rücknahmepreis ausschließlich von den Ausgabepreisen der bisherigen Tokens abhinge. Dies ist schlecht und ein sehr großes Problem der gängigen \textit{Bonding-Curves}-Implementierungen, da Koppelung des Rücknahmepreises - also des intrinsischen Werts des Tokens - ausschließlich an den Kaufpreis voriger Tokens - und die Wertentwicklung damit an den Kaufpreis etwaiger zukünftig ausgegebener Tokens, würde mathematisch alternativlos eine monoton steigende Ausgabepreis-Funktion $\mathcal{K} \left( i \right)$ erfordern. Ohne eine sehr stark fundierte projekt-bezogene Argumentation für ein monoton steigendes $\mathcal{K} \left( i \right)$ schrie das gesamte Modell nur so nach \textit{Pump \& Dump} und \textit{Hot Potatoe}. Und tatsächlich ist es so, dass nahezu alle \textit{Bonding-Curves}-Implementierungen Gebrauch von einer (streng) monoton steigenden Ausgabepreis-Kurve $\mathcal{K} \left( i \right)$ machen. Sie argumentieren mit anderem generierten Projekt-Value, der nicht durch die Contract-Treasury $\mathcal{T} \left( i \right)$ gemessen werden kann und diese Argumentation muss nicht zwingend falsch oder ungenügend sein. Uns reicht dies aber nicht - zumal es mehr als gute Gründe für ein monoton steigendes $\mathcal{K} \left( i \right)$ gibt (dazu später mehr). Somit bleibt uns nichts anderes als die - zumindest teilweise - Abkoppelung von $\mathcal{T} \left( i \right)$ und $\mathcal{K} \left( i \right)$, womit wir auch unsere obige Definition von $\mathcal{T} \left( i \right) = \sum_{j = 1}^{i} \mathcal{K} \left( j \right)$ wieder teils verwerfen müssen.
\end{itemize} 

\vspace{0.2cm}

Wir fassen zusammen:

\vspace{0.3cm}

\begin{Fazit}[Contract-Treasury als wichtigster Baustein zum Erfolg]

Wie sind der Überzeugung, der Contract-Treasury-Inhalt - gemessen als $\mathcal{T} \left( i \right)$ - sei der entscheidende Baustein für einen soliden \textit{Bonding-Curves}-Token. $\mathcal{T} \left( i \right)$ beeinflusst direkt den Rücknahmepreis $\mathcal{V} \left( i \right)$, verleiht dem Token damit einen echten geldwerten Value, was wiederum als Kaufargument für neue Tokens gilt und damit implizit auch zur Beurteilung des Ausgabepreises $\mathcal{K} \left( i \right)$ seitens etwaiger neuer Investoren hinzugezogen wird.

\vspace{0.2cm}

Konkret auf den Rücknahmepreis $\mathcal{V} \left( i \right)$ bezogen sehen wir kaum sinnvollere Alternativen als der gängigen Definition 

\begin{equation*}
\mathcal{V} \left( i \right) = \frac{\mathcal{T} \left( i \right)}{i}
\end{equation*}

zu folgen, was gleichbedeutend damit ist, dass der Besitz am Contract-Treasury-Inhalt pro rata auf die sich in Zirkulation befindenden Token verteilt wird, was sich konsequenterweise im Rücknahmepreis $\mathcal{V} \left( i \right)$ widerspiegelt. Mit der in Definition \ref{defBC} beschriebenen Treasury-Funktion $\mathcal{T} \left( i \right)$ ergibt sich damit für den Rücknahmepreis $\mathcal{V} \left( i \right)$

\begin{equation*}
\mathcal{V} \left( i \right) = \frac{\mathcal{T} \left( i \right)}{i} = \frac{\sum_{j = 1}^{i} \mathcal{K} \left( j \right)}{i}.
\end{equation*} 

\vspace{0.2cm}

Um einen Kauf bei aktuellem Supply von $i \in \mathbb{N}$ zu dem Preis von $\mathcal{K} \left( i \right)$ für sich als Investor zu rechtfertigen, muss man argumentieren, man könne den gekauften Token irgendwann zu einem höheren Preis wieder zurückgeben:

\begin{equation*}
\exists k > i, \textrm{  so dass } \mathcal{V} \left( k \right) > \mathcal{K} \left( i \right)
\end{equation*} 

\begin{equation*}
\Leftrightarrow \textrm{    } \sum_{j = 1}^{k} \mathcal{K} \left( j \right) > k \cdot \mathcal{K} \left( i \right)
\end{equation*} 

Gleichzeitig sollte $\mathcal{V} \left( i \right) \leq \mathcal{K} \left( i \right)$ für alle Supplies $i$ selbstverständlich angenommen werden können, da sonst Arbitrage entstünde, was sofort vom Markt aufgefressen werden würde.

\vspace{0.2cm}

Solche Anforderungen gehen alternativlos mit einem zwingend monoton steigenden $\mathcal{K} \left( i \right)$ einher (\todo{'monoton steigend' wäre hierbei noch zu präzisieren, da es nicht zu hundert Prozent dem mathematischen Verständnis gleicht und die Aussage wahrscheinlich noch zu beweisen; intuitiv ist es aber klar}).
Da eine zwingend monoton steigende Preiskurve als Investitionsmotivation - in einem abstrakt stehenden Kontext und ohne zusätzliche Argumente - schlichtweg unseriös ist, fordern wir für den \textit{Contract-Treasury-Inhalt}

\begin{equation*}
\mathcal{T} \left( i \right) > \sum_{j = 1}^{i} \mathcal{K} \left( j \right),
\end{equation*} 

bzw.

\begin{equation*}
\mathcal{T} \left( i \right) = \sum_{j = 1}^{i} \mathcal{K} \left( j \right) \textrm{  +  } f(t, i) \textrm{ mit } f(t, i) > 0,
\end{equation*} 

wobei $f(t, i)$ eine nicht genau präzisierte, positive \textit{Value-Funktion} unseres Projekts darstellt. Also so eine Art Gradmesser der (externen) Wertschöpfung/Wirtschaftlich\-keit unseres Projekts, die nicht in direktem Bezug zum Projekt-Kontrakt steht. Man könnte $f(t, i)$ vielleicht \textbf{\textit{Of-Contract-EBIT}} unseres Projekt nennen - wie auch immer dieses erwirtschaftet wird. 

Wie durch das $t$ angedeutet, bringt das $f(t, i)$ die zeitliche Dimension in unsere \textit{Contract-Treasury}-Rechnung, die externen (wirtschaftlichen) Einflüssen unterliegt und nicht mehr ausschließlich vom Gesamtsupply $i$ abhängt - gleichwohl der Gesamtsupply $i$ durchaus auch die genannten externen Einflüsse und damit auch $f(t, i)$ begünstigen kann.

\end{Fazit}

\vspace{0.3cm}

Der \textit{Of-Contract-EBIT} muss zwar nicht, kann aber durchaus auch aus anderen Smart-Contracts fließen. Solche Inter-Contract-Geldflüsse sind natürlich - falls möglich - eine stets bevorzugte Variante.  

Ist dies nicht möglich - und etwaige finanzielle Projekteinnahmen tatsächlich gänzlich \textit{off-chain}, bedarf es eines definierten und vertrauensvollen Mechanismus, der dafür sorge, dass die erzielten Projekterträge in die gemeinschaftliche Projekt-Treasury eingezahlt werden. Dies könnte entweder durch eine gesonderte Contract-Function dargestellt werden, oder aber durch den allgemeingültigen Weg mittels Tokenkäufen. Im zweiten Fall würden die gegen die extern erwirtschaftete Einlage ausgegebenen Tokens pro rata auf alle aktuellen Tokenhalter verteilt werden.

\vspace{0.2cm}

Mit einer nachweislich existenten \textit{externen projektbezogenen Profit-Funktion} $f(t, i)$ bekäme ein einem Projekt zugrunde liegendes \textit{Bonding-Curves-Token-Modell} das ihm noch fehlende - und aus unserer Sicht unverzichtbare - Merkmal: Nämlich eine gewisse Abkoppelung des realen Tokenwerts - gemessen durch $\mathcal{V} \left( i \right)$ - von der oft - zurecht als scheeball-artig kritisierten - Ausgabepreis-Kurve $\mathcal{K} \left( i \right)$.

Damit liefern wir aus unserer Sicht den bisher bei dem meisten \textit{Bonding-Curves-Implementierungen} fehlenden Baustein für eine solide und fundierte Token-Distribution mittels eines \textit{Bonding-Curves-Modells} und entgegnen damit - zumindest auf abstrakt Weise - dem bestehenden Hauptproblem der \textit{Bonding-Curves}: Nämlich \textit{Ponzi, Pump \& Dump oder Hot Potatoe} zu sein.

\vspace{0.2cm}

Wir behaupten hierbei keinesfalls, den heiligen Gral für das Funktionieren von \textit{Bonding-Curves-gestützten Token-Modellen} gefunden zu haben. Denn schließlich ist die Existenz des von uns eingeführten $f(t, i)$ alles andere als gegeben, äußerst projekt-abhängig und zudem in der Regel vermutlich nicht leicht zu formalisieren - geschweige denn ohne optimistische Annahmen zu beweisen. Denn nicht zuletzt argumentieren viele auf \textit{Bonding-Curves-Token-Modelle} gestützte Projekte genau wie wir hier an dieser Stelle: \textit{Es existiere ein externer Projekt-Value, von welchem die Tokeninhaber profitieren!} Nur geben diese Argumentationen diesem Value nicht den Namen $f(t, i)$ und verpassen damit eine formal abstrakte Verallgemeinerung.

Wir fassen zusammen:

\vspace{0.2cm}

\begin{Fazit}[Token-externer Projekt-Value]

Wir reduzieren das bestehende Problem von \textit{Bonding-Curves-Token-Modellen} hinsichtlich \textit{'Ponzi' und 'Pump \& Dump'} auf den Nachweis der Existenz einer \textbf{\textit{Of-Contract-EBIT-Funktion}} $f(t, i) > 0$ und deren Formalisierung anstatt von undefiniertem \textit{externen Projekt-Value} zu sprechen.

\vspace{0.2cm}

Idealerweise wird $f(t, i) > 0$ durch unbestreitbare Logik eines Smart-Contracts begründet, der in wertschöpfender Interaktion mit unserem Token-Contract steht (Profite in seine Treasury einzahlt). Andernfalls erschwert sich die Argumentation und muss solide begründet und formalisiert werden.

\end{Fazit}

\vspace{0.3cm}

Um das $f(t, i)$ nicht ganz als abstraktes Mysterium dastehen zu lassen und Zweifel an dessen Existenz zu zerstreuen: 

\vspace{0.2cm}

\begin{Example}[DeFi-Projekte besitzen stets ein $f(t, i) > 0$]

Jedes DeFi-Projekt besitzt ein $f(t, i) > 0$! Ob

\begin{itemize}
  \item DEX-Provider à la \textit{Balancer}, 
  \item Market-Places à la \textit{OpenSea},
  \item Lending-Platforms
\end{itemize}  

oder viele andere agieren wirtschaftlich, indem sie in der Regel Provisionen auf ihre Dienstleistung veranschlagen. Diese Gebühren sind unmissverständlich in ihrer Contract-Logik verankert und gänzlich transparent. 

Würden solche Projekte einen \textit{Bonding-Curves-gestützten} Utility-Token herausgeben und einen gewissen Abfluss ihrer Provisions-Profite in die Contract-Treasury dieses Tokens implementieren, würde solch ein Mechanismus auf ganz natürlich Weise das von uns geforderte $f(t, i) > 0$ definieren. Jeder Tokeninhaber würde damit jeder gebühren-pflichtigen Transaktion mit partizipieren.

\end{Example}

\vspace{0.3cm}

Wie nun bereits mehrfach betont, ist und bleibt das von uns eingeführte $f(t, i)$ projekt-bezogen. Wir wollen an dieser Stelle dagegen weiterhin möglichst abstrakt und allgemein bleiben und uns im folgenden der Ausgabepreis-Kurve $\mathcal{K} \left( i \right)$ widmen - insbesondere auch mit der Zielsetzung die Anforderungen an das $f(t, i)$ so gering zu gestalten, wie nur möglich.

Wie ebenfalls bereits herausgearbeitet, ist die verwendete Implementierung des $\mathcal{K} \left( i \right)$ ohne nachweisbare Existenz eines $f(t, i) > 0$ in vielen \textit{Bonding-Curves-Projekten} als ein teils unseriöser Preistreiber zu sehen. Dies resultierte aus der bereits oben formalisierten Kaufentscheidung eines jeden potenziellen Tokenkäufers:

\begin{equation*}
\textrm{Ich sollte kaufen   } \Leftrightarrow \exists k > i, \textrm{  so dass } \mathcal{V} \left( k \right) > \mathcal{K} \left( i \right)
\end{equation*} 

Und zwar \textbf{Preistreiber} deswegen, weil die Kaufentscheidung eines Tokens bei aktuellem Supply $i \in \mathbb{N}$ ohne fundiertes $f(t, i)$ von Kaufentscheidungen Anderer bei größerem Supply $i, i+1,...,k$ abhängt. Man sollte also genau dann kaufen, wenn man davon ausgeht, das andere später auch kaufen - und das noch zu einem höheren Preis. Diese müssten dann davon ausgehen, dass wiederum andere noch später zu einem noch höheren Kurs einsteigen und so weiter.

\vspace{0.2cm}

\todo{An bier WIP}

\vspace{0.5cm}

Und obwohl wir die Existenz eines $f(t, i) > 0$ nicht allgemein sicherstellen und es erst recht nicht konkret benennen können, so können wir zumindest eine Mindestanforderung herleiten und diese dabei zusätzlich ins Verhältnis zum Risiko eines potenziellen Investors setzen.

Dazu wollen wir das $\mathcal{V} \left( i \right)$ von etwaigen künftigen Käufen abkoppeln und seine Abhängigkeit von $\mathcal{K} \left( j \right)$ auf ausschließlich vergangene Käufe $\mathcal{K} \left( j \right)$ mit $j <= i$ reduzieren.

\vspace{0.3cm}

\begin{Praemisse}[Risiko eines Tokenkaufs]

Wir definieren eine zusätzliche Abhängigkeit zwischen Ausgabepreis-Kurve $\mathcal{K} \left( i \right)$ und Rücknahmepreis-Kurve $\mathcal{V} \left( i \right)$:

\begin{equation*}
\mathcal{K} \left( i \right) = \left( 1 + \rho \right) \cdot \mathcal{V} \left( i \right) = \left( 1 + \rho \right) \cdot \frac{\mathcal{T} \left( i \right)}{i} = \left( 1 + \rho \right) \cdot \frac{\sum_{j = 1}^{i} \mathcal{K} \left( j \right)}{i} \textrm{  mit } \rho > 0
\end{equation*}

Neu an dieser Forderung ist hierbei lediglich die erste Gleichheit. Die anderen sind Resultat unserer bereits weiter oben definierten Anforderungen an die Rücknahmepreis-Kurve $\mathcal{V} \left( i \right)$.

\vspace{0.3cm}

Das $\rho > 0$ beschreibt hierbei das Invest-Risiko eines Tokenkaufs bei Supply von $i \in \mathbb{N}$. Unsere Forderung ist also wie folgt zu interpretieren: $\rho$ beziffert den prozentualen Höchstverlust, den ein potenzieller Investor bei einem Kauf eines Tokens zum Preis von $\mathcal{K} \left( i \right)$ bei einem Supply von $i \in \mathbb{N}$ im schlimmsten Fall zu tragen hätte. Denn er könnte den gekauften Token jederzeit zum Preis von $\mathcal{V} \left( i \right) = \frac{\mathcal{K} \left( i \right)}{1 + \rho}$ wieder an den Contract zurückgeben und sich dafür auszahlen lassen.

\vspace{0.1cm}

Hinsichtlich dieser Interpretation wird das $\rho > 0$ in der Praxis also vermutlich tendenziell sehr klein zu wählen sein: $0 < \rho \ll 1$.

\vspace{0.1cm}

Die konkrete Wahl von $\rho$ wird hierbei in starker Wechselwirkung zum projekt-bezogenen $f(t, i) > 0$ stehen. Je vielversprechender der durch $f(t, i)$ beschriebe \textit{Of-Contract-EBIT} des Projekts ausfällt, desto größeres Risiko und damit $\rho$ ist einem Tokenkäufer zuzumuten und umgekehrt. 

\vspace{0.1cm}

Am Ende gilt wahrscheinlich sogar 

\begin{equation*}
\textrm{Es existiert kein  } f(t, i) > 0 \textrm{   } \Leftrightarrow \textrm{   } \rho = 0 \textrm{   } \Leftrightarrow \textrm{   } \mathcal{V} \left( i \right) = \mathcal{K} \left( i \right) \textrm{  } \forall i \in \mathbb{N}
\end{equation*}

\end{Praemisse}

\vspace{0.3cm}

Das Risiko ist mit der gemachten Vorgabe an die Rücknahmepreis-Kurve ist insofern mit gegebener Sicherheit gedeckelt als das $\mathcal{V} \left( i \right)$ für $\rho > 0$ stark monoton steigend ist

\vspace{0.3cm}

Zudem kann man bei gegebenem $\rho > 0$ seine durch

\begin{equation*}
\exists k > i, \textrm{  so dass } \mathcal{V} \left( k \right) > \mathcal{K} \left( i \right)
\end{equation*} 

gechallengte Kaufentscheidung validieren und nicht nur die Existenz eines solchen $k > i$ beweisen, sondern dieses $k$ auch konkret in Abhängigkeit des $\rho$ berechnen.

\vspace{0.3cm}

\todo{WIP}

\vspace{0.3cm}

\todo{Begünstigte formalisieren (mit zusätzlicher Funktion). Begünstigte erhalten einen Teil der Projekt-Treasury zu ihrer freien Verfügung.}

\vspace{0.5cm}
 % binde die Datei '[Anhang][Token-Economics][Bonding Curves].tex' ein
    % binde die Datei 'Anhang.tex' ein


 
\end{document}