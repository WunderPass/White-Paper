% !TEX root = paper.tex

Die in Definition \ref{defPool} erstmals eingeführte Größe $\mathcal{R}$ begegnete uns in den letzten Kapiteln unzählige Male. Wir nannten sie \textit{Regelset, Pool-} bzw. \textit{DAO-Vereinbarung} oder in Anlehnung an die in Annahme \ref{assumptionGbR} formulierte Analogie auch \textbf{\textit{Gesellschaftervereinbarung}} oder \textbf{\textit{-vertrag}}. 

\vspace{0.1cm}

So richtig formalisiert haben wir $\mathcal{R}$ jedoch bisher nirgends. Vielmehr wurde genannt, was $\mathcal{R}$ zu regeln hat oder was es enthalten kann oder muss. Die fehlende Formali\-sierung ist kein Versäumnis sonder schlichweg kaum möglich, was aber auch gleichzeitig kaum überraschen mag, wenn man bedenkt, wie beliebig komplex, heterogen und zahlenmäßig unbeschränkt die Regelungen in einem \textit{Gesellschaftervertrag} nur sein können. Man kann darin prinzipiell alles regeln oder fast gar nichts. 

\vspace{0.3cm}

\todo{Einfach Brainstorming als Bullet-Liste...Kein Anspruch auf Vollständigkeit...Definition eines SHA verlinken...}

\vspace{0.3cm}

\todo{Herausgearbeitete Dinge zu $\mathcal{R}$ zusammentragen}

\begin{itemize}
  \item Vorgabe zur Teilnehmer-Menge $\mathcal{U}$
  \item Vorgabe zur Pool-Treasury $\mathcal{T}$:
  \begin{itemize}
  	\item Währung (zB \textit{USDT})
  	\item Intervall $\mathcal{I}$ für $s_i \in \mathcal{I}$
  \end{itemize}
  \item Definition der \textit{Liquidierungsentscheidung-Oracle}
  \item Definition der \textit{Auszahlungsschlüssel-Oracle}
  \item Optionale Forderung $\varphi_i \geq 0$
  \item etc.
\end{itemize}

\vspace{0.5cm}

