% !TEX root = paper.tex

Eine (allgemeine) funktionale Beschreibung derjenigen WunderPool-Funktionalität, die der Überschrift der gegenständigen Sektion gerecht wird, ist insofern sehr schwierig, als dass sich diese deutlich schwerer auf unterschiedliche Pool-Cases verallgemeinern lässt. Wie anfangs in dem Kapitel \nameref{sec:pools-einleitung} ist die möglichste Verallgemeinerung aller Cases oberste Prämisse gewesen. Hier müssen wir versuchen zu verallgemeinern, was zu verallgemeinert geht, und den Rest eben Case-spezifisch lösen. 

\vspace{0.3cm}

Wir schauen zurück auf die anfangs in Kapitel \nameref{sec:pools-einleitung} hervorgehobenen Anwendungsfälle der WunderPools. Und zwar jetzt mit explizitem Blick auf ihre \textit{Lifetime}:

\begin{itemize}
  \item \textbf{\textit{Social Investing:}} Dies ist mit der klarste Case für eine relevante Lifetime eines Pools. Während der Lifetime werden mögliche Invests vorgeschlagen, zur Abstimmung gestellt und im Erfolgsfall abgewickelt. Die Möglichkeiten zur Erweiterung von Investmöglichkeiten (Staking, Lending, Liquidity-Providing, Yield Farming, Aktien, ETFs etc.) scheinen schier unendlich. In diesem Case unterliegt \textbf{die Dauer der Lifetime auch keinerlei natürlicher Grenzen} - diese Art von Pool kann theoretisch ewig existieren.
  \item \textbf{\textit{Geschenk-Pool:}} In diesem Case besteht die Daseinsberechtigung des Pools im Grunde lediglich darin, bequem und einfach Geld einzusammeln und evtl. bis zum Kauf des Geschenks \textit{"in Treuhand"} zu verwahren. Sind alle gewünschten Teilnehmer beigetreten (und damit gleichbedeutend deren Beitrag zum Geschenk entrichtet), hat der Pool de facto bereits seinen Zweck erfüllt. Man kann zwar argumentieren, man könne die Auswahl des Geschenks mit DAO-Mitteln zur Abstimmung stellen, dies bliebe jedoch an den Haaren herbeigezogen, solange das Geschenk kein auf der Blockchain erwerbbares Asset darstellt. \textbf{Die Dauer der Lifetime der Pools in diesem Case sind also klar begrenzt}: Spätestens bis zu dem Moment des Kaufs des Geschenks.
  \item \textbf{\textit{Kicktipp-Pool:}} Dies ist der Bilderbuch-Case für den Pool im Sinne der Treuhand-Verwahrung (eines Spieleinsatzes) über einen längeren Zeitraum. Hier wird einge\-zahlt, über einen gewissen Zeitraum (außerhalb des Pools) gespielt und am Ende - je nach Ergebnis - wieder ausgezahlt. Das Geld wird vom Pool also lediglich verwahrt und umverteilt. In der sogenannten \textit{Lifetime} des Pools passiert faktisch gar nichts. Man könnte sich sicherlich kreative Möglichkeiten zur Interaktion mit dem Pool überlegen (wie z.B. Abstimmungen über etwaige Regeländerungen oder über das Nachtragen von verspätet abgegebenen Tipps), dies beträfe aber nie die relevante Kernfunktionalität des Pools innerhalb dieses Cases. Die defacto \textit{'leere Lifetime'} des Pools endet in diesem Case mit Ablauf der Spielzeit, für die die Kicktipp-Runde eingerichtet wurde. Ihre \textbf{Dauer ist also begrenzt}.
  \item \textbf{\textit{Wetten:}} Dieser Case verhält sich sehr analog zum \textit{Kicktipp-Case}. Dazu muss jedoch klargestellt sein, dass wir den Case als eine einzige Wette (zwischen zwei oder mehr Leuten) verstehen, bei der der Pool der Treuhand-Verwahrung dient, und nicht etwa eine "Wett-Gruppe", wo immer mal wieder neue Wetten vorgeschlagen und umgesetzt werden. Der Pool dieses Cases bildet also eine einzige Wette ab und seine \textbf{Lifetime endet in dem Moment, wo das Ergebnis der Wette feststeht}.
  \item \textbf{\textit{Splitwise:}} Dies ist der außergewöhnlichste aller Cases. Hier existieren de facto weder eine echte Treasury noch eine Lifetime. Für Splitwise wird erst die Umverteil\-ung interessant, wobei hier genau genommen der Betrag von 0 auf die Teilnehmer umverteilt wird. Da hier aber - im Gegensatz zu allen obigen Cases - auch negative Withdraws zulässig sind (also genau genommen eine Einzahlung von denjenigen Teilnehmern, die anderen Teilnehmern etwas schulden), klingt die Umverteilung des Betrags 0 plötzlich doch nicht mehr so abwegig. Die 0 signalisiert nur die Forderung, die verteilten Beträge (Schulden und Auslagen mit entsprechendem Vorzeichen) müssten sich am Ende auf 0 summieren. Da der Pool in diesem Case faktisch gar keine Lifetime besitzt, ist \textbf{die Dauer der Lifetime konsequenterweise begrenzt}.
\end{itemize}

\vspace{0.3cm}

Zusammenfassend halten wir fest, die Dauer der Pool-Lifetime sei nur für den \textit{Social-Investing-Case} theoretisch unbegrenzt. Bei allen anderen Cases wird der Pool nach einer bestimmten Zeit oder bei Eintreten eines bestimmten Ereignisses obsolet und muss/sollte anschließend aufgelöst werden. Und auch hinsichtlich relevanter Funktionalität während der zugehörenden \textit{Lifetime} scheint der \textit{Social-Investing-Case} ebenfalls der einzig interessante zu sein. 

\vspace{0.1cm}

Eine Verallgemeinerung erscheint also - zumindest für die zuletzt genannten vier Cases - evtl. doch im Rahmen des Möglichen. 

\vspace{0.3cm}

Zu guter Letzt sollte der Umstand nicht unerwähnt bleiben, etwaiges Austreten bestehender Pool-Teilnehmer bzw. das Eintreten neuer stellten keine irelanten Szenarien dar, die sich ebenfalls während vermeintlichen \textit{Pool-Lifetime} abspielen würden.

\vspace{0.3cm}

Abschließend formalisieren wir erneut die erarbeiteten Gedanken - und zwar mit besonderem Blick auf Defintion \ref{defPool} und Annahme \ref{assumptionGbR}: 

\vspace{0.2cm}

\begin{Fazit}[Bestehen und Geschäftstätigkeit eines WunderPools als GbR (Lifetime)]
\label{lifetimeGbR} 

Sei $\mathcal{P} := \left( \mathcal{U}, \mathcal{R}, \mathcal{T}, \mathcal{G} \right)$ ein WunderPool wie in Definition \ref{defPool} und das Verständnis davon stark an Annahme \ref{assumptionGbR} angelehnt.

\vspace{0.2cm}

Wir möchten gerne auch die Lifetime eines WunderPools in die GbR-Analogie einbetten und unterscheiden dabei zwischen \textbf{der Geschäftstätigkeit / Unternehmens\-gegenstand der Gesellschaft selbst} auf der einen Seite und \textbf{den internen Gesellschaftsstrukturen} auf der anderen.

\noindent\hrulefill

\underline{Geschäftstätigkeit:}

\vspace{0.2cm}

Rein abstakt betrachtet, versuchen hierbei alle Gesellschafter $u \in \mathcal{U}$ - legiti\-miert durch deren Anteile aus $\mathcal{G}$ und restriktiert mittels Gesellschftervertrags $\mathcal{R}$ - durch strategisches Verhalten das gemeinschaftliche Gesellschaftsvermögen $\mathcal{T}$ zu optimieren bzw. dieses zumindest optimal für ein zweckgebundenes gemeinsam Ziel einzusetzen.

\vspace{0.2cm}

Die internen Gesellschter- und Gesellschaftstrukturen - repräsentiert durch die Größen $\mathcal{U}$, $\mathcal{G}$ und $\mathcal{R}$ - bleiben in diesem Kontext in aller Regel unberührt.

\noindent\hrulefill

\underline{Interne Strukturen:}

\vspace{0.2cm}

In diesem gegenwärtigen Kontext sind gegenteilig zum ersten genau die konträren Größen $\mathcal{U}$, $\mathcal{G}$ und $\mathcal{R}$ adressiert. Hierbei sind die Gesellschafter $u \in \mathcal{U}$ - legitimiert durch deren Anteile aus $\mathcal{G}$ und restriktiert mittels Gesellschftervertrags $\mathcal{R}$ - dazu befähigt und ggf. daran interessiert Veränderungen an $\mathcal{U}$, $\mathcal{G}$ und $\mathcal{R}$ erzwingen.

Beispielhaft sind folgende Cases denkbar:

\begin{itemize}
  \item \textbf{Kapitalerhöhung:} Hierbei würden die Gesellschafter anhand ihrer Stimmrechte aus $\mathcal{G}$ (und etwaiger Zusatzvereinbarungen aus $\mathcal{R}$) über die Aufnahme eines neuen Gesellschafters in die Gesellschaft abstimmen. Eine Zustimmung hätte mindestens eine Veränderung der Größen $\mathcal{U}$ und $\mathcal{G}$ zur Folge - und zwar in beiden Fällen eine Vergrößerung. In aller Regel würde bei einer Kapital\-erhöhung ebenso die Gesellschaftskasse $\mathcal{T}$ wachsen. Und in manchen Fällen wäre ebenso eine Veränderung des Gesellschftervertrags $\mathcal{R}$ zu erwarten.
  \item \textbf{Auszahlen eines bestehenden Gesellschafters:} Von der Logik her ein ähnlicher Case wie der erste, nur dass hierbei ein oder mehrere Gesellschafter die Gesellschaft verlassen, die Menge $\mathcal{U}$ also verkleinert wird. Anders als der erste Case ist der gegenständige jedoch differenzierter hinsichtlich des \textit{\textbf{Wie}} zu betrachten. Während bei der Kapitalerhöhung einfach neue Anteile kreiert werden, die schlichtweg von den neuen Gesellschaftern übernommen werden, gibt es hierbei mehrere gängige Varianten:
  \begin{itemize}
	\item Die Anteile des/der ausscheidenden Gesellschafter(s) werden von der Gesell\-schaft selbst übernommen und anschließend vernichtet. Hierbei verringert sich also $\sum_{g \in \mathcal{G}}g$ konsequenterweise. In diesem Fall muss in aller Regel die Gemeinschaftskasse $\mathcal{T}$ zur Auszahlung herangezogen werden. Dies kann insofern etwas komplex werden, als dass die Treasury $\mathcal{T}$ nicht zwingend ausschließlich aus Fiat-Vermögen bestehen muss, und Assets ggf. liqudiert werden müssten.
	\item Die Anteile des/der ausscheidenden Gesellschafter(s) werden von den verbleibenden Gesellschaftern übernommen. Hierbei bliebe $\sum_{g \in \mathcal{G}}g$ also unverändert. Die finanzielle Abwickung würde \textit{außerhalb der Gesellschaft} stattfinden. Insbesondere bliebe die Treasury $\mathcal{T}$ von der Transaktion unberührt, was die Implementierungslogik vehement vereinfachen würde.
	\item Der technisch simpleste Case wäre sicherlich der Verkauf der Anteile des/der ausscheidenden Gesellschafter(s) am Sekundärmarkt - die ja als ERC20-Tokens einfach handelbar wären. Auch hierbei bliebe sowie $\sum_{g \in \mathcal{G}}g$ als auch $\mathcal{T}$ als auch wahrscheinlich $\mathcal{R}$ unverändert. Genau genommen ist dieses Sub-Szenario ein Speziellfall des nächsten Cases.
  \end{itemize}
  \item \textbf{Anteilsverkauf / -übertragung:} In diesem Szenario würde ein Gesellschafter $u_{i} \in \mathcal{U}$ einen Teil oder alle seiner Anteile $g_{i} \in \mathcal{G}$ am Sekundärmarkt verkaufen. Der Gesellschaftervertrag $\mathcal{R}$ könnte ihn zwar theoretisch daran hindern bzw. die anderen Gesellschafter in eine solche Entscheidung einbeziehen. Dies würde jedoch mit erheblicher Komplexität einhergehen, weshalb wir \textit{per default} erst einmal davon ausgehen wollen, die Gesellschaftsanteile sind als ERC20-Gover\-nance-Tokens frei handelbar und unterliegen keinen Einschränkungen. In diesem Fall wäre die innere Gesellschaftsstruktur einem ausschließlich äußeren Einfluss unterstellt, dem sie nicht Herr wäre. (Unkontolliert) betroffen wären die Größen $\mathcal{U}$ und $\mathcal{G}$.
  \item \textbf{Änderung des Gesellschaftervertrags:} Hierbei würden die Gesellschafter anhand ihrer Stimmrechte aus $\mathcal{G}$ (und etwaiger Zusatzvereinbarungen aus $\mathcal{R}$) über gewisse Änderungen an $\mathcal{R}$ abstimmen. Welche Änderungen hierbei möglich wären, könnte wiederum mittels des vor der Änderung geltenden Regelsets $\mathcal{R}$ gemaßregelt sein. Aufgrund dieser Rekursivität müsste $\mathcal{R}$ wahr\-scheinlich einige unveränderliche Elemente enthalten. Zu der definitiv komplexesten Größe unserer WunderPools $\mathcal{R}$ siehe auch das Kapitel \nameref{sec:pools-vertrag}.
\end{itemize}

\end{Fazit}

\vspace{0.5cm}