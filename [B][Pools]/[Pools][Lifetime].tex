% !TEX root = paper.tex

Eine (allgemeine) funktionale Beschreibung derjenigen WunderPool-Funktionalität, die der Überschrift der gegenständigen Sektion gerecht wird, ist insofern sehr schwierig, als dass sich diese deutlich schwerer auf unterschiedliche Pool-Cases verallgemeinern lässt. Wie anfangs in dem Einführungskapitel \ref{sec:pools-einleitung} ist die möglichste Verallgemeinerung aller Cases oberste Prämisse gewesen. Hier müssen wir versuchen zu verallgemeinern, was nur geht, und den Rest eben Case-spezifisch lösen. 

\vspace{0.1cm}

Wir schauen auf die Anfangs in Kapitel \ref{sec:pools-einleitung} hervorgehobenen Anwendungsfälle für die WunderPools an - nun mit kurzer Skizzierung ihrer Lifetime:

\begin{itemize}
  \item \textbf{\textit{Social Investing:}} Das ist mit der klarste Case für eine relevante Lifetime eines Pools. Während der Lifetime werden mögliche Invests vorgeschlagen, zur Abstimmung gestellt und im Erfolgsfall abgewickelt. Die Möglichkeiten zur Erweiterung von Investmöglichkeiten (Staking, Lending, Liquidity-Providing, Yield Farming, Aktien, ETFs etc.) scheinen schier unendlich. In diesem Case unterliegt \textbf{die Dauer der Lifetime auch keinerlei natürlicher Grenzen} - diese Art von Pool kann theoretisch ewig existieren.
  \item \textbf{\textit{Geschenk-Pool:}} In diesem Case besteht die Daseinsberechtigung des Pools eigentlich lediglich darin, bequem und einfach Geld einzusammeln und evtl. bis zum Kauf des Geschenks "in Treuhand" zu verwahren. Sind alle gewünschten Teilnehmer beigetreten (und somit ihren Beitrag zum Geschenk entrichtet), hat der Pool eigentlich bereits seinen Zweck erfüllt. Man kann zwar argumentieren, man könne die Auswahl des Geschenks mit DAO-Mitteln zur Abstimmung stellen, dies bleibt jedoch an den Haaren herbeigezogen, solange das Geschenk kein auf der Blockchain erwerbbares Asset ist. \textbf{Die Dauer der Lifetime der Pools in diesem Case sind also klar begrenzt}: Spätestens bis zu dem Moment des Kaufs des Geschenks.
  \item \textbf{\textit{Kicktipp-Pool:}} Das ist der Bilderbuch-Case für den Pool im Sinne der Treuhand-Verwahrung (eines Spieleinsatzes) über einen längeren Zeitraum. Hier wird eingezahlt, über einen Zeitraum (außerhalb des Pools) gespielt und am Ende - je nach Ergebnis - wieder ausgezahlt. Das Geld wird vom Pool also lediglich verwahrt und umverteilt. In der sogenannten \textit{Lifetime} des Pools passiert faktisch gar nichts. Man könnte sich sicherlich kreative Möglichkeiten zur Interaktion mit dem Pool überlegen (wie z.B. Abstimmungen über etwaige Regeländerungen oder über das Nachtragen von verspätet abgegebenen Tipps), dies beträfe aber nie die relevante Kernfunktionalität des Pools innerhalb dieses Cases. Die defacto \textit{'leere Lifetime'} des Pools endet in diesem Case mit Ablauf der Spielzeit, für die die Kicktipp-Runde eingerichtet wurde. Ihre \textbf{Dauer ist also begrenzt}.
  \item \textbf{\textit{Wetten:}} Dieser Case verhält sich sehr analog zum \textit{Kicktipp-Case}. Dazu muss jedoch klargestellt sein, dass wir den Case als eine einzige Wette (zwischen zwei oder mehr Leuten) verstehen, bei der der Pool der Treuhand-Verwahrung dient, und nicht etwa eine "Wett-Gruppe", wo immer mal wieder neue Wetten vorgeschlagen und umgesetzt werden. Der Pool dieses Cases bildet also eine einzige Wette ab und seine \textbf{Lifetime endet in dem Moment, wo das Ergebnis der Wette feststeht}.
  \item \textbf{\textit{Splitwise:}} Dies ist der außergewöhnlichste aller Cases. Hier existieren de facto weder eine echte Treasury noch eine Lifetime. Für Splitwise wird erst die Umverteilung interessant, wobei hier genau genommen der Betrag von 0 auf die Teilnehmer umverteilt wird. Da hier aber - im Gegensatz zu allen obigen Cases - auch negative Withdraws zulässig sind (also genau genommen eine Einzahlung von denjenigen Teilnehmern, die anderen Teilnehmern etwas schulden), klingt die Umverteilung des Betrags 0 plötzlich doch nicht mehr so abwegig. Die 0 signalisiert nur die Forderung, die verteilten Beträge (Schulden und Auslagen mit entsprechendem Vorzeichen) müssen sich auf 0 summieren. Da der Pool in diesem Case faktisch gar keine Lifetime besitzt, ist \textbf{die Dauer der Lifetime konsequenterweise begrenzt}.
\end{itemize}

\vspace{0.3cm}

Zusammenfassend halten wir fest, die Dauer der Pool-Lifetime ist nur für den \textit{Social-Investing-Case} theoretisch unbegrenzt. Bei allen anderen Cases wird der Pool nach einer bestimmten Zeit oder bei Eintreten eines bestimmten Ereignisses obsolet und muss/sollte anschließend aufgelöst werden. Und auch hinsichtlich relevanter Funktionalität während der \textit{Lifetime} scheint der \textit{Social-Investing-Case} ebenfalls der einzig interessante zu sein. 

\vspace{0.1cm}

Eine Verallgemeinerung erscheint also - zumindest für die zuletzt genannten vier Cases - evtl. doch im Rahmen des Möglichen. 

\vspace{0.3cm}

\todo{Etwaiges Austreten bestehender Teilnehmer oder Eintreten neuer Teilnehmer würde sich während der 'Lifetime' abspielen.}

\vspace{0.5cm}