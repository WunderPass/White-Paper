% !TEX root = paper.tex


Dabei sollen gleichermaßen ein Monetarisierungmodell, ein zugehöriger Business-Plan sowie eine mögliche Utility-Token-Ökonomie, die diese Komponenten mittels \href{https://de.wikipedia.org/wiki/Mechanismus-Design-Theorie}{Mechanismus-Design} in Einklang zueinander bringt und in einem übergeordneten Ökonomie-Kreislauf verankert, gleichzeitig erarbeitet und miteinander verknüpft werden.

\vspace{0.2cm}

Am Ende soll idealerweise jede solcher Fragen wie,

\begin{itemize}
	\item \textit{Wer bezahlt den Pool-Service und wie viel?}
	\item \textit{Wer verdient am Pool-Service und wie viel?}
	\item \textit{Wie wird das Pool-Projekt finanziert und wie werden etwaige Investoren incentiviert und entlohnt?}
	\item \textit{Wie sieht der konkrete Business-Plan aus?}
	\item \textit{Wie wird der zugehörige Pool-Project-Token modelliert und in das übergeordnete Pool-Ökosystem integriert?}
	\item \textit{Wie sind Risiko und ROI von etwaigen Projekt-Invests zu beziffern?}
\end{itemize}

beantwortet sein.

\vspace{0.5cm}

Da der zentrale Bestandteil der eigentlichen Dienstleistung der Pools für seine Nutzer bereits in sehr starkem finanziellen Kontext - nämlich des \textit{Social-Investings} - steht, und wir uns im Folgenden mit dem finanziellen Gerüst des übergeordneten Pools-Projects beschäftigen möchten - das aber so gar nichts mit der Dienstleistung des \textit{Social-Investings} an sich gemein hat, müssen wir gleich zu Beginn eine essenzielle Abgrenzung ziehen, ohne deren unmissverständliches Bewusstsein beim Leser die folgenden Kapitel nur missverstanden werden können und werden.

\vspace{0.2cm}

\textbf{Man lese und verinnerliche also folgendes lieber gleich zehnmal:}

\vspace{0.2cm}

\begin{Abgrenzung}[Pools-Project-Economics haben nichts mit Invest/Economics eines einzelnen Pools (als Dienstleistung des Pools-Projects) zu tun.]

\vspace{0.2cm}

Die Dienstleistung unseres Pools-Projects hat im Sinne des \textit{Social-Investings} unausweichlich mit Geld zu tun. Die \textbf{Pools-Project-Economic} haben dies konsequenterweise ebenfalls.

\vspace{0.1cm}

\textbf{Dabei steht ausschließlich zweites im Fokus des gegenständigen Kapitels. Erstes dagegen bestenfalls beiläufig als Referenzgrundlage bis gar nicht.} Die User der Pools hantieren mit Geld, indem sie den Service nutzen. Projekt-Stakeholder verdienen idealerweise an der angebotenen Dienstleistung - wie sie es auch täten, falls die Dienstleistung keinerlei finanziellen Bezug hätte.

\vspace{0.75cm}

Wir wollen hier einige \textit{Fallstricke} für offensichtliche Missverständnisse und Verwechselungsgefahren ganz konkret beim Namen nennen:

\begin{itemize}
	\item Die Pools (als genutzte Dienstleistung) verfügen über Funds und Assets. Beides werden in aller Regel Tokens sein. Die Funds - als \textit{Fiat-Äquivalent} - vermutlich (aber auch nicht zwingend) mittels eines \textit{Stable-Coins} repräsentiert. Die Assets erst einmal nicht weiter spezifiziert. 
	
	\textbf{Diese finanziellen Mittel eines Pools stellen bestenfalls eine Referenzgrundlage zu anfallenden Service-Fees dar, sind kein direkter Bestandteil der Pools-Economics und verwenden ganz besonders NICHT den Pool-Project-Token als Basis-/Funding-Währung.}
	\item Die Monetarisierung des Pools-Service wird anhand von (prozentualen) Service-Fees erfolgen, die als Berechnungsgrundlage durchaus das Kapital des je\-weiligen Pools heranziehen kann und wird. 
	
	Konkret werden diese Fees in einer dafür definierten Währung anfallen, die ein \textit{Stable-Coin} UND/ODER der Pool-Project-Token sein kann. Die \textit{Monetarisier\-ungs-Währung} ist dabei zentraler Bestandteil der \textit{Pool-Economics}, die Währung der Pool-Funds eines Pools ist es dagegen absolut nicht und daher auch nicht maßgebend für die Fees-Abrechnung. Bei etwaigen Währungs-Diskrepanzen muss unter Umständen ein Umrechnungs- und Ad-Hoc-Umtausch-Mechanismus implementiert werden
	\item Ein in den folgenden Kapiteln definierter \textit{Token-\textbf{Staking}-Mechanismus} wird den Pool-Project-Token als Währung vorsehen und \textbf{hat dabei absolut nichts mit dem/den Pool-Kapital/-Funding/-Assets zu tun.}
	\item Jeder Pool wird eine \textbf{\textit{Pool-Treasury}} besitzen, die die Pool-Funds und die Pool-Assets verwaltet. Unser Pool-Project-Token wird gleichzeitig einem Mo\-dell folgen, bei dem eine sogenannte \textbf{\textit{Token-Contract-Treasury}} von großer Bedeutung sein wird, die wir künftig wahlweise auch als \textbf{\textit{Pools-Project-Treasury}}, \textbf{\textit{Pools-Token-Treasury}} oder als \textbf{\textit{Project-Token-Treasury}} be\-zeichnen. Ungeachtet der - nicht immer konsistenten Bezeichnung - ist diese dringend von der erstgenannten Treasury eines einzelnen Pools zu unterscheiden.
\end{itemize}

\end{Abgrenzung}

