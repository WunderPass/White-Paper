% !TEX root = paper.tex

Dies ist die \textit{Lightpaper-Version} des \href{https://github.com/WunderPass/White-Paper/blob/main/pools.pdf}{WunderPools-Papers}.

\vspace{0.3cm}


\begin{Problem}[Bedarf nach Gemeinschaftskassen und deren Verwahrung]

Es existieren zahlreiche Situationen, in denen - insbesondere im privaten Umfeld - Gemeinschaftskassen von Vorteil wären (z.B. Haushaltskasse in einer WG) bzw. in denen eine Gruppe von (Privat-)Personen, Geld für einen bestimmten (gemeinsammen) Zweck sammelt (z. B Geburtstagsgeschenk für einen Freund).

\vspace{0.3cm}

Digitale Lösungen für diesen Bedarf existieren de facto nicht. Der Grund dafür besteht im Wesentlichen in der Notwendigkeit einer Banklizenz (\textit{BaFin}) zur Geldverwahrung. Selbst \textit{PayPal} hat ihr früheres \textit{Pooling-Feature} abgeschaltet.

\vspace{0.1cm}

Die \textit{'analog'} Lösung - die darin bestehen, dass eine verantwortliche Person händisch Geld von den Anderen eintreiben muss - ist gelinde gesagt ziemlich müßig und des digitalen Zeitalters unwürdig! 

\end{Problem}

\vspace{0.3cm}


\begin{Example}[Anwendungsfälle sind zahlreich]

Im Folgenden eine nur kurze Liste von Anwendungsfällen:

\begin{itemize}
  \item Pool für ein gemeinsames (Geburtstags-)Geschenk.
  \item Haushaltskasse (z.B. in einer WG oder im Urlaub).
  \item Vereinskasse.
  \item Gemeinsame Invests in (Crypto-)Assets.
  \item Kicktipp-Pool (der über die gesamte Saison verwahrt werden muss).
  \item Wetten unter Freunden (z. B. 'Schaffe ich den Marathon?').
  \item Ausgleichspool für Auslagen von Geld an Freunde \& Bekannte (Splitwise).
\end{itemize}

\end{Example}

\vspace{0.3cm}


\begin{Solution}[Smart-Contracts brauchen keine BaFin-Lizenz]

Darum, was ein (digitaler) Service-Dienstleister ohne größte bürokratische Hürden nicht darf, muss ein Smart-Contract noch nicht einmal jemanden fragen: \textbf{Geldverwahrung für Dritte}.

\vspace{0.2cm}

Genau genommen machen zahlreiche der bedeutendsten heutigen Smart-Contracts bereits nichts anderes: Hantieren mit irgendwelchen Funds, was auch die Verwahrung inkludiert. 

\end{Solution}

\vspace{0.3cm}



\begin{Konzept}[von der GbR zur \textit{Mini-DAO}]

Die oben genannten Anwendungsfälle gelten laut deutschem Recht als eine \href{https://de.wikipedia.org/wiki/Gesellschaft_b%C3%BCrgerlichen_Rechts_(Deutschland)}{Gesellschaft bürgerlichen Rechts (GbR)}. Eine Gruppe von Privatpersonen, die für einen Freund ein gemeinsames Geschenk kaufen möchten, gelten demnach tatsächlich als eine \textit{GbR}.

\vspace{0.3cm}

Jeder WunderPool stellt also de facto nichts anderes dar als eine \textit{GbR}, die man im Web3-Jargon wiederum auch \textit{Mini-DAO} nennen könnte. Im Folgenden eine (plakative) Gegenüberstellung von analogen Bestandteilen: 

\vspace{0.5cm}

\begin{tabular}[h]{|c|c|c}
\hline
\textbf{WunderPool / Mini-DAO} & \textbf{GbR / Gesellschaft} \\
\hline
Pool- / DAO-Member & Gesellschafter \\
\hline
Pool- / DAO-Treasury & Gemeinschaftskasse / Gesellschaftskonto \\
\hline
parametrisierte Contract-Logik & Abmachungen / Gesellschaftervereinbarung \\
\hline
Pool- / DAO-Governance & Cap-Table \\
\hline
\end{tabular}

\end{Konzept}

\vspace{0.3cm}



\begin{Solution}[technische Konzeption]

Die technische Konzeption wird ausführlich in dem umfänglichen \href{https://github.com/WunderPass/White-Paper/blob/main/pools.pdf}{WunderPools-Paper} beschrieben. Dabei wird die Business-Logik maßgeblich in die folgenden drei Bereiche unterteilt:

\begin{itemize}
  \item \textbf{Pool-Erzeugung}
  \item \textbf{Pool-Lifetime}
  \item \textbf{Pool-Liquidierung}
\end{itemize}

\end{Solution}

\vspace{0.3cm}


\begin{Hypothese}[Abstraktion mittels Payout-Oracles]

Wir erwarten, dass sich unser \textit{Mini-DAO-Case} auf einen \textbf{wesentlich größeren Markt} abstrahieren lässt als kleine bis mittlere GbRs mit integriertem Geldpool. Und zwar auf jeden einzelnen Anwendungsfall, der eine praktive, sichere und unbüro\-kratische Aufbewahrung von Funds benötigt. Und insbesondere auch jene Anwendungsfälle, die eine solche Aufbewahrung eigentlich benötigen könnten/würden, in der Praxis jedoch aufgrund der BaFin-Regularien bisher darauf verzichten (müssen).

\vspace{0.3cm}

Bei einer solchen Abstaktion würde man wesentliche der obigen funktionalen Bestandteile der \textit{GbR} bzw. \textit{Mini-DAO} aus dem Funktionsumfang des Produkts herausnehmen und im Wesentlichen nur noch die \textit{Pool-Treasury} zum dezentralen Aufbewahren von Funds auf Protokoll-Ebene anbieten. Wie ein \textit{Safe} ohne viel Schnickschnack. Simple integrierbar in Fremdanwendungen und vor allem ohne Notwendigkeit einer BaFin-Lizenz. 

\vspace{0.3cm}

Die Auf- bzw. Umverteilung der aufbewahrten Funds bei Pool-Liquidierung würde dabei mittels abstrahierter \textit{Payout-Oracles} erfolgen. Zu Ideen und anschaulichen Beispielen zu dem Thema \textit{Payout-Oracles} siehe \textbf{Kapitel 5} des \href{https://github.com/WunderPass/White-Paper/blob/main/pools.pdf}{WunderPools-Papers}.

\end{Hypothese}

\vspace{0.3cm}





\vspace{0.5cm}