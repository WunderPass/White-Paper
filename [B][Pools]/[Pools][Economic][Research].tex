% !TEX root = paper.tex


\vspace{0.3cm}

\begin{Quelle*}[Professionelle Anbieter für Tokenomics]

\begin{itemize}
  \item \href{https://www.findas.org/}{FinDaS}
  \item \href{https://stobox.io/tokenization}{Stobox}
  \item
\end{itemize}

\end{Quelle*}

\vspace{0.3cm}

\begin{Zitat*}[Artikel zu guten Token-Economics (für Utility-Token)]

\href{https://blog.stobox.io/how-to-make-your-utility-token-economy-work/}{How to make your utility token economy work}

\vspace{0.2cm}

Tokenomics ist für das Überleben und den Erfolg von Projekten unerlässlich. Es ist entscheidend, die richtige Passung zwischen dem Geschäftsmodell und dem Token zu finden, die Nachfrage sorgfältig zu modellieren und die geeigneten Ansätze für das Management des Angebots auszuwählen. 

\vspace{0.2cm}

\underline{\textbf{Behauptungen:}}

\begin{itemize}
  \item Ein ordnungsgemäßes Token-Management im Einklang mit seinem Geschäftsmodell ist genauso wichtig wie die Qualität des Projekts/Produkts selbst.
  \item Im Grunde läuft die Tokenomik auf die Aufteilung des Token-Angebots hinaus (x Prozent Presale, y Prozent ICO, z Prozent Team etc. Dies erfolgt sehr häufig sehr willkürlich, muss jedoch gut durchdacht sein.
  \begin{itemize}
  	\item Ein solideres Verständnis der Verteilung ist essentiell.
  	\item Modellierung der organischen Nachfrage nach Token
  	\item Modellierung der spekulativen Nachfrage nach Token
  	\item proaktive Verwaltung des Token-Angebots
  	\item Kontrolle und etwaige Anpassung der Liquidität
  \end{itemize}
  \item Das Endziel von Tokenomics ist es, einen stabilen und idealerweise wachsenden Token-Preis und eine stabile Marktkapitalisierung aufrechtzuerhalten.
  \item Bedeutung von Marktkapitalisierung des Preises des Tokens:
  \begin{itemize}
  	\item Ein zu volatiler Token-Preis erschwert jegliche Art von Planung mit dem Token als Wert. Auf Unternehmensseite beispielsweise hinsichtlich Planung von Ausgaben und Wachstum . Und auf Userseite hinsichtlich Einsatz als Ressource/Utility.
  	\item Der Token bestimmt das Umsatzpotenzial des Projekts.
  	\item Tokens stellen ggf. einen relevante Anteil des Vergütungspakets von Mitarbeitern dar, sodass eine solide Wirtschaftlichkeit des Utility-Tokens die Bindung von Top-Talenten ermöglicht. 
  \end{itemize}
\end{itemize}

\vspace{0.3cm}


\underline{\textbf{Token-Preis steigern:}}

\begin{itemize}
  \item Nachfrage ankurbeln
  \begin{itemize}
  	\item Organische Nachfrage
	\begin{itemize}
		\item Organische Nachfrage ist durch den tatsächlichen Utility-Nutzen des Tokens getrieben. Man kauft ihn, um ihn zu benutzen.
		\item Die organische Nachfrage ist der Haupttreiber für die langfristige Preisentwicklung und Marktkapitalisierung.
		\item Eine starke organische Nachfrage ist eine notwendige Grundlage für eine gesunde Tokenomik.
		\item Um organische Nachfrage zu gewährleisten bzw. zu befeuern, gilt es eine guten Token-Utility maßzuschneidern, die den folgenden beiden Anforderungen genügt:
		\begin{itemize}
			\item Das zum Token gehörende Produkt muss ein überzeugendes Wertversprechen bieten, das die Kunden lieben.
			\item Das Token-Design sollte die User/Holder dazu ermutigen, die Tokens zu verwenden, um den Gesamtwert des Produkts bzw. der Dienstleistung zu erhalten/optimieren.
		\end{itemize}
		\item \textbf{Token als ergänzender Element des Produkts:} Teuebonus, Gamification etc.
		\item \textbf{'all-in':} Jeden denkbare Anwendungsfall des Produkts ist inhärent mit dem Token verknüpfen
		\item Mit der bedeutendste Aspekt bei der Konzeption der Token-Ökonomie ist die Modellierung der prognostizierten Nachfrage nach Tokens. Dies dient als Grundlage für die anfängliche Verteilung der Tokens, Staking-Programmen etc.
	\end{itemize}
  	\item Spekulative Nachfrage
	\begin{itemize}
		\item Spekulative Nachfrage wird von Einzelpersonen geschaffen, die einen Token kaufen, um ihn später zu einem besseren Preis wieder zu verkaufen.
		\item Spekulative Nachfrage ist normalerweise völlig unvorhersehbar und daher schwer zu kontrollieren.
		\item Die beste Methode, Spekulanten anzuziehen, besteht darin, ihr Profit-Spektrum und Handlungsspielraum zu vergrößern. Dies geschieht dadurch, dass der Token mehrere Blockchains supported, von möglichst vielen Börsen gelistet wird, Verwendung innerhalb von DeFi-Protokollen findet etc.
		\item Ein großer Nachteil dieses größeren Handlungsspielraums ist ein womöglich zu komplexes Ökosystem und eine zu starke Fragmentierung des anfangs ohnehin geringen Handelvolumens des Tokens, womit dieser praktisch nicht handelbar wird.
		\item Die spekulative Nachfrage ist deutlich weniger wichtig als die organische. Gleichwohl kann sie insbesondere am Anfang eine große Rolle für das Projekt spielen, um dieses zu finanzieren, bevor man mit dem Produkt selbst überzeugen kann.
	\end{itemize}
  \end{itemize}
  \item Angebot beeinflussen
  \begin{itemize}
  	\item Die Angebotsseite ist leichter zu steuern als die Nachfrageseite. Das Unternehmen kann in der Regel jederzeit neue Tokens an den Markt bringen und manchmal auch Tokens vom Markt wegkaufen, falls das Budget dafür vorhanden ist.
  	\item Man kann (von Unternehmensseite) das Angebot/Freigabe der Tokens entweder manuell oder algorithmisch steuern. Eine etwaige algorithmische Freigabe macht es noch wichtiger, die Token-Ökonomie richtig zu gestalten. 
	\item Gängige algorithmische Supply-Steuerungs-Mechanismen:
	\begin{itemize}
		\item \textbf{Token-Burn}
		\item \textbf{Lockup:} Effektive Lockups sind besonders dann wichtig, falls das Unternehmen eine verhältnismäßig kleine Anzahl an Token hält und somit die Kontrolle über das Angebot abgibt. So kann verhindert werden, dass der Markt mit einer zu großen Menge an Tokens geflutet werden kann. Die Lockup-Conditions hängen in der Regel von der prognostizierten Nachfrage ab, weshalb die richtige Nachfragemodellierung für korrekte Lockups essenziell ist.
		\item \textbf{Staking:} Man jedoch nicht alle (verfrühten) spekulativen Abverkäufe der Tokens durch Lockups unterbinden, da viele Spekulanten ihre Tokens vom Markt haben und nicht vom Unternehmen. Die Alternative zu Lockups ist also die Inzentivierung der Tokenbesitzer - auch der Spekulanten - ihre Tokens freiwillig zu halten, in dem man ihnen Rewards für das Halten zugesteht.
	\end{itemize}
  \end{itemize}
\end{itemize}

\vspace{0.3cm}
\underline{\textbf{Liquidity-Management:}}

\vspace{0.2cm}

Die Schaffung von Liquidität für den Token ist das letzte Stück des Token-Management-Puzzles. 

\begin{itemize}
  \item Um gewisse Preisstabilität sicherzustellen und Volatilität zu verhindern sowie ein bequemes Token-Handeln zu gewährleisten, bedarf es eines ausreichenden Markts für diesen Token (Liquidität).
  \item Mechanismen zur Sicherstellung von Liquidität:
  \begin{itemize}
	\item \textbf{Liquidity-Pools:} Idealerweise müssen LPs für die gängigsten Handelspaare auf den Blockchains erstellen, die die Community am häufigsten nutzt. In jedem Pool sollte genügend Liquidität vorhanden sein, damit der Token-Preis nicht zu volatil wird. Wenige Pools mit mehr Liquidität sind vielen Pools mit weniger Liquidität vorzuziehen.
	\item \textbf{zentrale Börsen}
	\item \textbf{Market-Maker:} (Manuelles) Platzieren von Kauf- und Verkaufsaufträge auf bestimmten Preisniveaus, sodass Token-Inhaber Token kaufen oder verkaufen können.
	\item 
  \end{itemize}
  \item
\end{itemize}

\end{Zitat*}

\vspace{0.5cm}
