% !TEX root = paper.tex


\vspace{0.3cm}

\begin{Zitat*}[Artikel zu guten Token-Economics (für Utility-Token)]

\href{https://blog.stobox.io/how-to-make-your-utility-token-economy-work/}{How to make your utility token economy work}

\vspace{0.2cm}

\underline{\textbf{Behauptungen:}}

\begin{itemize}
  \item Ein ordnungsgemäßes Token-Management im Einklang mit seinem Geschäftsmodell ist genauso wichtig wie die Qualität des Projekts/Produkts selbst.
  \item Im Grunde läuft die Tokenomik auf die Aufteilung des Token-Angebots hinaus (x Prozent Presale, y Prozent ICO, z Prozent Team etc. Dies erfolgt sehr häufig sehr willkürlich, muss jedoch gut durchdacht sein.
  \begin{itemize}
  	\item Ein solideres Verständnis der Verteilung ist essentiell.
  	\item Modellierung der organischen Nachfrage nach Token
  	\item Modellierung der spekulativen Nachfrage nach Token
  	\item proaktive Verwaltung des Token-Angebots
  	\item Kontrolle und etwaige Anpassung der Liquidität
  \end{itemize}
  \item Das Endziel von Tokenomics ist es, einen stabilen und idealerweise wachsenden Token-Preis und eine stabile Marktkapitalisierung aufrechtzuerhalten.
  \item Bedeutung von Marktkapitalisierung des Preises des Tokens:
  \begin{itemize}
  	\item Ein zu volatiler Token-Preis erschwert jegliche Art von Planung mit dem Token als Wert. Auf Unternehmensseite beispielsweise hinsichtlich Planung von Ausgaben und Wachstum . Und auf Userseite hinsichtlich Einsatz als Ressource/Utility.
  	\item Der Token bestimmt das Umsatzpotenzial des Projekts.
  	\item Tokens stellen ggf. einen relevante Anteil des Vergütungspakets von Mitarbeitern dar, sodass eine solide Wirtschaftlichkeit des Utility-Tokens die Bindung von Top-Talenten ermöglicht. 
  \end{itemize}
\end{itemize}

\end{Zitat*}

\vspace{0.5cm}
