% !TEX root = paper.tex

\todo{WIP}

\begin{itemize} 
	\item Business-Case (aus Investoren-Sicht) vorrechnen
	\begin{itemize}
		\item Wirtschaftlichkeit und Preisentwicklung
		\item (praktische) Obergrenze des eingebrachten Gesamtkapitals annehmen, mit der eine plausible und attraktive Rendite argumentiert werden kann.
	\end{itemize}
	\item Excel verlinken
\end{itemize}

\vspace{0.6cm}

Der \nameref{sec:fees} folgen einige Annahmen hinsichtlich des \textbf{Business-Plans} für eine Größenordnung von zwölf Monaten.

\vspace{0.3cm}

\begin{Assumption}[Business-Plan]\label{bp}

\vspace{0.75cm}

\todo{TODO: Zahlen an Excel anpassen}

\vspace{0.75cm}

Zunächst schätzen wir einige KPI ab, die es natürlich zu validieren gilt:

\begin{itemize}
	\item Wir gehen im Mittel von ca. 4-5 Teilnehmern je Pool aus.
	\item Wir gehen von einem durchschnittlichen Deposit von 200\$ je Teilnehmer und Pool aus - also einem durchschnittlichen initialen Pool-Kapital von 800-1000\$.
	\item Wir gehen des Weiteren von einer durchschnittlichen \textit{Pool-Lifetime} von ca. 6 Monaten aus,
	\item schätzen die durchschnittliche Anzahl an Tradings während der Pool-Lifetime auf 10-15,
	\item deren Trading-Volumen auf etwa $\frac{1}{3}$ des initialen Pool-Kapitals und schließlich 
	\item und einen daraus resultierenden konservativen mittleren Profit von 2.5 \% (auf die Pool-Lifetime von 6 Monaten also 5-6 \% p.a.).
	\item Zuletzt schätzen wir, jeder User betreibe im Mittel 2-3 Pools gleichzeitig.
\end{itemize}

\vspace{0.5cm}

Diesen geschätzten KPI zugrundeliegend setzen wir uns folgende Ziele hinsichtlich initiierter (gebührenpflichtiger) Pools - ungeachtet dessen, ob diese zu dem gegebenen Zeitpunkt noch existieren oder bereits liquidiert wurden:

\begin{itemize}
	\item 50 initiierte Pools nach 3 Monaten
	\item 150 initiierte Pools nach 6 Monaten
	\item 500 initiierte Pools nach 12 Monaten
\end{itemize}

\end{Assumption}

\vspace{0.5cm}

Damit ergeben sich folgende Business-Key-KPI:

\vspace{0.3cm}

\begin{Fazit}[Umsätze \& Forecast]

\vspace{0.75cm}

\todo{TODO: Zahlen an Excel anpassen}

\vspace{0.75cm}

Für einen durchschnittlichen Pool $\mathcal{P}$ mit dem initialen Pool-Kapital

\begin{equation*}
  vol^{\mathcal{P}} = 4.5 \cdot 200\$ = 900\$ 
\end{equation*}

approximieren wir die anfallenden Fees als Summe der Fee-Bestandteile

\begin{itemize}
	\item Grundgebühren: $fees_{G}^{\mathcal{P}} = \rho(nft) \cdot 0.019 \cdot (4.5 - 1) \cdot 200\$ $
	\item Trading-Gebühren: $fees_{T}^{\mathcal{P}} = \phi(nft) \cdot 0.001 \cdot 12.5 \cdot \frac{1}{3} \cdot vol^{\mathcal{P}} $
	\item Profit-Beteiligung: $fees_{P}^{\mathcal{P}} = \phi(nft) \cdot 0.099 \cdot 0.025 \cdot vol^{\mathcal{P}} $
\end{itemize}

wobei $\rho(nft)$ und $\phi(nft)$ Normierungsfaktoren darstellen, die die in Annahme \ref{fees} beschriebenen \textit{Benefits für PassNFT-Besitzer} berücksichtigen sollen, und von uns als 

\begin{itemize}
	\item $\rho(nft) \approx \frac{9}{10}$ und 
	\item $\phi(nft) \approx \frac{7}{8}$
\end{itemize}	

geschätzt werden sollen \todo{(für die Anfangsphase sind diese eher zu klein, im einge\-schwungenen Zustand viel zu groß)}.

\vspace{0.2cm}

Damit belaufen sich die einzelnen Fees-Bestandteile auf 

\begin{itemize}
	\item Grundgebühren: $fees_{G}^{\mathcal{P}} \approx 11.97\$ $
	\item Trading-Gebühren: $fees_{T}^{\mathcal{P}} \approx 3.28 \$ $
	\item Profit-Beteiligung: $fees_{P}^{\mathcal{P}} \approx 1.95 \$ $
\end{itemize}

und damit die im Mittel erwarteten Gesamt-Fees pro Pool auf

\begin{equation*}
  fees^{\mathcal{P}} = fees_{G}^{\mathcal{P}} + fees_{T}^{\mathcal{P}} + fees_{P}^{\mathcal{P}} \approx 17.20 \$. 
\end{equation*}

\vspace{0.67cm}

Bei einer Staking-Anforderung von 200 \% der geschätzten Pool-Fees \todo{(auf Staking verlinken)} und den in \ref{bp} getroffenen Business-Plan-Annahmen ergeben sich folgende näherungsweisen Forecasts:

\begin{itemize}
	\item Nach 3 Monaten: ca. 40 noch aktive und bereits ca. 10 liquidierte Pools.
	\begin{itemize}
		\item bereits \textit{umgesetzte Fees}: 266 \$ 
		\item \textit{Pending-Fees}: 1.064 \$ 
		\item \textit{Staked-Fees}: 2.129 \$ 
	\end{itemize}
	\item Nach 6 Monaten: ca. 100 noch aktive und bereits ca. 50 liquidierte Pools.
	\begin{itemize}
		\item bereits \textit{umgesetzte Fees}: 1.330 \$ 
		\item \textit{Pending-Fees}: 2.661 \$ 
		\item \textit{Staked-Fees}: 5.322 \$ 
	\end{itemize}
	\item Nach 12 Monaten: ca. 300 noch aktive und bereits ca. 200 liquidierte Pools.
	\begin{itemize}
		\item bereits \textit{umgesetzte Fees}: 5.322 \$ 
		\item \textit{Pending-Fees}: 7.983 \$ 
		\item \textit{Staked-Fees}: 15.966 \$ 
	\end{itemize}	 
\end{itemize}

\vspace{0.5cm}

Zu guter Letzt noch eine sehr bullishe Prognose:

\begin{itemize}
	\item Nach 5 Jahren: 450.000 noch aktive und bereits 550.000 liquidierte Pools.
	\begin{itemize}
		\item bereits \textit{umgesetzte Fees}: $\approx$ 15 Mio. \$ 
		\item \textit{Pending-Fees}: $\approx$ 12 Mio. \$ 
		\item \textit{Staked-Fees}: $\approx$ 24 Mio. \$ 
	\end{itemize}	 
\end{itemize}

\end{Fazit}

\vspace{0.5cm}


\todo{Ende WIP}