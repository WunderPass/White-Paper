% !TEX root = paper.tex


\subsubsection{Beispielrechnung}
\vspace{0.2cm}

\todo{WIP}

\begin{itemize}
	\item Daten, Zahlen, Fakten
	\item Profiterwartung aus Sicht eines Token-Holders/-Investors
\end{itemize}

\vspace{0.5cm}

Wir rechnen ein bisschen rum, um ein Gefühl für den nötigen Token-Supply zu bekommen:

\vspace{0.3cm}

\begin{Example}[Rechnerei zum Token-Supply]

\vspace{0.75cm}

\todo{TODO: Zahlen an Excel anpassen}

\vspace{0.75cm}

Wir peilen den Token-Contract so zu stricken, dass wir im eingeschwungenen Zustand einen Tokenwert des \textit{W-PLT} von $\approx$ 1 Cent anpeilen, aber gleichzeitig auch die Grenzen $[$0.5 Cent; 2 Cent$]$ im Auge behalten.

\vspace{0.5cm}

Wir forcieren beim Projekt-Fortschritt über die Zeit hinsichtlich des Tokens

\begin{itemize}
	\item Einen günstigen \textit{W-PLT}-Preis für die Gründer/Company ($\approx$ 0.25 Cent pro Token)
	\item Einen guten \textit{W-PLT}-Preis für ganz frühe Investoren ($<<$ 0.1 Cent)
	\item Einen \textit{W-PLT}-Preis von $<$ 1 Cent für die Early-Pool-User bis zum eingeschwungenen Zustand.
	\item Einen kontrollierten \textit{W-PLT}-Preis $<$ 2 Cent für die Pool-User im eingeschwungenen Zustand.
	\item Ein zunehmendes Ziel-Projekt-Invest mit Fortschritt des Projekts.
	\item Einen zunehmenden (aber kontrollierten) Ziel-Supply von \textit{W-PLT} mit Fortschritt des Projekts.
	\item Einen zunehmenden (aber kontrollierten) \textit{W-PLT}-Kurs mit Fortschritt des Projekts.
	\item Einen zunehmenden \textit{Utility-Koeffizient} (als Verhältnis zwischen mindest und Ziel-Supply) mit Fortschritt des Projekts bis zu Zielzustand des Koeffizienten von 50 \%.
\end{itemize}

\vspace{1.0cm}

Im Folgenden wieder die obige Forecast-Aufstellung - nun aus Token-Sicht:


\begin{itemize}
	\item Nach 3 Monaten: ca. 40 noch aktive und bereits ca. 10 liquidierte Pools.
	\begin{itemize}
		\item bereits \textit{umgesetzte Fees}: 25.000 \textit{W-PLT} 
		\item mindestens bereits geburnte Tokens: 12.500 \textit{W-PLT} 
		\item \textit{Pending-Fees}: 100.000 \textit{W-PLT}  
		\item \textit{Staked-Fees}: 200.000 \textit{W-PLT}
		\item min Supply: 300.000 \textit{W-PLT}
		\item Ziel-Projekt-Invest: 60.000 \$ \todo{(davon 30-40k durch Gründer/Company)}
		\item Projekt-Treasury: 60.000 \$ + 1.000 \$ Fees-Cash $\approx$ 61.000 \$
		\item Ziel-Supply: 12.0 Mio. \textit{W-PLT}
		\item \textit{Utility-Koeffizient}: $\frac{300.000}{12.000.000} = 2.5 \%$
		\item $\varnothing$ Kaufpreis pro \textit{W-PLT}: 0.5 Cent
		\item Mindest-Value pro \textit{W-PLT}: $\approx$ 0.51 Cent
	\end{itemize}
	\item Nach 6 Monaten: ca. 100 noch aktive und bereits ca. 50 liquidierte Pools.
	\begin{itemize}
		\item bereits \textit{umgesetzte Fees}: 130.000 \textit{W-PLT}
		\item mindestens bereits geburnte Tokens: 65.000 \textit{W-PLT}  
		\item \textit{Pending-Fees}: 250.000 \textit{W-PLT}
		\item \textit{Staked-Fees}: 500.000 \textit{W-PLT} 
		\item min Supply: 750.000 \textit{W-PLT}
		\item Ziel-Projekt-Invest: 120.000 \$
		\item Projekt-Treasury: 120.000 \$ + 2.500 \$ Fees-Cash $\approx$ 122.500 \$
		\item Ziel-Supply: 16.0 Mio. \textit{W-PLT}
		\item \textit{Utility-Koeffizient}: $\frac{750.000}{16.000.000} = 4.6875 \%$
		\item $\varnothing$ Kaufpreis pro \textit{W-PLT}: 0.75 Cent
		\item Mindest-Value pro \textit{W-PLT}: $\approx$ 0.77 Cent
	\end{itemize}
	\item Nach 12 Monaten: ca. 300 noch aktive und bereits ca. 200 liquidierte Pools.
	\begin{itemize}
		\item bereits \textit{umgesetzte Fees}: 500.000 \textit{W-PLT}  
		\item mindestens bereits geburnte Tokens: 250.000 \textit{W-PLT}
		\item \textit{Pending-Fees}: 800.000 \textit{W-PLT}  
		\item \textit{Staked-Fees}: 1.6 Mio. \textit{W-PLT} 
		\item min Supply: 2.4 Mio \textit{W-PLT} 
		\item Ziel-Projekt-Invest: 200.000 \$
		\item Projekt-Treasury: 200.000 \$ + 10.000 \$ Fees-Cash $\approx$ 210.000 \$
		\item Ziel-Supply: 20.0 Mio. \textit{W-PLT}
		\item \textit{Utility-Koeffizient}: $\frac{2.400.000}{20.000.000} = 12.0 \%$
		\item $\varnothing$ Kaufpreis pro \textit{W-PLT}: 1 Cent
		\item Mindest-Value pro \textit{W-PLT}: $\approx$ 1.05 Cent
	\end{itemize}
	\item Nach 3 Jahren: ca. 20.000 noch aktive und bereits ca. 20.000 liquidierte Pools.
	\begin{itemize}
		\item bereits \textit{umgesetzte Fees}: 50 Mio. \textit{W-PLT}  
		\item mindestens bereits geburnte Tokens: 25 Mio. \textit{W-PLT}
		\item \textit{Pending-Fees}: 50 Mio. \textit{W-PLT}  
		\item \textit{Staked-Fees}: 100 Mio. \textit{W-PLT} 
		\item min Supply: 150 Mio \textit{W-PLT} 
		\item Ziel-Projekt-Invest: 20.0 Mio. \$
		\item Projekt-Treasury: 20.0 Mio. \$ + 0.5 Mio \$ Fees-Cash $\approx$ 20.5 Mio. \$
		\item Ziel-Supply: 1.4 Mrd. \textit{W-PLT}
		\item \textit{Utility-Koeffizient}: $\frac{150.000.000}{1.400.000.000} = 12.0 \%$
		\item $\varnothing$ Kaufpreis pro \textit{W-PLT}: $\approx$ 1.43 Cent
		\item Mindest-Value pro \textit{W-PLT}: $\approx$ xxx Cent
	\end{itemize}
	\item Nach 5 Jahren: 450.000 noch aktive und bereits 550.000 liquidierte Pools.
	\begin{itemize}
		\item bereits \textit{umgesetzte Fees}: 1.5 Mrd. \textit{W-PLT} 
		\item mindestens bereits geburnte Tokens: 750 Mio. \textit{W-PLT}
		\item \textit{Pending-Fees}: 1.2 Mrd. \textit{W-PLT}
		\item \textit{Staked-Fees}: 2.4 Mrd. \textit{W-PLT} 
		\item min Supply: 3.6 Mrd. \textit{W-PLT} 
		\item Ziel-Projekt-Invest: 144 Mio. \$
		\item Projekt-Treasury: 144 Mio. \$ + 16 Mio. \$ Fees-Cash $\approx$ 160 Mio. \$
		\item Ziel-Supply: 7.2 Mrd. \textit{W-PLT}
		\item \textit{Utility-Koeffizient}: $\frac{3.6}{7.2} = 50.0 \%$
		\item $\varnothing$ Kaufpreis pro \textit{W-PLT}: 2 Cent
		\item Mindest-Value pro \textit{W-PLT}: $\approx$ 2.22 Cent		
	\end{itemize}	 
\end{itemize}

\end{Example}

\vspace{0.5cm}

\todo{Ende WIP}

\vspace{0.5cm}



\subsubsection{Recap \& Ausblick}
\vspace{0.2cm}

\todo{Einbettung in das Wunder-Ökosystem und Link zwischen WPT zu WUNDER}


\vspace{0.5cm}



\subsubsection{Unberücksichtigte Inhalte \& Ideen}

\vspace{0.3cm}
\todo{WIP}
\vspace{0.5cm}

\begin{Praemisse}[Cash-Flow]

\begin{itemize}
	\item Deposit/Invest erfolgt in einem Stable-Coin (z.B. \textit{USDT}). \todo{Es sind aber auch mehrere zulässige Währungen denkbar.}
	\item Cashout erfolgt in derselben Währung wie der Deposit \todo{(oder zumindest in einer der zulässigen Währungen)}. 
	\item Fees werden in aller Regel prozentual am Volumen (also in \textit{USDT}) berechnet, jedoch in \textbf{\textit{W-PLT}} veranschlagt, bei dem man von Kursschwankungen ausgehen muss und ausgehen will. \todo{(Das muss sowohl bei der Token-Modellierung als evtl. auch bei der Gebührenordnung berücksichtigt werden.)}
	\item Ein Teil der Fees soll direkt an den (Bonding-Curves-basierten) \textit{W-PLT}-Contract gehen und damit die \textit{W-PLT}-Investoren/-Hodler belohnen.
	\item Die Fees sollen \todo{(aufgrund des genauer zu erklärenden Stakings)} erst bei der Liquidierung des Pools entrichtet werden.
	\item Die Fees sollen von allen Pool-Teilnehmers außer des Pool-Creators getragen werden.
	\item Für den Pool-Creator soll folgendes gelten:
	\begin{itemize}
		\item \todo{(Muss einen PassNFT besitzen.)}
		\item Soll einen prozentual an den geschätzten gesamten Pool-Fees gemessenen Betrag $x$ als Sicherheit staken ($x \in [50 \%; 200 \%]$). \todo{Kann theoretisch auch einer absoluten oder relativen Obergrenze unterliegen.}
		\item Soll selbst keine Fees bezahlen.
		\item Soll für das Staken mit einem Teil der erwirtschafteten Gebühren entlohnt werden.
	\end{itemize}
\end{itemize}

\end{Praemisse}

\vspace{0.5cm}



\begin{Algo}[\textit{W-PLT}-Design]

\begin{itemize}
	\item Wir haben Szenarien vorgerechnet und dabei den zugehörigen \textit{Ziel-Supply} angegeben. Daran orientieren wir uns.
	\item Abhängig des Szenarios (gemappt auf den \textit{Ziel-Supply}) können wir einen aus Fees-Einnahmen resultierenden erwarteten Profit ableiten - und zwar pro Zeiteinheit, die genau der durchschnittlichen Pool-Lifetime entspricht.
	\item Wir legen einen Zeitraum als Vielfaches der durchschnittlichen Pool-Lifetime fest, die wir einem Token-Investor zumuten, bis sein Invest profital wird. Z. B. 4 $\cdot$ Pool-Lifetime.
	\item Aus den obigen Daten können wir für diesen Zeitraum (ab Invest beim entsprechenden Supply) den erwarteten relativen Profit pro Token $x$ approximieren.
	\item Diesen erwarteten relativen Profit pro Token müssen wir bei den obigen Rechnungen noch als $x(supply) = profit(zeitraum; supply)$ ergänzen.
	\item $kaufpreis(supply) := x(supply) \cdot \frac{treasury}{supply}$
	\item Den Early-Investoren darf durchaus ein größerer Faktor zugemutet werden.
	\item Wir errechnen für einige supply-Milestones den Kaufpreis auf der Grundlage des letzten Bullets und approximieren dann dazwischen.
\end{itemize}

\end{Algo}



\vspace{1.0cm}

\todo{WIP}

\vspace{0.3cm}

\begin{itemize}
	\item Wann bezahlen die User die Fees?
	\item In welcher Form/Währung dürfen die Fees von den Usern erbracht/verrechnet werden und wird dann alles im Hintergrund sofort in \textit{W-PLT} umgewandelt?
	\item Ist es denkbar die Fees aus dem Stake-Pool des Creators zu verwenden und diesem seinen Stake in einer anderen Währung zurückzuerstatten?
	\item Split der Gebühren auf Staker und Projekt-Treasury $(\sigma_{S}; \sigma_{T})$ mit $\sigma_{S} + \sigma_{T} = 1$ definieren. Dazu gibt es einige denkbare Varianten:
	\begin{itemize}
		\item fester, statischer Split
		\item fester, statischer Split mit eingebauten Unter- und Obergrenzen für den Gesamtertrag des Stakers $\sigma_{S} \cdot fees^{\mathcal{P}}$
		\item $fees^{\mathcal{P}}$-abhängiger (progressiver) Split, bei dem der Anteil des Stakers $\sigma_{S}$ mit zunehmendem $fees^{\mathcal{P}}$ stets kleiner wird. Dies unter Umständen ebenfalls unter Berücksichtigung eingebauter Unter- und Obergrenzen für den Staker.
		\item Begünstigung des Stakers in Abhängigkeit seines NFT-Pass-Status.
	\end{itemize}	
	\item Anfängliche Air-Drops von \textit{W-PLT} an erste Pool-User müssen einer Locking-Periode unterliegen.
	\item Problem: Wenn Token-Holder aussteigen, nehmen diese nicht nur ihren Anteil an den bisher erwirtschafteten Fees-Einnahmen mit, sondern drücken zudem auch noch den Token-Kurs nach unten. Das stellt einen sich selbst verstärkenden Effekt dar, der irgendwie in den Griff zu bekommen ist (evtl. Locking-Periode oder berücksichtigende Verkaufs-Kurve).
	\item Es macht eine Locking-Periode von der Dauer der durchschnittlichen Pool-Lifetime viel Sinn, da die beim Verkauf der Tokens mitgenommenen Gewinne aus Fees-Einnahmen durch neu generierte Fees-Einnahmen egalisiert werden und der Token-Value somit stabil bleibt.
	\item Die Einlage für den \textit{W-PLT} ist idealerweise in WUNDER zu erbringen \todo{(Es ist noch unklar, wie man an WUNDER kommt, wenn es vorher keinen Token-Sale gegeben hat. Ob der WUNDER ebenfalls mittels Bonding-Curves abzubilden wäre, sei hier erst einmal mehr als unklar.)}
	
\end{itemize}

\vspace{1.0cm}

\begin{Problem}[\textit{USDT} vs. \textit{W-PLT} als Berechnungsgrundlage für Fees, Staking etc.]
\vspace{0.2cm}

\todo{Was nehmen wir hier?}

\vspace{0.5cm}

\todo{Folgend übernommene alte Test-Passagen zu dem Thema:}

\vspace{0.5cm}

Ein weiterer sehr essenzieller Faktor für die Größe des zu stakenden Betrags könnte der Kurs des IPTs sein. Denn laut der \textbf{Bonding-Curves}-Implementierung würde der \textit{W-PLT}-Preis mit steigender Zirkulation steigen, was mit der Zunahme von existierende Pools geschähe. Damit wäre die Erstellung neuer Pools mit ihrer zahlen\-mäßigen Zunahme stets kapital-intensiver (aber nicht gleichbedeutend teurer). \textbf{Die Frage hierbei ist also, ob der zu erbringende Stake des Pool-Creators auf den \textit{Total-Supply des W-PLT} normiert werden sollte oder nicht}, die gänzlich mit der obigen Fragestellung einhergeht, ob der Pool-Creator eigentlich staken möchte oder das nur tun muss.
	
\begin{itemize}
	\item Gegen eine Normierung spricht die Annahme/Hoffnung, ein Pool-Creator sei gleichzeitig auch ein großer Supporter des gesamten Projekt und glaube daran. Wenn der \textit{W-PLT}-Preis steigt, ist dies gleichbedeutend mit der Zunahme an genutzten Pools, an denen der Pool-Creator als Staker, Besitzer von \textit{W-PLT} und damit Projekt-Investor auch selbst (finanziell) profitiert.
	\item Für eine Normierung spricht dagegen die potenzielle Gefahr, neue oder bestehende User durch eine zu hohe finanzielle Sicherheitseinlage davon abzuschrecken neue Pools zu erstellen.
\end{itemize}

\vspace{0.2cm}
	
Die Antwort auf diese Fragestellung könnte auch darin liegen, ob wir uns besonders viele oder lieber weniger aber besonders Teilnehmer-starke Pools wünschen.

\vspace{0.5cm}
	
Die \textit{Pool-Teilnehmer} (außer des Creators) können bei dieser Logik aber nicht wie nicht wie die Staker zusätzlich als Projekt-Investoren angesehen werden, weil sie \textit{W-PLT} kaufen, da die gekauften \textit{W-PLT} direkt als Gebühr entrichtet werden. Für die Pool-Teilnehmer stellt der \textit{W-PLT} also eher einen Utility- bzw. Purpose-Token dar weshalb die Höhe der zu entrichtenden Gebühr zweifelsfrei auf Basis von \textit{Total-Supply des W-PLT} normiert werden muss (die Gebühr darf keinesfalls mit Zunahme von Pools steigen).


\end{Problem}
