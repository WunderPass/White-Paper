% !TEX root = paper.tex


\subsubsection{Beispielrechnung}
\vspace{0.2cm}

\todo{WIP}

\begin{itemize}
	\item Daten, Zahlen, Fakten
	\item Profiterwartung aus Sicht eines Token-Holders/-Investors
\end{itemize}

\vspace{0.5cm}

Wir rechnen ein bisschen rum, um ein Gefühl für den nötigen Token-Supply zu bekommen:

\vspace{0.3cm}

\begin{Example}[Rechnerei zum Token-Supply]

\vspace{0.75cm}

\todo{TODO: Zahlen an Excel anpassen}

\vspace{0.75cm}

Wir peilen den Token-Contract so zu stricken, dass wir im eingeschwungenen Zustand einen Tokenwert des \textit{W-PLT} von $\approx$ 1 Cent anpeilen, aber gleichzeitig auch die Grenzen $[$0.5 Cent; 2 Cent$]$ im Auge behalten.

\vspace{0.5cm}

Wir forcieren beim Projekt-Fortschritt über die Zeit hinsichtlich des Tokens

\begin{itemize}
	\item Einen günstigen \textit{W-PLT}-Preis für die Gründer/Company ($\approx$ 0.25 Cent pro Token)
	\item Einen guten \textit{W-PLT}-Preis für ganz frühe Investoren ($<<$ 0.1 Cent)
	\item Einen \textit{W-PLT}-Preis von $<$ 1 Cent für die Early-Pool-User bis zum eingeschwungenen Zustand.
	\item Einen kontrollierten \textit{W-PLT}-Preis $<$ 2 Cent für die Pool-User im eingeschwungenen Zustand.
	\item Ein zunehmendes Ziel-Projekt-Invest mit Fortschritt des Projekts.
	\item Einen zunehmenden (aber kontrollierten) Ziel-Supply von \textit{W-PLT} mit Fortschritt des Projekts.
	\item Einen zunehmenden (aber kontrollierten) \textit{W-PLT}-Kurs mit Fortschritt des Projekts.
	\item Einen zunehmenden \textit{Utility-Koeffizient} (als Verhältnis zwischen mindest und Ziel-Supply) mit Fortschritt des Projekts bis zu Zielzustand des Koeffizienten von 50 \%.
\end{itemize}

\vspace{1.0cm}

Im Folgenden wieder die obige Forecast-Aufstellung - nun aus Token-Sicht:


\begin{itemize}
	\item Nach 3 Monaten: ca. 40 noch aktive und bereits ca. 10 liquidierte Pools.
	\begin{itemize}
		\item bereits \textit{umgesetzte Fees}: 25.000 \textit{W-PLT} 
		\item mindestens bereits geburnte Tokens: 12.500 \textit{W-PLT} 
		\item \textit{Pending-Fees}: 100.000 \textit{W-PLT}  
		\item \textit{Staked-Fees}: 200.000 \textit{W-PLT}
		\item min Supply: 300.000 \textit{W-PLT}
		\item Ziel-Projekt-Invest: 60.000 \$ \todo{(davon 30-40k durch Gründer/Company)}
		\item Projekt-Treasury: 60.000 \$ + 1.000 \$ Fees-Cash $\approx$ 61.000 \$
		\item Ziel-Supply: 12.0 Mio. \textit{W-PLT}
		\item \textit{Utility-Koeffizient}: $\frac{300.000}{12.000.000} = 2.5 \%$
		\item $\varnothing$ Kaufpreis pro \textit{W-PLT}: 0.5 Cent
		\item Mindest-Value pro \textit{W-PLT}: $\approx$ 0.51 Cent
	\end{itemize}
	\item Nach 6 Monaten: ca. 100 noch aktive und bereits ca. 50 liquidierte Pools.
	\begin{itemize}
		\item bereits \textit{umgesetzte Fees}: 130.000 \textit{W-PLT}
		\item mindestens bereits geburnte Tokens: 65.000 \textit{W-PLT}  
		\item \textit{Pending-Fees}: 250.000 \textit{W-PLT}
		\item \textit{Staked-Fees}: 500.000 \textit{W-PLT} 
		\item min Supply: 750.000 \textit{W-PLT}
		\item Ziel-Projekt-Invest: 120.000 \$
		\item Projekt-Treasury: 120.000 \$ + 2.500 \$ Fees-Cash $\approx$ 122.500 \$
		\item Ziel-Supply: 16.0 Mio. \textit{W-PLT}
		\item \textit{Utility-Koeffizient}: $\frac{750.000}{16.000.000} = 4.6875 \%$
		\item $\varnothing$ Kaufpreis pro \textit{W-PLT}: 0.75 Cent
		\item Mindest-Value pro \textit{W-PLT}: $\approx$ 0.77 Cent
	\end{itemize}
	\item Nach 12 Monaten: ca. 300 noch aktive und bereits ca. 200 liquidierte Pools.
	\begin{itemize}
		\item bereits \textit{umgesetzte Fees}: 500.000 \textit{W-PLT}  
		\item mindestens bereits geburnte Tokens: 250.000 \textit{W-PLT}
		\item \textit{Pending-Fees}: 800.000 \textit{W-PLT}  
		\item \textit{Staked-Fees}: 1.6 Mio. \textit{W-PLT} 
		\item min Supply: 2.4 Mio \textit{W-PLT} 
		\item Ziel-Projekt-Invest: 200.000 \$
		\item Projekt-Treasury: 200.000 \$ + 10.000 \$ Fees-Cash $\approx$ 210.000 \$
		\item Ziel-Supply: 20.0 Mio. \textit{W-PLT}
		\item \textit{Utility-Koeffizient}: $\frac{2.400.000}{20.000.000} = 12.0 \%$
		\item $\varnothing$ Kaufpreis pro \textit{W-PLT}: 1 Cent
		\item Mindest-Value pro \textit{W-PLT}: $\approx$ 1.05 Cent
	\end{itemize}
	\item Nach 3 Jahren: ca. 20.000 noch aktive und bereits ca. 20.000 liquidierte Pools.
	\begin{itemize}
		\item bereits \textit{umgesetzte Fees}: 50 Mio. \textit{W-PLT}  
		\item mindestens bereits geburnte Tokens: 25 Mio. \textit{W-PLT}
		\item \textit{Pending-Fees}: 50 Mio. \textit{W-PLT}  
		\item \textit{Staked-Fees}: 100 Mio. \textit{W-PLT} 
		\item min Supply: 150 Mio \textit{W-PLT} 
		\item Ziel-Projekt-Invest: 20.0 Mio. \$
		\item Projekt-Treasury: 20.0 Mio. \$ + 0.5 Mio \$ Fees-Cash $\approx$ 20.5 Mio. \$
		\item Ziel-Supply: 1.4 Mrd. \textit{W-PLT}
		\item \textit{Utility-Koeffizient}: $\frac{150.000.000}{1.400.000.000} = 12.0 \%$
		\item $\varnothing$ Kaufpreis pro \textit{W-PLT}: $\approx$ 1.43 Cent
		\item Mindest-Value pro \textit{W-PLT}: $\approx$ xxx Cent
	\end{itemize}
	\item Nach 5 Jahren: 450.000 noch aktive und bereits 550.000 liquidierte Pools.
	\begin{itemize}
		\item bereits \textit{umgesetzte Fees}: 1.5 Mrd. \textit{W-PLT} 
		\item mindestens bereits geburnte Tokens: 750 Mio. \textit{W-PLT}
		\item \textit{Pending-Fees}: 1.2 Mrd. \textit{W-PLT}
		\item \textit{Staked-Fees}: 2.4 Mrd. \textit{W-PLT} 
		\item min Supply: 3.6 Mrd. \textit{W-PLT} 
		\item Ziel-Projekt-Invest: 144 Mio. \$
		\item Projekt-Treasury: 144 Mio. \$ + 16 Mio. \$ Fees-Cash $\approx$ 160 Mio. \$
		\item Ziel-Supply: 7.2 Mrd. \textit{W-PLT}
		\item \textit{Utility-Koeffizient}: $\frac{3.6}{7.2} = 50.0 \%$
		\item $\varnothing$ Kaufpreis pro \textit{W-PLT}: 2 Cent
		\item Mindest-Value pro \textit{W-PLT}: $\approx$ 2.22 Cent		
	\end{itemize}	 
\end{itemize}

\end{Example}

\vspace{0.5cm}

\todo{Ende WIP}

\vspace{0.5cm}



\subsubsection{Recap \& Ausblick}
\vspace{0.2cm}

\todo{Einbettung in das Wunder-Ökosystem und Link zwischen WPT zu WUNDER}

\vspace{0.5cm}

Die Einlage für den \textit{WPT} ist idealerweise in WUNDER zu erbringen \todo{(Es ist noch unklar, wie man an WUNDER kommt, wenn es vorher keinen Token-Sale gegeben hat. Ob der WUNDER ebenfalls mittels Bonding-Curves abzubilden wäre, sei hier erst einmal mehr als unklar.)}

\vspace{0.5cm}



\subsubsection{Unberücksichtigte Inhalte \& Ideen}

\vspace{0.3cm}
\todo{WIP}
\vspace{0.5cm}


\begin{Algo}[\textit{W-PLT}-Design]
\label{approx}

\begin{itemize}
	\item Wir haben Szenarien vorgerechnet und dabei den zugehörigen \textit{Ziel-Supply} angegeben. Daran orientieren wir uns.
	\item Abhängig des Szenarios (gemappt auf den \textit{Ziel-Supply}) können wir einen aus Fees-Einnahmen resultierenden erwarteten Profit ableiten - und zwar pro Zeiteinheit, die genau der durchschnittlichen Pool-Lifetime entspricht.
	\item Wir legen einen Zeitraum als Vielfaches der durchschnittlichen Pool-Lifetime fest, die wir einem Token-Investor zumuten, bis sein Invest profital wird. Z. B. 4 $\cdot$ Pool-Lifetime.
	\item Aus den obigen Daten können wir für diesen Zeitraum (ab Invest beim entsprechenden Supply) den erwarteten relativen Profit pro Token $x$ approximieren.
	\item Diesen erwarteten relativen Profit pro Token müssen wir bei den obigen Rechnungen noch als $x(supply) = profit(zeitraum; supply)$ ergänzen.
	\item $kaufpreis(supply) := x(supply) \cdot \frac{treasury}{supply}$
	\item Den Early-Investoren darf durchaus ein größerer Faktor zugemutet werden.
	\item Wir errechnen für einige supply-Milestones den Kaufpreis auf der Grundlage des letzten Bullets und approximieren dann dazwischen.
\end{itemize}

\vspace{0.3cm}
\todo{\noindent\hrulefill}

\begin{itemize}
	\item Problem: Wenn Token-Holder aussteigen, nehmen diese nicht nur ihren Anteil an den bisher erwirtschafteten Fees-Einnahmen mit, sondern drücken zudem auch noch den Token-Kurs nach unten. Das stellt einen sich selbst verstärkenden Effekt dar, der irgendwie in den Griff zu bekommen ist (evtl. Locking-Periode oder berücksichtigende Verkaufs-Kurve).
	\item Es macht eine Locking-Periode von der Dauer der durchschnittlichen Pool-Lifetime viel Sinn, da die beim Verkauf der Tokens mitgenommenen Gewinne aus Fees-Einnahmen durch neu generierte Fees-Einnahmen egalisiert werden und der Token-Value somit stabil bleibt.	
\end{itemize}

\end{Algo}



