% !TEX root = paper.tex

\todo{Andere Kapitel müssen vorgezogen werden. Im Ergebnis leiten sich aber folgende Kurven für den Token ab:}

\vspace{0.3cm}

\begin{Solution}[Token-Curves]

Sei $s \in \mathbb{N}$ der Token-Supply und 

\begin{equation*}
p \approx 1.06
\end{equation*}

der \textit{Profit-Koeffizient} \todo{(erklären weshalb, wofür, warum)}.
Dann leiten sich Verkaufs- und Kaufpreis-Kurve wie folgt ab:

\vspace{0.2cm}

Verkaufspreis-Kurve:

\begin{equation*}
V(s) = V_{0} \cdot p^{ln\left(2 \cdot \frac{s}{s_{o}}\right) \cdot ln\left(\frac{s}{s_{o}}\right)}
\end{equation*}

\vspace{0.2cm}

Kaufpreis-Kurve:

\begin{align*}
K(s) &= K_{0} \cdot p^{ln\left(2 \cdot \frac{s}{s_{o}}\right) \cdot ln\left(\frac{s}{s_{o}}\right)} \cdot \left( ln(p) \cdot \left( 2 \cdot ln\left( \frac{s}{s_{o}} \right) + ln(2) \right) + 1 \right) \\
 &= V(s) \cdot \left( ln(p) \cdot \left( 2 \cdot ln\left( \frac{s}{s_{o}} \right) + ln(2) \right) + 1 \right)
\end{align*}

\vspace{0.4cm}

wobei $s_{0}$ für einen sehr kleinen (initialen) Supply und 

\begin{equation*}
K_{0} = K(s_{0}) = V(s_{0}) = V_{0}
\end{equation*}

für seinen initialen Kauf- und Verkaufskurs stehen und dabei übrigens ganz nebenbei 

\begin{equation*}
K(s) = \left( s \cdot V(s) \right)^{\prime}
\end{equation*}

gilt.

\end{Solution}
