% !TEX root = paper.tex

\todo{WIP}
\vspace{0.5cm}

\todo{Recap aus den vorigen Kapiteln und deren Input für das gegenständige Kapitel}
\vspace{0.5cm}

\paragraph{Token-Kurs-Kurven}
\textbf{ }
\vspace{0.3cm}

\todo{Hier müssen im Wesentlichen die Excel-Parameter erklärt werden}

\begin{itemize}
	\item die Annahme-Parameter in der Excel folgen im Wesentlichen aus den vorigen Kapiteln
	\item die einzelnen Kurven-Stufen erklären
	\item die tatsächlichen Bonding-Curves zu erklären wäre nicht ganz ohne. Kann man das einfach ausschweigen? Falls wir es genauer erklären wollen, wären die Ausführungen aus Algo \ref{approx} hilfreich.
\end{itemize}

\vspace{0.5cm}


\begin{Solution}[Token-Curves]

Sei $s \in \mathbb{N}$ der Token-Supply und 

\begin{equation*}
p \approx 1.06
\end{equation*}

der \textit{Profit-Koeffizient} \todo{(erklären weshalb, wofür, warum)}.
Dann leiten sich Verkaufs- und Kaufpreis-Kurve wie folgt ab:

\vspace{0.2cm}

Verkaufspreis-Kurve:

\begin{equation*}
V(s) = V_{0} \cdot p^{ln\left(2 \cdot \frac{s}{s_{o}}\right) \cdot ln\left(\frac{s}{s_{o}}\right)}
\end{equation*}

\vspace{0.2cm}

Kaufpreis-Kurve:

\begin{align*}
K(s) &= K_{0} \cdot p^{ln\left(2 \cdot \frac{s}{s_{o}}\right) \cdot ln\left(\frac{s}{s_{o}}\right)} \cdot \left( ln(p) \cdot \left( 2 \cdot ln\left( \frac{s}{s_{o}} \right) + ln(2) \right) + 1 \right) \\
 &= V(s) \cdot \left( ln(p) \cdot \left( 2 \cdot ln\left( \frac{s}{s_{o}} \right) + ln(2) \right) + 1 \right)
\end{align*}

\vspace{0.4cm}

wobei $s_{0}$ für einen sehr kleinen (initialen) Supply und 

\begin{equation*}
K_{0} = K(s_{0}) = V(s_{0}) = V_{0}
\end{equation*}

für seinen initialen Kauf- und Verkaufskurs stehen und dabei übrigens ganz nebenbei 

\begin{equation*}
K(s) = \left( s \cdot V(s) \right)^{\prime}
\end{equation*}

gilt.

\end{Solution}

\vspace{0.5cm}
\todo{Eine Abbildung der Preis-Kurve(n) wäre nichr verkehrt.}
\vspace{0.5cm}


\paragraph{Staking \& Gewinn-Split}
\textbf{ }
\vspace{0.3cm}

\todo{WIP}
\vspace{0.5cm}

Split der Gebühren auf Staker und Projekt-Treasury $(\sigma_{S}; \sigma_{T})$ mit $\sigma_{S} + \sigma_{T} = 1$ definieren. Dazu gibt es einige denkbare Varianten:

\begin{itemize}
	\item fester, statischer Split
	\item fester, statischer Split mit eingebauten Unter- und Obergrenzen für den Gesamtertrag des Stakers $\sigma_{S} \cdot fees^{\mathcal{P}}$
	\item $fees^{\mathcal{P}}$-abhängiger (progressiver) Split, bei dem der Anteil des Stakers $\sigma_{S}$ mit zunehmendem $fees^{\mathcal{P}}$ stets kleiner wird. Dies unter Umständen ebenfalls unter Berücksichtigung eingebauter Unter- und Obergrenzen für den Staker.
	\item Begünstigung des Stakers in Abhängigkeit seines NFT-Pass-Status.
\end{itemize}

\vspace{0.5cm}


\paragraph{Workflows}
\textbf{ }
\vspace{0.3cm}

\todo{WIP}
\vspace{0.5cm}

\begin{itemize}
	\item Wann bezahlen die User die Fees?
	\item In welcher Form/Währung dürfen die Fees von den Usern erbracht/verrechnet werden und wird dann alles im Hintergrund sofort in \textit{WPT} umgewandelt?
	\item Ist es denkbar die Fees aus dem Stake-Pool des Creators zu verwenden und diesem seinen Stake in einer anderen Währung zurückzuerstatten?
\end{itemize}

\vspace{0.5cm}

\paragraph{Problem}
\textbf{ }
\vspace{0.3cm}	

\begin{Problem}[\textit{USDT} vs. \textit{W-PLT} als Berechnungsgrundlage für Fees, Staking etc.]
\vspace{0.2cm}

\todo{Was nehmen wir hier?}

\vspace{0.5cm}

\todo{Folgend übernommene alte Test-Passagen zu dem Thema:}

\vspace{0.5cm}

Ein weiterer sehr essenzieller Faktor für die Größe des zu stakenden Betrags könnte der Kurs des IPTs sein. Denn laut der \textbf{Bonding-Curves}-Implementierung würde der \textit{W-PLT}-Preis mit steigender Zirkulation steigen, was mit der Zunahme von existierende Pools geschähe. Damit wäre die Erstellung neuer Pools mit ihrer zahlen\-mäßigen Zunahme stets kapital-intensiver (aber nicht gleichbedeutend teurer). \textbf{Die Frage hierbei ist also, ob der zu erbringende Stake des Pool-Creators auf den \textit{Total-Supply des W-PLT} normiert werden sollte oder nicht}, die gänzlich mit der obigen Fragestellung einhergeht, ob der Pool-Creator eigentlich staken möchte oder das nur tun muss.
	
\begin{itemize}
	\item Gegen eine Normierung spricht die Annahme/Hoffnung, ein Pool-Creator sei gleichzeitig auch ein großer Supporter des gesamten Projekt und glaube daran. Wenn der \textit{W-PLT}-Preis steigt, ist dies gleichbedeutend mit der Zunahme an genutzten Pools, an denen der Pool-Creator als Staker, Besitzer von \textit{W-PLT} und damit Projekt-Investor auch selbst (finanziell) profitiert.
	\item Für eine Normierung spricht dagegen die potenzielle Gefahr, neue oder bestehende User durch eine zu hohe finanzielle Sicherheitseinlage davon abzuschrecken neue Pools zu erstellen.
\end{itemize}

\vspace{0.2cm}
	
Die Antwort auf diese Fragestellung könnte auch darin liegen, ob wir uns besonders viele oder lieber weniger aber besonders Teilnehmer-starke Pools wünschen.

\vspace{0.5cm}
	
Die \textit{Pool-Teilnehmer} (außer des Creators) können bei dieser Logik aber nicht wie nicht wie die Staker zusätzlich als Projekt-Investoren angesehen werden, weil sie \textit{W-PLT} kaufen, da die gekauften \textit{W-PLT} direkt als Gebühr entrichtet werden. Für die Pool-Teilnehmer stellt der \textit{W-PLT} also eher einen Utility- bzw. Purpose-Token dar weshalb die Höhe der zu entrichtenden Gebühr zweifelsfrei auf Basis von \textit{Total-Supply des W-PLT} normiert werden muss (die Gebühr darf keinesfalls mit Zunahme von Pools steigen).

\end{Problem}

\vspace{0.5cm}


\paragraph{Fazit}
\textbf{ }
\vspace{0.3cm}

\todo{Gibt es noch Ungeklärtheiten, ohne die sich kein Token-Contract schreiben lässt?}

