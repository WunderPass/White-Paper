\documentclass[11pt]{scrartcl}


%---------Konfiguration------


%-----Sprache und Zeichen----
\usepackage[utf8]{inputenc}
% \usepackage{ucs}
% \usepackage[T1]{fontenc}

% Zeilenumbrüche der deutschen Sprache
% \usepackage[ngerman]{babel}


%---------Farben-------------
\usepackage[RGB]{xcolor}
\definecolor{dunkelgruen}{RGB}{0 136 0}
\newcommand\todo[1]{\textcolor{red}{#1}}


%---------Links/Refs---------
\usepackage[colorlinks=true, urlcolor=blue]{hyperref}


%---------Grafiken-----------
\usepackage{graphicx}
\usepackage{subfig}



%---------Sonstiges----------
\usepackage{parcolumns}
\usepackage{enumitem}



%---------Mathe--------------
\usepackage{amsthm, amssymb, mathtools}
% \usepackage{amsmath, amssymb, amstext ,lipsum}



%---------Umgebungen---------
\usepackage[framemethod=tikz]{mdframed}

%---------Business---------
\mdtheorem[
  linecolor=dunkelgruen,
  frametitlefont=\sffamily\bfseries\color{white},
  frametitlebackgroundcolor=dunkelgruen,
]{Business-Def}{Definition}

\mdtheorem[
  linecolor=dunkelgruen,
  frametitlefont=\sffamily\bfseries\color{white},
  frametitlebackgroundcolor=dunkelgruen,
]{Hypothese}{Hypothese}

\mdtheorem[
  linecolor=gray,
  frametitlefont=\sffamily\bfseries\color{white},
  frametitlebackgroundcolor=gray,
]{Praemisse}{Prämisse}

\mdtheorem[
  linecolor=red,
  frametitlefont=\sffamily\bfseries\color{white},
  frametitlebackgroundcolor=red,
]{Abgrenzung}{Abgrenzung}

\mdtheorem[
  linecolor=violet,
  frametitlefont=\sffamily\bfseries\color{white},
  frametitlebackgroundcolor=violet,
]{Quelle}{Quellen}

\mdtheorem[
  linecolor=violet,
  frametitlefont=\sffamily\bfseries\color{white},
  frametitlebackgroundcolor=violet,
]{Zitat}{Zitat}

\mdtheorem[
  linecolor=blue,
  frametitlefont=\sffamily\bfseries\color{white},
  frametitlebackgroundcolor=blue,
]{Fazit}{Conclusion}

\mdtheorem[
  linecolor=red,
  frametitlefont=\sffamily\bfseries\color{white},
  frametitlebackgroundcolor=red,
]{Problem}{Problem}

\mdtheorem[
  linecolor=cyan,
  frametitlefont=\sffamily\bfseries\color{black},
  frametitlebackgroundcolor=cyan,
]{Solution}{Lösung}

\mdtheorem[
  linecolor=dunkelgruen,
  frametitlefont=\sffamily\bfseries\color{white},
  frametitlebackgroundcolor=dunkelgruen,
]{Konzept}{Konzept}


\mdtheorem[
  linecolor=dunkelgruen,
  frametitlefont=\sffamily\bfseries\color{white},
  frametitlebackgroundcolor=dunkelgruen,
]{NFT-Prop}{NFT-Property}



%---------Mathe------------
\mdtheorem[
  linecolor=gray,
  frametitlefont=\sffamily\bfseries\color{white},
  frametitlebackgroundcolor=gray,
]{Def}{Definition}

\mdtheorem[
  linecolor=dunkelgruen,
  frametitlefont=\sffamily\bfseries\color{white},
  frametitlebackgroundcolor=dunkelgruen,
]{Theorem}{Theorem}

\mdtheorem[
  linecolor=blue,
  frametitlefont=\sffamily\bfseries\color{white},
  frametitlebackgroundcolor=blue,
]{Lemma}{Lemma}

\mdtheorem[
  linecolor=red,
  frametitlefont=\sffamily\bfseries\color{white},
  frametitlebackgroundcolor=red,
]{Assumption}{Annahme}

\mdtheorem[
  linecolor=violet,
  frametitlefont=\sffamily\bfseries\color{white},
  frametitlebackgroundcolor=violet,
]{Example}{Beispiel}

\mdtheorem[
  linecolor=cyan,
  frametitlefont=\sffamily\bfseries\color{black},
  frametitlebackgroundcolor=cyan,
]{Algo}{Algorithmus}




%---------Dokument-----------

%---------Titel--------------
\title{WunderPools}
\author{WunderTeam}
\date{\today{}, Berlin}
 
%---------Inhalt-------------
\begin{document}


\maketitle
\tableofcontents{}

\newpage



\section{Einleitung}
\label{sec:pools-einleitung}
\vspace{0.3cm}
% !TEX root = paper.tex

Die Idee hinter den sogenannten \textit{Wunder-Pools} ist das Bündeln von Liquidität mehrerer User/Teilnehmer bzw. eine Art 'Treuehandverwahrung' in einem gemeinsamen Pool. Die Anwendungsfälle solche Pools können sehr zahlreich sein. Um im Folgenden nur einige Beispiele zu nennen:  

\begin{itemize}
  \item Gemeinsame Invests in (Crypto-)Assets.
  \item Pool für ein gemeinsames (Geburtstags-)Geschenk.
  \item Kicktipp-Pool (der über die gesamte Saison verwahrt werden muss).
  \item Wetten unter Freunden (z. B. Sportereinisse wie ein WM-Finale).
  \item Ausgleichspool für Auslagen von Geld an Freunde (Splitwise).
\end{itemize}

\vspace{0.2cm}

Das besondere an dem in den folgenden Abschnitten genauer zu beschreibenden Mo\-dell ist sein sehr allgemein gehaltener Ansatz, mit dem sich gleichzeitig Cases umsetzen lassen, die auf den ersten Blick sehr verschieden zu sein scheinen. Genauer genommen lassen sich solche Pools mit speziellen \textit{DAO-Strukturen} beschreiben.

Abgesehen von der den Pools zugrundeliegenden Geschäftslogik besteht der zentrale Ansatz unserer \textit{Wunder-Pools} darin, dem User ein rundes Produkt anzubieten - und zwar gänzlich unabhängig davon, welcher der oben genannten Cases nun tatsächlich umgesetzt werden soll. An dieser Stelle möchten wir uns daher ganz explizit von dem Status quo der heute gängigen UX in der Web3-Welt abgrenzen.

\vspace{0.4cm}

Ganz grob beschrieben, streben wir in etwa folgende Geschäftslogik an:

\begin{itemize}
  \item Ein User erstellt einen Pool (in unserer eigens designten Wunder-Pool-UI).
  \item Derselbe User wählt die gewünschte \textit{Pool-Art}, ein etwaiges dazugehöriges Regelwerk und fordert andere User auf, dem Pool beizutreten. Idealerweise erfolgt die Einladung mittels Suche nach der Wunder-ID bzw. eines sprechenden Namens des einzuladenden Teilnehmers (und nicht etwa anhand seiner Ethereum-Adresse oder sonstigem).
  \item Die eingeladenen Teilnehmer erhalten die Einladung (in der WunderPass-App oder der Wunder-Pool-Applikation) und können entscheiden, ob sie dem Pool beitreten möchten oder nicht. 
  \item In der Regel ist der definierte Einsatz sofort beim Beitritt des Pools zu entrichten und geht direkt in die Pool-Treasury. In einigen Cases kann der Einsatz evtl. auch zu einem späteren Zeitpunkt erfolgen oder gar ganz entfallen (z.B. beim Case \textit{Splitwise}).
  \item Der eingerichtete Geldpool kann nun als Gemeinschaftsvermögen/ -konto zu u. a. folgenden Zwecken verwendet werden:
  \begin{itemize}
  	\item zum gemeinschaftlichen Investieren in (Crypto-)Assets,
  	\item zum Verwahren \textit{"in Treuhand"} bei einer oder mehreren abgeschlossenen Wetten (oder auch z. B. Kicktipp)
  	\item etc.
  \end{itemize}
  \item Der Pool wird liquidiert und das gemeinschaftliche Geld (nach einem aus dem vorher gemeinsam festgelegten Regelwerk folgenden Verteilungsschlüssel) auf alle Pool-Mitglieder verteilt. Die Liquidierung selbst kann entweder ebenfalls durch das Regelwerk auf einen bestimmten Zeitpunkt und/oder Ereignis terminiert sein (z.B. Ende einer BuLi-Saison beim Case \textit{Kicktipp}) oder aber durch die Teilnehmer beschlossen werden (mittels einer DAO-Abstimmung). Die Errechnung des genann\-ten Verteilungsschlüssels möchten wir möglichst allgemein halten und übertragen diese Verantwortlichkeit einem \textit{abstrakten Oracle}, welches es stets Case-spezifisch zu definieren (und zu implementieren) gilt.
\end{itemize}

\vspace{0.2cm}

\underline{\textbf{Product-Sicht}}

\vspace{0.2cm}

Abschließend sei noch einmal betont, dass wir das/die aus den Wunder-Pools hervorgehende(n) Product(s) (mittelfristig) alternativlos user-friendly sehen. Ohne notwendigen Bezug zur Crypto-Szene, ohne MetaMask und ohne kryptische hexadezimale Wallet-Adressen. Stattdessen clean und simpel.

\vspace{0.5cm}    % binde die Datei '[Pools][Einleitung].tex' ein


\section{Formalisierungen}
\vspace{0.3cm}
% !TEX root = paper.tex

Zunächst einmal benötigen wir einige formale Werkzeuge und bedienen uns dafür folgender Definition:

\vspace{0.2cm}

\begin{Def}\label{defPoolTeilnehmer}

Im folgenden setzen wir voraus, die Nutzung der Pools seitens der User erfordert zwingend den Besitz eines WunderPass (bzw. Wunder-ID) und betrachten von daher auch nur solche User.

\begin{equation*}
  U := \left\{ u_1; u_2;...; u_{n} \text{ } | \text{ } u_i \text{ besitzt eine Wunder-ID} \right\}
\end{equation*}

\vspace{0.2cm}

Wir unterstellen zudem, die geforderte Existenz einer Wunder-ID gehe mit dem Besitz von unterschiedlichen Wallets bzw. anderen durch die Wunder-ID implizierten Dingen einher. So hat jeder User $u_i$ z.B. eine Telefonnummer mit seiner Wunder-ID verknüpft (anhand derer er mittels Kontakte-Scan auf dem Smartphone als Inhaber einer Wunder-ID und damit potenzieller Pool-Teilnehmer erkannt werden kann und soll). Des weiteren kann $u_i$ einen NFT-Pass (siehe Kapitel \ref{sec:nft-pass}) besitzen und/oder unser ERC20-Utility-Token (im Folgenden als \textit{WPT} bezeichnet; siehe Kapitel \nameref{sec:wpt-design}). 

\vspace{0.2cm}

Wir formalisieren den in Kapitel \ref{sec:nft-pass} definierten NFT-Pass als die (geordnete) Menge aller bisher geminteter NFT-Pässe:

\begin{align*}
  WPN &:= \left\{ wpn_1; wpn_2;... \right\} \text{ mit} \\
  wpn_i &:= (s_i, w_i, m_i)
\end{align*}

\vspace{0.2cm}

Dabei repräsentiert $s_i$ den Status des NFT-Passes, $m_i$ sein Muster und $w_i$ das sich auf ihm abgebildete Weltwunder.

\vspace{0.2cm}

Den Besitz eines Pass-NFTs beschreiben wir durch die Funktion 

\vspace{0.2cm}

\begin{equation*}
  \omega : U \rightarrow \mathcal{P} \left( WPN \right)  
\end{equation*}

\begin{equation*}
  \omega(u):= \left\{ wpn \in WPN \text{ } | \text{ User u besitzt den Pass-NFT } wpn \right\}. 
\end{equation*}

$\Bigl( \omega(u) = \emptyset$ falls der User $u$ keinen Pass-NFT besitzt.$\Bigr)$

\vspace{0.4cm}

Analog dazu definieren wir auch den Besitz am \textit{WPT} eines Users - mit dem Unterschied, dass der Funktionsbereich dieser Funktion - aufgrund der Fungibilität - von einer Potenzmenge auf einen simplen numerischen Wert zusammenfällt:

\vspace{0.2cm}

\begin{equation*}
  \varphi : U \rightarrow \mathbb{Q}  
\end{equation*}

\begin{equation*}
  \varphi(u):= \text{ Balance des Users u am ERC20-Token WPT}. 
\end{equation*}

\end{Def}

\vspace{0.2cm}

Die Notwendigkeit bzw. die Vorteile der vorangegangenen Annahmen und Definitionen sollten sich größtenteils aus der Lektüre der Kapitel des übergeorneten \textit{WunderPass-Whitepapers} erschließen, dessen lediglich partiellen Bestandteil das hiesige \textit{Pools-Kapitel} ausmacht. 

\vspace{0.5cm}    % binde die Datei '[Pools][Formalisierung].tex' ein


\section{Pool-Erzeugung}
\vspace{0.3cm}
% !TEX root = paper.tex

Die Erzeugung des Pools findet in zwei Phasen statt: Der \textit{Initialisierungs-Phase} und der \textit{Teilnahme-Phase}

\vspace{0.2cm}

\underline{\textbf{Initialisierungs-Phase}}

\vspace{0.2cm}

Die Initialisierungs-Phase läuft in etwa in folgenden Schritten ab:

\begin{itemize}
	\item Ein Initiator (Admin) $u_A \in U$ erzeugt den Pool in einer dafür vorgesehenen \textit{Pool-Applikation} (vergleichbar mit der Erstellung einer WhatsApp-Gruppe). Der Initiator $u_A$ ist dabei selbst ein Teilnehmer des Pools. Unsere klare Absicht hierbei ist jedoch keine "gesonderten" Pool-Teilnehmer zu haben bzw. mit besonderen Rechten auszustatten. Die Unterscheidung zwischen dem Admin $u_A$ und anderen Pool-Teilnehmern $u \in U$ ist idealerweise - sofern es denn der spezielle Case zulässt - nur für die Initialisierungs-Phase von Nöten und kann anschließend entfallen.
	\item Der Admin definiert das Regelwerk für den zu erstellenden Pool:
	\begin{itemize}
		\item Art des Pools (Invest-Pool, Wette, Spende, Kicktipp, Splitwise etc.)
		\item privater oder öffentlich zugänglicher Pool
		\item etwaige Obergrenze an Teilnehmern
		\item Einsatz (minimaler, maximaler oder exakter Einsatz pro Teilnehmer und Währung des Einsatzes)
		\item Auszahlungslogik (per Abstimmung oder Adresse eines Oracle-Smart-Contracts, der abhängig seiner Contract-Logik einen Auszahlungsschlüssel bereitstellt)
	\end{itemize}
\end{itemize}

\vspace{0.3cm}

\underline{\textbf{Teilnahme-Phase}}

\vspace{0.2cm}

Die Teilnahme-Phase besteht grob aus folgenden Schritten:

\begin{itemize}
	\item Der Admin verliert seine Sonderstellung und wird stattdessen zum ersten Teilnehmer seines eigens initiierten Pools. 
	\item Der (ursprüngliche) Admin lädt Teilnehmer ein, sich am Pool zu beteiligen. Die Beteiligung erfordert dabei einen WunderPass (= Wunder-ID) seitens des Teilnehmers. Idealerweise sind die Wunder-IDs mit Telefonnummern verknüpft, sodass sich die einzuladenden User in den Kontakten des Admins erkennbar als potenzielle Teilnehmer wiederfinden.
	\item Alternativ kann der (ursprüngliche) Admin (oder auch jeder andere bereits beigetreten Teilnehmer) einen Teilnahme-Link an (weitere) potenzielle Teilnehmer verschicken.
	\item Jeder adressierte User erhält die Einladung mit allen relevanten Informationen zum eingeladenen Pool (insbesondere auch dem zu entwendenden Einsatz) in seiner Wunder-Pool-Applikation, und muss diese lediglich entweder bestätigen oder ablehnen (Pull-Prinzip). Insbesondere braucht der User für den Beitritt zum Pool kein MetaMask oder Sonstiges (Push-Prinzip wie aktuell bei DAOs üblich). 
	\item Auch der Einsatz des neuen Teilnehmers muss nicht aktiv entrichtet (an eine Wallet) werden, sondern wird stattdessen im Zuge des vorigen Schritts nach Bestätigung der Teilnahme am Pool automatisch eingezogen.
\end{itemize}

\vspace{0.2cm}

Wir fassen die bisher erzielten Ergebnisse etwas formaler zusammen:

\vspace{0.2cm}

\begin{Def}\label{defPool}

Ein (jungfräulicher) Pool im Sinne der oben Aufgezählten Eigenschaften und Anforderungen lässt sich formal schreiben als

\begin{equation*}
  Pool := \left( \mathcal{U}, \mathcal{R}, \mathcal{T}, \mathcal{G} \right) \text{ mit}
\end{equation*}

\begin{align*}
  & \mathcal{U} = \left\{ u_1; u_2;...;u_n \right\} \subseteq U \text{ die Menge der n Pool-Teilnehmer, } \\
  & \mathcal{R} \text{ das Regelset des Pools, was es gesondert zu formalisieren gilt, } \\
  & \mathcal{T} = \left\{ s_1...;s_n \right\} \text{ mit } s_i \in \mathbb{Q} \text{ die Treasury des Pools und} \\
  & \mathcal{G} = \left\{ g_1...;g_n \right\} \text{ mit } g_i \in \mathbb{N} \text{ die Governance des Pools.}
\end{align*}

\vspace{0.2cm}

Dabei beschreibt jedes $s_i$ den Einsatz des Teilnehmers $u_i \in \mathcal{U}$ ($s$ für Stake). Dieser Einsatz liegt dabei in einem vom Regelset $\mathcal{R}$ definierten Intervall $\mathcal{I} \subseteq \mathbb{Q}$. Damit haben wir bereits an dieser Stelle einen kleinen Teil der noch fehlenden Formalisierung von $\mathcal{R}$ identifiziert. Bei genauer Betrachtung fehlt uns noch die Einheit der Einsätze $s_i$. Diese wird sehr wahrscheinlich \textit{USDT} sein oder ein anderer Stable-Coin.

Zudem beachte man an dieser Stelle zusätzlich, die Definition von $\mathcal{T}$ werde im Verlaufe der Lifetime eines Pools nicht so simpel bleiben können, als nur aus dem eingebrachten Einsätzen der Teilnehmer zu bestehen. Die Pool-Treasury bedeutet nämlich mehr als nur die Menge der initialen Stakes. Etwaige Invests aus der Treasury heraus würden nämlich ebenfalls in der Treasury landen.

\vspace{0.2cm} 

Die $g_i$ dagegen beschreiben ganz simpel die Anzahl der Governance-Tokens pro User $u_i \in \mathcal{U}$. Man kann diese auch als Gesellschaftsanteile einer GbR betrachten. Das Stammkapital dieser Gesellschaft würde sich in diesem Vergleich auf

\begin{equation*}
  \kappa := \sum_{i=1}^{n} g_i 
\end{equation*}

belaufen. Der prozentuale Stimmrecht-Anteil eines Users $u_i \in \mathcal{U}$ ergäbe sich damit als 

\begin{equation*}
  \rho_i = \frac{g_i}{\kappa}, \text{   } \forall i = 1, 2, ...,n. 
\end{equation*}

\end{Def}

\vspace{0.3cm}

Die zusammengetragenen Anforderungen für die Initialisierung eines WunderPools lassen sofort deutlich erkennen, \textbf{diese Pools könnten mittels DAO-ähnlicher Strukturen implementiert werden}. Dies erscheint insofern noch logischer, nachdem wir erkannt haben, die Pools stellen gesellschaftsrechtlich GbRs dar - also Gesellschaften und/oder Organisationen. Diese Erkenntnis wollen wir noch einmal als eine formale Annahme formulieren: 

\vspace{0.2cm}

\begin{Assumption}[Ein WunderPool stellt eine GbR dar]
\label{assumptionGbR} 

Sei $\mathcal{P} := \left( \mathcal{U}, \mathcal{R}, \mathcal{T}, \mathcal{G} \right)$ ein WunderPool wie in Definition \ref{defPool} beschrieben. Wir ziehen die Analogie zu einer \textbf{Gesellschaft} bürgerlichen Rechts:

\begin{itemize}
	\item Die Menge $\mathcal{U}$ der Pool-Teilnehmer ist der \textbf{Gesellschafterkreis der Gesellschaft}.
	\item $\mathcal{G}$ bildet den \textbf{Cap-Table der Gesellschaft} ab.
	\item Das Pool-Regelwerk $\mathcal{R}$ ist der \textbf{Gesellschaftervertrag zur Gesellschaft}.
	\item Die Pool-Treasury $\mathcal{S}$ ist das \textbf{Gesellschaftskonto und/oder -depot der Gesellschaft}.
\end{itemize}

\end{Assumption}
 
\vspace{0.5cm}    % binde die Datei '[Pools][Erzeugung].tex' ein


\section{Pool-Lifetime}
\vspace{0.3cm}
% !TEX root = paper.tex

Eine (allgemeine) funktionale Beschreibung derjenigen WunderPool-Funktionalität, die der Überschrift der gegenständigen Sektion gerecht wird, ist insofern sehr schwierig, als dass sich diese deutlich schwerer auf unterschiedliche Pool-Cases verallgemeinern lässt. Wie anfangs in dem Kapitel \nameref{sec:pools-einleitung} ist die möglichste Verallgemeinerung aller Cases oberste Prämisse gewesen. Hier müssen wir versuchen zu verallgemeinern, was zu verallgemeinert geht, und den Rest eben Case-spezifisch lösen. 

\vspace{0.3cm}

Wir schauen zurück auf die anfangs in Kapitel \nameref{sec:pools-einleitung} hervorgehobenen Anwendungsfälle der WunderPools. Und zwar jetzt mit explizitem Blick auf ihre \textit{Lifetime}:

\begin{itemize}
  \item \textbf{\textit{Social Investing:}} Dies ist mit der klarste Case für eine relevante Lifetime eines Pools. Während der Lifetime werden mögliche Invests vorgeschlagen, zur Abstimmung gestellt und im Erfolgsfall abgewickelt. Die Möglichkeiten zur Erweiterung von Investmöglichkeiten (Staking, Lending, Liquidity-Providing, Yield Farming, Aktien, ETFs etc.) scheinen schier unendlich. In diesem Case unterliegt \textbf{die Dauer der Lifetime auch keinerlei natürlicher Grenzen} - diese Art von Pool kann theoretisch ewig existieren.
  \item \textbf{\textit{Geschenk-Pool:}} In diesem Case besteht die Daseinsberechtigung des Pools im Grunde lediglich darin, bequem und einfach Geld einzusammeln und evtl. bis zum Kauf des Geschenks \textit{"in Treuhand"} zu verwahren. Sind alle gewünschten Teilnehmer beigetreten (und damit gleichbedeutend deren Beitrag zum Geschenk entrichtet), hat der Pool de facto bereits seinen Zweck erfüllt. Man kann zwar argumentieren, man könne die Auswahl des Geschenks mit DAO-Mitteln zur Abstimmung stellen, dies bliebe jedoch an den Haaren herbeigezogen, solange das Geschenk kein auf der Blockchain erwerbbares Asset darstellt. \textbf{Die Dauer der Lifetime der Pools in diesem Case sind also klar begrenzt}: Spätestens bis zu dem Moment des Kaufs des Geschenks.
  \item \textbf{\textit{Kicktipp-Pool:}} Dies ist der Bilderbuch-Case für den Pool im Sinne der Treuhand-Verwahrung (eines Spieleinsatzes) über einen längeren Zeitraum. Hier wird einge\-zahlt, über einen gewissen Zeitraum (außerhalb des Pools) gespielt und am Ende - je nach Ergebnis - wieder ausgezahlt. Das Geld wird vom Pool also lediglich verwahrt und umverteilt. In der sogenannten \textit{Lifetime} des Pools passiert faktisch gar nichts. Man könnte sich sicherlich kreative Möglichkeiten zur Interaktion mit dem Pool überlegen (wie z.B. Abstimmungen über etwaige Regeländerungen oder über das Nachtragen von verspätet abgegebenen Tipps), dies beträfe aber nie die relevante Kernfunktionalität des Pools innerhalb dieses Cases. Die defacto \textit{'leere Lifetime'} des Pools endet in diesem Case mit Ablauf der Spielzeit, für die die Kicktipp-Runde eingerichtet wurde. Ihre \textbf{Dauer ist also begrenzt}.
  \item \textbf{\textit{Wetten:}} Dieser Case verhält sich sehr analog zum \textit{Kicktipp-Case}. Dazu muss jedoch klargestellt sein, dass wir den Case als eine einzige Wette (zwischen zwei oder mehr Leuten) verstehen, bei der der Pool der Treuhand-Verwahrung dient, und nicht etwa eine "Wett-Gruppe", wo immer mal wieder neue Wetten vorgeschlagen und umgesetzt werden. Der Pool dieses Cases bildet also eine einzige Wette ab und seine \textbf{Lifetime endet in dem Moment, wo das Ergebnis der Wette feststeht}.
  \item \textbf{\textit{Splitwise:}} Dies ist der außergewöhnlichste aller Cases. Hier existieren de facto weder eine echte Treasury noch eine Lifetime. Für Splitwise wird erst die Umverteil\-ung interessant, wobei hier genau genommen der Betrag von 0 auf die Teilnehmer umverteilt wird. Da hier aber - im Gegensatz zu allen obigen Cases - auch negative Withdraws zulässig sind (also genau genommen eine Einzahlung von denjenigen Teilnehmern, die anderen Teilnehmern etwas schulden), klingt die Umverteilung des Betrags 0 plötzlich doch nicht mehr so abwegig. Die 0 signalisiert nur die Forderung, die verteilten Beträge (Schulden und Auslagen mit entsprechendem Vorzeichen) müssten sich am Ende auf 0 summieren. Da der Pool in diesem Case faktisch gar keine Lifetime besitzt, ist \textbf{die Dauer der Lifetime konsequenterweise begrenzt}.
\end{itemize}

\vspace{0.3cm}

Zusammenfassend halten wir fest, die Dauer der Pool-Lifetime sei nur für den \textit{Social-Investing-Case} theoretisch unbegrenzt. Bei allen anderen Cases wird der Pool nach einer bestimmten Zeit oder bei Eintreten eines bestimmten Ereignisses obsolet und muss/sollte anschließend aufgelöst werden. Und auch hinsichtlich relevanter Funktionalität während der zugehörenden \textit{Lifetime} scheint der \textit{Social-Investing-Case} ebenfalls der einzig interessante zu sein. 

\vspace{0.1cm}

Eine Verallgemeinerung erscheint also - zumindest für die zuletzt genannten vier Cases - evtl. doch im Rahmen des Möglichen. 

\vspace{0.3cm}

Zu guter Letzt sollte der Umstand nicht unerwähnt bleiben, etwaiges Austreten be\-stehender Pool-Teilnehmer bzw. das Eintreten neuer stellten keine irelanten Szenarien dar, die sich ebenfalls während der vermeintlichen \textit{Pool-Lifetime} abspielen würden.

\vspace{0.3cm}

Abschließend formalisieren wir erneut die erarbeiteten Gedanken - und zwar mit besonderem Blick auf Defintion \ref{defPool} und Annahme \ref{assumptionGbR}: 

\vspace{0.2cm}

\begin{Fazit}[Bestehen und Geschäftstätigkeit eines WunderPools als GbR (Lifetime)]
\label{lifetimeGbR} 

Sei $\mathcal{P} := \left( \mathcal{U}, \mathcal{R}, \mathcal{T}, \mathcal{G} \right)$ ein WunderPool wie in Definition \ref{defPool} und das Verständnis davon stark an Annahme \ref{assumptionGbR} angelehnt.

\vspace{0.2cm}

Wir möchten gerne auch die Lifetime eines WunderPools in die GbR-Analogie überführen und unterscheiden dabei zwischen \textbf{der Geschäftstätigkeit / Unterneh\-mensgegenstand der Gesellschaft selbst} auf der einen Seite und \textbf{den internen Gesellschaftsstrukturen} auf der anderen.

\noindent\hrulefill

\textbf{\underline{Geschäftstätigkeit:}}

\vspace{0.2cm}

Rein abstakt betrachtet, versuchen hierbei alle Gesellschafter $u \in \mathcal{U}$ - legiti\-miert durch deren Anteile aus $\mathcal{G}$ und restriktiert mittels Gesellschaftervertrags $\mathcal{R}$ - durch strategisches Verhalten das gemeinschaftliche Gesellschaftsvermögen $\mathcal{T}$ zu optimieren bzw. dieses zumindest optimal für ein zweckgebundenes gemeinsames Ziel einzusetzen.

\vspace{0.2cm}

Die internen Gesellschafter- und Gesellschaftstrukturen - repräsentiert durch die Größen $\mathcal{U}$, $\mathcal{G}$ und $\mathcal{R}$ - bleiben in diesem Kontext in aller Regel unberührt.

\noindent\hrulefill

\textbf{\underline{Interne Strukturen:}}

\vspace{0.2cm}

In diesem speziellen Kontext sind gegenteilig zum ersten genau die konträren Größen $\mathcal{U}$, $\mathcal{G}$ und $\mathcal{R}$ adressiert. Hierbei sind die Gesellschafter $u \in \mathcal{U}$ - erneut legitimiert durch deren Anteile aus $\mathcal{G}$ und wieder restriktiert mittels Gesellschaftervertrags $\mathcal{R}$ - dazu befähigt und ggf. daran interessiert, Veränderungen an $\mathcal{U}$, $\mathcal{G}$ und $\mathcal{R}$ zu erzwingen. Beispielhaft sind dabei folgende Cases denkbar:

\begin{itemize}
  \item \textbf{Kapitalerhöhung:} Hierbei würden die Gesellschafter anhand ihrer Stimmrechte aus $\mathcal{G}$ (und etwaiger Zusatzvereinbarungen aus $\mathcal{R}$) über die Aufnahme eines neuen Gesellschafters in die Gesellschaft abstimmen. Eine Zustimmung hätte mindestens eine Veränderung der Größen $\mathcal{U}$ und $\mathcal{G}$ zur Folge - und zwar in beiden Fällen eine Vergrößerung. In aller Regel würde bei einer Kapital\-erhöhung ebenso die Gesellschaftskasse $\mathcal{T}$ wachsen. Und in manchen Fällen wäre ebenso eine Veränderung des Gesellschaftervertrags $\mathcal{R}$ zu erwarten.
  \item \textbf{Auszahlen eines bestehenden Gesellschafters:} Von der Logik her ein ähnlicher Case wie der erste, nur dass hierbei ein oder mehrere Gesellschafter die Gesellschaft verlassen, die Menge $\mathcal{U}$ also schrumpft. Anders als der erste Case ist der gegenständige jedoch differenzierter hinsichtlich des \textit{\textbf{Wie}} zu betrachten. Während bei der Kapitalerhöhung einfach neue Anteile kreiert werden, die schlichtweg von neuen Gesellschaftern übernommen werden, gibt es im aktuellen Fall mehrere gängige Varianten:
  \begin{itemize}
	\item Die Anteile des/der ausscheidenden Gesellschafter(s) werden von der Gesell\-schaft selbst übernommen und anschließend vernichtet. Hierbei verringert sich also $\sum_{g \in \mathcal{G}}g$ konsequenterweise. In diesem Fall muss in aller Regel die Gemeinschaftskasse $\mathcal{T}$ zur Auszahlung herangezogen werden. Dies kann insofern etwas tricky werden, als dass die Treasury $\mathcal{T}$ nicht zwingend ausschließlich aus Fiat-Vermögen bestehen muss, und stattdessen ggf. Assets liqudiert werden müssten.
	\item Die Anteile des/der ausscheidenden Gesellschafter(s) werden von den verbleibenden Gesellschaftern übernommen. Hierbei bliebe $\sum_{g \in \mathcal{G}}g$ unverändert. Die finanzielle Abwickung würde de facto \textit{außerhalb der Gesell\-schaft} stattfinden. Insbesondere bliebe die Treasury $\mathcal{T}$ von der Transaktion unberührt, was die Implementierungslogik vehement vereinfacht.
	\item Der technisch simpleste Case wäre sicherlich der Verkauf der Anteile des/der ausscheidenden Gesellschafter(s) am Sekundärmarkt - die ja als ERC20-Tokens einfach handelbar wären. Auch hierbei bliebe sowie $\sum_{g \in \mathcal{G}}g$ als auch $\mathcal{T}$ als auch wahrscheinlich $\mathcal{R}$ unverändert. Genau genommen ist dieses Sub-Szenario ein Speziellfall des nächsten Case.
  \end{itemize}
  \item \textbf{Anteilsverkauf / -übertragung:} In diesem Szenario würde ein Gesellschafter $u_{i} \in \mathcal{U}$ einen Teil oder alle seiner $g_{i} \in \mathcal{G}$ Anteile am Sekundärmarkt verkaufen. Der Gesellschaftervertrag $\mathcal{R}$ könnte ihn zwar theoretisch daran hindern bzw. die anderen Gesellschafter in eine solche Entscheidung einbeziehen. Dies würde jedoch mit erheblicher Komplexität einhergehen, weshalb wir \textit{per default} erst einmal davon ausgehen wollen, die Gesellschaftsanteile sind als ERC20-Gover\-nance-Tokens frei handelbar und unterliegen keinen Einschränkungen. In diesem Fall wäre die innere Gesellschaftsstruktur einem ausschließlich äußeren Einfluss unterstellt, dem sie nicht Herr wäre. (Unkontolliert) betroffen wären die Größen $\mathcal{U}$ und $\mathcal{G}$.
  \item \textbf{Änderung des Gesellschaftervertrags:} Hierbei würden die Gesellschafter anhand ihrer Stimmrechte aus $\mathcal{G}$ (und etwaiger Zusatzvereinbarungen aus $\mathcal{R}$) über gewisse Änderungen an $\mathcal{R}$ abstimmen. Welche Änderungen hierbei möglich wären, könnte wiederum mittels des vor der Änderung geltenden Regelsets $\mathcal{R}$ gemaßregelt sein. Aufgrund dieser Rekursivität müsste $\mathcal{R}$ wahr\-scheinlich einige unveränderliche Elemente beinhalten. Zu der definitiv komplexesten Größe unserer WunderPools $\mathcal{R}$ siehe auch das Kapitel \nameref{sec:pools-vertrag}.
\end{itemize}

\end{Fazit}

\vspace{0.5cm}    % binde die Datei '[Pools][Lifetime].tex' ein


\section{Pool-Liquidierung}
\label{sec:pools-liquidierung}
\vspace{0.3cm}
% !TEX root = paper.tex

Für eine etwaige Pool-Liquidierung stellen sich exakt zwei Fragen: "\textbf{Wann} wird liquidiert?" und "\textbf{Wie} wird liquidiert?" Das \textbf{Wann} ist hierbei schnell geklärt. Es gibt grob folgende drei Möglichkeiten, von denen \textbf{genau eine} durch das in Definition \ref{defPool} definierte Regelset $\mathcal{R}$ zu benennen ist:

\begin{itemize}
  \item $\mathcal{R}$ legt einen exakten Zeitpunkt fest, zu dem der Pool liquidiert werden soll.
  \item $\mathcal{R}$ definiert ein bestimmtes Ereignis, bei deren Eintreten der Pool liquidiert werden soll.
  \item $\mathcal{R}$ regelt, dass die Pool-Liquidierung per (DAO-)Abstimmung beschlossen werden muss.
\end{itemize}

\vspace{0.3cm}

Bullet 2 klingt hier leider noch nicht ausreichend abstrakt. Daher abstrahieren wir die genannten drei Forderungen in einer einzigen:

\vspace{0.2cm}

\begin{Fazit}[Liquidierungsentscheidung-Oracle]

Das in Definition \ref{defPool} definierte Regelset $\mathcal{R}$ definiert ein Oracle, welches zu jedem Zeitpunkt die Frage beantworten kann, ob der Pool zum jetzigen Zeitpunkt liquidiert werden soll oder nicht. 

\end{Fazit}
 
\vspace{0.1cm}
 
Dieses Oracle kann beliebig einfach gestrickt sein (z.B. im Falle des obigen Bullet 1 einfach anhand $"SYSDATE <= T_{END}"$ über das Fortbestehen des Pools entscheidet) oder aber auch beliebig komplex. Dies braucht uns aber an dieser Stelle nicht weiter interessieren.

\vspace{0.3cm}

Und da die Abstraktion mittels Oracle so bequem scheint, tun wir das Gleiche ebenfalls für das oben genannte \textbf{Wie}:

\vspace{0.2cm}

\begin{Fazit}[Auszahlungsschlüssel-Oracle]

Seien $\mathcal{P} = \left( \mathcal{U}, \mathcal{R}, \mathcal{T}, \mathcal{G} \right)$ 
der Pool und $\mathcal{U} = \left\{ u_1; u_2;...;u_n \right\}$ die Menge seiner $n$ Teilnehmer wie in Definition \ref{defPool} beschrieben und $v_{\mathcal{T}}$ der sich zum Liquidierungszeitpunkt in der Pool-Treasury $\mathcal{T}$ befindende Value. 

Falls der Pool lediglich als Treuhand-Verwahrung diente (also über die Zeit keine Veränderung der Treasury stattfand) ergibt sich $v_{\mathcal{T}}$ als  

\vspace{0.1cm}

\begin{equation*}
  v_{\mathcal{T}} = \sum_{i=1}^{n} s_i \text{ mit } s_i \text{ wie in Definition \ref{defPool}}
\end{equation*}

\vspace{0.2cm}

Wir definieren einen Auszahlungsvektor als

\begin{equation*}
  \varphi_{\mathcal{P}} = [\varphi_1, \varphi_1, ..., \varphi_n] \text{ mit } \sum_{i=1}^{n} \varphi_i = v_{\mathcal{T}} 
\end{equation*}

\vspace{0.2cm}

Die $\varphi_i$ beschreiben also die absoluten Anteile der Teilnehmers $u_i$ an der Pool-Treasury $\mathcal{T}$. Und $\frac{\varphi_i}{v_{\mathcal{T}}}$ dann logischerweise die prozentualen.

\end{Fazit}


Definiert/Konkretisiert werden müssten die \textit{Liquidierungsentscheidung-} und \textit{Auszahl\-ungsschlüssel-Oracle} in dem Pool-Regelset $\mathcal{R}$

\vspace{0.3cm}

Im Folgenden einige Beispiele für denkbare \textit{Auszahlungsschlüssel-Oracle}. Zur Vereinfachung nehmen wir dazu an, die Pool-Treasury $\mathcal{T}$ enthielte ausschließlich Funds derselben Fiat-Währung, weshalb
der oben genannte Value $v_{\mathcal{T}}$ gänzlich intuitiv ersichtlich sei.  


\vspace{0.3cm}

\begin{Example}[Pro-Rata-Auszahlung]

Dies stellt eigentlich den intuitivsten aller denkbaren Auszahlungsschlüssel dar. Jeder Pool-Teilnehmer bekommt genau den prozentualen Anteil an $v_{\mathcal{T}}$ ausgezahlt, der seinem Anteil an der Pool-Governance $\mathcal{G}$ entspricht.

\vspace{0.2cm}

Der obige Auszahlungsvektor würde sich in diesem Fall ganz simple als

\begin{equation*}
  \varphi_i = v_{\mathcal{T}} \cdot \frac{g_i}{\sum_{j=1}^{n} g_j}
\end{equation*}

ergeben. Wobei $\mathcal{G} = \left\{ g_1; g_2;...;g_n \right\}$ wäre.

\end{Example}

\vspace{0.2cm}


\begin{Example}[Auszahlung nach abgeschlossener Kicktipp-Tipprunde]

Angenommen so ein WunderPool würde für die Verwahrung der Spieleinsätze einer Kicktipp-Tipprunde verwendet werden. Nach abgeschlossene Spielzeit sollte der gesamte Wettpool an die besten Tipper ausgezahlt werden. Wer die besten Tipper waren und wie das Geld konkret unter diesen verteilt wird, mocken wir hinter einem \textit{Kicktipp-Oracle}, der diese Daten wie auch immer (von extern) beschafft. Bei einem Wettpool von ingesamt 1000 Euro und acht Mitspielern könnte das vom Oracle gelieferte Ergebnis z. B. wie folgt aussehen:

\begin{equation*}
  \varphi_{\mathcal{P}} = [0, 100, 0, 0, 700, 0, 200, 0]
\end{equation*}

\end{Example}

\vspace{0.2cm}

\begin{Example}[Splitwise-Abrechnung]

Nutzt man die WunderPools-Abstahierung, um eine Splitwise-Abrechnung unter Wunder-Usern innerhalb derselben Splitwise-Gruppe abzubilden, könnte die Splitwise-API dazu genutzt werden, ein \textit{Splitwise-Oracle} zu implementieren. Dieses könnte bei acht Spitwise-Usern z. B. folgendes Ergebnis liefern:

\begin{equation*}
  \varphi_{\mathcal{P}} = [80, 50, -50, -30, 0, -40, 20, -30]
\end{equation*}

Man beachte, dass bei diesem Case die $\varphi_i$ auch gegativ sein können und sich in Summe auf 0 addieren: $\sum_{i=1}^n \varphi_i = 0$.

\end{Example}

\vspace{0.2cm}

\begin{Example}[Random Gambling]

Keine besonders sinvolle aber theoretisch dennoch denkbare Auszahlungsstrategie wäre eine völlig zufällige Verteilung von $v_{\mathcal{T}}$ unter allen Pool-Teilnehmern. Hierbei wäre unser Oracle ein einfacher Zufallsgenerator, der uns eine Zufallsverteilung $P = \left\{ P_1; P_2;...;P_n \right\}$ für unsere $n$ Pool-Teilnehmer mit $\sum_{i=1}^n P_i = 1$ liefert.

\vspace{0.2cm}

Der Auszahlungsvektor würde sich in diesem Fall ganz simple als

\begin{equation*}
  \varphi_i = v_{\mathcal{T}} \cdot P_i
\end{equation*}

ergeben. 

\end{Example}


\vspace{0.5cm}    % binde die Datei '[Pools][Liquidierung].tex' ein


\section{Pool-Vertrag}
\label{sec:pools-vertrag}
\vspace{0.3cm}
% !TEX root = paper.tex

\todo{Herausgearbeitete Dinge zu $\mathcal{R}$ zusammentragen}

\begin{itemize}
  \item Vorgabe zur Teilnehmer-Menge $\mathcal{U}$
  \item Vorgabe zur Pool-Treasury $\mathcal{S}$:
  \begin{itemize}
  	\item Währung (zB \textit{USDT})
  	\item Intervall $\mathcal{I}$ für $s_i \in \mathcal{I}$
  \end{itemize}
  \item Definition der \textit{Liquidierungsentscheidung-Oracle}
  \item Definition der \textit{Auszahlungsschlüssel-Oracle}
  \item Optionale Forderung $\varphi_i \geq 0$
  \item etc.
\end{itemize}

\vspace{0.5cm}

    % binde die Datei '[Pools][Vertrag].tex' ein


%---------Pool-Economics------------------
\section{WunderPools-Economics}
\vspace{0.3cm}


\subsection{Einleitung}
\vspace{0.2cm}
% !TEX root = paper.tex


Dabei sollen gleichermaßen ein Monetarisierungmodell, ein zugehöriger Business-Plan sowie eine mögliche Utility-Token-Ökonomie, die diese Komponenten mittels \href{https://de.wikipedia.org/wiki/Mechanismus-Design-Theorie}{Mechanismus-Design} in Einklang zueinander bringt und in einem übergeordneten Ökonomie-Kreislauf verankert, gleichzeitig erarbeitet und miteinander verknüpft werden.

\vspace{0.2cm}

Am Ende soll idealerweise jede solcher Fragen wie,

\begin{itemize}
	\item \textit{Wer bezahlt den Pool-Service und wie viel?}
	\item \textit{Wer verdient am Pool-Service und wie viel?}
	\item \textit{Wie wird das Pool-Projekt finanziert und wie werden etwaige Investoren incentiviert und entlohnt?}
	\item \textit{Wie sieht der konkrete Business-Plan aus?}
	\item \textit{Wie wird der zugehörige Pool-Project-Token modelliert und in das übergeordnete Pool-Ökosystem integriert?}
	\item \textit{Wie sind Risiko und ROI von etwaigen Projekt-Invests zu beziffern?}
\end{itemize}

beantwortet sein.

\vspace{0.5cm}

Da der zentrale Bestandteil der eigentlichen Dienstleistung der Pools für seine Nutzer bereits in sehr starkem finanziellen Kontext - nämlich des \textit{Social-Investings} - steht, und wir uns im Folgenden mit dem finanziellen Gerüst des übergeordneten Pools-Projects beschäftigen möchten - das aber so gar nichts mit der Dienstleistung des \textit{Social-Investings} an sich gemein hat, müssen wir gleich zu Beginn eine essenzielle Abgrenzung ziehen, ohne deren unmissverständliches Bewusstsein beim Leser die folgenden Kapitel nur missverstanden werden können und werden.

\vspace{0.2cm}

\textbf{Man lese und verinnerliche also folgendes lieber gleich zehnmal:}

\vspace{0.2cm}

\begin{Abgrenzung}[Pools-Project-Economics haben nichts mit Invest/Economics eines einzelnen Pools (als Dienstleistung des Pools-Projects) zu tun.]

\vspace{0.2cm}

Die Dienstleistung unseres Pools-Projects hat im Sinne des \textit{Social-Investings} unausweichlich mit Geld zu tun. Die \textbf{Pools-Project-Economic} haben dies konsequenterweise ebenfalls.

\vspace{0.1cm}

\textbf{Dabei steht ausschließlich zweites im Fokus des gegenständigen Kapitels. Erstes dagegen bestenfalls beiläufig als Referenzgrundlage bis gar nicht.} Die User der Pools hantieren mit Geld, indem sie den Service nutzen. Projekt-Stakeholder verdienen idealerweise an der angebotenen Dienstleistung - wie sie es auch täten, falls die Dienstleistung keinerlei finanziellen Bezug hätte.

\vspace{0.75cm}

Wir wollen hier einige \textit{Fallstricke} für offensichtliche Missverständnisse und Verwechselungsgefahren ganz konkret beim Namen nennen:

\begin{itemize}
	\item Die Pools (als genutzte Dienstleistung) verfügen über Funds und Assets. Beides werden in aller Regel Tokens sein. Die Funds - als \textit{Fiat-Äquivalent} - vermutlich (aber auch nicht zwingend) mittels eines \textit{Stable-Coins} repräsentiert. Die Assets erst einmal nicht weiter spezifiziert. 
	
	\textbf{Diese finanziellen Mittel eines Pools stellen bestenfalls eine Referenzgrundlage zu anfallenden Service-Fees dar, sind kein direkter Bestandteil der Pools-Economics und verwenden ganz besonders NICHT den Pool-Project-Token als Basis-/Funding-Währung.}
	\item Die Monetarisierung des Pools-Service wird anhand von (prozentualen) Service-Fees erfolgen, die als Berechnungsgrundlage durchaus das Kapital des je\-weiligen Pools heranziehen kann und wird. 
	
	Konkret werden diese Fees in einer dafür definierten Währung anfallen, die ein \textit{Stable-Coin} UND/ODER der Pool-Project-Token sein kann. Die \textit{Monetarisier\-ungs-Währung} ist dabei zentraler Bestandteil der \textit{Pool-Economics}, die Währung der Pool-Funds eines Pools ist es dagegen absolut nicht und daher auch nicht maßgebend für die Fees-Abrechnung. Bei etwaigen Währungs-Diskrepanzen muss unter Umständen ein Umrechnungs- und Ad-Hoc-Umtausch-Mechanismus implementiert werden
	\item Ein in den folgenden Kapiteln definierter \textit{Token-\textbf{Staking}-Mechanismus} wird den Pool-Project-Token als Währung vorsehen und \textbf{hat dabei absolut nichts mit dem/den Pool-Kapital/-Funding/-Assets zu tun.}
	\item Jeder Pool wird eine \textbf{\textit{Pool-Treasury}} besitzen, die die Pool-Funds und die Pool-Assets verwaltet. Unser Pool-Project-Token wird gleichzeitig einem Mo\-dell folgen, bei dem eine sogenannte \textbf{\textit{Token-Contract-Treasury}} von großer Bedeutung sein wird, die wir künftig wahlweise auch als \textbf{\textit{Pools-Project-Treasury}}, \textbf{\textit{Pools-Token-Treasury}} oder als \textbf{\textit{Project-Token-Treasury}} be\-zeichnen. Ungeachtet der - nicht immer konsistenten Bezeichnung - ist diese dringend von der erstgenannten Treasury eines einzelnen Pools zu unterscheiden.
\end{itemize}

\end{Abgrenzung}

    % binde die Datei '[Pools][Economic][Einleitung].tex' ein
\vspace{0.5cm}

\subsection{WPT - Die grundlegende Idee eines Pools-Project-Utility-Tokens}
\label{sec:wpt-design}
\vspace{0.2cm}
% !TEX root = paper.tex

\paragraph{Monetarisierung \& Tokenisierung}
\textbf{ }
\vspace{0.3cm}

Der abstrakt gehaltenen Einleitung zum finanziellen Grundgerüst unseres Pool-Projekts wollen wir in diesem Abschnitt nun den konzeptuell gedanklichen Grundstein zur dessen tatsächlichen Economics-Realisierung legen, auf dem dann im Anschluss die folgenden Kapitel aufbauen.

\vspace{0.2cm}

Dazu folgen zunächst einige - mehr oder minder erklärungsbedürftige - rohe Aussagen: 

\vspace{0.2cm} 

\begin{Praemisse}[Monetarisierung]
\label{monetarisierung}
\vspace{0.2cm}

Die Monetarisierung unseres Pool-Service soll auf Basis (prozentualer) Fees (siehe \nameref{sec:fees}) - gemessen am (finanziellen) Volumen der erbrachten Dienst\-leistung - erfolgen. Für den Moment sehr plakativ betrachtet, ist dies gleichbedeutend mit: 

\vspace{0.2cm} 

\textbf{\textit{Mit je mehr Kohle die Pools hantieren, desto größer sollen die anfallenden Fees sein!}}

\end{Praemisse}

\vspace{0.5cm}

\begin{Praemisse}[Utility-Token als Monetarisierungs-Tool für alle Stakeholder]
\label{fees-for-token}
\vspace{0.2cm}

\textbf{Die Fees sollen mittels eines dafür geschaffenen Utility-Tokens abgerechnet, erhoben und erbracht werden!}

\vspace{0.5cm} 

Für den - unbestreitbar verkomplizierenden und technisch teils nicht unerheblich umständlichen - Umweg der Monetarisierung über einen Token sehen wir folgende schlagende Argumente, die auch in den anschließend folgenden Kapiteln immer mal wieder argumentativ zum Vorschein kommen werden:

\begin{itemize}
	\item Die Nutzung des Pools-Service kann als ein echtes \textit{\textbf{Gut}} - eine \textit{Utility} - angesehen werden, was unter Umständen nicht endlos verfügbar sei (begrenzte Skalierung auf der Blockchain), besonders begehrt (bei exzellenter Service-Qualität) oder im Übermaß vorhanden (bei anfänglicher Unbekanntheit des Service) sei. 
	
	Durch die Tokenisierung der Dienstleistung einverleibt man dieser den Stellenwert einer \textit{Ressource}, mit zugehörigen Eigenschaften wie \textbf{Verfügbarkeit}, \textbf{Qualität} und \textbf{Nachhaltigkeitsgedanken}, was bei digitalen Dienstleistungen oft unberücksichtigt bleibt. 
	
	Mit diesem Ansatz kommt das \textit{Marktwirtschaftsprinzip von Angebot \& Nachfrage} auch bei digitalen Services zum Tragen, was in der digitalen Welt heutzutage ausschließlich auf \textit{Nachfrage} reduziert wurde, da das \textit{Angebot} de facto als unendlich betrachtet wird.
	\item Die Tokenisierung eines Business-Modells eröffnet einem das sehr mächtige spieltheoretische Werkzeug des \href{https://de.wikipedia.org/wiki/Mechanismus-Design-Theorie}{Mechanismus-Design}, um sämtliche Projekt-Beteiligte bzw. -Stakeholder in ihrem Verhalten hinsichtlich des übergeordneten Projekterfolgs zu beeinflussen/incentivieren. Oder simple ausgedrückt: Das zu tun, was wir aus strategischen Überlegungen möchten, dass er/sie tut.
	\item \textbf{Direkte \& unbürokratische Projekt-Finanzierung}.
	
	Durch die Tokenisierung der Dienstleistung muss ein potenzieller Investor beim Kauf von Utility-Tokens lediglich vom Erfolg der Dientleistung=Utility selbst überzeugt sein (da eine Nachfrage nach der Dienstleistung direkt an die Nachfrage nach dem zugehörigen Utility-Token gekoppelt ist), anstatt bei seiner ROI-Evaluierung herkömmliche bürokratisch geregelte Venture-Capital-Aspekte wie etwaige Shareholders-Agreements und Exit-Szenarien hinzuziehen zu müssen.
	\item Technische und konzeptuelle Vereinfachung, Flexibilität und Direktheit bei \textit{Customer-Akquise} und \textit{CRM} mittels des Utility-Tokens, da
	\begin{itemize}
		\item die \textit{Marketing-Währung} in Form von Tokens die \textbf{Utility} selbst statt \textit{Fiat} in den Vordergrund rückt.
		\item Der \textit{Project-Owner} (in dem Fall also WunderPass) in aller Regel selbst ein großer Token-Holder sein wird und somit über die Mittel verfügt, das Marketing-Volumen zu erbringen (ohne dabei zusätzlich finanziell belastet zu werden).
	\end{itemize}
	\item Uneingeschränkte Transparenz für alle Projekt-Teilnehmer über Stake, Cash-Flows, Handlungen, Strategien etc. aller anderen Projekt-Teilnehmer und damit ihrer Position und Interessen innerhalb des Projekts mittels jederzeit offen einsehbarer dezentraler Smart-Contract-Logik.
	\item Uneingeschränkte Transparenz und Eliminierung von Interpretationsspielraum hinsichtlich des Business-Plans.
	\item Zu guter Letzt sei noch das - weniger auf harten Fakten als auf dem \textit{Opportunitiy-Gedanken} begründete - Argument des vermeintlichen \textit{Tokenisierungs-Trends} zu nennen, welches ein rein selbstzweck-getriebenes Interesse bei potenziellen Token-Investoren wecken könnte.
\end{itemize}

\end{Praemisse}

\vspace{0.5cm}

\paragraph{Die entscheidende Idee}
\textbf{ }
\vspace{0.3cm}

Allen relevanten Erklärungen vorweggreifend folgt unser fundamentale \\
\textit{Token-Economics}-Ansatz für die Pools-Project-Token:

\vspace{0.2cm}

\begin{Konzept}[Dividende auf den Pools-Project-Token]
\label{token-usp}
\vspace{0.2cm}

Zusätzlich zur \textit{Utility}-Beschaffenheit unseres Pools-Project-Tokens möchten \\
wir diesem noch eine gewisse \textit{Equity}-Eigenschaft einverleiben:

\vspace{0.2cm}

\textbf{Ein Token-Besitzer soll mittels des Tokens nicht nur den Pools-Service nutzen können oder an der steigenden Nachfrage nach diesem - durch eine positive Kursentwicklung - profitieren, sondern zusätzlich DIREKT an den generierten Erträgen des gesamten Pools-Projects beteiligt werden.}

\vspace{0.2cm}

Er soll demnach de facto als Anteilseigner des Pools-Projects gelten und an etwaigen Gewinnen des Projekts - in Form einer gewissen \textit{Dividende} - pro rata seines Token-Volumens partizipieren.

\vspace{0.2cm}

Die Implementierung dieses \textit{Equity}-Mechanismus soll selbstverständlich mittels eines Smart-Contracts sichergestellt sein, was unseren Token stark von anderen \\ \textit{Equity}-Tokens abhebt.

\vspace{0.2cm}

Durch diesen zusätzlichen Kniff, schaffen wir eine sich selbst verstärkende Synergie zwischen den \textit{Utility}- und \textit{Equity}-Eigenschaften unseres Pools-Project-Tokens, indem wir einen potenziellen User des Pools-Service (besitzt \textit{Utility} in Form des Tokens) gleichzeitig zu einem Projekt-Investor machen (besitzt \textit{Equity} in Form desselben Tokens). Dieser doppelte Synergieeffekt weitet sich auch unmittelbar auf die Kursentwicklung aus. DENN: Wachsende Nutzung des Pools-Services impliziert zwangsläufig eine steigende Token-Zirkulation (im Sinne der \textit{Utility}-Beschaffenheit) und steigenden Bedarf und somit Nachfrage nach dem Token UND generiert gleich\-zeitig zunehmenden Ertrag durch Service-Fees, was wiederum eine Wertsteigerung des Tokens aus seiner \textit{Equity}-Beschaffenheit nach sich zieht.

\end{Konzept}

\vspace{0.3cm}

Wie genau wir uns das eben formulierte Vorhaben in der Umsetzung planen, wird etwas weiter unten vertieft. Zunächst bleiben wir beim ökonomischen Teil des Token-Designs und erarbeiten einige relevante Mechanismen.

\vspace{0.5cm}


\paragraph{Token-Design}
\textbf{ }
\vspace{0.3cm}

Beim Design unseres Pools-Project-Tokens wollen wir uns stark an den Gedanken des spieltheoretischen Gebiets des \href{https://de.wikipedia.org/wiki/Mechanismus-Design-Theorie}{Mechanismus-Design} orientieren.

Dieses Wissenschaftsgebiet befasst sich im Wesentlichen damit als \textit{höhere Instanz eines Spiels} - also in dem Fall wir als Project-Owner - mittels Regelgestaltung und Incentivierungs-Mechanismen - also in unserem Fall mittels Token-Design - Einfluss auf das Verhalten der Spieler - also in dem Fall Nutzer des Pool-Service und Investoren - im Sinne des Spiels nehmen kann.

\vspace{0.1cm}

Entscheidend hinsichtlich letzter Formulierung ist dabei das \textit{"... im Sinne des Spiels..."} genaust möglich zu präzisieren und idealerweise zu quantifizieren und formalisieren.

\vspace{0.5cm}

\textbf{Was möchten wir also genau wie, wann und womit erreichen für unser Pools-Projekt?}

\vspace{0.5cm}

Dabei bewegen sich die \href{https://de.wikipedia.org/wiki/Mechanismus-Design-Theorie}{Mechanismus-Design}-Werkzeuge tendenziell auf einer granularen Ebene, weshalb die Antwort \textit{"Pools-Project to the moon!"} auf obige Frage nicht in deren Sinne stünde. Viel mehr ist obige Frage daher als

\begin{itemize}
	\item Welche Etappenziele möchten wir erreichen (Projekt-Funding, Wachstum, Exit etc.)?
	\item Welche Projekt-Stakeholder (Gründer, Project-Owner, Investoren, User etc.) werden gebraucht und wie können diese gewonnen und deren Interessen gewahrt werden?
	\item Welche Hebel und designte Einflussmöglichkeiten möchten wir mittels von Token-Mechanismen besonders stark in eigener Hand behalten, anstatt sie dem Zufall oder Markt-Gesetzen zu überlassen?
	\item Welche Synergien möchten wir schaffen/verstärken bzw. verhindern/bremsen?
	\item Letzeres ist nicht nur aus Sicht des Pools-Projekts für sich alleinstehend zu betrachten sondern insbesondere auch im Hinblick auf ein etwaiges künftiges Wunder-Ökosystem. 
	\item Wie können wir als Gründer/Project-Owner (finanziell) profitieren?
\end{itemize}

zu verstehen. Um das ganze nicht ausufern zu lassen, wollen wir diese Fragestellungen stark auf das Pools-Projekt, seinen Projekt-Token und insbesondere dessen erhofften Effekte fokussieren:

\vspace{0.3cm}

\begin{Assumption}[Erwünschte Effekte des Pools-Project-Tokens]
\label{token-anforderungen}
\vspace{0.2cm}

Folgende Anforderungen, Erwartungen und Absichten verfolgen wir mit dem zu designenden Projekt-Token und/oder beabsichtigen zu erfüllen:

\begin{itemize}
	\item Selbstverständlich stellt ein gewisses initiales Projekt-Funding mittels Token-Sale eine der ausschlaggebendsten Motivationen für den Token dar, um z. B. auch Entwicklungskosten zu decken. 
	\item Gleichzeitig müssen aber eben die initialen Kapitalgeber angemessen für ihr Risiko entlohnt werden und signifikant stärker an ihrem Token-Invest profitieren als spätere Token-Käufer.
	\item Nicht verkehrt wäre gleiches für die Gründer ;)
	\item Nicht nur für die zuletzt genannten early Investors sondern generell für alle Token-Investoren möchten wir einen transparenten, berechenbaren und vertrauenswürdigen Token schaffen, 
	\begin{itemize}
		\item dessen Kursentwicklung keiner künstlichen PR-getriebenen Hysterie mit anschließendem Crash unterliegt (\textit{Pump \& Dump}),
		\item dessen Value transparenten und idealerweise durch Smart-Contracts ge\-steuerten Mechanismen und Projekt-Entwicklungen folgt,
		\item dessen Value einen \textit{Utility-}Bezug hat und
		\item der idealerweise mittels eines AMMs (\textit{Automated Market Maker}) jederzeit handelbar sein soll.
	\end{itemize}
	\item Nicht ganz so essenziell wie das initiale Projekt-Funding jedoch ebenfalls nicht zu vernachlässigen ist die fortlaufende (operative) Projekt-Finanzierung, die gänzlich oder zumindest teilweise durch den Projekt-Token mitfinanziert werden könnte.
	\item Gleichwohl der oben skizzierte USP unseres Tokens (siehe \ref{token-usp}) \textit{Equity}-techni\-scher Natur ist, ist und bleibt unserer Pools-Project-Token substanziell ein \textbf{\textit{Utility-Token}}.
	\begin{itemize}
		\item Grundsätzlich wird die Zirkulation eines \textit{Utility-Tokens} stets stark korreliert mit der Nutzung/Nachfrage der Utility - also in unserem Fall dem Pools-Service - sein. Wie solch eine Korrelation konkret aussieht, haben wir mittels des Token-Designs maßgeblich in eigener Hand. So kann man mit Mitteln wie z. B. \textit{Staking} oder \textit{Locking} die Zirkulation künstlich verlangsamen bzw. eine künstliche Verknappung an zirkulierenden Tokens induzieren.
		\item In gewisser Überzeugung, ein echter \textit{Utility-Token} repräsentiere eine nur endlich verfügbare Ressource, streben wir einen deflationären Token an. Oder zumindest einen \textit{pseudo-deflationären} (also einen, der zwar theoretisch unendlich lange weitergemintet werden kann, dies jedoch ab einem bestimmten Moment absolut unwirtschaftlich wird).
	\end{itemize}
\end{itemize}

\end{Assumption}

\vspace{0.5cm}

An der Abarbeitung dieser Liste werden wir uns - nicht zwingend die Reihenfolge wahrend - durch das restliche Kapitel hangeln. Bevor wir uns gleich im Anschluss etwas detaillierter dem letzten Punkt der obigen Liste - nämlich der Einflussnahme auf die Token-Zirkulation - widmen, zunächst ein sich sofort ersichtlicher \textit{Quick-Win} hinsichtlich Bullet 4 der obigen Liste:

\vspace{0.3cm}

\begin{Konzept}[Das \textit{Bonding-Curves-Modell} als vielversprechendes Mittel für unseren Pool-Project-Token]
\label{bcm}
\vspace{0.2cm}

Der Wunsch nach einem \textbf{transparenten, berechenbaren und vertrauenswürdig\-en Token} aus der Anforderungsliste \ref{token-anforderungen} suggeriert, das \textit{Bonding-Curves-Modell} als Grundlage zur Modellierung unseres Pool-Project-Token in Betracht zu ziehen, da der \textit{Bonding-Curves-Ansatz}

\begin{itemize}
	\item mittels Einsatzes eines Smart-Contract-AMMs, \textbf{Transparenz und Berechenbarkeit} des Tokens garantiert,
	\item durch im Token-Contract vorgehaltene \textbf{Kapital-Deckung pro ausgegebenem Token} das Investrisiko deckelt und damit die gewünschte \textbf{Vertrauenswürdig\-keit} abbildet und
	\item letztendlich durch seinen Basis-Mechanismus zwingend einen in seiner Logik verankerten AMM mitliefert.
\end{itemize}

\vspace{0.5cm}

Zur Einführung und Motivation des \textit{Bonding-Curves-Modells} sei repräsentativ auf folgende zwei Blog-Artikel verwiesen:

\begin{itemize}
	\item \href{https://medium.com/coinmonks/token-bonding-curves-explained-7a9332198e0e}{Token Bonding Curves Explained}
	\item \href{https://blog.goodaudience.com/rewriting-the-story-of-human-collaboration-c33a8a4cd5b8}{Rewriting the Story of Human Collaboration}
\end{itemize}

\vspace{0.5cm}

Tatsächlich werden wir den \textit{Bonding-Curves-Ansatz} für unseren Pool-Project-Token später wieder aufgreifen und uns seiner Anwendung - den Gedanken aus dem Anhang zu \nameref{sec:bonding-curves} folgend - bemühen.

\vspace{0.3cm}

Der Vorgriff darauf erfolgte an dieser Stelle lediglich aufgrund des direkten Kontext-Bezugs zu Bullet 4 aus Anforderungsliste \ref{token-anforderungen}.

\end{Konzept}

%\newpage
\vspace{0.5cm}


Nun kommen wir - wie bereits angekündigt zu Mechanismen der \textbf{Token-Zirkulation}:

\vspace{0.3cm}


\begin{Konzept}[Token-Zirkulation-Mechanismen]
\label{circulation}
\vspace{0.2cm}

Eines sofort vorweg:

\vspace{0.2cm}
\todo{\noindent\hrulefill}

\todo{Die gleich vorgestellten Gedanken und Konzepte sind als noch nicht sehr ausgereifte initiale Ideen und Entwurfsmuster zu verstehen, die es noch zu erforschen und besser zu verstehen gilt. Mögen diese vielleicht in ihren grundlegenden Ansätzen noch so fundiert und durchdacht sein, wäre ein Anspruch ihrer perfekten Ausformulierung in einem - nicht auf fundierten praktischen Produkt-Erfahrung aufbauenden - White-Paper - wie es dieses aktuell ist - nur anmaßend und eine Vortäuschung einer pseudo-fundierten Theorie, die es aber ohne praktische Erprobung nicht ist.}

\vspace{0.2cm}

\todo{Vielmehr gilt es, die folgenden Ideen und Ansätze in ihrem Grundsatz zu verinnerlichen, und dabei gleichzeitig, die etwaigen Konkretisierungen mit Augenmaß \textit{weich} zu deuten, um diese mit zunehmender praktischer Anwendung zu validieren, zu justieren oder zu verwerfen.}

\vspace{0.2cm}

\todo{Dieser Teil des White-Papers ist also mit voller Absicht bis auf weiteres als \textbf{WIP} anzusehen und soll hier als solches markiert sein.}

\todo{\noindent\hrulefill}
\vspace{0.5cm}


\textbf{Die folgenden Ausführungen betrachten den anvisierten Pool-Project-Token in seinem Dasein als \textit{Utility-Token}.}

\vspace{0.2cm}

Es bedarf wahrscheinlich keiner weiteren Erklärung, wir verfolgten im Großen und Ganzen einen sich \textbf{positiv entwickelnden Token-Kurs} und richteten unsere \textit{Mechanism-Design}-Überlegungen genau diesem Ziel folgend aus.

\vspace{0.2cm}

Den Markt-Gesetzen folgend geht ein steigender Kurs mit \textbf{steigender Nachfrage und/oder knapper werdendem Angebot} der durch den Token repräsentierten \textit{Utility} einher.

\vspace{0.2cm}

Da wir die \textit{Utility} unseres Pool-Project-Tokens als Zahlungsmittel für die anfallenden Service-Fees des Pools-Service definiert haben, stellt uns die Gegenüberstellung der gewünschten \textbf{Kurssteigerung des Tokens} vs. des \textbf{Angebot-Nachfrage-Prinzips} vor ein nicht unerhebliches Problem:

\vspace{0.4cm}

\textbf{Die Nutzung des Pools-Service erfolgt über einen gewissen (längeren) Zeitraum. Die Entrichtung der Fees geschieht dagegen in einem einzigen Moment, was die Nachfrage nach dem Pools-Service von der Nachfrage nach dem zugehörigen \textit{Utility-Token} nahezu gänzlich voneinander ent\-koppelt - wenn nicht gar das gesamte Verständnis von einer Nachfrage nach dem \textit{Utility-Token} in sich zusammenfallen lässt.} 

\vspace{0.4cm}

Um genau diesem Problem entgegenzuwirken und die \textit{Utility} - die wir unverändert bei der Service-Fee-Abrechnung belassen wollen - zeitlich auf die übergeordnete Pools-Nutzung-Dienstleistung auszudehnen und dabei gleichzeitig eine \textbf{künstliche Verknappung} der zirkulierenden Pool-Project-Tokens zu induzieren, bedienen wir uns zweier entscheidender Design-Mechanismen:

\vspace{0.5cm}

\underline{\textbf{Pending-Fees:}}
\vspace{0.3cm}

Der simpelst denkbare Mechanismus dem oben aufgeworfenen Problem zu entgegnen, ist die künstliche Streckung des Zeitraums zwischen dem Moment, zu dem die Service-Fees anfallen und dem Moment, wo diese tatsächlich fließen.

\vspace{0.2cm}

Gleichwohl der Großteil der Service-Fees bereits in der initialen Phase eines \\ \textit{Investing-Pools} anfallen (siehe \nameref{sec:fees}), kann ihre Abrechnung - quasi die \textit{"Überweisung"} - durchaus (deutlich) später erfolgen. Dabei würden die Fees zwar trotzdem zum Fälligkeitszeitpunkt eingezogen werden, im Anschluss jedoch bis zum Abrechnungszeitpunkt - als \textit{Token-Stake} - bis zu ihrer Auszahlung an die Begünstigten in einer Art \textit{Treuhand} verweilen.

\vspace{0.1cm}

Auf diese Weise wäre der Service-Fee-äquivalente \textbf{Token-Betrag \textit{gelockt}} und damit ein kursfördernder - sich zwar in Zirkulation befindender aber nicht liquider - Bestandteil des gesamten Token-Supplys.

\vspace{0.2cm}

Der Abrechnungszeitpunkt wäre hierbei natürlich noch zu definieren. Dieser könnte z. B. entweder der Zeitpunkt der \nameref{sec:pools-liquidierung} sein, oder aber - zwecks besserer Planung und Vorbeugung \textit{"toter Pools"} - zeitlich wiederkehrende \textit{Abrechnungs-Stichtage}.

\vspace{0.75cm}

\underline{\textbf{Staking:}}

\vspace{0.3cm}

Der sogenannte \textbf{\textit{Staking-Mechanismus}} ist ein - besonders im \textit{DeFi}-Umfeld - bekanntes und sehr gängiges \textit{Mechanism-Design}-Mittel, seinen - häufig \textit{Utility}-angelehnten - Token, der Markt-Liquidität zu entziehen und damit eine künstliche Verknappung des Markt-Angebots zu erzeugen, womit eine positive Kurs-Entwicklung des Tokens befeuert werden soll. Der dabei auftretende Protagonist - \textbf{\textit{der Staker}} - wird hierbei mittels sogenannter \textbf{\textit{Staking-Reward}} - in aller Regel mindest garantiert\-en Zinssatzes - zum \textbf{\textit{Staking}} incentiviert.

\vspace{0.2cm}

Problematisch an diesem - generell sehr sinnvollen Ansatz - ist die Tatsache, diese eigne sich auch unheimlich gut als Hebel eines \textit{"Pump \& Dump"-Scams}: Ist die \textit{FOMO} eines \textit{"Pump \& Dump"-Token-Sales} erst gesät, implementiert man obendrauf noch einen \textit{Staking-Mechanismus} mit horrend hohen - jeglicher Realität entbehrenden - \textit{Staking-Rewards}, verknappt - in dem ohnehin kurzen Augenblick der extremen \textit{FOMO-Phase} - noch künstlich \textit{"pumpend"} das Token-Angebot und nimmt den Stakern zusätzlich die Handlungsfähigkeit, rechtzeitig auf den anstehenden \textbf{Dump} zu reagieren. 

Nach erfolgtem \textbf{Dump} bleiben die garantierten horrend \textit{Staking-Rewards} (teils in Größenordnungen von zig Prozent AM TAG) zwar weiter garantiert, nur sind diese - genauso wie der zugrundeliegende Token - plötzlich nichts mehr wert.

\noindent\hrulefill

\vspace{0.5cm}

Dennoch möchten wir bei unserem Pools-Project-Token nicht auf besagten \textbf{\textit{Staking-Mechanismus}} verzichten und haben auch für das adressierte Problem eine wasser\-dichte Lösung:

\vspace{0.2cm}
 
\textbf{Anstatt einen gewissen \textit{Staking-Zins} einfach unbegründet zu garantieren, wollen wir diesen in Bezug zu dem durch den \textit{erbrachten Stake} generierten Value setzen.}

\vspace{0.5cm}

Konkretisierend folgt nun en Detail unser anvisierter \textbf{\textit{Staking-Mechanismus}}:

\begin{itemize}
	\item Der Pool-Creator wird dazu verpflichtet, einen gewissen \textbf{\textit{Staking-Betrag}} - in Form des \textbf{Pools-Project-Tokens} - zu erbringen, um den Pool überhaupt erst eröffnen zu dürfen.
	\item Der zu leistende \textbf{\textit{Staking-Betrag}} richtet sich an den \textbf{\textit{zu erwartenden Service-Fees}}, die in Gänze über die gesamte \textit{Pool-Lifetime} anfallen werden/könnten. Hierbei ist ein etwaiger \textit{Multiplier} auf die Service-Fees anzunehmen.
	\item Der zu erbringende \textbf{\textit{Staking-Betrag}} (in Token) bleibt während der gesamten \textit{Pool-Lifetime} in einer Art \textbf{\textit{Treuhand}} verwahrt, ist damit \textbf{\textit{dem Markt-Angebot entzogen}} und wirkt damit kurs-befeuernd.
	\item Da die Service-Fees generell in Relation zum den im Pool bewegten finanziellen Mitteln stehen (siehe \nameref{sec:fees}) - also tendenziell in \textit{USDT} errechnet werden - der \textbf{\textit{Staking-Betrag}} jedoch in Pools-Project-Tokens zu erbringen ist, bleibt hierbei noch ein angemessenes \textit{Umrechnungs-Design} nachzuliefern.
	\item Der \textbf{\textit{Staking-Betrag}} kann und soll als eine Art \textbf{\textit{Sicherheit}} argumentiert werden, aber auch als eine Art \textbf{\textit{Preepaid-Fees-Konto}} des Pools verwendet werden können. 
	\item Zwar kann der \textbf{\textit{Staker}} in der Theorie seinen \textbf{\textit{Stake}} (oder einen Teil davon) verlieren - falls der Pool z. B. ungenutzt bleibt (\todo{hier bieten sich uns weitere Mechanism-Design-Möglichkeiten, das Verhalten der Pool-Teilnehmer zu be\-einflussen}) - soll dieses Szenario jedoch einen tendenziell ungewollten Edge-Case darstellen und der \textbf{\textit{Staker}} in aller Regel zu keinem \textbf{\textit{Payer}} werden.
	\item Den letzten Punkt aufgreifend soll der \textbf{\textit{Staker}} idealerweise von jegliche anfallenden Service-Fees befreit werden. Stattdessen sollen die Fees durch die \textbf{\textit{"passiven" Pool-Member}} getragen werden.
	\item Zusätzlich zur Befreiung von den Service-Fees, soll der \textbf{\textit{Staker}} einen Teil der anfallenden Fees als \textbf{\textit{Staking-Reward}} erhalten. Die Konkretisierung des genauen Anteils bzw. der Berechnungsgrundlage dieser erfolgt zu einem späteren Zeitpunkt.
	\item Der Staker ist damit nicht nur User des \textit{Pool-Service} sondern gleichzeitig auch ein Investor (Token-Holder) des übergeordneten Pools-Projects (da er gezwungen ist, die Tokens über einen längeren Zeitraum zu halten). Er avanciert damit zu der spannendsten Rolle innerhalb des \textbf{\textit{Pools-Ökosystems}}, da sich die Motivation seines Token-Besitzes besonders stark streut:
	\begin{itemize}
		\item Er braucht den Token als \textit{Utility} zu Nutzung der Pools-Dienstleistung.
		\item Er braucht den Token als \textbf{\textit{Staking-Betrag}} zur Erstellung eines neuen Pools, verdient aber gleichzeitig an dieser im Form einer Gewinnbeteiligung an den generierten Service-Fees. Damit ist er konsequenterweise incentiviert, neue Pools zu erstellen und diese aktiv zu bewerben.
		\item Als Token-Holder ist er gleichzeitig ein Projekt-Investor, Gewinnbeteiligter und damit Interessent und Werbetreibender \textbf{\textit{(word-of-mouth)}} für Wachs\-tum - also viele neue Pools, auch an denen er nicht aktiv beteiligt ist.
	\end{itemize}
\end{itemize}

\end{Konzept}

\vspace{0.5cm}



\paragraph{Umsetzung}
\textbf{ }
\vspace{0.3cm}

Gleichwohl noch nicht richtig quantifizierbar, jedoch konzeptuell bereits solide Formen annehmend, wollen wir an dieser Stelle endlich unseren angestrebten \textit{Pool-Project-Token} einführen und fortan an einem konkreten anstatt wie bisher abstrakt gehaltenem Gebilde weiterarbeiten:

\vspace{0.3cm}

\begin{Solution}[WunderPool-Token (\textit{WPT})]
\label{wpt}
\vspace{0.2cm}

Folgenden bisher erarbeiteten wesentlichen Ergebnissen folgend definieren wir den WunderPool-Token (\textbf{WPT}) als den anvisierten \textit{Pool-Project-Token}:

\begin{itemize}
	\item Die Monetarisierung des Pools-Projekt erfolgt durch Service-Fees, die Mittels des \textit{Utility-Tokens} \textbf{WPT} veranschlagt und abgerechnet werden (Prämissen \ref{monetarisierung} und \ref{fees-for-token}).
	\item Die Venture-Invests der \textbf{WPT}-Käufer werden durch echte Kapital-Rücklagen innerhalb des \textbf{WPT}-Token-Contracts gedeckt (Design-Merkmal \ref{bcm}).
	\item Alle \textbf{WPT}-Holder werden finanziell am etwaigen Projekt-Erfolg beteiligt \\ (Design-Merkmal \ref{token-usp}).
	\item Ein AMM zur \textbf{WPT}-Distribution (und initialem Token-Sale) wird bereitgestellt (Design-Merkmal \ref{bcm}).
	\item Der \textbf{WPT}-Supply und -Kurs wird zwecks Vermeidung von Inflation des \textit{Utility-Tokens} in gewissem Rahmen kontrolliert (Design-Merkmal \ref{bcm}).
	\item Bei gegebenem Demand nach dem Pools-Service wird die aktuelle Zirkulation des \textbf{WPT-Utility-Tokens} künstlich aufrecht erhalten und der Anteil der verfügbaren an sich in Zirkulation befindenden \textbf{WPT} künstlich verknappt (Design-Merkmal \ref{circulation}).
\end{itemize}

\vspace{0.5cm}

Zur Motivation des Einsatzes von \textbf{Bonding-Curves} beim hier besonders prägenden Design-Merkmal \ref{bcm} sei zur allgemeinen Einführung unter anderem auf die Artikel \href{https://medium.com/coinmonks/token-bonding-curves-explained-7a9332198e0e}{Token Bonding Curves Explained} und \href{https://blog.goodaudience.com/rewriting-the-story-of-human-collaboration-c33a8a4cd5b8}{Rewriting the Story of Human Collaboration}, zur Inspiration auf den Artikel \href{https://medium.com/atchai/can-we-save-the-utility-token-55ef639370cf}{Utility-Token als Bunding-Curves-Modell} und hinsichtlich Umsetzung auf unseren eigenen Content aus dem Anhang zu \nameref{sec:bonding-curves} verwiesen.

\vspace{0.3cm}
\todo{TODO: Welcher der obigen Punkte wird in welchen der folgenden Kapitel en Detail aufgegriffen und weiter vertiegt, um die noch fehlende Quantifizierung darzustellen?}

\end{Solution}

\vspace{0.5cm}

\todo{WIP}

\begin{itemize}
	\item Bonding-Curves
	\begin{itemize}
		\item Warum?
		\item Modellierung mit den Denkansätzen aus dem Anhang zu \nameref{sec:bonding-curves} unter Einbeziehung der obigen Gewinnbeteiligung an Fees
	\end{itemize}	
\end{itemize}

\vspace{0.5cm}



\paragraph{Ausblick}
\textbf{ }
\vspace{0.3cm}

\todo{WIP}

\begin{itemize}
	\item Der erste und größere Investor für das Pool-Projekt wäre WunderPass selbst. Für die erfolgte Einlage in den Projekt-Pool bekäme WunderPass WPT, die es für Incentivierungen und Rewards für die Nutzung von Pools verwenden könnte. Dieses Invest könnte (im Gegensatz zu den Einlagen anderer Investoren) zB. auch einem Locking unterliegen, um eine gewisse Preisstabilität des WPT zu gewährleisten.	
	\item Erste Andeutung, dass die Token-Contract-Treasury zwar im ersten Schritt in \textit{USDT} modelliert, jedoch in \textit{WUNDER} geplant ist.
	\item Ausblick auf die anschließenden Kapitel, die das präsentierte Konzept umsetzen.
	\item Erstmals auf \textit{Economics-Excel} verweisen.
\end{itemize}
    % binde die Datei '[Pools][Economic][WPT].tex' ein
%% !TEX root = paper.tex

\paragraph{Monetarisierung \& Tokenisierung}
\textbf{ }
\vspace{0.3cm}

Der abstrakt gehaltenen Einleitung zum finanziellen Grundgerüst unseres Pool-Projekts wollen wir in diesem Abschnitt nun den konzeptuell gedanklichen Grundstein zur dessen tatsächlichen Economics-Realisierung legen, auf dem dann im Anschluss die folgenden Kapitel aufbauen.

\vspace{0.2cm}

Dazu folgen zunächst einige - mehr oder minder erklärungsbedürftige - rohe Aussagen: 

\vspace{0.2cm} 

\begin{Praemisse}[Monetarisierung]
\label{monetarisierung}
\vspace{0.2cm}

Die Monetarisierung unseres Pool-Service soll auf Basis (prozentualer) Fees (siehe \nameref{sec:fees}) - gemessen am (finanziellen) Volumen der erbrachten Dienst\-leistung - erfolgen. Für den Moment sehr plakativ betrachtet, ist dies gleichbedeutend mit: 

\vspace{0.2cm} 

\textbf{\textit{Mit je mehr Kohle die Pools hantieren, desto größer sollen die anfallenden Fees sein!}}

\end{Praemisse}

\vspace{0.5cm}

\begin{Praemisse}[Utility-Token als Monetarisierungs-Tool für alle Stakeholder]
\label{fees-for-token}
\vspace{0.2cm}

\textbf{Die Fees sollen mittels eines dafür geschaffenen Utility-Tokens abgerechnet, erhoben und erbracht werden!}

\vspace{0.5cm} 

Für den - unbestreitbar verkomplizierenden und technisch teils nicht unerheblich umständlichen - Umweg der Monetarisierung über einen Token sehen wir folgende schlagende Argumente, die auch in den anschließend folgenden Kapiteln immer mal wieder argumentativ zum Vorschein kommen werden:

\begin{itemize}
	\item Die Nutzung des Pools-Service kann als ein echtes \textit{\textbf{Gut}} - eine \textit{Utility} - angesehen werden, was unter Umständen nicht endlos verfügbar sei (begrenzte Skalierung auf der Blockchain), besonders begehrt (bei exzellenter Service-Qualität) oder im Übermaß vorhanden (bei anfänglicher Unbekanntheit des Service) sei. 
	
	Durch die Tokenisierung der Dienstleistung einverleibt man dieser den Stellenwert einer \textit{Ressource}, mit zugehörigen Eigenschaften wie \textbf{Verfügbarkeit}, \textbf{Qualität} und \textbf{Nachhaltigkeitsgedanken}, was bei digitalen Dienstleistungen oft unberücksichtigt bleibt. 
	
	Mit diesem Ansatz kommt das \textit{Marktwirtschaftsprinzip von Angebot \& Nachfrage} auch bei digitalen Services zum Tragen, was in der digitalen Welt heutzutage ausschließlich auf \textit{Nachfrage} reduziert wurde, da das \textit{Angebot} de facto als unendlich betrachtet wird.
	\item Die Tokenisierung eines Business-Modells eröffnet einem das sehr mächtige spieltheoretische Werkzeug des \href{https://de.wikipedia.org/wiki/Mechanismus-Design-Theorie}{Mechanismus-Design}, um sämtliche Projekt-Beteiligte bzw. -Stakeholder in ihrem Verhalten hinsichtlich des übergeordneten Projekterfolgs zu beeinflussen/incentivieren. Oder simple ausgedrückt: Das zu tun, was wir aus strategischen Überlegungen möchten, dass er/sie tut.
	\item \textbf{Direkte \& unbürokratische Projekt-Finanzierung}.
	
	Durch die Tokenisierung der Dienstleistung muss ein potenzieller Investor beim Kauf von Utility-Tokens lediglich vom Erfolg der Dientleistung=Utility selbst überzeugt sein (da eine Nachfrage nach der Dienstleistung direkt an die Nachfrage nach dem zugehörigen Utility-Token gekoppelt ist), anstatt bei seiner ROI-Evaluierung herkömmliche bürokratisch geregelte Venture-Capital-Aspekte wie etwaige Shareholders-Agreements und Exit-Szenarien hinzuziehen zu müssen.
	\item Technische und konzeptuelle Vereinfachung, Flexibilität und Direktheit bei \textit{Customer-Akquise} und \textit{CRM} mittels des Utility-Tokens, da
	\begin{itemize}
		\item die \textit{Marketing-Währung} in Form von Tokens die \textbf{Utility} selbst statt \textit{Fiat} in den Vordergrund rückt.
		\item Der \textit{Project-Owner} (in dem Fall also WunderPass) in aller Regel selbst ein großer Token-Holder sein wird und somit über die Mittel verfügt, das Marketing-Volumen zu erbringen (ohne dabei zusätzlich finanziell belastet zu werden).
	\end{itemize}
	\item Uneingeschränkte Transparenz für alle Projekt-Teilnehmer über Stake, Cash-Flows, Handlungen, Strategien etc. aller anderen Projekt-Teilnehmer und damit ihrer Position und Interessen innerhalb des Projekts mittels jederzeit offen einsehbarer dezentraler Smart-Contract-Logik.
	\item Uneingeschränkte Transparenz und Eliminierung von Interpretationsspielraum hinsichtlich des Business-Plans.
	\item Zu guter Letzt sei noch das - weniger auf harten Fakten als auf dem \textit{Opportunitiy-Gedanken} begründete - Argument des vermeintlichen \textit{Tokenisierungs-Trends} zu nennen, welches ein rein selbstzweck-getriebenes Interesse bei potenziellen Token-Investoren wecken könnte.
\end{itemize}

\end{Praemisse}

\vspace{0.5cm}

\paragraph{Die entscheidende Idee}
\textbf{ }
\vspace{0.3cm}

Allen relevanten Erklärungen vorweggreifend folgt unser fundamentale \\
\textit{Token-Economics}-Ansatz für die Pools-Project-Token:

\vspace{0.2cm}

\begin{Konzept}[Dividende auf den Pools-Project-Token]
\label{token-usp}
\vspace{0.2cm}

Zusätzlich zur \textit{Utility}-Beschaffenheit unseres Pools-Project-Tokens möchten \\
wir diesem noch eine gewisse \textit{Equity}-Eigenschaft einverleiben:

\vspace{0.2cm}

\textbf{Ein Token-Besitzer soll mittels des Tokens nicht nur den Pools-Service nutzen können oder an der steigenden Nachfrage nach diesem - durch eine positive Kursentwicklung - profitieren, sondern zusätzlich DIREKT an den generierten Erträgen des gesamten Pools-Projects beteiligt werden.}

\vspace{0.2cm}

Er soll demnach de facto als Anteilseigner des Pools-Projects gelten und an etwaigen Gewinnen des Projekts - in Form einer gewissen \textit{Dividende} - pro rata seines Token-Volumens partizipieren.

\vspace{0.2cm}

Die Implementierung dieses \textit{Equity}-Mechanismus soll selbstverständlich mittels eines Smart-Contracts sichergestellt sein, was unseren Token stark von anderen \\ \textit{Equity}-Tokens abhebt.

\vspace{0.2cm}

Durch diesen zusätzlichen Kniff, schaffen wir eine sich selbst verstärkende Synergie zwischen den \textit{Utility}- und \textit{Equity}-Eigenschaften unseres Pools-Project-Tokens, indem wir einen potenziellen User des Pools-Service (besitzt \textit{Utility} in Form des Tokens) gleichzeitig zu einem Projekt-Investor machen (besitzt \textit{Equity} in Form desselben Tokens). Dieser doppelte Synergieeffekt weitet sich auch unmittelbar auf die Kursentwicklung aus. DENN: Wachsende Nutzung des Pools-Services impliziert zwangsläufig eine steigende Token-Zirkulation (im Sinne der \textit{Utility}-Beschaffenheit) und steigenden Bedarf und somit Nachfrage nach dem Token UND generiert gleich\-zeitig zunehmenden Ertrag durch Service-Fees, was wiederum eine Wertsteigerung des Tokens aus seiner \textit{Equity}-Beschaffenheit nach sich zieht.

\end{Konzept}

\vspace{0.3cm}

Wie genau wir uns das eben formulierte Vorhaben in der Umsetzung planen, wird etwas weiter unten vertieft. Zunächst bleiben wir beim ökonomischen Teil des Token-Designs und erarbeiten einige relevante Mechanismen.

\vspace{0.5cm}


\paragraph{Token-Design}
\textbf{ }
\vspace{0.3cm}

Beim Design unseres Pools-Project-Tokens wollen wir uns stark an den Gedanken des spieltheoretischen Gebiets des \href{https://de.wikipedia.org/wiki/Mechanismus-Design-Theorie}{Mechanismus-Design} orientieren.

Dieses Wissenschaftsgebiet befasst sich im Wesentlichen damit als \textit{höhere Instanz eines Spiels} - also in dem Fall wir als Project-Owner - mittels Regelgestaltung und Incentivierungs-Mechanismen - also in unserem Fall mittels Token-Design - Einfluss auf das Verhalten der Spieler - also in dem Fall Nutzer des Pool-Service und Investoren - im Sinne des Spiels nehmen kann.

\vspace{0.1cm}

Entscheidend hinsichtlich letzter Formulierung ist dabei das \textit{"... im Sinne des Spiels..."} genaust möglich zu präzisieren und idealerweise zu quantifizieren und formalisieren.

\vspace{0.5cm}

\textbf{Was möchten wir also genau wie, wann und womit erreichen für unser Pools-Projekt?}

\vspace{0.5cm}

Dabei bewegen sich die \href{https://de.wikipedia.org/wiki/Mechanismus-Design-Theorie}{Mechanismus-Design}-Werkzeuge tendenziell auf einer granularen Ebene, weshalb die Antwort \textit{"Pools-Project to the moon!"} auf obige Frage nicht in deren Sinne stünde. Viel mehr ist obige Frage daher als

\begin{itemize}
	\item Welche Etappenziele möchten wir erreichen (Projekt-Funding, Wachstum, Exit etc.)?
	\item Welche Projekt-Stakeholder (Gründer, Project-Owner, Investoren, User etc.) werden gebraucht und wie können diese gewonnen und deren Interessen gewahrt werden?
	\item Welche Hebel und designte Einflussmöglichkeiten möchten wir mittels von Token-Mechanismen besonders stark in eigener Hand behalten, anstatt sie dem Zufall oder Markt-Gesetzen zu überlassen?
	\item Welche Synergien möchten wir schaffen/verstärken bzw. verhindern/bremsen?
	\item Letzeres ist nicht nur aus Sicht des Pools-Projekts für sich alleinstehend zu betrachten sondern insbesondere auch im Hinblick auf ein etwaiges künftiges Wunder-Ökosystem. 
	\item Wie können wir als Gründer/Project-Owner (finanziell) profitieren?
\end{itemize}

zu verstehen. Um das ganze nicht ausufern zu lassen, wollen wir diese Fragestellungen stark auf das Pools-Projekt, seinen Projekt-Token und insbesondere dessen erhofften Effekte fokussieren:

\vspace{0.3cm}

\begin{Assumption}[Erwünschte Effekte des Pools-Project-Tokens]
\label{token-anforderungen}
\vspace{0.2cm}

Folgende Anforderungen, Erwartungen und Absichten verfolgen wir mit dem zu designenden Projekt-Token und/oder beabsichtigen zu erfüllen:

\begin{itemize}
	\item Selbstverständlich stellt ein gewisses initiales Projekt-Funding mittels Token-Sale eine der ausschlaggebendsten Motivationen für den Token dar, um z. B. auch Entwicklungskosten zu decken. 
	\item Gleichzeitig müssen aber eben die initialen Kapitalgeber angemessen für ihr Risiko entlohnt werden und signifikant stärker an ihrem Token-Invest profitieren als spätere Token-Käufer.
	\item Nicht verkehrt wäre gleiches für die Gründer ;)
	\item Nicht nur für die zuletzt genannten early Investors sondern generell für alle Token-Investoren möchten wir einen transparenten, berechenbaren und vertrauenswürdigen Token schaffen, 
	\begin{itemize}
		\item dessen Kursentwicklung keiner künstlichen PR-getriebenen Hysterie mit anschließendem Crash unterliegt (\textit{Pump \& Dump}),
		\item dessen Value transparenten und idealerweise durch Smart-Contracts ge\-steuerten Mechanismen und Projekt-Entwicklungen folgt,
		\item dessen Value einen \textit{Utility-}Bezug hat und
		\item der idealerweise mittels eines AMMs (\textit{Automated Market Maker}) jederzeit handelbar sein soll.
	\end{itemize}
	\item Nicht ganz so essenziell wie das initiale Projekt-Funding jedoch ebenfalls nicht zu vernachlässigen ist die fortlaufende (operative) Projekt-Finanzierung, die gänzlich oder zumindest teilweise durch den Projekt-Token mitfinanziert werden könnte.
	\item Gleichwohl der oben skizzierte USP unseres Tokens (siehe \ref{token-usp}) \textit{Equity}-techni\-scher Natur ist, ist und bleibt unserer Pools-Project-Token substanziell ein \textbf{\textit{Utility-Token}}.
	\begin{itemize}
		\item Grundsätzlich wird die Zirkulation eines \textit{Utility-Tokens} stets stark korreliert mit der Nutzung/Nachfrage der Utility - also in unserem Fall dem Pools-Service - sein. Wie solch eine Korrelation konkret aussieht, haben wir mittels des Token-Designs maßgeblich in eigener Hand. So kann man mit Mitteln wie z. B. \textit{Staking} oder \textit{Locking} die Zirkulation künstlich verlangsamen bzw. eine künstliche Verknappung an zirkulierenden Tokens induzieren.
		\item In gewisser Überzeugung, ein echter \textit{Utility-Token} repräsentiere eine nur endlich verfügbare Ressource, streben wir einen deflationären Token an. Oder zumindest einen \textit{pseudo-deflationären} (also einen, der zwar theoretisch unendlich lange weitergemintet werden kann, dies jedoch ab einem bestimmten Moment absolut unwirtschaftlich wird).
	\end{itemize}
\end{itemize}

\end{Assumption}

\vspace{0.5cm}

An der Abarbeitung dieser Liste werden wir uns - nicht zwingend die Reihenfolge wahrend - durch das restliche Kapitel hangeln. Bevor wir uns gleich im Anschluss etwas detaillierter dem letzten Punkt der obigen Liste - nämlich der Einflussnahme auf die Token-Zirkulation - widmen, zunächst ein sich sofort ersichtlicher \textit{Quick-Win} hinsichtlich Bullet 4 der obigen Liste:

\vspace{0.3cm}

\begin{Konzept}[Das \textit{Bonding-Curves-Modell} als vielversprechendes Mittel für unseren Pool-Project-Token]
\label{bcm}
\vspace{0.2cm}

Der Wunsch nach einem \textbf{transparenten, berechenbaren und vertrauenswürdig\-en Token} aus der Anforderungsliste \ref{token-anforderungen} suggeriert, das \textit{Bonding-Curves-Modell} als Grundlage zur Modellierung unseres Pool-Project-Token in Betracht zu ziehen, da der \textit{Bonding-Curves-Ansatz}

\begin{itemize}
	\item mittels Einsatzes eines Smart-Contract-AMMs, \textbf{Transparenz und Berechenbarkeit} des Tokens garantiert,
	\item durch im Token-Contract vorgehaltene \textbf{Kapital-Deckung pro ausgegebenem Token} das Investrisiko deckelt und damit die gewünschte \textbf{Vertrauenswürdig\-keit} abbildet und
	\item letztendlich durch seinen Basis-Mechanismus zwingend einen in seiner Logik verankerten AMM mitliefert.
\end{itemize}

\vspace{0.5cm}

Zur Einführung und Motivation des \textit{Bonding-Curves-Modells} sei repräsentativ auf folgende zwei Blog-Artikel verwiesen:

\begin{itemize}
	\item \href{https://medium.com/coinmonks/token-bonding-curves-explained-7a9332198e0e}{Token Bonding Curves Explained}
	\item \href{https://blog.goodaudience.com/rewriting-the-story-of-human-collaboration-c33a8a4cd5b8}{Rewriting the Story of Human Collaboration}
\end{itemize}

\vspace{0.5cm}

Tatsächlich werden wir den \textit{Bonding-Curves-Ansatz} für unseren Pool-Project-Token später wieder aufgreifen und uns seiner Anwendung - den Gedanken aus dem Anhang zu \nameref{sec:bonding-curves} folgend - bemühen.

\vspace{0.3cm}

Der Vorgriff darauf erfolgte an dieser Stelle lediglich aufgrund des direkten Kontext-Bezugs zu Bullet 4 aus Anforderungsliste \ref{token-anforderungen}.

\end{Konzept}

%\newpage
\vspace{0.5cm}


Nun kommen wir - wie bereits angekündigt zu Mechanismen der \textbf{Token-Zirkulation}:

\vspace{0.3cm}


\begin{Konzept}[Token-Zirkulation-Mechanismen]
\label{circulation}
\vspace{0.2cm}

Eines sofort vorweg:

\vspace{0.2cm}
\todo{\noindent\hrulefill}

\todo{Die gleich vorgestellten Gedanken und Konzepte sind als noch nicht sehr ausgereifte initiale Ideen und Entwurfsmuster zu verstehen, die es noch zu erforschen und besser zu verstehen gilt. Mögen diese vielleicht in ihren grundlegenden Ansätzen noch so fundiert und durchdacht sein, wäre ein Anspruch ihrer perfekten Ausformulierung in einem - nicht auf fundierten praktischen Produkt-Erfahrung aufbauenden - White-Paper - wie es dieses aktuell ist - nur anmaßend und eine Vortäuschung einer pseudo-fundierten Theorie, die es aber ohne praktische Erprobung nicht ist.}

\vspace{0.2cm}

\todo{Vielmehr gilt es, die folgenden Ideen und Ansätze in ihrem Grundsatz zu verinnerlichen, und dabei gleichzeitig, die etwaigen Konkretisierungen mit Augenmaß \textit{weich} zu deuten, um diese mit zunehmender praktischer Anwendung zu validieren, zu justieren oder zu verwerfen.}

\vspace{0.2cm}

\todo{Dieser Teil des White-Papers ist also mit voller Absicht bis auf weiteres als \textbf{WIP} anzusehen und soll hier als solches markiert sein.}

\todo{\noindent\hrulefill}
\vspace{0.5cm}


\textbf{Die folgenden Ausführungen betrachten den anvisierten Pool-Project-Token in seinem Dasein als \textit{Utility-Token}.}

\vspace{0.2cm}

Es bedarf wahrscheinlich keiner weiteren Erklärung, wir verfolgten im Großen und Ganzen einen sich \textbf{positiv entwickelnden Token-Kurs} und richteten unsere \textit{Mechanism-Design}-Überlegungen genau diesem Ziel folgend aus.

\vspace{0.2cm}

Den Markt-Gesetzen folgend geht ein steigender Kurs mit \textbf{steigender Nachfrage und/oder knapper werdendem Angebot} der durch den Token repräsentierten \textit{Utility} einher.

\vspace{0.2cm}

Da wir die \textit{Utility} unseres Pool-Project-Tokens als Zahlungsmittel für die anfallenden Service-Fees des Pools-Service definiert haben, stellt uns die Gegenüberstellung der gewünschten \textbf{Kurssteigerung des Tokens} vs. des \textbf{Angebot-Nachfrage-Prinzips} vor ein nicht unerhebliches Problem:

\vspace{0.4cm}

\textbf{Die Nutzung des Pools-Service erfolgt über einen gewissen (längeren) Zeitraum. Die Entrichtung der Fees geschieht dagegen in einem einzigen Moment, was die Nachfrage nach dem Pools-Service von der Nachfrage nach dem zugehörigen \textit{Utility-Token} nahezu gänzlich voneinander ent\-koppelt - wenn nicht gar das gesamte Verständnis von einer Nachfrage nach dem \textit{Utility-Token} in sich zusammenfallen lässt.} 

\vspace{0.4cm}

Um genau diesem Problem entgegenzuwirken und die \textit{Utility} - die wir unverändert bei der Service-Fee-Abrechnung belassen wollen - zeitlich auf die übergeordnete Pools-Nutzung-Dienstleistung auszudehnen und dabei gleichzeitig eine \textbf{künstliche Verknappung} der zirkulierenden Pool-Project-Tokens zu induzieren, bedienen wir uns zweier entscheidender Design-Mechanismen:

\vspace{0.5cm}

\underline{\textbf{Pending-Fees:}}
\vspace{0.3cm}

Der simpelst denkbare Mechanismus dem oben aufgeworfenen Problem zu entgegnen, ist die künstliche Streckung des Zeitraums zwischen dem Moment, zu dem die Service-Fees anfallen und dem Moment, wo diese tatsächlich fließen.

\vspace{0.2cm}

Gleichwohl der Großteil der Service-Fees bereits in der initialen Phase eines \\ \textit{Investing-Pools} anfallen (siehe \nameref{sec:fees}), kann ihre Abrechnung - quasi die \textit{"Überweisung"} - durchaus (deutlich) später erfolgen. Dabei würden die Fees zwar trotzdem zum Fälligkeitszeitpunkt eingezogen werden, im Anschluss jedoch bis zum Abrechnungszeitpunkt - als \textit{Token-Stake} - bis zu ihrer Auszahlung an die Begünstigten in einer Art \textit{Treuhand} verweilen.

\vspace{0.1cm}

Auf diese Weise wäre der Service-Fee-äquivalente \textbf{Token-Betrag \textit{gelockt}} und damit ein kursfördernder - sich zwar in Zirkulation befindender aber nicht liquider - Bestandteil des gesamten Token-Supplys.

\vspace{0.2cm}

Der Abrechnungszeitpunkt wäre hierbei natürlich noch zu definieren. Dieser könnte z. B. entweder der Zeitpunkt der \nameref{sec:pools-liquidierung} sein, oder aber - zwecks besserer Planung und Vorbeugung \textit{"toter Pools"} - zeitlich wiederkehrende \textit{Abrechnungs-Stichtage}.

\vspace{0.75cm}

\underline{\textbf{Staking:}}

\vspace{0.3cm}

Der sogenannte \textbf{\textit{Staking-Mechanismus}} ist ein - besonders im \textit{DeFi}-Umfeld - bekanntes und sehr gängiges \textit{Mechanism-Design}-Mittel, seinen - häufig \textit{Utility}-angelehnten - Token, der Markt-Liquidität zu entziehen und damit eine künstliche Verknappung des Markt-Angebots zu erzeugen, womit eine positive Kurs-Entwicklung des Tokens befeuert werden soll. Der dabei auftretende Protagonist - \textbf{\textit{der Staker}} - wird hierbei mittels sogenannter \textbf{\textit{Staking-Reward}} - in aller Regel mindest garantiert\-en Zinssatzes - zum \textbf{\textit{Staking}} incentiviert.

\vspace{0.2cm}

Problematisch an diesem - generell sehr sinnvollen Ansatz - ist die Tatsache, diese eigne sich auch unheimlich gut als Hebel eines \textit{"Pump \& Dump"-Scams}: Ist die \textit{FOMO} eines \textit{"Pump \& Dump"-Token-Sales} erst gesät, implementiert man obendrauf noch einen \textit{Staking-Mechanismus} mit horrend hohen - jeglicher Realität entbehrenden - \textit{Staking-Rewards}, verknappt - in dem ohnehin kurzen Augenblick der extremen \textit{FOMO-Phase} - noch künstlich \textit{"pumpend"} das Token-Angebot und nimmt den Stakern zusätzlich die Handlungsfähigkeit, rechtzeitig auf den anstehenden \textbf{Dump} zu reagieren. 

Nach erfolgtem \textbf{Dump} bleiben die garantierten horrend \textit{Staking-Rewards} (teils in Größenordnungen von zig Prozent AM TAG) zwar weiter garantiert, nur sind diese - genauso wie der zugrundeliegende Token - plötzlich nichts mehr wert.

\noindent\hrulefill

\vspace{0.5cm}

Dennoch möchten wir bei unserem Pools-Project-Token nicht auf besagten \textbf{\textit{Staking-Mechanismus}} verzichten und haben auch für das adressierte Problem eine wasser\-dichte Lösung:

\vspace{0.2cm}
 
\textbf{Anstatt einen gewissen \textit{Staking-Zins} einfach unbegründet zu garantieren, wollen wir diesen in Bezug zu dem durch den \textit{erbrachten Stake} generierten Value setzen.}

\vspace{0.5cm}

Konkretisierend folgt nun en Detail unser anvisierter \textbf{\textit{Staking-Mechanismus}}:

\begin{itemize}
	\item Der Pool-Creator wird dazu verpflichtet, einen gewissen \textbf{\textit{Staking-Betrag}} - in Form des \textbf{Pools-Project-Tokens} - zu erbringen, um den Pool überhaupt erst eröffnen zu dürfen.
	\item Der zu leistende \textbf{\textit{Staking-Betrag}} richtet sich an den \textbf{\textit{zu erwartenden Service-Fees}}, die in Gänze über die gesamte \textit{Pool-Lifetime} anfallen werden/könnten. Hierbei ist ein etwaiger \textit{Multiplier} auf die Service-Fees anzunehmen.
	\item Der zu erbringende \textbf{\textit{Staking-Betrag}} (in Token) bleibt während der gesamten \textit{Pool-Lifetime} in einer Art \textbf{\textit{Treuhand}} verwahrt, ist damit \textbf{\textit{dem Markt-Angebot entzogen}} und wirkt damit kurs-befeuernd.
	\item Da die Service-Fees generell in Relation zum den im Pool bewegten finanziellen Mitteln stehen (siehe \nameref{sec:fees}) - also tendenziell in \textit{USDT} errechnet werden - der \textbf{\textit{Staking-Betrag}} jedoch in Pools-Project-Tokens zu erbringen ist, bleibt hierbei noch ein angemessenes \textit{Umrechnungs-Design} nachzuliefern.
	\item Der \textbf{\textit{Staking-Betrag}} kann und soll als eine Art \textbf{\textit{Sicherheit}} argumentiert werden, aber auch als eine Art \textbf{\textit{Preepaid-Fees-Konto}} des Pools verwendet werden können. 
	\item Zwar kann der \textbf{\textit{Staker}} in der Theorie seinen \textbf{\textit{Stake}} (oder einen Teil davon) verlieren - falls der Pool z. B. ungenutzt bleibt (\todo{hier bieten sich uns weitere Mechanism-Design-Möglichkeiten, das Verhalten der Pool-Teilnehmer zu be\-einflussen}) - soll dieses Szenario jedoch einen tendenziell ungewollten Edge-Case darstellen und der \textbf{\textit{Staker}} in aller Regel zu keinem \textbf{\textit{Payer}} werden.
	\item Den letzten Punkt aufgreifend soll der \textbf{\textit{Staker}} idealerweise von jegliche anfallenden Service-Fees befreit werden. Stattdessen sollen die Fees durch die \textbf{\textit{"passiven" Pool-Member}} getragen werden.
	\item Zusätzlich zur Befreiung von den Service-Fees, soll der \textbf{\textit{Staker}} einen Teil der anfallenden Fees als \textbf{\textit{Staking-Reward}} erhalten. Die Konkretisierung des genauen Anteils bzw. der Berechnungsgrundlage dieser erfolgt zu einem späteren Zeitpunkt.
	\item Der Staker ist damit nicht nur User des \textit{Pool-Service} sondern gleichzeitig auch ein Investor (Token-Holder) des übergeordneten Pools-Projects (da er gezwungen ist, die Tokens über einen längeren Zeitraum zu halten). Er avanciert damit zu der spannendsten Rolle innerhalb des \textbf{\textit{Pools-Ökosystems}}, da sich die Motivation seines Token-Besitzes besonders stark streut:
	\begin{itemize}
		\item Er braucht den Token als \textit{Utility} zu Nutzung der Pools-Dienstleistung.
		\item Er braucht den Token als \textbf{\textit{Staking-Betrag}} zur Erstellung eines neuen Pools, verdient aber gleichzeitig an dieser im Form einer Gewinnbeteiligung an den generierten Service-Fees. Damit ist er konsequenterweise incentiviert, neue Pools zu erstellen und diese aktiv zu bewerben.
		\item Als Token-Holder ist er gleichzeitig ein Projekt-Investor, Gewinnbeteiligter und damit Interessent und Werbetreibender \textbf{\textit{(word-of-mouth)}} für Wachs\-tum - also viele neue Pools, auch an denen er nicht aktiv beteiligt ist.
	\end{itemize}
\end{itemize}

\end{Konzept}

\vspace{0.5cm}



\paragraph{Umsetzungskonzept}
\textbf{ }
\vspace{0.3cm}

Gleichwohl noch nicht richtig quantifizierbar, jedoch konzeptuell bereits solide Formen annehmend, wollen wir an dieser Stelle endlich unseren angestrebten \textit{Pool-Project-Token} einführen und fortan an einem konkreten anstatt wie bisher abstrakt gehaltenem Gebilde weiterarbeiten:

\vspace{0.3cm}

\begin{Solution}[WunderPool-Token (\textit{WPT})]
\label{wpt}
\vspace{0.2cm}

Folgenden bisher erarbeiteten wesentlichen Ergebnissen folgend definieren wir den WunderPool-Token (\textbf{WPT}) als den anvisierten \textit{Pool-Project-Token}:

\begin{enumerate}
	\item Die Monetarisierung des Pools-Projekt erfolgt durch Service-Fees, die Mittels des \textit{Utility-Tokens} \textbf{WPT} veranschlagt und abgerechnet werden (Prämissen \ref{monetarisierung} und \ref{fees-for-token}).
	\item Die Venture-Invests der \textbf{WPT}-Käufer werden durch echte Kapital-Rücklagen innerhalb des \textbf{WPT}-Token-Contracts gedeckt (Design-Merkmal \ref{bcm}).
	\item Alle \textbf{WPT}-Holder werden finanziell am etwaigen Projekt-Erfolg beteiligt \\ (Design-Merkmal \ref{token-usp}).
	\item Ein AMM zur \textbf{WPT}-Distribution (und initialem Token-Sale) wird bereitgestellt (Design-Merkmal \ref{bcm}).
	\item Der \textbf{WPT}-Supply und -Kurs wird zwecks Vermeidung von Inflation des \textit{Utility-Tokens} in gewissem Rahmen kontrolliert (Design-Merkmal \ref{bcm}).
	\item Bei gegebenem Demand nach dem Pools-Service wird die aktuelle Zirkulation des \textbf{WPT-Utility-Tokens} künstlich aufrechterhalten und der Anteil der verfügbaren sich in Zirkulation befindenden \textbf{WPT} künstlich verknappt (Design-Merkmal \ref{circulation}).
\end{enumerate}

\vspace{0.5cm}

Zur Motivation des Einsatzes von \textbf{Bonding-Curves} beim hier besonders prägenden Design-Merkmal \ref{bcm} sei zur allgemeinen Einführung unter anderem auf die Artikel \href{https://medium.com/coinmonks/token-bonding-curves-explained-7a9332198e0e}{Token Bonding Curves Explained} und \href{https://blog.goodaudience.com/rewriting-the-story-of-human-collaboration-c33a8a4cd5b8}{Rewriting the Story of Human Collaboration}, zur Inspiration auf den Artikel \href{https://medium.com/atchai/can-we-save-the-utility-token-55ef639370cf}{Utility-Token als Bunding-Curves-Modell} und hinsichtlich Umsetzung auf unseren eigenen Content aus dem Anhang zu \nameref{sec:bonding-curves} verwiesen.

\end{Solution}

\vspace{0.5cm}

Konzeptuell können wir an dieser Stelle - bis auf etwaige kleinere Justierungen - einen Haken an das Design unseres \textbf{WPT-Tokens} setzen und uns im Folgenden an die quantitative, konkrete Modellierung desselben wagen. Bezugnehmend auf die eben erfolgte Definition des \textbf{WPT} werden wir

\begin{itemize}
	\item in Abschnitt \nameref{sec:fees} die Grundlage für obigen Punkt (1) schaffen,
	\item in Abschnitt \nameref{sec:bp} die Berechnungsgrundlage und das Potenzial hinsichtlich obigen Punkts (3) beleuchten,
	\item uns bei obigen Punkten (2), (4) und (5) auf das Vertrautsein des Lesers zu \textbf{\textit{Bonding-Curves}} generell und der halbwegs verstandenen Lektüre des - teils sehr mathematischen - Anhangs zu unserem Blickwinkel auf \nameref{sec:bonding-curves} berufen,
	\item von einer (vortäuschenden) Genauigkeit bezüglich obigen Punkts (6) zum aktuellen Zeitpunkt im White-Paper absehen, die zugehörigen Parameter initial nach bestem Wissen und Gewissen schätzen und erst mit zunehmenden praktischen Erkenntnissen eine Justierung vornehmen, die auch ihren Platz im White-Paper findet, und
	\item schlussendlich in Abschnitt \nameref{sec:wpt} alle Ergebnisse einfließenlassend ein Token-Modell formuliert, das sich ohne offene Fragen in einen Token-Contract gießen lassen sollte. 
\end{itemize}

\vspace{0.3cm}

Bevor es also in den folgenden Abschnitten dann letztendlich an die quantitative Modellierung geht, bleiben noch einige letzte konzeptuelle Gedanken zu formulieren, die bisher untergegangen sein könnten.

\vspace{0.5cm}


\paragraph{Ausblick}
\textbf{ }
\vspace{0.3cm}

Um die ohnehin nicht ganz geringe Komplexität unseres \textit{WPT} nicht noch mehr ausufern zu lassen, haben wir in allen obigen Gedanken und Ausführungen einen sehr entscheidenden Punkt geschickt ausgespart. Leicht angedeutet wurde dies stets durch die verwendete Bezeichnung \textit{Pools-\textbf{Project}-Token}. Und wahrhaftig stellt der \textit{WPT} weder unseren \textbf{\textit{'Main'}}-Token dar, noch bleibt er der einzige.

\vspace{0.3cm}

\begin{Abgrenzung}[Der \textit{WPT} ist \textbf{nicht} der \textit{WUNDER}]
\vspace{0.2cm}

An der Stelle sei die Erinnerung daran, dass es bei dem \textit{Pool-Projekt} lediglich um ein Teilprojekt der übergeordneten großen \textbf{WunderPass-Vision} handelt, sehr angebracht. Um ein Teilprojekt als kleiner Baustein des anvisierten größeren \textbf{WunderPass-Ökosystems} - und als solches versehen mit seinem eigenen Projekt-Token. 

\vspace{0.2cm}

Der \textbf{\textit{WUNDER}} soll dagegen den Ökosystem-übergreifenden \textit{'Main'}-Token dar\-stellen. \todo{$[ \rightarrow Verlinken]$}

\vspace{0.2cm}

Eine Begründung für die Trennung der \textit{Project}-Tokens vom \textit{'Main'}-Token würde an dieser Stelle den Rahmen sprengen (\todo{hier wäre eine Verlinkung zu bezugnehmenden Kapiteln des White-Papers sehr wünschenswert; ist aktuell noch etwas chaotisch}).

\vspace{0.2cm}

Gleichzeitig ist es einleuchtend und nicht weiter erklärungsbedürftig, dass der \textit{WPT} in den Kontext des \textit{WUNDER} eingeordnet werden muss, wie sich das \textit{Pool-Projekt} in den Kontext des \textbf{WunderPass-Ökosystems} einordnet. Das soll aber nicht Gegenstand dieses Kapitels sein.

\vspace{0.3cm}
\todo{\noindent\hrulefill}

\todo{Grundsätzlich ist die Verknüpfung zwischen \textit{WPT} und \textit{WUNDER} konzeptuell auch noch nicht abgeschlossen. Dies wird auch noch größerer Design- und Entwicklungsblöcke erfordern und Zeit brauchen. Vermutlich wird dies gar ein fortlaufenden Erprobungsprozess werden. Der \textit{WUNDER} ist daher zum aktuellen Moment auch noch sehr vage und unfinal gehalten. Das ist durchaus so beabsichtigt, um sich möglichst keines Potenzials durch zu frühe Entscheidungen zu berauben.}

\vspace{0.2cm}

\todo{Für den \textit{WPT} sind diese gewissen \textit{Unfertigkeit} aber größtenteils von keinem Nachteil oder in irgendeiner Weise problematisch. Er kann sich aktuell einfach auf einen \textit{abstrakten WUNDER} referenzieren.}

\vspace{0.2cm}

\todo{Wichtig ist es lediglich, dies an dieser Stelle deutlich klargestellt und genannt zu haben. Und bei gegeben Fortschritt hier textuelle Anpassungen vorzunehmen}

\todo{\noindent\hrulefill}
\vspace{0.3cm}

\todo{WIP}

\begin{itemize}
	\item Erste Andeutung, dass die Token-Contract-Treasury zwar im ersten Schritt in \textit{USDT} modelliert, jedoch in \textit{WUNDER} geplant ist.
\end{itemize}

\end{Abgrenzung}


\vspace{0.3cm}
\todo{WIP}

\begin{itemize}
	\item Der erste und größere Investor für das Pool-Projekt wäre WunderPass selbst. Für die erfolgte Einlage in den Projekt-Pool bekäme WunderPass WPT, die es für Incentivierungen und Rewards für die Nutzung von Pools verwenden könnte. Dieses Invest könnte (im Gegensatz zu den Einlagen anderer Investoren) zB. auch einem Locking unterliegen, um eine gewisse Preisstabilität des WPT zu gewährleisten.	
	\item Erstmals auf \textit{Economics-Excel} verweisen.
\end{itemize}

\vspace{0.5cm}

\subsection{Begleitende Excel}
\label{sec:excel}
\vspace{0.2cm}
% !TEX root = paper.tex

Die Rechnerei der anschließenden Kapitel wird stets durch die Excel-Datei \href{pool-economics-V3.xlsx}{Pools-Economics} begleitet, auf die sich insbesondere auch die zahlenmäßigen Werte der folgenden Kapitel beziehen.

\vspace{0.2cm}

Dabei können die Werte der Excel unter Umständen einen aktuellen Fortschrittsstatus widerspiegeln als das bezugnehmende Äquivalent der folgenden Kapitel.    % binde die Datei '[Pools][Economic][Excel].tex' ein
\vspace{0.5cm}

\subsection{Gebühren-Ordnung}
\label{sec:fees}
\vspace{0.2cm}
%% !TEX root = paper.tex

\paragraph{Gebühren-Modell}
\textbf{ }
\vspace{0.2cm}

\todo{Es muss noch zeitliche Service-Pauschale (z.B. nach Ablauf jeden Jahres) veranschlagt werden, um gänzlich inaktive Pool (mit Geld drauf) zu verhindern.}

\vspace{0.3cm}

\begin{Assumption}[Gebühren]\label{fees}

Es sollen in etwa folgende \textit{Basic-Fees} anfallen:

\begin{itemize}
	\item Grundgebühr von 1.9 \% auf den Deposit (für jeden Pool-Teilnehmer außer des Pool-Creators).
	\item Tradinggebühr von 0.1 \% auf jede Kauf- oder Verkaufsorder.
	\item Gewinnprovision von 9.9 \% auf einen durch den Pool erwirtschafteten \textbf{positiven} EBIT (bei Liquidierung des Pools).
\end{itemize}

\vspace{0.2cm}

Ergänzt werde diese durch etwaige \textit{Service-Fees}: 

\begin{itemize}
	\item Erweiterte Grundgebühr von zusätzlichen 1.5 \% auf den Deposit bei einem späteren Pool-Beitritt (additiv zu der obigen Basis-Grundgebühr).
	\item \textit{Leaving-Gebühr} von 6.9 \% auf den Cashout-Betrag bei vorzeitigem Verlassen des Pools und Cashout seitens eines Pool-Teilnehmers, falls der Cashout über den Pool-Contract erfolgt (und nicht z.B. mittels Verkaufs der Shares an einen anderen Pool-Teilnehmer oder am Sekundär-Markt).
\end{itemize}

\vspace{0.2cm}

Zudem sind folgende \textit{Benefits} hinsichtlich der Gebührenordnung für Inhaber eines Pass-NFTs \todo{(Kapitel verlinken)} vorgesehen:

\begin{itemize}
	\item Wegfall der Deposit-Grundgebühr für Inhaber eines PassNFTs des Status \textit{Diamond} und \textit{Black}.
	\item Reduzierung sämtlicher Gebühren, die auf User- und nicht Pool-Basis anfallen um
	\begin{itemize}
		\item 50 \% für Teilnehmer mit PassNFT-Status \textit{Diamond},
		\item 30 \% für Teilnehmer mit PassNFT-Status \textit{Black},
		\item 20 \% für Teilnehmer mit PassNFT-Status \textit{Pearl},
		\item 10 \% für Teilnehmer mit PassNFT-Status \textit{Platin}.
	\end{itemize}
\end{itemize}

\vspace{0.5cm}	

Aktuell nicht berücksichtigt jedoch grundsätzlich spannend sind die folgenden Gebühren-Aspekte und -Varianten:

\begin{itemize}
	\item Eine mögliche \textit{Trial-vs-Pro-Gebührenordnung}, bei der (stark) limitierte Pools (sowohl finanziell als auch feature-technisch) gänzlich kostenlos bleiben könnten, während eine unlimitierte Nutzung mit höheren Gebühren als den obigen einhergehen würde.
	\item \textit{Managed-Pools}: Pools, die von einem erprobten und erfolgreichen Pool-Creator hinsichtlich der Invests gesteuert, könnten eine höhere Teilnahme-Gebühr erfordern, an der auch der Creator maßgeblich beteiligt wird. 
\end{itemize}	

\end{Assumption}

\vspace{0.5cm}



\paragraph{Gebühren-Abrechnung}
\textbf{ }
\vspace{0.2cm}




\todo{WIP}

\begin{itemize}
	\item Klarstellung und Erklärung, dass die Gebühren zunächst in Fiat berechnet, jedoch am Ende in WPT veranschlagt werden.
	\item Erste Andeutung, dass die Token-Contract-Treasury zwar im ersten Schritt in \textit{USDT} modelliert, jedoch in \textit{WUNDER} geplant ist.
	\item Cash-Flows hinsichtlich der Fees:
	\begin{itemize}
		\item Wann fallen die Gebühren an? 
		\item Wann und wie werden diese in WPT transferiert? 
		\item Wann werden diese ausgezahlt? $\rightarrow$ \textit{Pending-Fees}
	\end{itemize}
	\item Handling inaktiver Pool über die Zeit - inklusive einer automatischen Eliminierung nach einem gewissen Inaktivitätszeitraum (um auch reale Finanzmittel nicht in toten Pools verloren gehen zu lassen, wenn die Teilnehmer - wie auch immer geartet - "kryptografisch tot" sind).
\end{itemize}

\vspace{0.5cm}

Abgerechnet werden die auf den Pool anfallenden Fees (selbst die ausschließlich User-basierten) aufgrund von \textit{Mechanism-Design}-Überlegungen erst bei seiner Liquidierung.

\vspace{0.3cm}

\begin{Praemisse}[Abrechnung]

Sämtliche für einen Pool angefallenen Fees werden (ungeachtet ihres Fälligkeitszeit\-punkts) fließen erst bei seiner Liquidierung und werden zwischen Fälligkeit und Entrichtung in einem gesonderten Teil der \textit{Pool-Treasury} vorgehalten (ähnlich dessen, wo der gestakte Betrag des Pool-Creators verwahrt wird).

\vspace{0.2cm}

Wir werden diese finanziellen Mittel im weiteren Verlauf auch als \textbf{\textit{Pending-Fees}} bezeichnen.

\vspace{0.2cm}

In Analogie dazu werden wir an geeigneter Stelle folgend auch von \textbf{\textit{Staked-Fees}} sprechen - gleichwohl es sich dabei eher um eine Sicherheit als um tatsächliche Fees handelt.

\end{Praemisse}

\vspace{0.5cm}

\todo{Ende WIP}    % binde die Datei '[Pools][Economic][Fees].tex' ein
% !TEX root = paper.tex

\paragraph{Gebühren-Modell}
\textbf{ }
\vspace{0.2cm}

\todo{Es muss noch zeitliche Service-Pauschale (z.B. nach Ablauf jeden Jahres) veranschlagt werden, um gänzlich inaktive Pool (mit Geld drauf) zu verhindern.}

\vspace{0.3cm}

\begin{Assumption}[Gebühren]\label{fees}

Es sollen in etwa folgende \textit{Basic-Fees} anfallen:

\begin{itemize}
	\item Grundgebühr von 1.9 \% auf den Deposit (für jeden Pool-Teilnehmer außer des Pool-Creators).
	\item Tradinggebühr von 0.1 \% auf jede Kauf- oder Verkaufsorder.
	\item Gewinnprovision von 9.9 \% auf einen durch den Pool erwirtschafteten \textbf{positiven} EBIT (bei Liquidierung des Pools).
\end{itemize}

\vspace{0.2cm}

Ergänzt werde diese durch etwaige \textit{Service-Fees}: 

\begin{itemize}
	\item Erweiterte Grundgebühr von zusätzlichen 1.5 \% auf den Deposit bei einem späteren Pool-Beitritt (additiv zu der obigen Basis-Grundgebühr).
	\item \textit{Leaving-Gebühr} von 6.9 \% auf den Cashout-Betrag bei vorzeitigem Verlassen des Pools und Cashout seitens eines Pool-Teilnehmers, falls der Cashout über den Pool-Contract erfolgt (und nicht z.B. mittels Verkaufs der Shares an einen anderen Pool-Teilnehmer oder am Sekundär-Markt).
\end{itemize}

\vspace{0.2cm}

Zudem sind folgende \textit{Benefits} hinsichtlich der Gebührenordnung für Inhaber eines Pass-NFTs \todo{(Kapitel verlinken)} vorgesehen:

\begin{itemize}
	\item Wegfall der Deposit-Grundgebühr für Inhaber eines PassNFTs des Status \textit{Diamond} und \textit{Black}.
	\item Reduzierung sämtlicher Gebühren, die auf User- und nicht Pool-Basis anfallen um
	\begin{itemize}
		\item 50 \% für Teilnehmer mit PassNFT-Status \textit{Diamond},
		\item 30 \% für Teilnehmer mit PassNFT-Status \textit{Black},
		\item 20 \% für Teilnehmer mit PassNFT-Status \textit{Pearl},
		\item 10 \% für Teilnehmer mit PassNFT-Status \textit{Platin}.
	\end{itemize}
\end{itemize}

\vspace{0.5cm}	

Aktuell nicht berücksichtigt jedoch grundsätzlich spannend sind die folgenden Gebühren-Aspekte und -Varianten:

\begin{itemize}
	\item Eine mögliche \textit{Trial-vs-Pro-Gebührenordnung}, bei der (stark) limitierte Pools (sowohl finanziell als auch feature-technisch) gänzlich kostenlos bleiben könnten, während eine unlimitierte Nutzung mit höheren Gebühren als den obigen einhergehen würde.
	\item \textit{Managed-Pools}: Pools, die von einem erprobten und erfolgreichen Pool-Creator hinsichtlich der Invests gesteuert, könnten eine höhere Teilnahme-Gebühr erfordern, an der auch der Creator maßgeblich beteiligt wird. 
\end{itemize}	

\end{Assumption}

\vspace{0.5cm}



\paragraph{Gebühren-Abrechnung}
\textbf{ }
\vspace{0.2cm}




\todo{WIP}

\begin{itemize}
	\item Klarstellung und Erklärung, dass die Gebühren zunächst in Fiat berechnet, jedoch am Ende in WPT veranschlagt werden.
	\item Erste Andeutung, dass die Token-Contract-Treasury zwar im ersten Schritt in \textit{USDT} modelliert, jedoch in \textit{WUNDER} geplant ist.
	\item Cash-Flows hinsichtlich der Fees:
	\begin{itemize}
		\item Wann fallen die Gebühren an? 
		\item Wann und wie werden diese in WPT transferiert? 
		\item Wann werden diese ausgezahlt? $\rightarrow$ \textit{Pending-Fees}
	\end{itemize}
	\item Handling inaktiver Pool über die Zeit - inklusive einer automatischen Eliminierung nach einem gewissen Inaktivitätszeitraum (um auch reale Finanzmittel nicht in toten Pools verloren gehen zu lassen, wenn die Teilnehmer - wie auch immer geartet - "kryptografisch tot" sind).
\end{itemize}

\vspace{0.5cm}

Abgerechnet werden die auf den Pool anfallenden Fees (selbst die ausschließlich User-basierten) aufgrund von \textit{Mechanism-Design}-Überlegungen erst bei seiner Liquidierung.

\vspace{0.3cm}

\begin{Praemisse}[Abrechnung]

Sämtliche für einen Pool angefallenen Fees werden (ungeachtet ihres Fälligkeitszeit\-punkts) fließen erst bei seiner Liquidierung und werden zwischen Fälligkeit und Entrichtung in einem gesonderten Teil der \textit{Pool-Treasury} vorgehalten (ähnlich dessen, wo der gestakte Betrag des Pool-Creators verwahrt wird).

\vspace{0.2cm}

Wir werden diese finanziellen Mittel im weiteren Verlauf auch als \textbf{\textit{Pending-Fees}} bezeichnen.

\vspace{0.2cm}

In Analogie dazu werden wir an geeigneter Stelle folgend auch von \textbf{\textit{Staked-Fees}} sprechen - gleichwohl es sich dabei eher um eine Sicherheit als um tatsächliche Fees handelt.

\end{Praemisse}

\vspace{0.5cm}

\todo{Ende WIP}
\vspace{0.5cm}

%\subsection{Business-Plan}
%\label{sec:bp}
%\vspace{0.2cm}
%% !TEX root = paper.tex

\todo{WIP}

\begin{itemize} 
	\item Business-Case (aus Investoren-Sicht) vorrechnen
	\begin{itemize}
		\item Wirtschaftlichkeit und Preisentwicklung
		\item (praktische) Obergrenze des eingebrachten Gesamtkapitals annehmen, mit der eine plausible und attraktive Rendite argumentiert werden kann.
	\end{itemize}
	\item Excel verlinken
\end{itemize}

\vspace{0.6cm}

Der \nameref{sec:fees} folgen einige Annahmen hinsichtlich des \textbf{Business-Plans} für eine Größenordnung von zwölf Monaten.

\vspace{0.3cm}

\begin{Assumption}[Business-Plan]\label{bp}

\vspace{0.75cm}

\todo{TODO: Zahlen an Excel anpassen}

\vspace{0.75cm}

Zunächst schätzen wir einige KPI ab, die es natürlich zu validieren gilt:

\begin{itemize}
	\item Wir gehen im Mittel von ca. 4-5 Teilnehmern je Pool aus.
	\item Wir gehen von einem durchschnittlichen Deposit von 200\$ je Teilnehmer und Pool aus - also einem durchschnittlichen initialen Pool-Kapital von 800-1000\$.
	\item Wir gehen des Weiteren von einer durchschnittlichen \textit{Pool-Lifetime} von ca. 6 Monaten aus,
	\item schätzen die durchschnittliche Anzahl an Tradings während der Pool-Lifetime auf 10-15,
	\item deren Trading-Volumen auf etwa $\frac{1}{3}$ des initialen Pool-Kapitals und schließlich 
	\item und einen daraus resultierenden konservativen mittleren Profit von 2.5 \% (auf die Pool-Lifetime von 6 Monaten also 5-6 \% p.a.).
	\item Zuletzt schätzen wir, jeder User betreibe im Mittel 2-3 Pools gleichzeitig.
\end{itemize}

\vspace{0.5cm}

Diesen geschätzten KPI zugrundeliegend setzen wir uns folgende Ziele hinsichtlich initiierter (gebührenpflichtiger) Pools - ungeachtet dessen, ob diese zu dem gegebenen Zeitpunkt noch existieren oder bereits liquidiert wurden:

\begin{itemize}
	\item 50 initiierte Pools nach 3 Monaten
	\item 150 initiierte Pools nach 6 Monaten
	\item 500 initiierte Pools nach 12 Monaten
\end{itemize}

\end{Assumption}

\vspace{0.5cm}

Damit ergeben sich folgende Business-Key-KPI:

\vspace{0.3cm}

\begin{Fazit}[Umsätze \& Forecast]

\vspace{0.75cm}

\todo{TODO: Zahlen an Excel anpassen}

\vspace{0.75cm}

Für einen durchschnittlichen Pool $\mathcal{P}$ mit dem initialen Pool-Kapital

\begin{equation*}
  vol^{\mathcal{P}} = 4.5 \cdot 200\$ = 900\$ 
\end{equation*}

approximieren wir die anfallenden Fees als Summe der Fee-Bestandteile

\begin{itemize}
	\item Grundgebühren: $fees_{G}^{\mathcal{P}} = \rho(nft) \cdot 0.019 \cdot (4.5 - 1) \cdot 200\$ $
	\item Trading-Gebühren: $fees_{T}^{\mathcal{P}} = \phi(nft) \cdot 0.001 \cdot 12.5 \cdot \frac{1}{3} \cdot vol^{\mathcal{P}} $
	\item Profit-Beteiligung: $fees_{P}^{\mathcal{P}} = \phi(nft) \cdot 0.099 \cdot 0.025 \cdot vol^{\mathcal{P}} $
\end{itemize}

wobei $\rho(nft)$ und $\phi(nft)$ Normierungsfaktoren darstellen, die die in Annahme \ref{fees} beschriebenen \textit{Benefits für PassNFT-Besitzer} berücksichtigen sollen, und von uns als 

\begin{itemize}
	\item $\rho(nft) \approx \frac{9}{10}$ und 
	\item $\phi(nft) \approx \frac{7}{8}$
\end{itemize}	

geschätzt werden sollen \todo{(für die Anfangsphase sind diese eher zu klein, im einge\-schwungenen Zustand viel zu groß)}.

\vspace{0.2cm}

Damit belaufen sich die einzelnen Fees-Bestandteile auf 

\begin{itemize}
	\item Grundgebühren: $fees_{G}^{\mathcal{P}} \approx 11.97\$ $
	\item Trading-Gebühren: $fees_{T}^{\mathcal{P}} \approx 3.28 \$ $
	\item Profit-Beteiligung: $fees_{P}^{\mathcal{P}} \approx 1.95 \$ $
\end{itemize}

und damit die im Mittel erwarteten Gesamt-Fees pro Pool auf

\begin{equation*}
  fees^{\mathcal{P}} = fees_{G}^{\mathcal{P}} + fees_{T}^{\mathcal{P}} + fees_{P}^{\mathcal{P}} \approx 17.20 \$. 
\end{equation*}

\vspace{0.67cm}

Bei einer Staking-Anforderung von 200 \% der geschätzten Pool-Fees \todo{(auf Staking verlinken)} und den in \ref{bp} getroffenen Business-Plan-Annahmen ergeben sich folgende näherungsweisen Forecasts:

\begin{itemize}
	\item Nach 3 Monaten: ca. 40 noch aktive und bereits ca. 10 liquidierte Pools.
	\begin{itemize}
		\item bereits \textit{umgesetzte Fees}: 266 \$ 
		\item \textit{Pending-Fees}: 1.064 \$ 
		\item \textit{Staked-Fees}: 2.129 \$ 
	\end{itemize}
	\item Nach 6 Monaten: ca. 100 noch aktive und bereits ca. 50 liquidierte Pools.
	\begin{itemize}
		\item bereits \textit{umgesetzte Fees}: 1.330 \$ 
		\item \textit{Pending-Fees}: 2.661 \$ 
		\item \textit{Staked-Fees}: 5.322 \$ 
	\end{itemize}
	\item Nach 12 Monaten: ca. 300 noch aktive und bereits ca. 200 liquidierte Pools.
	\begin{itemize}
		\item bereits \textit{umgesetzte Fees}: 5.322 \$ 
		\item \textit{Pending-Fees}: 7.983 \$ 
		\item \textit{Staked-Fees}: 15.966 \$ 
	\end{itemize}	 
\end{itemize}

\vspace{0.5cm}

Zu guter Letzt noch eine sehr bullishe Prognose:

\begin{itemize}
	\item Nach 5 Jahren: 450.000 noch aktive und bereits 550.000 liquidierte Pools.
	\begin{itemize}
		\item bereits \textit{umgesetzte Fees}: $\approx$ 15 Mio. \$ 
		\item \textit{Pending-Fees}: $\approx$ 12 Mio. \$ 
		\item \textit{Staked-Fees}: $\approx$ 24 Mio. \$ 
	\end{itemize}	 
\end{itemize}

\end{Fazit}

\vspace{0.5cm}


\todo{Ende WIP}    % binde die Datei '[Pools][Economic][Business-Plan].tex' ein
%\vspace{0.5cm}

\subsection{WPT - ready to launch}
\label{sec:wpt}
\vspace{0.2cm}
% !TEX root = paper.tex

\todo{WIP}
\vspace{0.5cm}

\todo{Recap aus den vorigen Kapiteln und deren Input für das gegenständige Kapitel}
\vspace{0.5cm}

\paragraph{Token-Kurs-Kurven}
\textbf{ }
\vspace{0.3cm}

\todo{Hier müssen im Wesentlichen die Excel-Parameter erklärt werden}

\begin{itemize}
	\item die Annahme-Parameter in der Excel folgen im Wesentlichen aus den vorigen Kapiteln
	\item die einzelnen Kurven-Stufen erklären
	\item die tatsächlichen Bonding-Curves zu erklären wäre nicht ganz ohne. Kann man das einfach ausschweigen? Falls wir es genauer erklären wollen, wären die Ausführungen aus Algo \ref{approx} hilfreich.
\end{itemize}

\vspace{0.5cm}


\begin{Solution}[Token-Curves]

Sei $s \in \mathbb{N}$ der Token-Supply und 

\begin{equation*}
p \approx 1.06
\end{equation*}

der \textit{Profit-Koeffizient} \todo{(erklären weshalb, wofür, warum)}.
Dann leiten sich Verkaufs- und Kaufpreis-Kurve wie folgt ab:

\vspace{0.2cm}

Verkaufspreis-Kurve:

\begin{equation*}
V(s) = V_{0} \cdot p^{ln\left(2 \cdot \frac{s}{s_{o}}\right) \cdot ln\left(\frac{s}{s_{o}}\right)}
\end{equation*}

\vspace{0.2cm}

Kaufpreis-Kurve:

\begin{align*}
K(s) &= K_{0} \cdot p^{ln\left(2 \cdot \frac{s}{s_{o}}\right) \cdot ln\left(\frac{s}{s_{o}}\right)} \cdot \left( ln(p) \cdot \left( 2 \cdot ln\left( \frac{s}{s_{o}} \right) + ln(2) \right) + 1 \right) \\
 &= V(s) \cdot \left( ln(p) \cdot \left( 2 \cdot ln\left( \frac{s}{s_{o}} \right) + ln(2) \right) + 1 \right)
\end{align*}

\vspace{0.4cm}

wobei $s_{0}$ für einen sehr kleinen (initialen) Supply und 

\begin{equation*}
K_{0} = K(s_{0}) = V(s_{0}) = V_{0}
\end{equation*}

für seinen initialen Kauf- und Verkaufskurs stehen und dabei übrigens ganz nebenbei 

\begin{equation*}
K(s) = \left( s \cdot V(s) \right)^{\prime}
\end{equation*}

gilt.

\end{Solution}

\vspace{0.5cm}
\todo{Eine Abbildung der Preis-Kurve(n) wäre nichr verkehrt.}
\vspace{0.5cm}


\paragraph{Staking \& Gewinn-Split}
\textbf{ }
\vspace{0.3cm}

\todo{WIP}
\vspace{0.5cm}

Split der Gebühren auf Staker und Projekt-Treasury $(\sigma_{S}; \sigma_{T})$ mit $\sigma_{S} + \sigma_{T} = 1$ definieren. Dazu gibt es einige denkbare Varianten:

\begin{itemize}
	\item fester, statischer Split
	\item fester, statischer Split mit eingebauten Unter- und Obergrenzen für den Gesamtertrag des Stakers $\sigma_{S} \cdot fees^{\mathcal{P}}$
	\item $fees^{\mathcal{P}}$-abhängiger (progressiver) Split, bei dem der Anteil des Stakers $\sigma_{S}$ mit zunehmendem $fees^{\mathcal{P}}$ stets kleiner wird. Dies unter Umständen ebenfalls unter Berücksichtigung eingebauter Unter- und Obergrenzen für den Staker.
	\item Begünstigung des Stakers in Abhängigkeit seines NFT-Pass-Status.
\end{itemize}

\vspace{0.5cm}


\paragraph{Workflows}
\textbf{ }
\vspace{0.3cm}

\todo{WIP}
\vspace{0.5cm}

\begin{itemize}
	\item Wann bezahlen die User die Fees?
	\item In welcher Form/Währung dürfen die Fees von den Usern erbracht/verrechnet werden und wird dann alles im Hintergrund sofort in \textit{WPT} umgewandelt?
	\item Ist es denkbar die Fees aus dem Stake-Pool des Creators zu verwenden und diesem seinen Stake in einer anderen Währung zurückzuerstatten?
\end{itemize}

\vspace{0.5cm}

\paragraph{Problem}
\textbf{ }
\vspace{0.3cm}	

\begin{Problem}[\textit{USDT} vs. \textit{W-PLT} als Berechnungsgrundlage für Fees, Staking etc.]
\vspace{0.2cm}

\todo{Was nehmen wir hier?}

\vspace{0.5cm}

\todo{Folgend übernommene alte Test-Passagen zu dem Thema:}

\vspace{0.5cm}

Ein weiterer sehr essenzieller Faktor für die Größe des zu stakenden Betrags könnte der Kurs des IPTs sein. Denn laut der \textbf{Bonding-Curves}-Implementierung würde der \textit{W-PLT}-Preis mit steigender Zirkulation steigen, was mit der Zunahme von existierende Pools geschähe. Damit wäre die Erstellung neuer Pools mit ihrer zahlen\-mäßigen Zunahme stets kapital-intensiver (aber nicht gleichbedeutend teurer). \textbf{Die Frage hierbei ist also, ob der zu erbringende Stake des Pool-Creators auf den \textit{Total-Supply des W-PLT} normiert werden sollte oder nicht}, die gänzlich mit der obigen Fragestellung einhergeht, ob der Pool-Creator eigentlich staken möchte oder das nur tun muss.
	
\begin{itemize}
	\item Gegen eine Normierung spricht die Annahme/Hoffnung, ein Pool-Creator sei gleichzeitig auch ein großer Supporter des gesamten Projekt und glaube daran. Wenn der \textit{W-PLT}-Preis steigt, ist dies gleichbedeutend mit der Zunahme an genutzten Pools, an denen der Pool-Creator als Staker, Besitzer von \textit{W-PLT} und damit Projekt-Investor auch selbst (finanziell) profitiert.
	\item Für eine Normierung spricht dagegen die potenzielle Gefahr, neue oder bestehende User durch eine zu hohe finanzielle Sicherheitseinlage davon abzuschrecken neue Pools zu erstellen.
\end{itemize}

\vspace{0.2cm}
	
Die Antwort auf diese Fragestellung könnte auch darin liegen, ob wir uns besonders viele oder lieber weniger aber besonders Teilnehmer-starke Pools wünschen.

\vspace{0.5cm}
	
Die \textit{Pool-Teilnehmer} (außer des Creators) können bei dieser Logik aber nicht wie nicht wie die Staker zusätzlich als Projekt-Investoren angesehen werden, weil sie \textit{W-PLT} kaufen, da die gekauften \textit{W-PLT} direkt als Gebühr entrichtet werden. Für die Pool-Teilnehmer stellt der \textit{W-PLT} also eher einen Utility- bzw. Purpose-Token dar weshalb die Höhe der zu entrichtenden Gebühr zweifelsfrei auf Basis von \textit{Total-Supply des W-PLT} normiert werden muss (die Gebühr darf keinesfalls mit Zunahme von Pools steigen).

\end{Problem}

\vspace{0.5cm}


\paragraph{Fazit}
\textbf{ }
\vspace{0.3cm}

\todo{Gibt es noch Ungeklärtheiten, ohne die sich kein Token-Contract schreiben lässt?}

    % binde die Datei '[Pools][Economic][Bonding-Curves].tex' ein
\vspace{0.5cm}


%% !TEX root = paper.tex


\subsubsection{Beispielrechnung}
\vspace{0.2cm}

\todo{WIP}

\begin{itemize}
	\item Daten, Zahlen, Fakten
	\item Profiterwartung aus Sicht eines Token-Holders/-Investors
\end{itemize}

\vspace{0.5cm}

Wir rechnen ein bisschen rum, um ein Gefühl für den nötigen Token-Supply zu bekommen:

\vspace{0.3cm}

\begin{Example}[Rechnerei zum Token-Supply]

\vspace{0.75cm}

\todo{TODO: Zahlen an Excel anpassen}

\vspace{0.75cm}

Wir peilen den Token-Contract so zu stricken, dass wir im eingeschwungenen Zustand einen Tokenwert des \textit{W-PLT} von $\approx$ 1 Cent anpeilen, aber gleichzeitig auch die Grenzen $[$0.5 Cent; 2 Cent$]$ im Auge behalten.

\vspace{0.5cm}

Wir forcieren beim Projekt-Fortschritt über die Zeit hinsichtlich des Tokens

\begin{itemize}
	\item Einen günstigen \textit{W-PLT}-Preis für die Gründer/Company ($\approx$ 0.25 Cent pro Token)
	\item Einen guten \textit{W-PLT}-Preis für ganz frühe Investoren ($<<$ 0.1 Cent)
	\item Einen \textit{W-PLT}-Preis von $<$ 1 Cent für die Early-Pool-User bis zum eingeschwungenen Zustand.
	\item Einen kontrollierten \textit{W-PLT}-Preis $<$ 2 Cent für die Pool-User im eingeschwungenen Zustand.
	\item Ein zunehmendes Ziel-Projekt-Invest mit Fortschritt des Projekts.
	\item Einen zunehmenden (aber kontrollierten) Ziel-Supply von \textit{W-PLT} mit Fortschritt des Projekts.
	\item Einen zunehmenden (aber kontrollierten) \textit{W-PLT}-Kurs mit Fortschritt des Projekts.
	\item Einen zunehmenden \textit{Utility-Koeffizient} (als Verhältnis zwischen mindest und Ziel-Supply) mit Fortschritt des Projekts bis zu Zielzustand des Koeffizienten von 50 \%.
\end{itemize}

\vspace{1.0cm}

Im Folgenden wieder die obige Forecast-Aufstellung - nun aus Token-Sicht:


\begin{itemize}
	\item Nach 3 Monaten: ca. 40 noch aktive und bereits ca. 10 liquidierte Pools.
	\begin{itemize}
		\item bereits \textit{umgesetzte Fees}: 25.000 \textit{W-PLT} 
		\item mindestens bereits geburnte Tokens: 12.500 \textit{W-PLT} 
		\item \textit{Pending-Fees}: 100.000 \textit{W-PLT}  
		\item \textit{Staked-Fees}: 200.000 \textit{W-PLT}
		\item min Supply: 300.000 \textit{W-PLT}
		\item Ziel-Projekt-Invest: 60.000 \$ \todo{(davon 30-40k durch Gründer/Company)}
		\item Projekt-Treasury: 60.000 \$ + 1.000 \$ Fees-Cash $\approx$ 61.000 \$
		\item Ziel-Supply: 12.0 Mio. \textit{W-PLT}
		\item \textit{Utility-Koeffizient}: $\frac{300.000}{12.000.000} = 2.5 \%$
		\item $\varnothing$ Kaufpreis pro \textit{W-PLT}: 0.5 Cent
		\item Mindest-Value pro \textit{W-PLT}: $\approx$ 0.51 Cent
	\end{itemize}
	\item Nach 6 Monaten: ca. 100 noch aktive und bereits ca. 50 liquidierte Pools.
	\begin{itemize}
		\item bereits \textit{umgesetzte Fees}: 130.000 \textit{W-PLT}
		\item mindestens bereits geburnte Tokens: 65.000 \textit{W-PLT}  
		\item \textit{Pending-Fees}: 250.000 \textit{W-PLT}
		\item \textit{Staked-Fees}: 500.000 \textit{W-PLT} 
		\item min Supply: 750.000 \textit{W-PLT}
		\item Ziel-Projekt-Invest: 120.000 \$
		\item Projekt-Treasury: 120.000 \$ + 2.500 \$ Fees-Cash $\approx$ 122.500 \$
		\item Ziel-Supply: 16.0 Mio. \textit{W-PLT}
		\item \textit{Utility-Koeffizient}: $\frac{750.000}{16.000.000} = 4.6875 \%$
		\item $\varnothing$ Kaufpreis pro \textit{W-PLT}: 0.75 Cent
		\item Mindest-Value pro \textit{W-PLT}: $\approx$ 0.77 Cent
	\end{itemize}
	\item Nach 12 Monaten: ca. 300 noch aktive und bereits ca. 200 liquidierte Pools.
	\begin{itemize}
		\item bereits \textit{umgesetzte Fees}: 500.000 \textit{W-PLT}  
		\item mindestens bereits geburnte Tokens: 250.000 \textit{W-PLT}
		\item \textit{Pending-Fees}: 800.000 \textit{W-PLT}  
		\item \textit{Staked-Fees}: 1.6 Mio. \textit{W-PLT} 
		\item min Supply: 2.4 Mio \textit{W-PLT} 
		\item Ziel-Projekt-Invest: 200.000 \$
		\item Projekt-Treasury: 200.000 \$ + 10.000 \$ Fees-Cash $\approx$ 210.000 \$
		\item Ziel-Supply: 20.0 Mio. \textit{W-PLT}
		\item \textit{Utility-Koeffizient}: $\frac{2.400.000}{20.000.000} = 12.0 \%$
		\item $\varnothing$ Kaufpreis pro \textit{W-PLT}: 1 Cent
		\item Mindest-Value pro \textit{W-PLT}: $\approx$ 1.05 Cent
	\end{itemize}
	\item Nach 3 Jahren: ca. 20.000 noch aktive und bereits ca. 20.000 liquidierte Pools.
	\begin{itemize}
		\item bereits \textit{umgesetzte Fees}: 50 Mio. \textit{W-PLT}  
		\item mindestens bereits geburnte Tokens: 25 Mio. \textit{W-PLT}
		\item \textit{Pending-Fees}: 50 Mio. \textit{W-PLT}  
		\item \textit{Staked-Fees}: 100 Mio. \textit{W-PLT} 
		\item min Supply: 150 Mio \textit{W-PLT} 
		\item Ziel-Projekt-Invest: 20.0 Mio. \$
		\item Projekt-Treasury: 20.0 Mio. \$ + 0.5 Mio \$ Fees-Cash $\approx$ 20.5 Mio. \$
		\item Ziel-Supply: 1.4 Mrd. \textit{W-PLT}
		\item \textit{Utility-Koeffizient}: $\frac{150.000.000}{1.400.000.000} = 12.0 \%$
		\item $\varnothing$ Kaufpreis pro \textit{W-PLT}: $\approx$ 1.43 Cent
		\item Mindest-Value pro \textit{W-PLT}: $\approx$ xxx Cent
	\end{itemize}
	\item Nach 5 Jahren: 450.000 noch aktive und bereits 550.000 liquidierte Pools.
	\begin{itemize}
		\item bereits \textit{umgesetzte Fees}: 1.5 Mrd. \textit{W-PLT} 
		\item mindestens bereits geburnte Tokens: 750 Mio. \textit{W-PLT}
		\item \textit{Pending-Fees}: 1.2 Mrd. \textit{W-PLT}
		\item \textit{Staked-Fees}: 2.4 Mrd. \textit{W-PLT} 
		\item min Supply: 3.6 Mrd. \textit{W-PLT} 
		\item Ziel-Projekt-Invest: 144 Mio. \$
		\item Projekt-Treasury: 144 Mio. \$ + 16 Mio. \$ Fees-Cash $\approx$ 160 Mio. \$
		\item Ziel-Supply: 7.2 Mrd. \textit{W-PLT}
		\item \textit{Utility-Koeffizient}: $\frac{3.6}{7.2} = 50.0 \%$
		\item $\varnothing$ Kaufpreis pro \textit{W-PLT}: 2 Cent
		\item Mindest-Value pro \textit{W-PLT}: $\approx$ 2.22 Cent		
	\end{itemize}	 
\end{itemize}

\end{Example}

\vspace{0.5cm}

\todo{Ende WIP}

\vspace{0.5cm}



\subsubsection{Recap \& Ausblick}
\vspace{0.2cm}

\todo{Einbettung in das Wunder-Ökosystem und Link zwischen WPT zu WUNDER}

\vspace{0.5cm}

Die Einlage für den \textit{WPT} ist idealerweise in WUNDER zu erbringen \todo{(Es ist noch unklar, wie man an WUNDER kommt, wenn es vorher keinen Token-Sale gegeben hat. Ob der WUNDER ebenfalls mittels Bonding-Curves abzubilden wäre, sei hier erst einmal mehr als unklar.)}

\vspace{0.5cm}



\subsubsection{Unberücksichtigte Inhalte \& Ideen}

\vspace{0.3cm}
\todo{WIP}
\vspace{0.5cm}


\begin{Algo}[\textit{W-PLT}-Design]
\label{approx}

\begin{itemize}
	\item Wir haben Szenarien vorgerechnet und dabei den zugehörigen \textit{Ziel-Supply} angegeben. Daran orientieren wir uns.
	\item Abhängig des Szenarios (gemappt auf den \textit{Ziel-Supply}) können wir einen aus Fees-Einnahmen resultierenden erwarteten Profit ableiten - und zwar pro Zeiteinheit, die genau der durchschnittlichen Pool-Lifetime entspricht.
	\item Wir legen einen Zeitraum als Vielfaches der durchschnittlichen Pool-Lifetime fest, die wir einem Token-Investor zumuten, bis sein Invest profital wird. Z. B. 4 $\cdot$ Pool-Lifetime.
	\item Aus den obigen Daten können wir für diesen Zeitraum (ab Invest beim entsprechenden Supply) den erwarteten relativen Profit pro Token $x$ approximieren.
	\item Diesen erwarteten relativen Profit pro Token müssen wir bei den obigen Rechnungen noch als $x(supply) = profit(zeitraum; supply)$ ergänzen.
	\item $kaufpreis(supply) := x(supply) \cdot \frac{treasury}{supply}$
	\item Den Early-Investoren darf durchaus ein größerer Faktor zugemutet werden.
	\item Wir errechnen für einige supply-Milestones den Kaufpreis auf der Grundlage des letzten Bullets und approximieren dann dazwischen.
\end{itemize}

\vspace{0.3cm}
\todo{\noindent\hrulefill}

\begin{itemize}
	\item Problem: Wenn Token-Holder aussteigen, nehmen diese nicht nur ihren Anteil an den bisher erwirtschafteten Fees-Einnahmen mit, sondern drücken zudem auch noch den Token-Kurs nach unten. Das stellt einen sich selbst verstärkenden Effekt dar, der irgendwie in den Griff zu bekommen ist (evtl. Locking-Periode oder berücksichtigende Verkaufs-Kurve).
	\item Es macht eine Locking-Periode von der Dauer der durchschnittlichen Pool-Lifetime viel Sinn, da die beim Verkauf der Tokens mitgenommenen Gewinne aus Fees-Einnahmen durch neu generierte Fees-Einnahmen egalisiert werden und der Token-Value somit stabil bleibt.	
\end{itemize}

\end{Algo}



    % binde die Datei '[Pools][Economic][Rest].tex' ein




\end{document}

