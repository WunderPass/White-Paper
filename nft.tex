\documentclass[11pt]{scrartcl}


%---------Konfiguration------


%-----Sprache und Zeichen----
\usepackage[utf8]{inputenc}
% \usepackage{ucs}
% \usepackage[T1]{fontenc}

% Zeilenumbrüche der deutschen Sprache
% \usepackage[ngerman]{babel}


%---------Farben-------------
\usepackage[RGB]{xcolor}
\definecolor{dunkelgruen}{RGB}{0 136 0}
\newcommand\todo[1]{\textcolor{red}{#1}}


%---------Links/Refs---------
\usepackage[colorlinks=true, urlcolor=blue]{hyperref}


%---------Grafiken-----------
\usepackage{graphicx}
\usepackage{subfig}



%---------Sonstiges----------
\usepackage{parcolumns}
\usepackage{enumitem}



%---------Mathe--------------
\usepackage{amsthm, amssymb, mathtools}
% \usepackage{amsmath, amssymb, amstext ,lipsum}



%---------Umgebungen---------
\usepackage[framemethod=tikz]{mdframed}

%---------Business---------
\mdtheorem[
  linecolor=dunkelgruen,
  frametitlefont=\sffamily\bfseries\color{white},
  frametitlebackgroundcolor=dunkelgruen,
]{Business-Def}{Definition}

\mdtheorem[
  linecolor=dunkelgruen,
  frametitlefont=\sffamily\bfseries\color{white},
  frametitlebackgroundcolor=dunkelgruen,
]{Hypothese}{Hypothese}

\mdtheorem[
  linecolor=gray,
  frametitlefont=\sffamily\bfseries\color{white},
  frametitlebackgroundcolor=gray,
]{Praemisse}{Prämisse}

\mdtheorem[
  linecolor=violet,
  frametitlefont=\sffamily\bfseries\color{white},
  frametitlebackgroundcolor=violet,
]{Quelle}{Quellen}

\mdtheorem[
  linecolor=violet,
  frametitlefont=\sffamily\bfseries\color{white},
  frametitlebackgroundcolor=violet,
]{Zitat}{Zitat}

\mdtheorem[
  linecolor=blue,
  frametitlefont=\sffamily\bfseries\color{white},
  frametitlebackgroundcolor=blue,
]{Fazit}{Conclusion}

\mdtheorem[
  linecolor=red,
  frametitlefont=\sffamily\bfseries\color{white},
  frametitlebackgroundcolor=red,
]{Problem}{Problem}

\mdtheorem[
  linecolor=cyan,
  frametitlefont=\sffamily\bfseries\color{black},
  frametitlebackgroundcolor=cyan,
]{Solution}{Lösung}


\mdtheorem[
  linecolor=dunkelgruen,
  frametitlefont=\sffamily\bfseries\color{white},
  frametitlebackgroundcolor=dunkelgruen,
]{NFT-Prop}{NFT-Property}



%---------Mathe------------
\mdtheorem[
  linecolor=gray,
  frametitlefont=\sffamily\bfseries\color{white},
  frametitlebackgroundcolor=gray,
]{Def}{Definition}

\mdtheorem[
  linecolor=dunkelgruen,
  frametitlefont=\sffamily\bfseries\color{white},
  frametitlebackgroundcolor=dunkelgruen,
]{Theorem}{Theorem}

\mdtheorem[
  linecolor=blue,
  frametitlefont=\sffamily\bfseries\color{white},
  frametitlebackgroundcolor=blue,
]{Lemma}{Lemma}

\mdtheorem[
  linecolor=red,
  frametitlefont=\sffamily\bfseries\color{white},
  frametitlebackgroundcolor=red,
]{Assumption}{Annahme}

\mdtheorem[
  linecolor=violet,
  frametitlefont=\sffamily\bfseries\color{white},
  frametitlebackgroundcolor=violet,
]{Example}{Beispiel}

\mdtheorem[
  linecolor=cyan,
  frametitlefont=\sffamily\bfseries\color{black},
  frametitlebackgroundcolor=cyan,
]{Algo}{Algorithmus}






 
%---------Dokument-----------

%---------Titel--------------
\title{WunderPass-NFT}
\author{WunderTeam}
\date{\today{}, Berlin}
 
%---------Inhalt-------------
\begin{document}


\maketitle
\tableofcontents{}



%------------------------------------------------Einleitende Worte---------------------------------

\newpage

Ein exzellentes Mittel, um \textit{WunderPass} als Geschäftsmodell, Unternehmung und Unternehmen ein symbolisches - gewissermaßen plastisches - Sinnbild einzuverleiben, ist die Repräsentation von \textit{WunderPass} als Service/Protokoll mittels eines - eigens dafür kreierten - NFTs: \textbf{"Des WunderPass"} (im Folgenden auch \textit{NFT-Pass} genannt)

\vspace{0.3cm}

\begin{Fazit}[\textit{WunderPass} deabstrahiert durch \textbf{"den WunderPass"} als NFT]

"Ich nutze \textit{WunderPass}" wird symbolisiert durch "Ich besitze \textbf{meinen WunderPass}"!

\end{Fazit}

\vspace{0.3cm}

%------------------------------------------------Konzept-Kapitel-----------------------------------

% -----------------Einleitende Worte-------------
% !TEX root = paper.tex

Unser Anspruch an den zu modellierenden \textit{NFT-Pass} ist grob der folgende:

\vspace{0.2cm}

\begin{itemize}
  \item Der \textit{NFT-Pass} muss sich ganz klar von dem Großteil der heutigen - in größter Regel als Sammlerstück verstandenen - den Markt überflutenden NFTs abgrenzen. Er braucht einen klar ersichtlichen \textbf{intrinsischen Wert}. Man muss also "etwas mit dem \textit{NFT-Pass} anfangen/machen können" und diesen nicht "lediglich besitzen", um ihn ausschließlich mit einer gewissen Wahrscheinlichkeit gewinnbringend weiterverkaufen zu können ("Hot Potato"). Der Token bedarf also gewisser Eigenschaften eines \textit{Governance-Tokens} (DAO) oder Ähnlichem.
  \item 
  \begin{sloppypar}  
  Der \textit{NFT-Pass} braucht ungeachtet des vorigen Bullet-Points jedoch trotzdem zusätzlich ebenso eine ähnliche Beschaffenheit - wie solche der aktuell üblichen marktbeherrschenden NFTs - als Sammlerstück - gleichwohl nicht erstrangig. 
  \end{sloppypar}
  \item Anders als die aktuell gängigen NFTs soll unser \textit{NFT-Pass} \textbf{nicht begrenzt} in der Anzahl seiner Stücke sein. Stattdessen sollen theoretisch beliebig viele \textit{NFT-Pässe} existieren können. Nichtsdestotrotz soll unser \textit{NFT-Pass} ebenso die Eigenschaft der "nicht inflationären Begehrtheit" einverleibt bekommen. Dies möchten wir mittels einer ausgeklügelten Minting-Logik abbilden, die ein \textbf{endliches Sub-Set} an raren und begehrten \textit{NFT-Pässen} innerhalb des \textbf{unendlichen Gesamt-Sets} der \textit{NFT-Pässe} sicherstellt. Soll heißen: Es werden einerseits \textit{NFT-Pässe} exis\-tieren, die den heutigen NFTs - im Sinne ihres Sammlerwertes - gleichkommen, während die restlichen andererseits mit ihrer steigenden Gesamtanzahl zunehmend entwerten, bis sie irgendwann (als NFT betrachtet) nahezu wertlos und lediglich "funktional" werden.
  \item Die Rarität und Begehrtheit unseres \textit{NFT-Pass} soll Gamification-Mechanismen folgen:
  \begin{itemize}
    \item Wir brauchen an etwaigen Stellen ein (wertbestimmendes) \textit{first-come-first-serve-Prinzip}.
    \item Wir brauchen an anderen Stellen ein (ebenso wertbestimmendes) Zufallsprinzip.
    \item Wir brauchen irgendwo ebenso ein (geringes) Maß an persönlicher Individua\-lisierung des \textit{NFT-Pass} - ausschließlich durch den User gesteuert.
    \item Abrundend könnte ein \textbf{gemeinnützig wertbestimmendes} (randomisiertes) Merkmal wirken. (Beispiel: Wenn die \textit{NFT-Pässe} irgendwann inflationär geworden sind, könnte der zehn-millionste plötzlich wieder richtig krass sein.)
  \end{itemize}
  \item Der \textit{NFT-Pass} muss gänzlich transparent und vor allem verständlich für den interessierten - gleichwohl vielleicht technisch nicht bewandertsten - User sein.
\end{itemize}

\vspace{0.3cm}

In den kommenden Abschnitten folgt ein initialer Abriss unserer Vorstellung des \textit{NFT-Pass}:

\vspace{0.3cm}    % binde die Datei '[NFT-Pass][Konzept][Einleitung].tex' ein


% -----------------NFT-Status--------------------
\section{Status-Property}
\vspace{0.2cm}
% !TEX root = paper.tex

Diese NFT-Property - die wir gleichzeitig als die Main-Property unseres \textit{NFT-Pass} ansehen - soll der oben formulierten Anforderung nach einem first-come-first-serve-Prinzip Rechnung tragen. Zeitlich früher ausgestellte NFT-Pässe sollen einen rareren und begehrteren \textit{Pass-Status} inne haben als die späteren. Und vor allem sollen die \textit{NFT-Pässe} eines bestimmten ausgestellten Status in ihrer Anzahl begrenzt sein und nach Erreichen einer zu definierenden Höchstgrenze fortan nie wieder ausgestellt (ge\-mintet) werden können.

\vspace{0.3cm}

\begin{NFT-Prop}[Pass-Status]

Wir definieren folgende \textit{NFT-Pass-Status} mit den dazugehörenden Eigenschaften:

\begin{itemize}
    \item Status A (\textbf{Diamond})
    \begin{itemize}
    	\item Anzahl Pässe: 200
    	\item Gemintet an Nummer: 1 bis 200
    \end{itemize}
    \item Status B (\textbf{Black})
    \begin{itemize}
    	\item Anzahl Pässe: 1.600
    	\item Gemintet an Nummer: 201 bis 1800
    \end{itemize}
    \item Status C (\textbf{Pearl})
    \begin{itemize}
    	\item Anzahl Pässe: 12.800
    	\item Gemintet an Nummer: 1801 bis 14.600
    \end{itemize}
    \item Status D (\textbf{Platin})
    \begin{itemize}
    	\item Anzahl Pässe: 102.400
    	\item Gemintet an Nummer: 14.601 bis 117.000
    \end{itemize}
    \item Status E (\textbf{Ruby})
    \begin{itemize}
    	\item Anzahl Pässe: 819.200
    	\item Gemintet an Nummer: 117.001 bis 936.200
    \end{itemize}
    \item Status F (\textbf{Gold})
    \begin{itemize}
    	\item Anzahl Pässe: 6.553.600
    	\item Gemintet an Nummer: 936.201 bis 7.489.800
    \end{itemize}
    \item Status G (\textbf{Silver})
    \begin{itemize}
    	\item Anzahl Pässe: 52.428.800
    	\item Gemintet an Nummer: 7.489.801 bis 59.918.600
    \end{itemize}
    \item Status H (\textbf{Bronze})
    \begin{itemize}
    	\item Anzahl Pässe: 419.430.400
    	\item Gemintet an Nummer: 59.918.601 bis 479.349.000
    \end{itemize}
    \item Status I (\textbf{White})
    \begin{itemize}
    	\item Anzahl Pässe: $\infty$
    	\item Gemintet an Nummer: 479.349.001 bis $\infty$
    \end{itemize}
\end{itemize}

\end{NFT-Prop}

\vspace{0.3cm}

Diese NFT-Property ist per Definition trivialerweise \textbf{deterministisch}: Es ist stets zweifellos klar, welchen Status ein an x-ter Stelle geminteter \textit{NFT-Pass} haben wird. Die hinzugezogene "Reverse-Halving-Logik" \textbf{belohnt die Early-Adopter} mit einem begehrten NFT, dessen Rarität per Protokoll mit der Zeit stets abnimmt.

\vspace{0.1cm}


    % binde die Datei '[NFT-Pass][Konzept][Status].tex' ein


% -----------------Überleitende Worte------------
Die Beschaffenheit dieser \textit{first-come-first-serve-Property} soll jedoch einzigartig bleiben. Die folgenden Properties werden nicht mehr deterministisch sein, um unserem \textit{NFT-Pass} ein \textbf{unvorherbestimmbares "Eigenleben"} einzuverleiben. 


% -----------------NFT-Hologramm-----------------
\section{Hologramm}
\label{sec:hologramm}
\vspace{0.3cm}
% !TEX root = paper.tex

Diese NFT-Property soll zwar einem ähnlichen abstufenden Raritätsprinzip zu Grunde liegen wie die Main-Property, dies jedoch nicht mehr einem first-come-first-serve- sondern stattdessen einem Zufallsprinzip folgend.

Ebenfalls abweichend von der Beschaffenheit der Main-Property soll bei dieser Pro\-perty die Rarität nicht mittels einer absoluten Obergrenze abgebildet werden, sondern mittels einer relativen. (Dies zahlt auf die oben formulierte Anforderung nach einem \textbf{gemeinnützig gewinnbringendem Value} unseres \textit{NFT-Pass} ein.

\vspace{0.3cm}

\begin{NFT-Prop}[Hologramm (Welt-Wunder)]

Wir definieren folgende \textit{NFT-Pass-Hologramme} mit den dazugehörenden Eigenschaften:

\begin{itemize}
    \item WW1
    \begin{itemize}
    	\item Mögliche Ausprägung: \textbf{Pyramids of Giza}
    	\item Anteil Pässe: 0,390625\% $\left( \frac{1}{256} \right)$
    \end{itemize}
    \item WW2
    \begin{itemize}
    	\item Mögliche Ausprägung: \textbf{Great Wall of China}
    	\item Anteil Pässe: 0,78125\% $\left( \frac{1}{128} \right)$
    \end{itemize}
    \item WW3
    \begin{itemize}
    	\item Mögliche Ausprägung: \textbf{Petra} 
    	\item Anteil Pässe: 1,5625\% $\left( \frac{1}{64} \right)$
    \end{itemize}
    \item WW4
    \begin{itemize}
    	\item Mögliche Ausprägung: \textbf{Colosseum} 
    	\item Anteil Pässe: 3,125\% $\left( \frac{1}{32} \right)$
    \end{itemize}
    \item WW5
    \begin{itemize}
    	\item Mögliche Ausprägung: \textbf{Chichén Itzá} 
    	\item Anteil Pässe: 6,25\% $\left( \frac{1}{16} \right)$
    \end{itemize}
    \item WW6
    \begin{itemize}
    	\item Mögliche Ausprägung: \textbf{Machu Picchu} 
    	\item Anteil Pässe: 12,5\% $\left( \frac{1}{8} \right)$
    \end{itemize}
    \item WW7
    \begin{itemize}
    	\item Mögliche Ausprägung: \textbf{Taj Mahal} 
    	\item Anteil Pässe: 25\% $\left( \frac{1}{4} \right)$
    \end{itemize}
    \item WW8
    \begin{itemize}
    	\item Mögliche Ausprägung: \textbf{Christ the Redeemer} 
    	\item Anteil Pässe: 50\% + x $\left( \frac{1}{2} + \frac{1}{256} \right)$
    \end{itemize}
\end{itemize}

\end{NFT-Prop}

\vspace{0.3cm}

Das Besondere an dieser Property spiegelt sich in der Tatsache wider, gewisse rar beschaffene Ausprägungen seien nur "zeitweise" ausgeschöpft, da sich ihre (rare) Anzahl lediglich \textbf{relativ} an der Gesamtzahl der aktuell \textit{ausgestellten NFT-Pässe} bemisst und nicht wie die Main-Property einer absoluten Obergrenze obliegt, deren Erreichung unumkehrbar ist. Soll heißen: Ist die prozentuale Obergrenze an Pässen mit einer bestimmten Ausprägung der gegenwärtigen Property zu einem be\-stimmten Zeitpunkt erreicht, kann zwar für einen gewissen Zeitraum kein Pass mit dieser Ausprägung mehr ausgestellt werden. Sobald jedoch die Gesamtanzahl der \textit{ausgestellten NFT-Pässe} wieder groß genug ist - sodass die Anzahl der vorhandenen \textit{NFT-Pässe} mit der betroffenen Ausprägung wieder die prozentuale Obergrenze unterschreitet - werden Pässe der besagten Ausprägung "wieder verfügbar".

\vspace{0.3cm}

\begin{Algo}[Verlosungs-Mechanismus für Hologramm-Property]

\begin{itemize}
    \item Zunächst bestimme man die Gesamtanzahl aller bisher geminteter Pässe $n$.
    \item Gleiches tue man nun für die Counts der geminteten Pässe pro Ausprägung der Hologramm-Property WW1 bis WW8 als entsprechende Größen $n_1, n_2,...,n_8$.
    \item Und damit anschließend die aktuelle prozentuale Verteilung der Ausprägung auf die aktuell geminteten Pässe als $\sigma_i:= \frac{n_i}{n}$ für $i \in \lbrace 1,...,8 \rbrace$ berechnen.
    \item Seien $\Theta_i$ für $i \in \lbrace 1,...,8 \rbrace$ die oben definierten \textbf{relativen} Obergrenzen der \newline Ausprägungen der Hologramm-Property WW1 bis WW8.
    \item Alle Ausprägungen mit $\sigma_i \geq \Theta_i$ können zum aktuellen Zeitpunkt nicht vergeben werden und damit auch nicht beim Minting eines neuen Pass berücksichtigt werden.
    \item Für die Ausprägungen mit $\sigma_i < \Theta_i$ berechnen wir den Normierungsfaktor
\end{itemize} 

\begin{equation*}
\omega := \sum_{\sigma_i < \Theta_i} \Theta_i \textrm{ } \leq 1
\end{equation*} 

\begin{itemize}
    \item Damit errechnen wir die aktuell vorliegenden Wahrscheinlichkeiten $\rho_i$ für unsere Hologramm-Ausprägungen als
\end{itemize} 

\[
\rho_i:=\left\{%
\begin{array}{ll}
    0, & \hbox{falls $\sigma_i \geq \Theta_i$} \\[0,3cm]
    \hbox{\LARGE $\frac{\Theta_i}{\omega}$,} & \hbox{falls $\sigma_i < \Theta_i$}. \\
\end{array}%
\right.
\] 

Man vergewissere sich an dieser Stelle gedanklich, auch für die neuen \newline Wahrscheinlichkeiten gelte \[\sum_{i = 1}^7 \rho_i \textrm{ } = 1.\]

\begin{itemize}
    \item Am Ende bestimme man mittels Randomisierung anhand der Wahrscheinlichkeiten $\rho_i$ für $i \in \lbrace 1,...7 \rbrace$ die zu vergebende Hologramm-Ausprägung. 
\end{itemize}

\end{Algo}

\vspace{0.3cm}

Was hier so kompliziert klingt, lässt sich aber super simpel veranschaulichen:

Die \textit{Verlosung} der Wunder erfolgt in einem periodischen 256er-Turnus ($256 = 2^{n}$ mit $n=8$ für die acht bereitgestellten Hologramme). Nach jedem 256. geminteten Pass schmeißt man 256 Lose in eine Lostrommel: Ein Los für die \textit{Pyramiden}, zwei für die \textit{Chinesische Mauer}, vier für \textit{Petra} etc. Die \textit{Jesus-Statue} kommt letztendlich mit 129 Losen in die Trommel.

Nun ziehen wir blind ein Los und vergeben das gezogenen Hologramm an den nächsten zu mintenden NFT-Pass. Wir tun dies solange, bis die Trommel leer ist. Anschließend fangen wir wieder von Vorne an und befüllen die Trommel erneut mit denselben 256 Losen.

\textbf{Achtung:} Wir befüllen die Trommel ausschließlich nachdem sie komplett leer geworden ist und nicht etwa zwischendurch mal.

\vspace{0.3cm}

    % binde die Datei '[NFT-Pass][Konzept][Wunder].tex' ein
%% !TEX root = paper.tex

Diese NFT-Property soll zwar einem ähnlichen abstufenden Raritätsprinzip zu Grunde liegen wie die Main-Property, dies jedoch nicht mehr einem first-come-first-serve- sondern stattdessen einem Zufallsprinzip folgend.

Ebenfalls abweichend von der Beschaffenheit der Main-Property soll bei dieser Pro\-perty die Rarität nicht mittels einer absoluten Obergrenze abgebildet werden, sondern mittels einer relativen. (Dies zahlt auf die oben formulierte Anforderung nach einem \textbf{gemeinnützig gewinnbringendem Value} unseres \textit{NFT-Pass} ein.

\vspace{0.3cm}

\begin{NFT-Prop}[Hologramm (Welt-Wunder)]

Wir definieren folgende \textit{NFT-Pass-Hologramme} mit den dazugehörenden Eigenschaften:

\begin{itemize}
    \item WW1
    \begin{itemize}
    	\item Mögliche Ausprägung: \textbf{Pyramids of Giza}
    	\item Anteil Pässe: 0,390625\% $\left( \frac{1}{256} \right)$
    \end{itemize}
    \item WW2
    \begin{itemize}
    	\item Mögliche Ausprägung: \textbf{Great Wall of China}
    	\item Anteil Pässe: 0,78125\% $\left( \frac{1}{128} \right)$
    \end{itemize}
    \item WW3
    \begin{itemize}
    	\item Mögliche Ausprägung: \textbf{Petra} 
    	\item Anteil Pässe: 1,5625\% $\left( \frac{1}{64} \right)$
    \end{itemize}
    \item WW4
    \begin{itemize}
    	\item Mögliche Ausprägung: \textbf{Colosseum} 
    	\item Anteil Pässe: 3,125\% $\left( \frac{1}{32} \right)$
    \end{itemize}
    \item WW5
    \begin{itemize}
    	\item Mögliche Ausprägung: \textbf{Chichén Itzá} 
    	\item Anteil Pässe: 6,25\% $\left( \frac{1}{16} \right)$
    \end{itemize}
    \item WW6
    \begin{itemize}
    	\item Mögliche Ausprägung: \textbf{Machu Picchu} 
    	\item Anteil Pässe: 12,5\% $\left( \frac{1}{8} \right)$
    \end{itemize}
    \item WW7
    \begin{itemize}
    	\item Mögliche Ausprägung: \textbf{Taj Mahal} 
    	\item Anteil Pässe: 25\% $\left( \frac{1}{4} \right)$
    \end{itemize}
    \item WW8
    \begin{itemize}
    	\item Mögliche Ausprägung: \textbf{Christ the Redeemer} 
    	\item Anteil Pässe: 50\% + x $\left( \frac{1}{2} + \frac{1}{256} \right)$
    \end{itemize}
\end{itemize}

\end{NFT-Prop}

\vspace{0.3cm}

Das Besondere an dieser Property spiegelt sich in der Tatsache wider, gewisse rar beschaffene Ausprägungen seien nur "zeitweise" ausgeschöpft, da sich ihre (rare) Anzahl lediglich \textbf{relativ} an der Gesamtzahl der aktuell \textit{ausgestellten NFT-Pässe} bemisst und nicht wie die Main-Property einer absoluten Obergrenze obliegt, deren Erreichung unumkehrbar ist. Soll heißen: Ist die prozentuale Obergrenze an Pässen mit einer bestimmten Ausprägung der gegenwärtigen Property zu einem be\-stimmten Zeitpunkt erreicht, kann zwar für einen gewissen Zeitraum kein Pass mit dieser Ausprägung mehr ausgestellt werden. Sobald jedoch die Gesamtanzahl der \textit{ausgestellten NFT-Pässe} wieder groß genug ist - sodass die Anzahl der vorhandenen \textit{NFT-Pässe} mit der betroffenen Ausprägung wieder die prozentuale Obergrenze unterschreitet - werden Pässe der besagten Ausprägung "wieder verfügbar".

\vspace{0.3cm}

\begin{Algo}[Verlosungs-Mechanismus für Hologramm-Property]

\begin{itemize}
    \item Zunächst bestimme man die Gesamtanzahl aller bisher geminteter Pässe $n$.
    \item Gleiches tue man nun für die Counts der geminteten Pässe pro Ausprägung der Hologramm-Property WW1 bis WW8 als entsprechende Größen $n_1, n_2,...,n_8$.
    \item Und damit anschließend die aktuelle prozentuale Verteilung der Ausprägung auf die aktuell geminteten Pässe als $\sigma_i:= \frac{n_i}{n}$ für $i \in \lbrace 1,...,8 \rbrace$ berechnen.
    \item Seien $\Theta_i$ für $i \in \lbrace 1,...,8 \rbrace$ die oben definierten \textbf{relativen} Obergrenzen der \newline Ausprägungen der Hologramm-Property WW1 bis WW8.
    \item Alle Ausprägungen mit $\sigma_i \geq \Theta_i$ können zum aktuellen Zeitpunkt nicht vergeben werden und damit auch nicht beim Minting eines neuen Pass berücksichtigt werden.
    \item Für die Ausprägungen mit $\sigma_i < \Theta_i$ berechnen wir den Normierungsfaktor
\end{itemize} 

\begin{equation*}
\omega := \sum_{\sigma_i < \Theta_i} \Theta_i \textrm{ } \leq 1
\end{equation*} 

\begin{itemize}
    \item Damit errechnen wir die aktuell vorliegenden Wahrscheinlichkeiten $\rho_i$ für unsere Hologramm-Ausprägungen als
\end{itemize} 

\[
\rho_i:=\left\{%
\begin{array}{ll}
    0, & \hbox{falls $\sigma_i \geq \Theta_i$} \\[0,3cm]
    \hbox{\LARGE $\frac{\Theta_i}{\omega}$,} & \hbox{falls $\sigma_i < \Theta_i$}. \\
\end{array}%
\right.
\] 

Man vergewissere sich an dieser Stelle gedanklich, auch für die neuen \newline Wahrscheinlichkeiten gelte \[\sum_{i = 1}^7 \rho_i \textrm{ } = 1.\]

\begin{itemize}
    \item Am Ende bestimme man mittels Randomisierung anhand der Wahrscheinlichkeiten $\rho_i$ für $i \in \lbrace 1,...7 \rbrace$ die zu vergebende Hologramm-Ausprägung. 
\end{itemize}

\end{Algo}

\vspace{0.3cm}

Was hier so kompliziert klingt, lässt sich aber super simpel veranschaulichen:

Die \textit{Verlosung} der Wunder erfolgt in einem periodischen 256er-Turnus ($256 = 2^{n}$ mit $n=8$ für die acht bereitgestellten Hologramme). Nach jedem 256. geminteten Pass schmeißt man 256 Lose in eine Lostrommel: Ein Los für die \textit{Pyramiden}, zwei für die \textit{Chinesische Mauer}, vier für \textit{Petra} etc. Die \textit{Jesus-Statue} kommt letztendlich mit 129 Losen in die Trommel.

Nun ziehen wir blind ein Los und vergeben das gezogenen Hologramm an den nächsten zu mintenden NFT-Pass. Wir tun dies solange, bis die Trommel leer ist. Anschließend fangen wir wieder von Vorne an und befüllen die Trommel erneut mit denselben 256 Losen.

\textbf{Achtung:} Wir befüllen die Trommel ausschließlich nachdem sie komplett leer geworden ist und nicht etwa zwischendurch mal.

\vspace{0.3cm}




% -----------------NFT-Pattern-------------------
\section{Pattern-Property}
\vspace{0.3cm}
% !TEX root = C:/Users/Slava/White-Paper/[06][NFT-Pass]/[NFT-Pass][Konzept].tex

\subsubsection{Pattern-Property}

\vspace{0.3cm}

\begin{sloppypar}
Diese NFT-Property soll ebenso wie die beiden vorigen einem abstufenden Raritätsprinzip zu Grunde liegen - und zwar ausschließlich dem Zufall folgend.
\end{sloppypar}

Im Gegensatz zu den beiden vorigen Properties obliegt die \textit{Pattern-Property} keiner absoluten Obergrenze - insbesondere auch dann nicht, falls einige Pattern zu einem Zeitpunkt verhältnismäßig unter- oder überrepräsentiert sind.

\vspace{0.3cm}

\begin{NFT-Prop}[Background (Pattern)]

Wir definieren folgende \textit{NFT-Pass-Background-Muster} mit den dazugehörenden Eigenschaften:

\begin{itemize}
    \item P1
    \begin{itemize}
    	\item Mögliche Ausprägung: \textbf{Safari Fun} 
    	\item Wahrscheinlichkeit: 0,1953125\% $\left( \frac{1}{512} \right)$
    \end{itemize}
    \item P2
    \begin{itemize}
    	\item Mögliche Ausprägung: \textbf{Triangular Bars} 
    	\item Wahrscheinlichkeit: 0,390625\% $\left( \frac{1}{256} \right)$
    \end{itemize}
    \item P3
    \begin{itemize}
    	\item Mögliche Ausprägung: \textbf{Pointillism} 
    	\item Wahrscheinlichkeit: 0,78125\% $\left( \frac{1}{128} \right)$
    \end{itemize}
    \item P4
    \begin{itemize}
    	\item Mögliche Ausprägung: \textbf{Wavy waves} 
    	\item Wahrscheinlichkeit: 1,5625\% $\left( \frac{1}{64} \right)$
    \end{itemize}
    \item P5
    \begin{itemize}
    	\item Mögliche Ausprägung: \textbf{Stony desert} 
    	\item Wahrscheinlichkeit: 3,125\% $\left( \frac{1}{32} \right)$
    \end{itemize}
    \item P6
    \begin{itemize}
    	\item Mögliche Ausprägung: \textbf{WunderPass} 
    	\item Wahrscheinlichkeit: 6,25\% $\left( \frac{1}{16} \right)$
    \end{itemize}
    \item P7
    \begin{itemize}
    	\item Mögliche Ausprägung: \textbf{Zigzag} 
    		\item Wahrscheinlichkeit: 12,5\% $\left( \frac{1}{8} \right)$
    \end{itemize}
    \item P8
    \begin{itemize}
    	\item Mögliche Ausprägung: \textbf{Linear}  
    	\item Wahrscheinlichkeit: 25\% $\left( \frac{1}{4} \right)$
    \end{itemize}
    \item P9
    \begin{itemize}
    	\item Mögliche Ausprägung: \textbf{Curves}
    	\item Wahrscheinlichkeit: 50,1953125\% $\left( \frac{257}{512} \right)$
    \end{itemize}
\end{itemize}

\end{NFT-Prop}

\vspace{0.3cm}

    % binde die Datei '[NFT-Pass][Konzept][Pattern].tex' ein
%% !TEX root = C:/Users/Slava/White-Paper/[06][NFT-Pass]/[NFT-Pass][Konzept].tex

\subsubsection{Pattern-Property}

\vspace{0.3cm}

\begin{sloppypar}
Diese NFT-Property soll ebenso wie die beiden vorigen einem abstufenden Raritätsprinzip zu Grunde liegen - und zwar ausschließlich dem Zufall folgend.
\end{sloppypar}

Im Gegensatz zu den beiden vorigen Properties obliegt die \textit{Pattern-Property} keiner absoluten Obergrenze - insbesondere auch dann nicht, falls einige Pattern zu einem Zeitpunkt verhältnismäßig unter- oder überrepräsentiert sind.

\vspace{0.3cm}

\begin{NFT-Prop}[Background (Pattern)]

Wir definieren folgende \textit{NFT-Pass-Background-Muster} mit den dazugehörenden Eigenschaften:

\begin{itemize}
    \item M1
    \begin{itemize}
    	\item Mögliche Ausprägung: \textbf{Safari} 
    	\item maximale Anzahl Pässe: 256
    	\item Wahrscheinlichkeit falls noch nicht aufgebraucht: 1,5625\% $\left( \frac{1}{64} \right)$
    \end{itemize}
    \item M2
    \begin{itemize}
    	\item Mögliche Ausprägung: \textbf{Bars} 
    	\item maximale Anzahl Pässe: 4.096
    	\item Wahrscheinlichkeit falls noch nicht aufgebraucht: 3,125\% $\left( \frac{1}{32} \right)$
    \end{itemize}
    \item M3
    \begin{itemize}
    	\item Mögliche Ausprägung: \textbf{Dots} 
    	\item maximale Anzahl Pässe: 65.536
    	\item Wahrscheinlichkeit falls noch nicht aufgebraucht: 6,25\% $\left( \frac{1}{16} \right)$
    \end{itemize}
    \item M4
    \begin{itemize}
    	\item Mögliche Ausprägung: \textbf{Muster festlegen} 
    	\item maximale Anzahl Pässe: 1.048.576
    	\item Wahrscheinlichkeit falls noch nicht aufgebraucht: 12,5\% $\left( \frac{1}{8} \right)$
    \end{itemize}
    \item M5
    \begin{itemize}
    	\item Mögliche Ausprägung: \textbf{Muster festlegen}  
    	\item maximale Anzahl Pässe: 16.777.216
    	\item Wahrscheinlichkeit falls noch nicht aufgebraucht: 25\% $\left( \frac{1}{4} \right)$
    \end{itemize}
    \item M6
    \begin{itemize}
    	\item Mögliche Ausprägung: \textbf{Muster festlegen}
    	\item maximale Anzahl Pässe: unbegrenzt  
    	\item Wahrscheinlichkeit: 50\% + x mit stets größer werdendem $x \in \left[ \frac{1}{64}; \frac{1}{2} \right]$
    \end{itemize}
\end{itemize}

\vspace{0.2cm}

Werden Pässe der Ausprägungen M1 bis M5 (aufgrund ihres Caps) im Laufe der Zeit aufgebraucht, geht deren Wahrscheinlichkeit auf die Property-Ausprägung M6 über. Für das x aus der Beschreibung der Ausprägung M6 gilt also:

\vspace{0.2cm}

\begin{equation*}
x = \frac{1}{64} + \sum_{aufgebrauchte \textrm{ } M \in \lbrace M1;...;M5 \rbrace} Wahrscheinlichkeit(M)
\end{equation*}

\end{NFT-Prop}

\vspace{0.3cm}

\begin{Algo}[Verlosungs-Mechanismus für Background-Property]

\begin{itemize}
    \item Zunächst bestimme man die Gesamtanzahl aller bisher geminteter Pässe $n$.
    \item Gleiches tue man nun für die Counts der geminteten Pässe pro Ausprägung der Background-Property M1 bis M6 als entsprechende Größen $n_1, n_2,...,n_6$.
    \item Seien $\Theta_i$ für $i \in \lbrace 1,...6 \rbrace$ die oben definierten \textbf{absoluten} Obergrenzen der \newline Ausprägungen der Background-Property M1 bis M6.
    \item Seien $\rho_i$ für $i \in \lbrace 1,...6 \rbrace$ die oben definierten Wahrscheinlichkeiten der \newline Ausprägungen der Background-Property M1 bis M6, deren Auswahl wir hier zusätzlich formalisieren wollen:
    
\[
\rho_i:=\left\{%
\begin{array}{ll}
    \hbox{\LARGE $\frac{1}{2^{7 - i}}$,} & \hbox{für $i = 1,...,5$} \\[0,3cm]
    \hbox{$\frac{1}{2} + \frac{1}{2^6}$,} & \hbox{für $i = 6$} \\
\end{array}%
\right.
\]    
    
    \item Alle Ausprägungen mit $n_i \geq \Theta_i$ sind aufgebraucht und können weder zum aktuellen Zeitpunkt noch in der Zukunft vergeben werden und damit fortan auch nicht beim Minting eines neuen Pass berücksichtigt werden.
    \item Wir unterteilen die Ausprägungen der Background-Property in \textit{"verbraucht"} und \textit{"verfügbar"}:
    
\begin{align*}
\overline{M} &:= \lbrace i \in \lbrace 1,...,6 \rbrace \textrm{ } | \textrm{ } n_i \geq \Theta_i \rbrace \textrm{     [verbraucht]} \\
M &:= \lbrace i \in \lbrace 1,...,6 \rbrace \textrm{ } | \textrm{ } n_i < \Theta_i \rbrace \textrm{     [verfügbar]}
\end{align*} 

Man vergewissere sich an dieser Stelle gedanklich, dass $M \cap \overline{M} = \emptyset$, $M \cup \overline{M} = \lbrace 1,...,6 \rbrace$ und $6 \in M$ gelten.
    
    \item Ist eine bestimmte Ausprägung $i \in \lbrace 1,...,5 \rbrace$ verbraucht, soll ihre Wahrscheinlichkeit $\rho_i$ auf die Ausprägung M6 übertragen werden (damit wir bei 100\% bleiben).

    \item Damit errechnen wir die aktuell vorliegenden neuen Wahrscheinlichkeiten $\widehat{\rho}_i$ für unsere Background-Ausprägungen als
\end{itemize} 

\[
\widehat{\rho}_i:=\left\{%
\begin{array}{ll}
	\hbox{0,} & \hbox{für $i \in \overline{M}$} \\[0,1cm] 
    \hbox{$\rho_i$,} & \hbox{für $i \in M$, $i \neq 6$} \\[0,3cm]
    \displaystyle \rho_1 + \sum_{i \in \overline{M}} \rho_i, & \hbox{für $i = 6$} \\
\end{array}%
\right.
\]

Man vergewissere sich an dieser Stelle gedanklich, dass auch für die neuen \newline Wahrscheinlichkeiten \[\sum_{i = 1}^6 \widehat{\rho}_i \textrm{ } = 1\] gilt.

\begin{itemize}
    \item Am Ende bestimme man mittels Randomisierung anhand der Wahrscheinlichkeiten $\widehat{\rho}_i$ für $i \in \lbrace 1,...6 \rbrace$ die zu vergebende Background-Ausprägung. 
\end{itemize}

\end{Algo}

\vspace{0.3cm}




% -----------------NFT-Edition-------------------
\section{Edition}
\vspace{0.3cm}
% !TEX root = paper.tex

Die Edition unseres WunderPasses soll als Property auf die anfangs geforderte Möglichkeit einer gewissen Individualisierung des WunderPasses durch seinen Besitzer einzahlen. Zu individuell darf eine solche NFT-Property aber auch nicht sein, da der NFT zwingend seinen Eigentümer wechseln können soll, da das ganze Unterfangen mit dem NFT-Pass andernfalls ad absurdum führte.

Um die Edition-Property noch etwas interessanter zu gestalten, sollen Exemplare jeder Edition nicht endlos verfügbar sein, sondern stattdessen irgendwann einmal \textit{aufgebraucht}. In solch einem Fall soll sich der User aber nicht einfach irgendeiner anderer Edition bedienen, sondern erhält die \textit{"Oberedition"} (Parent) seiner ursprünglich gewünschten Edition. Und sollte auch diese \textit{aufgebraucht} sein, dann wiederum die \textit{"Oberedition"} der \textit{"Oberedition"} usw. 

\vspace{0.2cm}

\begin{NFT-Prop}[Edition]

Als Ausprägung der WunderPass-NFT-\textbf{Edition} haben wir uns für \textbf{Städte} der Welt entschieden. Die \textbf{Parent-Edition} einer Stadt ist das dazugehörige \textbf{Land}, deren 
Parent-Edition wiederum der entsprechende \textbf{Kontinent} und als \textbf{oberste Editions-Ebene} dann die \textbf{Welt-Edition}. Letztere unterliegt folglich dann auch keiner stückweisen Obergrenze mehr.

\vspace{0.2cm}

\underline{\textbf{\textit{Beispiel einer Edition-Kette:}}}

\vspace{0.2cm}

\begin{equation*}
\textrm{Berlin } \rightarrow \textrm{ Germany } \rightarrow \textrm{ Europe } \rightarrow \textrm{ World }
\end{equation*} 

\vspace{0.2cm}

Es gilt das folgende grobe Regel-Set, was jedoch explizit auch nach Launch modifizierbar bleiben soll:

\begin{itemize}
    \item Die möglichen Editionen werden von uns bestimmt. Diese müssen nicht zwingend beim Launch des NFT vollständig benannt werden, sondern können stattdessen auch nachträglich eingepflegt werden. User-Wünsche (in welcher Form auch immer) sind dabei explizit erwünscht.
    \item Jede berücksichtigte \textit{Städte-Edition} ist genau \textbf{100} Mal verfügbar. Sind alle 100 Exemplare einer \textit{Städte-Edition} bereits gemintet (verbraucht), erhält die nächste Mint-Anfrage nach einem WunderPass derselben Edition automatisch die zu dieser Städte-Edition gehörende \textit{Landes-Edition}.
    \item Die \textit{Landes-Editionen} sind in einer maximalen Stückzahl von je \textbf{10.000} pro berücksichtigtem Land verfügbar. Sind auch diese aufgebraucht, wird die durch den User ausgewählte Stadt auf die ihrem Land übergeordnete \textit{Kontinent-Edition} gemappt.
    \item Die \textit{Kontinent-Editionen} sind in einer maximalen Stückzahl von je \textbf{1.000.000} für jeden Kontinent (außer der Antarktis) vorgesehen. Sollte auch diese Menge irgendwann erschöpfen, greifen wir zu der übergeordneten \textit{Welt-Edition}.
\end{itemize} 

\end{NFT-Prop}

\vspace{0.4cm}

\underline{\textbf{Quantitative Daten zu den Editionen:}}

\begin{itemize}
    \item Nach aktuellem Stand sind mindestens 693 \textit{Städte-Edition} vorgesehen.
    \item Die genannten \textit{Städte-Edition} verteilen sich dabei aktuell auf 179 \textit{Landes-Edition}.
    \item Die unterschiedlichen \textit{Kontinent-Edition} belaufen sich auf 6 (Nord- und Südamerika, Europa, Afrika, Asien und Australien).
    \item Die übergeordnete \textit{Welt-Edition} ist in ihrer Stückzahl unbegrenzt. 
    \item Die Auswahl der angebotenen \textit{Städte-Editionen} folgt (mit Augenmaß) in etwa folgender Logik:
    \begin{itemize}
    	\item Die Hauptstadt eines jeden mit einer \textit{Landes-Edition} versehenen Landes ist gleichzeitig auch eine verfügbare \textit{Städte-Edition}.
    	\item Mit Ausnahme der Hauptstädte erfordert die Größe einer Stadt (nach Einwohnern) ein Mindestmaß $m_1$, um als \textit{Städte-Edition} aufgenommen zu werden.
    	\item Sofern es das vorige Kriterium hergibt, sollen nach Möglichkeit für jedes Land mit einer eigenen \textit{Landes-Edition} mindestens seine 5 größten Städte mit einer eigenen \textit{Städte-Edition} versehen werden.
    	\item Überschreiten die $n$ größten Städte eines in die \textit{Landes-Editionen} aufgenommenen Landes eine bestimmte Mindestgröße $m_2$ (nach Einwohnern), werden alle $n$ Städte in die verfügbaren \textit{Städte-Editionen} aufgenommen. Dieses Kriterium wird aufgrund des vorigen ausschließlich für $n > 5$ relevant.
    	\item Städte der G7-Länder werden (ungeachtet etwaiger Mindestgröße) vermehrt in die \textit{Städte-Editionen} aufgenommen (bis zu 25 \textit{Städte-Editionen} pro G7-Land).
    \end{itemize} 
    \item Einzelne Städte können bei Bedarf auch bei Missachtung aller vorigen Kriterien aufgenommen werden.
\end{itemize}

\vspace{0.5cm}




    % binde die Datei '[NFT-Pass][Konzept][Edition].tex' ein

% -----------------Design------------------------
\section{Design}
\vspace{0.3cm}
% !TEX root = paper.tex

\subsubsection{Design}

\vspace{0.2cm}

\todo{TODO: Design}

\vspace{0.3cm}    % binde die Datei '[NFT-Pass][Konzept][Design].tex' ein
%% !TEX root = paper.tex

\subsubsection{Design}

\vspace{0.2cm}

\todo{TODO: Design}

\vspace{0.3cm}
\newpage


% -----------------Beispiel----------------------
\section{Beispielhafte Analyse der Collection}
\vspace{0.3cm}
% !TEX root = C:/Users/Slava/White-Paper/[06][NFT-Pass]/[NFT-Pass][Konzept].tex

\subsubsection{Beispiel}

\vspace{0.2cm}

\todo{TODO: Beispielrechnung für geminteten NFT-Pass mit der Nummer x}

Angenommen x sei 1.005.965.

\begin{itemize}
  \item vorrechnet, welche ersten 1.005.964 NFT-Pässe schon weggemintet sein könnten und Wahrscheinlichkeiten für den neu zu mintenden NFT-Pass erklären.
  \item neuen NFT-Pass unter Einbindung der Wahrscheinlichkeiten und vorgegaukelten Zufalls errechnet.
  \item geminteten neuen NFT-Pass als exakte Grafik in unserem Design hier abbilden.
\end{itemize}

\vspace{0.3cm}    % binde die Datei '[NFT-Pass][Konzept][Beispiel].tex' ein
%% !TEX root = C:/Users/Slava/White-Paper/[06][NFT-Pass]/[NFT-Pass][Konzept].tex

\subsubsection{Beispiel}

\vspace{0.2cm}

\todo{TODO: Beispielrechnung für geminteten NFT-Pass mit der Nummer x}

Angenommen x sei 1.005.965.

\begin{itemize}
  \item vorrechnet, welche ersten 1.005.964 NFT-Pässe schon weggemintet sein könnten und Wahrscheinlichkeiten für den neu zu mintenden NFT-Pass erklären.
  \item neuen NFT-Pass unter Einbindung der Wahrscheinlichkeiten und vorgegaukelten Zufalls errechnet.
  \item geminteten neuen NFT-Pass als exakte Grafik in unserem Design hier abbilden.
\end{itemize}

\vspace{0.3cm} 


% -----------------Intrinsischer Wert------------
%\section{Intrinsischer Wert}
%\vspace{0.3cm}
%% !TEX root = paper.tex
\vspace{0.2cm}

\todo{TODO: intrinsischer Wert mittels Berechtigungen als Governance-Token}

\vspace{0.3cm}    % binde die Datei '[NFT-Pass][Konzept][Intrinsischer Wert].tex' ein


 
\end{document}