% !TEX root = paper.tex

% https://de.wikibooks.org/wiki/LaTeX-Kompendium:_Schnellkurs:_Grafiken

Dem teils trockenen Text der vorigen Kapitel sollen hier einfach wortlos einige denkbare Ausprägungen unseres WunderPasses in Bild folgen:

\begin{figure}[h]
  \centering
  \subfloat[][]{\includegraphics[width=0.4\linewidth]{{"[A][NFT-Pass]/images/diamand 1"}}}%
  \qquad
  \subfloat[][]{\includegraphics[width=0.4\linewidth]{{"[A][NFT-Pass]/images/diamand 2"}}}%
  \caption{zwei \textit{diamond} Pässe mit je unterschiedlichen Hologrammen und Pattern}%
\end{figure}

\begin{figure}[h]
  \centering
  \subfloat[][]{\includegraphics[width=0.4\linewidth]{{"[A][NFT-Pass]/images/black"}}}%
  \qquad
  \subfloat[][]{\includegraphics[width=0.4\linewidth]{{"[A][NFT-Pass]/images/pearl"}}}%
  \caption{Pässe des Status \textit{black} und \textit{pearl}}%
\end{figure}

\begin{figure}[h]
  \centering
  \subfloat[][]{\includegraphics[width=0.4\linewidth]{{"[A][NFT-Pass]/images/gold"}}}%
  \qquad
  \subfloat[][]{\includegraphics[width=0.4\linewidth]{{"[A][NFT-Pass]/images/bronze"}}}%
  \caption{rechts ein bronzener Pass mit den sehr sehr seltenen \textit{Pyramiden von Gizeh} als Hologramm}%
\end{figure}
