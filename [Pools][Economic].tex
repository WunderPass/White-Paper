% !TEX root = paper.tex

\vspace{0.3cm}

\href{https://medium.com/atchai/can-we-save-the-utility-token-55ef639370cf}{So könnte ein Pool-Utility-Token umgesetzt werden.}

\vspace{0.3cm}

\begin{itemize}
	\item Utility-Token: WunderPool-Tolen (PLT)
	\item Bonding-Curves-Modell
	\item Bezug zum WunderToken herstellen
	\item Daten, Zahlen, Fakten 
	\item Business-Case (aus Investoren-Sicht)
	\begin{itemize}
		\item Wirtschaftlichkeit und Preisentwicklung vorrechnen
		\item (praktische) Obergrenze des eingebrachten Gesamtkapitals annehmen, mit der eine plausible und attraktive Rendite vorgerechnet werden kann.
	\end{itemize}
\end{itemize}

\newpage

\todo{Einleitung: Einführung des WunderPool-Tokens: W-PLT}

\vspace{0.5cm}

\begin{Praemisse}[Cash-Flow]

\begin{itemize}
	\item Deposit/Invest erfolgt in einem Stable-Coin (z.B. \textit{USDT}). \todo{Es sind aber auch mehrere zulässige Währungen zulässig.}
	\item Cashout erfolgt in derselben Währung wie der Deposit \todo{(oder zumindest in einer der zulässigen Währungen)}. 
	\item Fees werden in aller Regel prozentual am Volumen (also in \textit{USDT}) berechnet, jedoch in \textbf{\textit{W-PLT}} veranschlagt, bei dem man von Kursschwankungen ausgehen muss und ausgehen will. \todo{(Das muss sowohl bei der Token-Modellierung als evtl. auch bei der Gebührenordnung berücksichtigt werden.)}
	\item Ein Teil der Fees soll direkt an den (Bonding-Curves-basierten) \textit{W-PLT}-Contract gehen und damit die \textit{W-PLT}-Investoren/-Hodler belohnen.
	\item Die Fees sollen \todo{(aufgrund des genauer zu erklärenden Stakings)} erst bei der Liquidierung des Pools entrichtet werden.
	\item Die Fees sollen von allen Pool-Teilnehmers außer des Pool-Creators getragen werden.
	\item Für den Pool-Creator soll folgendes gelten:
	\begin{itemize}
		\item \todo{(Muss einen PassNFT besitzen.)}
		\item Soll einen prozentual an den geschätzten gesamten Pool-Fees gemessenen Betrag $x$ als Sicherheit staken ($x \in [50 \%; 200 \%]$).
		\item Soll selbst keine Fees bezahlen.
		\item Soll für das Staken mit einem Teil der erwirtschafteten Gebühren entlohnt werden.
	\end{itemize}
\end{itemize}

\end{Praemisse}

\vspace{0.5cm}

\begin{Assumption}[Gebühren]\label{fees}

Es sollen in etwa folgende \textit{Basic-Fees} anfallen:

\begin{itemize}
	\item Grundgebühr von 1 \% auf den Deposit (für jeden Pool-Teilnehmer außer des Pool-Creators).
	\item Tradinggebühr von 0.1 \% auf jede Kauf- oder Verkaufsorder.
	\item Gewinnprovision von 4.9 \% auf einen durch den Pool erwirtschafteten \textbf{positiven} EBIT (bei Liquidierung des Pools).
\end{itemize}

\vspace{0.2cm}

Ergänzt werde diese durch etwaige \textit{Service-Fees}: 

\begin{itemize}
	\item Erweiterte Grundgebühr von zusätzlichen 1.5 \% auf den Deposit bei einem späteren Pool-Beitritt (additiv zu der obigen Basis-Grundgebühr).
	\item \textit{Leaving-Gebühr} von 6.9 \% auf den Cashout-Betrag bei vorzeitigem Verlassen des Pools und Cashout seitens eines Pool-Teilnehmers, falls der Cashout über den Pool-Contract erfolgt (und nicht z.B. mittels Verkaufs der Shares an einen anderen Pool-Teilnehmer oder am Sekundär-Markt).
\end{itemize}

\vspace{0.2cm}

Zudem sind folgende \textit{Benefits} hinsichtlich der Gebührenordnung für Inhaber eines Pass-NFTs \todo{(Kapitel verlinken)} vorgesehen:

\begin{itemize}
	\item Wegfall der Deposit-Grundgebühr für Inhaber eines PassNFTs des Status \textit{Diamond}.
	\item Reduzierung sämtlicher Gebühren, die auf User- und nicht Pool-Basis anfallen um
	\begin{itemize}
		\item 50 \% für Teilnehmer mit PassNFT-Status \textit{Diamond},
		\item 30 \% für Teilnehmer mit PassNFT-Status \textit{Black},
		\item 20 \% für Teilnehmer mit PassNFT-Status \textit{Pearl},
		\item 10 \% für Teilnehmer mit PassNFT-Status \textit{Platin}.
	\end{itemize}
\end{itemize}

\vspace{0.5cm}	

Aktuell nicht berücksichtigt jedoch grundsätzlich spannend sind die folgenden Gebühren-Aspekte und -Varianten:

\begin{itemize}
	\item Eine mögliche \textit{Trial-vs-Pro-Gebührenordnung}, bei der (stark) limitierte Pools (sowohl finanziell als auch feature-technisch) gänzlich kostenlos bleiben könnten, während eine unlimitierte Nutzung mit höheren Gebühren als den obigen einhergehen würde.
	\item \textit{Managed-Pools}: Pools, die von einem erprobten und erfolgreichen Pool-Creator hinsichtlich der Invests gesteuert, könnten eine höhere Teilnahme-Gebühr erfordern, an der auch der Creator maßgeblich beteiligt wird. 
\end{itemize}	

\end{Assumption}

\vspace{0.5cm}

Abgerechnet werden die auf den Pool anfallenden Fees (selbst die ausschließlich User-basierten) aufgrund von \textit{Mechanism-Design}-Überlegungen erst bei seiner Liquidierung.

\vspace{0.3cm}

\begin{Praemisse}[Abrechnung]

Sämtliche für einen Pool angefallenen Fees werden (ungeachtet ihres Fälligkeitszeit\-punkts) fließen erst bei seiner Liquidierung und werden zwischen Fälligkeit und Entrichtung in einem gesonderten Teil der \textit{Pool-Treasury} vorgehalten (ähnlich dessen, wo der gestakte Betrag des Pool-Creators verwahrt wird).

\vspace{0.2cm}

Wir werden diese finanziellen Mittel im weiteren Verlauf auch als \textbf{\textit{Pending-Fees}} bezeichnen.

\vspace{0.2cm}

In Analogie dazu werden wir an geeigneter Stelle folgend auch von \textbf{\textit{Staked-Fees}} sprechen - gleichwohl es sich dabei eher um eine Sicherheit als um tatsächliche Fees handelt.

\end{Praemisse}

\vspace{0.5cm}

Der Gebührenordnung folgen einige Annahmen hinsichtlich des \textbf{Business-Plans} für eine Größenordnung von zwölf Monaten.

\vspace{0.3cm}

\begin{Assumption}[Business-Plan]

Zunächst schätzen wir einige KPI ab, die es natürlich zu validieren gilt:

\begin{itemize}
	\item Wir gehen im Mittel von ca. 4-5 Teilnehmern je Pool aus.
	\item Wir gehen von einem durchschnittlichen Deposit von 200\$ je Teilnehmer und Pool aus - also einem durchschnittlichen initialen Pool-Kapital von 800-1000\$.
	\item Wir gehen des Weiteren von einer durchschnittlichen \textit{Pool-Lifetime} von ca. 6 Monaten aus,
	\item schätzen die durchschnittliche Anzahl an Tradings während der Pool-Lifetime auf 10-15,
	\item deren Trading-Volumen auf etwa $\frac{1}{3}$ des initialen Pool-Kapitals und schließlich 
	\item und einen daraus resultierenden konservativen mittleren Profit von 2 \% (auf die Pool-Lifetime von 6 Monaten also 4-5 \% p.a.).
	\item Zuletzt schätzen wir, jeder User betreibe im Mittel 2-3 Pools gleichzeitig.
\end{itemize}

\vspace{0.5cm}

Diesen geschätzten KPI zugrundeliegend setzen wir uns folgende Ziele hinsichtlich initiierter (gebührenpflichtiger) Pools - ungeachtet dessen, ob diese zu dem gegebenen Zeitpunkt noch existieren oder bereits liquidiert wurden:

\begin{itemize}
	\item 50 initiierte Pools nach 3 Monaten
	\item 150 initiierte Pools nach 6 Monaten
	\item 500 initiierte Pools nach 12 Monaten
\end{itemize}

\end{Assumption}

\vspace{0.5cm}

Damit ergeben sich folgende Business-Key-KPI:

\vspace{0.3cm}

\begin{Fazit}[Umsätze \& Forecast]

Für einen durchschnittlichen Pool $\mathcal{P}$ mit dem initialen Pool-Kapital

\begin{equation*}
  vol^{\mathcal{P}} = 4.5 \cdot 200\$ = 900\$ 
\end{equation*}

approximieren wir die anfallenden Fees als Summe der Fee-Bestandteile

\begin{itemize}
	\item Grundgebühren: $fees_{G}^{\mathcal{P}} = \rho(nft) \cdot 0.01 \cdot (4.5 - 1) \cdot 200\$ $
	\item Trading-Gebühren: $fees_{T}^{\mathcal{P}} = \phi(nft) \cdot 0.001 \cdot 12.5 \cdot vol^{\mathcal{P}} $
	\item Profit-Beteiligung: $fees_{P}^{\mathcal{P}} = \phi(nft) \cdot 0.02 \cdot vol^{\mathcal{P}} $
\end{itemize}

wobei $\rho(nft)$ und $\phi(nft)$ Normierungsfaktoren darstellen, die die in Annahme \ref{fees} beschriebenen \textit{Benefits für PassNFT-Besitzer} berücksichtigen sollen, und von uns als 

\begin{itemize}
	\item $\rho(nft) \approx \frac{2}{3}$ und 
	\item $\phi(nft) \approx \frac{3}{4}$
\end{itemize}	

geschätzt werden sollen.

\vspace{0.2cm}

Damit belaufen sich die einzelnen Fees-Bestandteile auf 

\begin{itemize}
	\item Grundgebühren: $fees_{G}^{\mathcal{P}} \approx 4.67\$ $
	\item Trading-Gebühren: $fees_{T}^{\mathcal{P}} \approx 8.44 \$ $
	\item Profit-Beteiligung: $fees_{P}^{\mathcal{P}} \approx 13.5 \$ $
\end{itemize}

und damit die im Mittel erwarteten Gesamt-Fees pro Pool auf

\begin{equation*}
  fees^{\mathcal{P}} = fees_{G}^{\mathcal{P}} + fees_{T}^{\mathcal{P}} + fees_{P}^{\mathcal{P}} \approx 26.61 \$. 
\end{equation*}

\vspace{1.0cm}

\todo{Forecast}


\end{Fazit}

\vspace{0.3cm}

\todo{Staking von WUNDER}

\todo{Pass-NFT-Status-Rewards}

\todo{Pass-NFT-Wunder-Rewards}

\vspace{0.5cm}

