% !TEX root = paper.tex

\subsubsection{Einleitung}

\vspace{0.2cm}

Im gegenständigen Abschnitt erfolgt eine grundlegende finanzielle Beleuchtung des in den letzten Abschnitten beschriebenen Pools-Projekt - und zwar aus Sicht aller beteiligten Parteien - also aus Sicht von WunderPass, aus Sicht der User und aus Sicht etwaiger Projekt-Investoren und anderer -Stakeholder.

\vspace{0.1cm}

Dabei sollen gleichermaßen ein Monetarisierungmodell, ein zugehöriger Business-Plan sowie eine mögliche Utility-Token-Ökonomie, die diese Komponenten mittels \href{https://de.wikipedia.org/wiki/Mechanismus-Design-Theorie}{Mechanismus-Design} in Einklang zueinander bringt und in einem übergeordneten Ökonomie-Kreislauf verankert, gleichzeitig erarbeitet und miteinander verknüpft werden.

\vspace{0.2cm}

Am Ende soll idealerweise jede solcher Fragen wie,

\begin{itemize}
	\item \textit{Wer bezahlt den Pool-Service und wie viel?}
	\item \textit{Wer verdient am Pool-Service und wie viel?}
	\item \textit{Wie wird das Pool-Projekt finanziert und wie werden etwaige Investoren incentiviert und entlohnt?}
	\item \textit{Wie sieht der konkrete Business-Plan aus?}
	\item \textit{Wie wird der zugehörige Pool-Project-Token modelliert und in das übergeordnete Pool-Ökosystem integriert?}
	\item \textit{Wie sind Risiko und ROI von etwaigen Projekt-Invests zu beziffern?}
\end{itemize}

beantwortet sein.

\vspace{0.5cm}

Da der zentrale Bestandteil der eigentlichen Dienstleistung der Pools für seine Nutzer bereits in sehr starkem finanziellen Kontext - nämlich des \textit{Social-Investings} - steht, und wir uns im Folgenden mit dem finanziellen Gerüst des übergeordneten Pools-Projects beschäftigen möchten - das aber so gar nichts mit der Dienstleistung des \textit{Social-Investings} an sich gemein hat, müssen wir gleich zu Beginn eine essenzielle Abgrenzung ziehen, ohne deren unmissverständliches Bewusstsein beim Leser die folgenden Kapitel nur missverstanden werden können und werden.

\vspace{0.2cm}

\textbf{Man lese und verinnerliche also folgendes lieber gleich zehnmal:}

\vspace{0.2cm}

\begin{Abgrenzung}[Pools-Project-Economics haben nichts mit Invest/Economics eines einzelnen Pools (als Dienstleistung des Pools-Projects) zu tun.]

\vspace{0.2cm}

Die Dienstleistung unseres Pools-Projects hat im Sinne des \textit{Social-Investings} unausweichlich mit Geld zu tun. Die \textbf{Pools-Project-Economic} haben dies konsequenterweise ebenfalls.

\vspace{0.1cm}

\textbf{Dabei steht ausschließlich zweites im Fokus des gegenständigen Kapitels. Erstes dagegen bestenfalls beiläufig als Referenzgrundlage bis gar nicht.} Die User der Pools hantieren mit Geld, indem sie den Service nutzen. Projekt-Stakeholder verdienen idealerweise an der angebotenen Dienstleistung - wie sie es auch täten, falls die Dienstleistung keinerlei finanziellen Bezug hätte.

\vspace{0.75cm}

Wir wollen hier einige \textit{Fallstricke} für offensichtliche Missverständnisse und Verwechselungsgefahren ganz konkret beim Namen nennen:

\begin{itemize}
	\item Die Pools (als genutzte Dienstleistung) verfügen über Funds und Assets. Beides werden in aller Regel Tokens sein. Die Funds - als \textit{Fiat-Äquivalent} - vermutlich (aber auch nicht zwingend) mittels eines \textit{Stable-Coins} repräsentiert. Die Assets erst einmal nicht weiter spezifiziert. 
	
	\textbf{Diese finanziellen Mittel eines Pools stellen bestenfalls eine Referenzgrundlage zu anfallenden Service-Fees dar, sind kein direkter Bestandteil der Pools-Economics und verwenden ganz besonders NICHT den Pool-Project-Token als Basis-/Funding-Währung.}
	\item Die Monetarisierung des Pools-Service wird anhand von (prozentualen) Service-Fees erfolgen, die als Berechnungsgrundlage durchaus das Kapital des je\-weiligen Pools heranziehen kann und wird. 
	
	Konkret werden diese Fees in einer dafür definierten Währung anfallen, die ein \textit{Stable-Coin} UND/ODER der Pool-Project-Token sein kann. Die \textit{Monetarisier\-ungs-Währung} ist dabei zentraler Bestandteil der \textit{Pool-Economics}, die Währung der Pool-Funds eines Pools ist es dagegen absolut nicht und daher auch nicht maßgebend für die Fees-Abrechnung. Bei etwaigen Währungs-Diskrepanzen muss unter Umständen ein Umrechnungs- und Ad-Hoc-Umtausch-Mechanismus implementiert werden
	\item Ein in den folgenden Kapiteln definierter \textit{Token-\textbf{Staking}-Mechanismus} wird den Pool-Project-Token als Währung vorsehen und \textbf{hat dabei absolut nichts mit dem/den Pool-Kapital/-Funding/-Assets zu tun.}
	\item Jeder Pool wird eine \textbf{\textit{Pool-Treasury}} besitzen, die die Pool-Funds und die Pool-Assets verwaltet. Unser Pool-Project-Token wird gleichzeitig einem Mo\-dell folgen, bei dem eine sogenannte \textbf{\textit{Token-Contract-Treasury}} von großer Bedeutung sein wird, die wir künftig wahlweise auch als \textbf{\textit{Pools-Project-Treasury}}, \textbf{\textit{Pools-Token-Treasury}} oder als \textbf{\textit{Project-Token-Treasury}} be\-zeichnen. Ungeachtet der - nicht immer konsistenten Bezeichnung - ist diese dringend von der erstgenannten Treasury eines einzelnen Pools zu unterscheiden.
\end{itemize}

\end{Abgrenzung}

\vspace{0.5cm}



\subsubsection{WPT - Die grundlegende Idee eines Pools-Project-Utility-Tokens}
\vspace{0.2cm}

\paragraph{Monetarisierung \& Tokenisierung}
\textbf{ }
\vspace{0.3cm}

Der abstrakt gehaltenen Einleitung zum finanziellen Grundgerüst unseres Pool-Projekts wollen wir in diesem Abschnitt nun den konzeptuell gedanklichen Grundstein zur dessen tatsächlichen Economics-Realisierung legen, auf dem dann im Anschluss die folgenden Kapitel aufbauen.

\vspace{0.2cm}

Dazu folgen zunächst einige - mehr oder minder erklärungsbedürftige - rohe Aussagen: 

\vspace{0.2cm} 

\begin{Praemisse}[Monetarisierung]
\label{monetarisierung}
\vspace{0.2cm}

Die Monetarisierung unseres Pool-Service soll auf Basis (prozentualer) Fees (siehe \nameref{sec:fees}) - gemessen am (finanziellen) Volumen der erbrachten Dienst\-leistung - erfolgen. Für den Moment sehr plakativ betrachtet, ist dies gleichbedeutend mit: 

\vspace{0.2cm} 

\textbf{\textit{Mit je mehr Kohle die Pools hantieren, desto größer sollen die anfallenden Fees sein!}}

\end{Praemisse}

\vspace{0.5cm}

\begin{Praemisse}[Utility-Token als Monetarisierungs-Tool für alle Stakeholder]
\label{fees-for-token}
\vspace{0.2cm}

\textbf{Die Fees sollen mittels eines dafür geschaffenen Utility-Tokens abgerechnet, erhoben und erbracht werden!}

\vspace{0.5cm} 

Für den - unbestreitbar verkomplizierenden und technisch teils nicht unerheblich umständlichen - Umweg der Monetarisierung über einen Token sehen wir folgende schlagende Argumente, die auch in den anschließend folgenden Kapiteln immer mal wieder argumentativ zum Vorschein kommen werden:

\begin{itemize}
	\item Die Nutzung des Pools-Service kann als ein echtes \textit{\textbf{Gut}} - eine \textit{Utility} - angesehen werden, was unter Umständen nicht endlos verfügbar sei (begrenzte Skalierung auf der Blockchain), besonders begehrt (bei exzellenter Service-Qualität) oder im Übermaß vorhanden (bei anfänglicher Unbekanntheit des Service) sei. 
	
	Durch die Tokenisierung der Dienstleistung einverleibt man dieser den Stellenwert einer \textit{Ressource}, mit zugehörigen Eigenschaften wie \textbf{Verfügbarkeit}, \textbf{Qualität} und \textbf{Nachhaltigkeitsgedanken}, was bei digitalen Dienstleistungen oft unberücksichtigt bleibt. 
	
	Mit diesem Ansatz kommt das \textit{Marktwirtschaftsprinzip von Angebot \& Nachfrage} auch bei digitalen Services zum Tragen, was in der digitalen Welt heutzutage ausschließlich auf \textit{Nachfrage} reduziert wurde, da das \textit{Angebot} de facto als unendlich betrachtet wird.
	\item Die Tokenisierung eines Business-Modells eröffnet einem das sehr mächtige spieltheoretische Werkzeug des \href{https://de.wikipedia.org/wiki/Mechanismus-Design-Theorie}{Mechanismus-Design}, um sämtliche Projekt-Beteiligte bzw. -Stakeholder in ihrem Verhalten hinsichtlich des übergeordneten Projekterfolgs zu beeinflussen/incentivieren. Oder simple ausgedrückt: Das zu tun, was wir aus strategischen Überlegungen möchten, dass er/sie tut.
	\item \textbf{Direkte \& unbürokratische Projekt-Finanzierung}.
	
	Durch die Tokenisierung der Dienstleistung muss ein potenzieller Investor beim Kauf von Utility-Tokens lediglich vom Erfolg der Dientleistung=Utility selbst überzeugt sein (da eine Nachfrage nach der Dienstleistung direkt an die Nachfrage nach dem zugehörigen Utility-Token gekoppelt ist), anstatt bei seiner ROI-Evaluierung herkömmliche bürokratisch geregelte Venture-Capital-Aspekte wie etwaige Shareholders-Agreements und Exit-Szenarien hinzuziehen zu müssen.
	\item Technische und konzeptuelle Vereinfachung, Flexibilität und Direktheit bei \textit{Customer-Akquise} und \textit{CRM} mittels des Utility-Tokens, da
	\begin{itemize}
		\item die \textit{Marketing-Währung} in Form von Tokens die \textbf{Utility} selbst statt \textit{Fiat} in den Vordergrund rückt.
		\item Der \textit{Project-Owner} (in dem Fall also WunderPass) in aller Regel selbst ein großer Token-Holder sein wird und somit über die Mittel verfügt, das Marketing-Volumen zu erbringen (ohne dabei zusätzlich finanziell belastet zu werden).
	\end{itemize}
	\item Uneingeschränkte Transparenz für alle Projekt-Teilnehmer über Stake, Cash-Flows, Handlungen, Strategien etc. aller anderen Projekt-Teilnehmer und damit ihrer Position und Interessen innerhalb des Projekts mittels jederzeit offen einsehbarer dezentraler Smart-Contract-Logik.
	\item Uneingeschränkte Transparenz und Eliminierung von Interpretationsspielraum hinsichtlich des Business-Plans.
	\item Zu guter Letzt sei noch das - weniger auf harten Fakten als auf dem \textit{Opportunitiy-Gedanken} begründete - Argument des vermeintlichen \textit{Tokenisierungs-Trends} zu nennen, welches ein rein selbstzweck-getriebenes Interesse bei potenziellen Token-Investoren wecken könnte.
\end{itemize}

\end{Praemisse}

\vspace{0.5cm}

\paragraph{Die entscheidende Idee}
\textbf{ }
\vspace{0.3cm}

Allen relevanten Erklärungen vorweggreifend folgt unser fundamentale \\
\textit{Token-Economics}-Ansatz für die Pools-Project-Token:

\vspace{0.2cm}

\begin{Konzept}[Dividende auf den Pools-Project-Token]
\label{token-usp}
\vspace{0.2cm}

Zusätzlich zur \textit{Utility}-Beschaffenheit unseres Pools-Project-Tokens möchten \\
wir diesem noch eine gewisse \textit{Equity}-Eigenschaft einverleiben:

\vspace{0.2cm}

\textbf{Ein Token-Besitzer soll mittels des Tokens nicht nur den Pools-Service nutzen können oder an der steigenden Nachfrage nach diesem - durch eine positive Kursentwicklung - profitieren, sondern zusätzlich DIREKT an den generierten Erträgen des gesamten Pools-Projects beteiligt werden.}

\vspace{0.2cm}

Er soll demnach de facto als Anteilseigner des Pools-Projects gelten und an etwaigen Gewinnen des Projekts - in Form einer gewissen \textit{Dividende} - pro rata seines Token-Volumens partizipieren.

\vspace{0.2cm}

Die Implementierung dieses \textit{Equity}-Mechanismus soll selbstverständlich mittels eines Smart-Contracts sichergestellt sein, was unseren Token stark von anderen \\ \textit{Equity}-Tokens abhebt.

\vspace{0.2cm}

Durch diesen zusätzlichen Kniff, schaffen wir eine sich selbst verstärkende Synergie zwischen den \textit{Utility}- und \textit{Equity}-Eigenschaften unseres Pools-Project-Tokens, indem wir einen potenziellen User des Pools-Service (besitzt \textit{Utility} in Form des Tokens) gleichzeitig zu einem Projekt-Investor machen (besitzt \textit{Equity} in Form desselben Tokens). Dieser doppelte Synergieeffekt weitet sich auch unmittelbar auf die Kursentwicklung aus. DENN: Wachsende Nutzung des Pools-Services impliziert zwangsläufig eine steigende Token-Zirkulation (im Sinne der \textit{Utility}-Beschaffenheit) und steigenden Bedarf und somit Nachfrage nach dem Token UND generiert gleich\-zeitig zunehmenden Ertrag durch Service-Fees, was wiederum eine Wertsteigerung des Tokens aus seiner \textit{Equity}-Beschaffenheit nach sich zieht.

\end{Konzept}

\vspace{0.3cm}

Wie genau wir uns das eben formulierte Vorhaben in der Umsetzung planen, wird etwas weiter unten vertieft. Zunächst bleiben wir beim ökonomischen Teil des Token-Designs und erarbeiten einige relevante Mechanismen.

\vspace{0.5cm}


\paragraph{Token-Design}
\textbf{ }
\vspace{0.3cm}

Beim Design unseres Pools-Project-Tokens wollen wir uns stark an den Gedanken des spieltheoretischen Gebiets des \href{https://de.wikipedia.org/wiki/Mechanismus-Design-Theorie}{Mechanismus-Design} orientieren.

Dieses Wissenschaftsgebiet befasst sich im Wesentlichen damit als \textit{höhere Instanz eines Spiels} - also in dem Fall wir als Project-Owner - mittels Regelgestaltung und Incentivierungs-Mechanismen - also in unserem Fall mittels Token-Design - Einfluss auf das Verhalten der Spieler - also in dem Fall Nutzer des Pool-Service und Investoren - im Sinne des Spiels nehmen kann.

\vspace{0.1cm}

Entscheidend hinsichtlich letzter Formulierung ist dabei das \textit{"... im Sinne des Spiels..."} genaust möglich zu präzisieren und idealerweise zu quantifizieren und formalisieren.

\vspace{0.5cm}

\textbf{Was möchten wir also genau wie, wann und womit erreichen für unser Pools-Projekt?}

\vspace{0.5cm}

Dabei bewegen sich die \href{https://de.wikipedia.org/wiki/Mechanismus-Design-Theorie}{Mechanismus-Design}-Werkzeuge tendenziell auf einer granularen Ebene, weshalb die Antwort \textit{"Pools-Project to the moon!"} auf obige Frage nicht in deren Sinne stünde. Viel mehr ist obige Frage daher als

\begin{itemize}
	\item Welche Etappenziele möchten wir erreichen (Projekt-Funding, Wachstum, Exit etc.)?
	\item Welche Projekt-Stakeholder (Gründer, Project-Owner, Investoren, User etc.) werden gebraucht und wie können diese gewonnen und deren Interessen gewahrt werden?
	\item Welche Hebel und designte Einflussmöglichkeiten möchten wir mittels von Token-Mechanismen besonders stark in eigener Hand behalten, anstatt sie dem Zufall oder Markt-Gesetzen zu überlassen?
	\item Welche Synergien möchten wir schaffen/verstärken bzw. verhindern/bremsen?
	\item Letzeres ist nicht nur aus Sicht des Pools-Projekts für sich alleinstehend zu betrachten sondern insbesondere auch im Hinblick auf ein etwaiges künftiges Wunder-Ökosystem. 
	\item Wie können wir als Gründer/Project-Owner (finanziell) profitieren?
\end{itemize}

zu verstehen. Um das ganze nicht ausufern zu lassen, wollen wir diese Fragestellungen stark auf das Pools-Projekt, seinen Projekt-Token und insbesondere dessen erhofften Effekte fokussieren:

\vspace{0.3cm}

\begin{Assumption}[Erwünschte Effekte des Pools-Project-Tokens]
\label{token-anforderungen}
\vspace{0.2cm}

Folgende Anforderungen, Erwartungen und Absichten verfolgen wir mit dem zu designenden Projekt-Token und/oder beabsichtigen zu erfüllen:

\begin{itemize}
	\item Selbstverständlich stellt ein gewisses initiales Projekt-Funding mittels Token-Sale eine der ausschlaggebendsten Motivationen für den Token dar, um z. B. auch Entwicklungskosten zu decken. 
	\item Gleichzeitig müssen aber eben die initialen Kapitalgeber angemessen für ihr Risiko entlohnt werden und signifikant stärker an ihrem Token-Invest profitieren als spätere Token-Käufer.
	\item Nicht verkehrt wäre gleiches für die Gründer ;)
	\item Nicht nur für die zuletzt genannten early Investors sondern generell für alle Token-Investoren möchten wir einen transparenten, berechenbaren und vertrauenswürdigen Token schaffen, 
	\begin{itemize}
		\item dessen Kursentwicklung keiner künstlichen PR-getriebenen Hysterie mit anschließendem Crash unterliegt (\textit{Pump \& Dump}),
		\item dessen Value transparenten und idealerweise durch Smart-Contracts ge\-steuerten Mechanismen und Projekt-Entwicklungen folgt,
		\item dessen Value einen \textit{Utility-}Bezug hat und
		\item der idealerweise mittels eines AMMs (\textit{Automated Market Maker}) jederzeit handelbar sein soll.
	\end{itemize}
	\item Nicht ganz so essenziell wie das initiale Projekt-Funding jedoch ebenfalls nicht zu vernachlässigen ist die fortlaufende (operative) Projekt-Finanzierung, die gänzlich oder zumindest teilweise durch den Projekt-Token mitfinanziert werden könnte.
	\item Gleichwohl der oben skizzierte USP unseres Tokens (siehe \ref{token-usp}) \textit{Equity}-techni\-scher Natur ist, ist und bleibt unserer Pools-Project-Token substanziell ein \textbf{\textit{Utility-Token}}.
	\begin{itemize}
		\item Grundsätzlich wird die Zirkulation eines \textit{Utility-Tokens} stets stark korreliert mit der Nutzung/Nachfrage der Utility - also in unserem Fall dem Pools-Service - sein. Wie solch eine Korrelation konkret aussieht, haben wir mittels des Token-Designs maßgeblich in eigener Hand. So kann man mit Mitteln wie z. B. \textit{Staking} oder \textit{Locking} die Zirkulation künstlich verlangsamen bzw. eine künstliche Verknappung an zirkulierenden Tokens induzieren.
		\item In gewisser Überzeugung, ein echter \textit{Utility-Token} repräsentiere eine nur endlich verfügbare Ressource, streben wir einen deflationären Token an. Oder zumindest einen \textit{pseudo-deflationären} (also einen, der zwar theoretisch unendlich lange weitergemintet werden kann, dies jedoch ab einem bestimmten Moment absolut unwirtschaftlich wird).
	\end{itemize}
\end{itemize}

\end{Assumption}

\vspace{0.5cm}

An der Abarbeitung dieser Liste werden wir uns - nicht zwingend die Reihenfolge wahrend - durch das restliche Kapitel hangeln. Bevor wir uns gleich im Anschluss etwas detaillierter dem letzten Punkt der obigen Liste - nämlich der Einflussnahme auf die Token-Zirkulation - widmen, zunächst ein sich sofort ersichtlicher \textit{Quick-Win} hinsichtlich Bullet 4 der obigen Liste:

\vspace{0.3cm}

\begin{Konzept}[Das \textit{Bonding-Curves-Modell} als vielversprechendes Mittel für unseren Pool-Project-Token]
\label{bcm}
\vspace{0.2cm}

Der Wunsch nach einem \textbf{transparenten, berechenbaren und vertrauenswürdig\-en Token} aus der Anforderungsliste \ref{token-anforderungen} suggeriert, das \textit{Bonding-Curves-Modell} als Grundlage zur Modellierung unseres Pool-Project-Token in Betracht zu ziehen, da der \textit{Bonding-Curves-Ansatz}

\begin{itemize}
	\item mittels Einsatzes eines Smart-Contract-AMMs, \textbf{Transparenz und Berechenbarkeit} des Tokens garantiert,
	\item durch im Token-Contract vorgehaltene \textbf{Kapital-Deckung pro ausgegebenem Token} das Investrisiko deckelt und damit die gewünschte \textbf{Vertrauenswürdig\-keit} abbildet und
	\item letztendlich durch seinen Basis-Mechanismus zwingend einen in seiner Logik verankerten AMM mitliefert.
\end{itemize}

\vspace{0.3cm}

\todo{1-2 allgemeine fortführende Links zu \textit{Bonding-Curves} referenzieren.}

\vspace{0.5cm}

Tatsächlich werden wir den \textit{Bonding-Curves-Ansatz} für unseren Pool-Project-Token später wieder aufgreifen und uns seiner Anwendung - den Gedanken aus dem Anhang zu \nameref{sec:bonding-curves} folgend - bemühen.

\vspace{0.3cm}

Der Vorgriff darauf erfolgte an dieser Stelle lediglich aufgrund des direkten Kontext-Bezugs zu Bullet 4 aus Anforderungsliste \ref{token-anforderungen}.

\end{Konzept}

%\newpage
\vspace{0.5cm}


Nun kommen wir - wie bereits angekündigt zu Mechanismen der \textbf{Token-Zirkulation}:

\vspace{0.3cm}


\begin{Konzept}[Token-Zirkulation-Mechanismen]
\label{circulation}
\vspace{0.2cm}

Eines sofort vorweg:

\vspace{0.2cm}
\todo{\noindent\hrulefill}

\todo{Die gleich vorgestellten Gedanken und Konzepte sind als noch nicht sehr ausgereifte initiale Ideen und Entwurfsmuster zu verstehen, die es noch zu erforschen und besser zu verstehen gilt. Mögen diese vielleicht in ihren grundlegenden Ansätzen noch so fundiert und durchdacht sein, wäre ein Anspruch ihrer perfekten Ausformulierung in einem - nicht auf fundierten praktischen Produkt-Erfahrung aufbauenden - White-Paper - wie es dieses aktuell ist - nur anmaßend und eine Vortäuschung einer pseudo-fundierten Theorie, die es aber ohne praktische Erprobung nicht ist.}

\vspace{0.2cm}

\todo{Vielmehr gilt es, die folgenden Ideen und Ansätze in ihrem Grundsatz zu verinnerlichen, und dabei gleichzeitig, die etwaigen Konkretisierungen mit Augenmaß \textit{weich} zu deuten, um diese mit zunehmender praktischer Anwendung zu validieren, zu justieren oder zu verwerfen.}

\vspace{0.2cm}

\todo{Dieser Teil des White-Papers ist also mit voller Absicht bis auf weiteres als \textbf{WIP} anzusehen und soll hier als solches markiert sein.}

\todo{\noindent\hrulefill}
\vspace{0.5cm}


\textbf{Die folgenden Ausführungen betrachten den anvisierten Pool-Project-Token in seinem Dasein als \textit{Utility-Token}.}

\vspace{0.2cm}

Es bedarf wahrscheinlich keiner weiteren Erklärung, wir verfolgten im Großen und Ganzen einen sich \textbf{positiv entwickelnden Token-Kurs} und richteten unsere \textit{Mechanism-Design}-Überlegungen genau diesem Ziel folgend aus.

\vspace{0.2cm}

Den Markt-Gesetzen folgend geht ein steigender Kurs mit \textbf{steigender Nachfrage und/oder knapper werdendem Angebot} der durch den Token repräsentierten \textit{Utility} einher.

\vspace{0.2cm}

Da wir die \textit{Utility} unseres Pool-Project-Tokens als Zahlungsmittel für die anfallenden Service-Fees des Pools-Service definiert haben, stellt uns die Gegenüberstellung der gewünschten \textbf{Kurssteigerung des Tokens} vs. des \textbf{Angebot-Nachfrage-Prinzips} vor ein nicht unerhebliches Problem:

\vspace{0.4cm}

\textbf{Die Nutzung des Pools-Service erfolgt über einen gewissen (längeren) Zeitraum. Die Entrichtung der Fees geschieht dagegen in einem einzigen Moment, was die Nachfrage nach dem Pools-Service von der Nachfrage nach dem zugehörigen \textit{Utility-Token} nahezu gänzlich voneinander ent\-koppelt - wenn nicht gar das gesamte Verständnis von einer Nachfrage nach dem \textit{Utility-Token} in sich zusammenfallen lässt.} 

\vspace{0.4cm}

Um genau diesem Problem entgegenzuwirken und die \textit{Utility} - die wir unverändert bei der Service-Fee-Abrechnung belassen wollen - zeitlich auf die übergeordnete Pools-Nutzung-Dienstleistung auszudehnen und dabei gleichzeitig eine \textbf{künstliche Verknappung} der zirkulierenden Pool-Project-Tokens zu induzieren, bedienen wir uns zweier entscheidender Design-Mechanismen:

\vspace{0.5cm}

\underline{\textbf{Pending-Fees}}

\vspace{0.3cm}
\todo{WIP}

\vspace{0.3cm}

\todo{Gebühren-Zahler:} künstlichen Token-Bedarf über längeren Zeitraum mittels \textit{Pending-Fees} forcieren


\vspace{0.75cm}

\underline{\textbf{Staking}}

\vspace{0.3cm}

\vspace{0.3cm}
\todo{WIP}

\vspace{0.3cm}


\begin{itemize}
	\item Der Pool-Initiator müsste bei der Pool-Eröffnung WPT staken, die er unter bestimmten Umständen verlieren könnte, wenn sein Pool zB. ungenutzt bleibt. So könnte man sicherstellen, dass ernste Absichten hinter den Pools stecken und diese auch genutzt werden. 	
	\item Dazu müssen Tokens gekauft und über einen längeren Zeitraum gehalten werden.
	\item Dies reduziert den Token-Umlauf und erhöht damit den Kurs.
	\item Der Staker erhält einen Anteil an den veranschlagten Fees.
	\item Der Staker ist damit nicht nur User des \textit{Pool-Service} sondern gleichzeitig auch ein Investor (Token-Holder) in das übergeordnete Pools-Project (da er gezwungen ist, die Tokens über einen längeren Zeitraum zu halten)
	\item Der Staker ist damit mehrfach incentiviert durch 
	\begin{itemize}
		\item Nutzung der Dienstleistung als solches
		\item Erstellung des/mehreren eigenen Pools aufgrund der Fees-Gewinn-Beteiligung als Staker
		\item Erstellung vieler Pools allgemein aufgrund der Fees-Gewinn-Beteiligung als Token-Holder \textit{(word-of-mouth)}
	\item In jedem Fall sollte der Staker im Normalfall (falls er nicht irgendwie Scheiße baut) bei der Auflösung des Pools mindestens seinen Einsatz zurückerhalten (also keinerlei Gebühren für die Nutzung des Pool-Service zahlen). In aller Regel sollte er mit mehr als dem ursprünglich gestakten Betrag rausgehen.
	\end{itemize}
\end{itemize}


\end{Konzept}
\vspace{0.5cm}




\paragraph{Umsetzung}
\textbf{ }
\vspace{0.3cm}

Gleichwohl noch nicht richtig quantifizierbar, jedoch konzeptuell bereits solide Formen annehmend, wollen wir an dieser Stelle endlich unseren angestrebten \textit{Pool-Project-Token} einführen und fortan an einem konkreten anstatt wie bisher abstrakt gehaltenem Gebilde weiterarbeiten:

\vspace{0.3cm}

\begin{Solution}[WunderPool-Token (\textit{WPT})]
\label{wpt}
\vspace{0.2cm}

Folgenden bisher erarbeiteten wesentlichen Ergebnissen folgend definieren wir den WunderPool-Token (\textbf{WPT}) als den anvisierten \textit{Pool-Project-Token}:

\begin{itemize}
	\item Die Monetarisierung des Pools-Projekt erfolgt durch Service-Fees, die Mittels des \textit{Utility-Tokens} \textbf{WPT} veranschlagt und abgerechnet werden (Prämissen \ref{monetarisierung} und \ref{fees-for-token}).
	\item Die Venture-Invests der \textbf{WPT}-Käufer werden durch echte Kapital-Rücklagen innerhalb des \textbf{WPT}-Token-Contracts gedeckt (Design-Merkmal \ref{bcm}).
	\item Alle \textbf{WPT}-Holder werden finanziell am etwaigen Projekt-Erfolg beteiligt \\ (Design-Merkmal \ref{token-usp}).
	\item Ein AMM zur \textbf{WPT}-Distribution (und initialem Token-Sale) wird bereitgestellt (Design-Merkmal \ref{bcm}).
	\item Der \textbf{WPT}-Supply und -Kurs wird zwecks Vermeidung von Inflation des \textit{Utility-Tokens} in gewissem Rahmen kontrolliert (Design-Merkmal \ref{bcm}).
	\item Bei gegebenem Demand nach dem Pools-Service wird die aktuelle Zirkulation des \textbf{WPT-Utility-Tokens} künstlich aufrecht erhalten und der Anteil der verfügbaren an sich in Zirkulation befindenden \textbf{WPT} künstlich verknappt (Design-Merkmal \ref{circulation}).
\end{itemize}

\vspace{0.5cm}

Zur Motivation des Einsatzes von \textbf{Bonding-Curves} beim hier besonders prägenden Design-Merkmal \ref{bcm} sei zur allgemeinen Einführung auf den Artikel \todo{Link zu einem guten Artikel}, zur Inspiration auf den Artikel \href{https://medium.com/atchai/can-we-save-the-utility-token-55ef639370cf}{Utility-Token als Bunding-Curves-Modell} und hinsichtlich Umsetzung auf unseren eigenen Content aus dem Anhang zu \nameref{sec:bonding-curves} verwiesen.

\vspace{0.3cm}
\todo{TODO: Welcher der obigen Punkte wird in welchen der folgenden Kapitel en Detail aufgegriffen und weiter vertiegt, um die noch fehlende Quantifizierung darzustellen?}

\end{Solution}

\vspace{0.5cm}

\todo{WIP}

\begin{itemize}
	\item Bonding-Curves
	\begin{itemize}
		\item Warum?
		\item Modellierung mit den Denkansätzen aus dem Anhang zu \nameref{sec:bonding-curves} unter Einbeziehung der obigen Gewinnbeteiligung an Fees
	\end{itemize}	
\end{itemize}

\vspace{0.5cm}



\paragraph{Ausblick}
\textbf{ }
\vspace{0.3cm}

\todo{WIP}

\begin{itemize}
	\item Der erste und größere Investor für das Pool-Projekt wäre WunderPass selbst. Für die erfolgte Einlage in den Projekt-Pool bekäme WunderPass WPT, die es für Incentivierungen und Rewards für die Nutzung von Pools verwenden könnte. Dieses Invest könnte (im Gegensatz zu den Einlagen anderer Investoren) zB. auch einem Locking unterliegen, um eine gewisse Preisstabilität des WPT zu gewährleisten.	
	\item Erste Andeutung, dass die Token-Contract-Treasury zwar im ersten Schritt in \textit{USDT} modelliert, jedoch in \textit{WUNDER} geplant ist.
	\item Ausblick auf die anschließenden Kapitel, die das präsentierte Konzept umsetzen.
	\item Erstmals auf \textit{Economics-Excel} verweisen.
\end{itemize}


\vspace{0.5cm}



\subsubsection{Gebühren-Ordnung}
\label{sec:fees}
\vspace{0.2cm}

\paragraph{Gebühren-Modell}
\textbf{ }
\vspace{0.2cm}

\begin{Assumption}[Gebühren]\label{fees}

Es sollen in etwa folgende \textit{Basic-Fees} anfallen:

\begin{itemize}
	\item Grundgebühr von 1.9 \% auf den Deposit (für jeden Pool-Teilnehmer außer des Pool-Creators).
	\item Tradinggebühr von 0.1 \% auf jede Kauf- oder Verkaufsorder.
	\item Gewinnprovision von 9.9 \% auf einen durch den Pool erwirtschafteten \textbf{positiven} EBIT (bei Liquidierung des Pools).
\end{itemize}

\vspace{0.2cm}

Ergänzt werde diese durch etwaige \textit{Service-Fees}: 

\begin{itemize}
	\item Erweiterte Grundgebühr von zusätzlichen 1.5 \% auf den Deposit bei einem späteren Pool-Beitritt (additiv zu der obigen Basis-Grundgebühr).
	\item \textit{Leaving-Gebühr} von 6.9 \% auf den Cashout-Betrag bei vorzeitigem Verlassen des Pools und Cashout seitens eines Pool-Teilnehmers, falls der Cashout über den Pool-Contract erfolgt (und nicht z.B. mittels Verkaufs der Shares an einen anderen Pool-Teilnehmer oder am Sekundär-Markt).
\end{itemize}

\vspace{0.2cm}

Zudem sind folgende \textit{Benefits} hinsichtlich der Gebührenordnung für Inhaber eines Pass-NFTs \todo{(Kapitel verlinken)} vorgesehen:

\begin{itemize}
	\item Wegfall der Deposit-Grundgebühr für Inhaber eines PassNFTs des Status \textit{Diamond} und \textit{Black}.
	\item Reduzierung sämtlicher Gebühren, die auf User- und nicht Pool-Basis anfallen um
	\begin{itemize}
		\item 50 \% für Teilnehmer mit PassNFT-Status \textit{Diamond},
		\item 30 \% für Teilnehmer mit PassNFT-Status \textit{Black},
		\item 20 \% für Teilnehmer mit PassNFT-Status \textit{Pearl},
		\item 10 \% für Teilnehmer mit PassNFT-Status \textit{Platin}.
	\end{itemize}
\end{itemize}

\vspace{0.5cm}	

Aktuell nicht berücksichtigt jedoch grundsätzlich spannend sind die folgenden Gebühren-Aspekte und -Varianten:

\begin{itemize}
	\item Eine mögliche \textit{Trial-vs-Pro-Gebührenordnung}, bei der (stark) limitierte Pools (sowohl finanziell als auch feature-technisch) gänzlich kostenlos bleiben könnten, während eine unlimitierte Nutzung mit höheren Gebühren als den obigen einhergehen würde.
	\item \textit{Managed-Pools}: Pools, die von einem erprobten und erfolgreichen Pool-Creator hinsichtlich der Invests gesteuert, könnten eine höhere Teilnahme-Gebühr erfordern, an der auch der Creator maßgeblich beteiligt wird. 
\end{itemize}	

\end{Assumption}

\vspace{0.5cm}



\paragraph{Gebühren-Abrechnung}
\textbf{ }
\vspace{0.2cm}




\todo{WIP}

\begin{itemize}
	\item Klarstellung und Erklärung, dass die Gebühren zunächst in Fiat berechnet, jedoch am Ende in WPT veranschlagt werden.
	\item Erste Andeutung, dass die Token-Contract-Treasury zwar im ersten Schritt in \textit{USDT} modelliert, jedoch in \textit{WUNDER} geplant ist.
	\item Cash-Flows hinsichtlich der Fees:
	\begin{itemize}
		\item Wann fallen die Gebühren an? 
		\item Wann und wie werden diese in WPT transferiert? 
		\item Wann werden diese ausgezahlt? $\rightarrow$ \textit{Pending-Fees}
	\end{itemize}
\end{itemize}

\vspace{0.5cm}

Abgerechnet werden die auf den Pool anfallenden Fees (selbst die ausschließlich User-basierten) aufgrund von \textit{Mechanism-Design}-Überlegungen erst bei seiner Liquidierung.

\vspace{0.3cm}

\begin{Praemisse}[Abrechnung]

Sämtliche für einen Pool angefallenen Fees werden (ungeachtet ihres Fälligkeitszeit\-punkts) fließen erst bei seiner Liquidierung und werden zwischen Fälligkeit und Entrichtung in einem gesonderten Teil der \textit{Pool-Treasury} vorgehalten (ähnlich dessen, wo der gestakte Betrag des Pool-Creators verwahrt wird).

\vspace{0.2cm}

Wir werden diese finanziellen Mittel im weiteren Verlauf auch als \textbf{\textit{Pending-Fees}} bezeichnen.

\vspace{0.2cm}

In Analogie dazu werden wir an geeigneter Stelle folgend auch von \textbf{\textit{Staked-Fees}} sprechen - gleichwohl es sich dabei eher um eine Sicherheit als um tatsächliche Fees handelt.

\end{Praemisse}

\vspace{0.5cm}

\todo{Ende WIP}

\vspace{0.5cm}


\subsubsection{Business-Plan}
\vspace{0.2cm}

\todo{WIP}

\begin{itemize} 
	\item Business-Case (aus Investoren-Sicht) vorrechnen
	\begin{itemize}
		\item Wirtschaftlichkeit und Preisentwicklung
		\item (praktische) Obergrenze des eingebrachten Gesamtkapitals annehmen, mit der eine plausible und attraktive Rendite argumentiert werden kann.
	\end{itemize}
	\item Excel verlinken
\end{itemize}

\vspace{0.6cm}

Der \nameref{sec:fees} folgen einige Annahmen hinsichtlich des \textbf{Business-Plans} für eine Größenordnung von zwölf Monaten.

\vspace{0.3cm}

\begin{Assumption}[Business-Plan]\label{bp}

\vspace{0.75cm}

\todo{TODO: Zahlen an Excel anpassen}

\vspace{0.75cm}

Zunächst schätzen wir einige KPI ab, die es natürlich zu validieren gilt:

\begin{itemize}
	\item Wir gehen im Mittel von ca. 4-5 Teilnehmern je Pool aus.
	\item Wir gehen von einem durchschnittlichen Deposit von 200\$ je Teilnehmer und Pool aus - also einem durchschnittlichen initialen Pool-Kapital von 800-1000\$.
	\item Wir gehen des Weiteren von einer durchschnittlichen \textit{Pool-Lifetime} von ca. 6 Monaten aus,
	\item schätzen die durchschnittliche Anzahl an Tradings während der Pool-Lifetime auf 10-15,
	\item deren Trading-Volumen auf etwa $\frac{1}{3}$ des initialen Pool-Kapitals und schließlich 
	\item und einen daraus resultierenden konservativen mittleren Profit von 2.5 \% (auf die Pool-Lifetime von 6 Monaten also 5-6 \% p.a.).
	\item Zuletzt schätzen wir, jeder User betreibe im Mittel 2-3 Pools gleichzeitig.
\end{itemize}

\vspace{0.5cm}

Diesen geschätzten KPI zugrundeliegend setzen wir uns folgende Ziele hinsichtlich initiierter (gebührenpflichtiger) Pools - ungeachtet dessen, ob diese zu dem gegebenen Zeitpunkt noch existieren oder bereits liquidiert wurden:

\begin{itemize}
	\item 50 initiierte Pools nach 3 Monaten
	\item 150 initiierte Pools nach 6 Monaten
	\item 500 initiierte Pools nach 12 Monaten
\end{itemize}

\end{Assumption}

\vspace{0.5cm}

Damit ergeben sich folgende Business-Key-KPI:

\vspace{0.3cm}

\begin{Fazit}[Umsätze \& Forecast]

\vspace{0.75cm}

\todo{TODO: Zahlen an Excel anpassen}

\vspace{0.75cm}

Für einen durchschnittlichen Pool $\mathcal{P}$ mit dem initialen Pool-Kapital

\begin{equation*}
  vol^{\mathcal{P}} = 4.5 \cdot 200\$ = 900\$ 
\end{equation*}

approximieren wir die anfallenden Fees als Summe der Fee-Bestandteile

\begin{itemize}
	\item Grundgebühren: $fees_{G}^{\mathcal{P}} = \rho(nft) \cdot 0.019 \cdot (4.5 - 1) \cdot 200\$ $
	\item Trading-Gebühren: $fees_{T}^{\mathcal{P}} = \phi(nft) \cdot 0.001 \cdot 12.5 \cdot \frac{1}{3} \cdot vol^{\mathcal{P}} $
	\item Profit-Beteiligung: $fees_{P}^{\mathcal{P}} = \phi(nft) \cdot 0.099 \cdot 0.025 \cdot vol^{\mathcal{P}} $
\end{itemize}

wobei $\rho(nft)$ und $\phi(nft)$ Normierungsfaktoren darstellen, die die in Annahme \ref{fees} beschriebenen \textit{Benefits für PassNFT-Besitzer} berücksichtigen sollen, und von uns als 

\begin{itemize}
	\item $\rho(nft) \approx \frac{9}{10}$ und 
	\item $\phi(nft) \approx \frac{7}{8}$
\end{itemize}	

geschätzt werden sollen \todo{(für die Anfangsphase sind diese eher zu klein, im einge\-schwungenen Zustand viel zu groß)}.

\vspace{0.2cm}

Damit belaufen sich die einzelnen Fees-Bestandteile auf 

\begin{itemize}
	\item Grundgebühren: $fees_{G}^{\mathcal{P}} \approx 11.97\$ $
	\item Trading-Gebühren: $fees_{T}^{\mathcal{P}} \approx 3.28 \$ $
	\item Profit-Beteiligung: $fees_{P}^{\mathcal{P}} \approx 1.95 \$ $
\end{itemize}

und damit die im Mittel erwarteten Gesamt-Fees pro Pool auf

\begin{equation*}
  fees^{\mathcal{P}} = fees_{G}^{\mathcal{P}} + fees_{T}^{\mathcal{P}} + fees_{P}^{\mathcal{P}} \approx 17.20 \$. 
\end{equation*}

\vspace{0.67cm}

Bei einer Staking-Anforderung von 200 \% der geschätzten Pool-Fees \todo{(auf Staking verlinken)} und den in \ref{bp} getroffenen Business-Plan-Annahmen ergeben sich folgende näherungsweisen Forecasts:

\begin{itemize}
	\item Nach 3 Monaten: ca. 40 noch aktive und bereits ca. 10 liquidierte Pools.
	\begin{itemize}
		\item bereits \textit{umgesetzte Fees}: 266 \$ 
		\item \textit{Pending-Fees}: 1.064 \$ 
		\item \textit{Staked-Fees}: 2.129 \$ 
	\end{itemize}
	\item Nach 6 Monaten: ca. 100 noch aktive und bereits ca. 50 liquidierte Pools.
	\begin{itemize}
		\item bereits \textit{umgesetzte Fees}: 1.330 \$ 
		\item \textit{Pending-Fees}: 2.661 \$ 
		\item \textit{Staked-Fees}: 5.322 \$ 
	\end{itemize}
	\item Nach 12 Monaten: ca. 300 noch aktive und bereits ca. 200 liquidierte Pools.
	\begin{itemize}
		\item bereits \textit{umgesetzte Fees}: 5.322 \$ 
		\item \textit{Pending-Fees}: 7.983 \$ 
		\item \textit{Staked-Fees}: 15.966 \$ 
	\end{itemize}	 
\end{itemize}

\vspace{0.5cm}

Zu guter Letzt noch eine sehr bullishe Prognose:

\begin{itemize}
	\item Nach 5 Jahren: 450.000 noch aktive und bereits 550.000 liquidierte Pools.
	\begin{itemize}
		\item bereits \textit{umgesetzte Fees}: $\approx$ 15 Mio. \$ 
		\item \textit{Pending-Fees}: $\approx$ 12 Mio. \$ 
		\item \textit{Staked-Fees}: $\approx$ 24 Mio. \$ 
	\end{itemize}	 
\end{itemize}

\end{Fazit}

\vspace{0.5cm}


\todo{Ende WIP}


\vspace{0.5cm}

\subsubsection{Herleitung Bonding-Curves}
\vspace{0.2cm}

\todo{Andere Kapitel müssen vorgezogen werden. Im Ergebnis leiten sich aber folgende Kurven für den Token ab:}

\vspace{0.3cm}

\begin{Solution}[Token-Curves]

Sei $s \in \mathbb{N}$ der Token-Supply und 

\begin{equation*}
p \approx 1.06
\end{equation*}

der \textit{Profit-Koeffizient} \todo{(erklären weshalb, wofür, warum)}.
Dann leiten sich Verkaufs- und Kaufpreis-Kurve wie folgt ab:

\vspace{0.2cm}

Verkaufspreis-Kurve:

\begin{equation*}
V(s) = V_{0} \cdot p^{ln\left(2 \cdot \frac{s}{s_{o}}\right) \cdot ln\left(\frac{s}{s_{o}}\right)}
\end{equation*}

\vspace{0.2cm}

Kaufpreis-Kurve:

\begin{align*}
K(s) &= K_{0} \cdot p^{ln\left(2 \cdot \frac{s}{s_{o}}\right) \cdot ln\left(\frac{s}{s_{o}}\right)} \cdot \left( ln(p) \cdot \left( 2 \cdot ln\left( \frac{s}{s_{o}} \right) + ln(2) \right) + 1 \right) \\
 &= V(s) \cdot \left( ln(p) \cdot \left( 2 \cdot ln\left( \frac{s}{s_{o}} \right) + ln(2) \right) + 1 \right)
\end{align*}

\vspace{0.4cm}

wobei $s_{0}$ für einen sehr kleinen (initialen) Supply und 

\begin{equation*}
K_{0} = K(s_{0}) = V(s_{0}) = V_{0}
\end{equation*}

für seinen initialen Kauf- und Verkaufskurs stehen und dabei übrigens ganz nebenbei 

\begin{equation*}
K(s) = \left( s \cdot V(s) \right)^{\prime}
\end{equation*}

gilt.

\end{Solution}

\vspace{0.5cm}


\subsubsection{Beispielrechnung}
\vspace{0.2cm}

\todo{WIP}

\begin{itemize}
	\item Daten, Zahlen, Fakten
	\item Profiterwartung aus Sicht eines Token-Holders/-Investors
\end{itemize}

\vspace{0.5cm}

Wir rechnen ein bisschen rum, um ein Gefühl für den nötigen Token-Supply zu bekommen:

\vspace{0.3cm}

\begin{Example}[Rechnerei zum Token-Supply]

\vspace{0.75cm}

\todo{TODO: Zahlen an Excel anpassen}

\vspace{0.75cm}

Wir peilen den Token-Contract so zu stricken, dass wir im eingeschwungenen Zustand einen Tokenwert des \textit{W-PLT} von $\approx$ 1 Cent anpeilen, aber gleichzeitig auch die Grenzen $[$0.5 Cent; 2 Cent$]$ im Auge behalten.

\vspace{0.5cm}

Wir forcieren beim Projekt-Fortschritt über die Zeit hinsichtlich des Tokens

\begin{itemize}
	\item Einen günstigen \textit{W-PLT}-Preis für die Gründer/Company ($\approx$ 0.25 Cent pro Token)
	\item Einen guten \textit{W-PLT}-Preis für ganz frühe Investoren ($<<$ 0.1 Cent)
	\item Einen \textit{W-PLT}-Preis von $<$ 1 Cent für die Early-Pool-User bis zum eingeschwungenen Zustand.
	\item Einen kontrollierten \textit{W-PLT}-Preis $<$ 2 Cent für die Pool-User im eingeschwungenen Zustand.
	\item Ein zunehmendes Ziel-Projekt-Invest mit Fortschritt des Projekts.
	\item Einen zunehmenden (aber kontrollierten) Ziel-Supply von \textit{W-PLT} mit Fortschritt des Projekts.
	\item Einen zunehmenden (aber kontrollierten) \textit{W-PLT}-Kurs mit Fortschritt des Projekts.
	\item Einen zunehmenden \textit{Utility-Koeffizient} (als Verhältnis zwischen mindest und Ziel-Supply) mit Fortschritt des Projekts bis zu Zielzustand des Koeffizienten von 50 \%.
\end{itemize}

\vspace{1.0cm}

Im Folgenden wieder die obige Forecast-Aufstellung - nun aus Token-Sicht:


\begin{itemize}
	\item Nach 3 Monaten: ca. 40 noch aktive und bereits ca. 10 liquidierte Pools.
	\begin{itemize}
		\item bereits \textit{umgesetzte Fees}: 25.000 \textit{W-PLT} 
		\item mindestens bereits geburnte Tokens: 12.500 \textit{W-PLT} 
		\item \textit{Pending-Fees}: 100.000 \textit{W-PLT}  
		\item \textit{Staked-Fees}: 200.000 \textit{W-PLT}
		\item min Supply: 300.000 \textit{W-PLT}
		\item Ziel-Projekt-Invest: 60.000 \$ \todo{(davon 30-40k durch Gründer/Company)}
		\item Projekt-Treasury: 60.000 \$ + 1.000 \$ Fees-Cash $\approx$ 61.000 \$
		\item Ziel-Supply: 12.0 Mio. \textit{W-PLT}
		\item \textit{Utility-Koeffizient}: $\frac{300.000}{12.000.000} = 2.5 \%$
		\item $\varnothing$ Kaufpreis pro \textit{W-PLT}: 0.5 Cent
		\item Mindest-Value pro \textit{W-PLT}: $\approx$ 0.51 Cent
	\end{itemize}
	\item Nach 6 Monaten: ca. 100 noch aktive und bereits ca. 50 liquidierte Pools.
	\begin{itemize}
		\item bereits \textit{umgesetzte Fees}: 130.000 \textit{W-PLT}
		\item mindestens bereits geburnte Tokens: 65.000 \textit{W-PLT}  
		\item \textit{Pending-Fees}: 250.000 \textit{W-PLT}
		\item \textit{Staked-Fees}: 500.000 \textit{W-PLT} 
		\item min Supply: 750.000 \textit{W-PLT}
		\item Ziel-Projekt-Invest: 120.000 \$
		\item Projekt-Treasury: 120.000 \$ + 2.500 \$ Fees-Cash $\approx$ 122.500 \$
		\item Ziel-Supply: 16.0 Mio. \textit{W-PLT}
		\item \textit{Utility-Koeffizient}: $\frac{750.000}{16.000.000} = 4.6875 \%$
		\item $\varnothing$ Kaufpreis pro \textit{W-PLT}: 0.75 Cent
		\item Mindest-Value pro \textit{W-PLT}: $\approx$ 0.77 Cent
	\end{itemize}
	\item Nach 12 Monaten: ca. 300 noch aktive und bereits ca. 200 liquidierte Pools.
	\begin{itemize}
		\item bereits \textit{umgesetzte Fees}: 500.000 \textit{W-PLT}  
		\item mindestens bereits geburnte Tokens: 250.000 \textit{W-PLT}
		\item \textit{Pending-Fees}: 800.000 \textit{W-PLT}  
		\item \textit{Staked-Fees}: 1.6 Mio. \textit{W-PLT} 
		\item min Supply: 2.4 Mio \textit{W-PLT} 
		\item Ziel-Projekt-Invest: 200.000 \$
		\item Projekt-Treasury: 200.000 \$ + 10.000 \$ Fees-Cash $\approx$ 210.000 \$
		\item Ziel-Supply: 20.0 Mio. \textit{W-PLT}
		\item \textit{Utility-Koeffizient}: $\frac{2.400.000}{20.000.000} = 12.0 \%$
		\item $\varnothing$ Kaufpreis pro \textit{W-PLT}: 1 Cent
		\item Mindest-Value pro \textit{W-PLT}: $\approx$ 1.05 Cent
	\end{itemize}
	\item Nach 3 Jahren: ca. 20.000 noch aktive und bereits ca. 20.000 liquidierte Pools.
	\begin{itemize}
		\item bereits \textit{umgesetzte Fees}: 50 Mio. \textit{W-PLT}  
		\item mindestens bereits geburnte Tokens: 25 Mio. \textit{W-PLT}
		\item \textit{Pending-Fees}: 50 Mio. \textit{W-PLT}  
		\item \textit{Staked-Fees}: 100 Mio. \textit{W-PLT} 
		\item min Supply: 150 Mio \textit{W-PLT} 
		\item Ziel-Projekt-Invest: 20.0 Mio. \$
		\item Projekt-Treasury: 20.0 Mio. \$ + 0.5 Mio \$ Fees-Cash $\approx$ 20.5 Mio. \$
		\item Ziel-Supply: 1.4 Mrd. \textit{W-PLT}
		\item \textit{Utility-Koeffizient}: $\frac{150.000.000}{1.400.000.000} = 12.0 \%$
		\item $\varnothing$ Kaufpreis pro \textit{W-PLT}: $\approx$ 1.43 Cent
		\item Mindest-Value pro \textit{W-PLT}: $\approx$ xxx Cent
	\end{itemize}
	\item Nach 5 Jahren: 450.000 noch aktive und bereits 550.000 liquidierte Pools.
	\begin{itemize}
		\item bereits \textit{umgesetzte Fees}: 1.5 Mrd. \textit{W-PLT} 
		\item mindestens bereits geburnte Tokens: 750 Mio. \textit{W-PLT}
		\item \textit{Pending-Fees}: 1.2 Mrd. \textit{W-PLT}
		\item \textit{Staked-Fees}: 2.4 Mrd. \textit{W-PLT} 
		\item min Supply: 3.6 Mrd. \textit{W-PLT} 
		\item Ziel-Projekt-Invest: 144 Mio. \$
		\item Projekt-Treasury: 144 Mio. \$ + 16 Mio. \$ Fees-Cash $\approx$ 160 Mio. \$
		\item Ziel-Supply: 7.2 Mrd. \textit{W-PLT}
		\item \textit{Utility-Koeffizient}: $\frac{3.6}{7.2} = 50.0 \%$
		\item $\varnothing$ Kaufpreis pro \textit{W-PLT}: 2 Cent
		\item Mindest-Value pro \textit{W-PLT}: $\approx$ 2.22 Cent		
	\end{itemize}	 
\end{itemize}

\end{Example}

\vspace{0.5cm}

\todo{Ende WIP}

\vspace{0.5cm}



\subsubsection{Recap \& Ausblick}
\vspace{0.2cm}

\todo{Einbettung in das Wunder-Ökosystem und Link zwischen WPT zu WUNDER}


\vspace{0.5cm}



\subsubsection{Unberücksichtigte Inhalte \& Ideen}

\vspace{0.3cm}
\todo{WIP}
\vspace{0.5cm}

\begin{Praemisse}[Cash-Flow]

\begin{itemize}
	\item Deposit/Invest erfolgt in einem Stable-Coin (z.B. \textit{USDT}). \todo{Es sind aber auch mehrere zulässige Währungen denkbar.}
	\item Cashout erfolgt in derselben Währung wie der Deposit \todo{(oder zumindest in einer der zulässigen Währungen)}. 
	\item Fees werden in aller Regel prozentual am Volumen (also in \textit{USDT}) berechnet, jedoch in \textbf{\textit{W-PLT}} veranschlagt, bei dem man von Kursschwankungen ausgehen muss und ausgehen will. \todo{(Das muss sowohl bei der Token-Modellierung als evtl. auch bei der Gebührenordnung berücksichtigt werden.)}
	\item Ein Teil der Fees soll direkt an den (Bonding-Curves-basierten) \textit{W-PLT}-Contract gehen und damit die \textit{W-PLT}-Investoren/-Hodler belohnen.
	\item Die Fees sollen \todo{(aufgrund des genauer zu erklärenden Stakings)} erst bei der Liquidierung des Pools entrichtet werden.
	\item Die Fees sollen von allen Pool-Teilnehmers außer des Pool-Creators getragen werden.
	\item Für den Pool-Creator soll folgendes gelten:
	\begin{itemize}
		\item \todo{(Muss einen PassNFT besitzen.)}
		\item Soll einen prozentual an den geschätzten gesamten Pool-Fees gemessenen Betrag $x$ als Sicherheit staken ($x \in [50 \%; 200 \%]$). \todo{Kann theoretisch auch einer absoluten oder relativen Obergrenze unterliegen.}
		\item Soll selbst keine Fees bezahlen.
		\item Soll für das Staken mit einem Teil der erwirtschafteten Gebühren entlohnt werden.
	\end{itemize}
\end{itemize}

\end{Praemisse}

\vspace{0.5cm}



\begin{Algo}[\textit{W-PLT}-Design]

\begin{itemize}
	\item Wir haben Szenarien vorgerechnet und dabei den zugehörigen \textit{Ziel-Supply} angegeben. Daran orientieren wir uns.
	\item Abhängig des Szenarios (gemappt auf den \textit{Ziel-Supply}) können wir einen aus Fees-Einnahmen resultierenden erwarteten Profit ableiten - und zwar pro Zeiteinheit, die genau der durchschnittlichen Pool-Lifetime entspricht.
	\item Wir legen einen Zeitraum als Vielfaches der durchschnittlichen Pool-Lifetime fest, die wir einem Token-Investor zumuten, bis sein Invest profital wird. Z. B. 4 $\cdot$ Pool-Lifetime.
	\item Aus den obigen Daten können wir für diesen Zeitraum (ab Invest beim entsprechenden Supply) den erwarteten relativen Profit pro Token $x$ approximieren.
	\item Diesen erwarteten relativen Profit pro Token müssen wir bei den obigen Rechnungen noch als $x(supply) = profit(zeitraum; supply)$ ergänzen.
	\item $kaufpreis(supply) := x(supply) \cdot \frac{treasury}{supply}$
	\item Den Early-Investoren darf durchaus ein größerer Faktor zugemutet werden.
	\item Wir errechnen für einige supply-Milestones den Kaufpreis auf der Grundlage des letzten Bullets und approximieren dann dazwischen.
\end{itemize}

\end{Algo}



\vspace{1.0cm}

\todo{WIP}

\vspace{0.3cm}

\begin{itemize}
	\item Wann bezahlen die User die Fees?
	\item In welcher Form/Währung dürfen die Fees von den Usern erbracht/verrechnet werden und wird dann alles im Hintergrund sofort in \textit{W-PLT} umgewandelt?
	\item Ist es denkbar die Fees aus dem Stake-Pool des Creators zu verwenden und diesem seinen Stake in einer anderen Währung zurückzuerstatten?
	\item Split der Gebühren auf Staker und Projekt-Treasury $(\sigma_{S}; \sigma_{T})$ mit $\sigma_{S} + \sigma_{T} = 1$ definieren. Dazu gibt es einige denkbare Varianten:
	\begin{itemize}
		\item fester, statischer Split
		\item fester, statischer Split mit eingebauten Unter- und Obergrenzen für den Gesamtertrag des Stakers $\sigma_{S} \cdot fees^{\mathcal{P}}$
		\item $fees^{\mathcal{P}}$-abhängiger (progressiver) Split, bei dem der Anteil des Stakers $\sigma_{S}$ mit zunehmendem $fees^{\mathcal{P}}$ stets kleiner wird. Dies unter Umständen ebenfalls unter Berücksichtigung eingebauter Unter- und Obergrenzen für den Staker.
		\item Begünstigung des Stakers in Abhängigkeit seines NFT-Pass-Status.
	\end{itemize}	
	\item Anfängliche Air-Drops von \textit{W-PLT} an erste Pool-User müssen einer Locking-Periode unterliegen.
	\item Problem: Wenn Token-Holder aussteigen, nehmen diese nicht nur ihren Anteil an den bisher erwirtschafteten Fees-Einnahmen mit, sondern drücken zudem auch noch den Token-Kurs nach unten. Das stellt einen sich selbst verstärkenden Effekt dar, der irgendwie in den Griff zu bekommen ist (evtl. Locking-Periode oder berücksichtigende Verkaufs-Kurve).
	\item Es macht eine Locking-Periode von der Dauer der durchschnittlichen Pool-Lifetime viel Sinn, da die beim Verkauf der Tokens mitgenommenen Gewinne aus Fees-Einnahmen durch neu generierte Fees-Einnahmen egalisiert werden und der Token-Value somit stabil bleibt.
	\item Die Einlage für den \textit{W-PLT} ist idealerweise in WUNDER zu erbringen \todo{(Es ist noch unklar, wie man an WUNDER kommt, wenn es vorher keinen Token-Sale gegeben hat. Ob der WUNDER ebenfalls mittels Bonding-Curves abzubilden wäre, sei hier erst einmal mehr als unklar.)}
	
\end{itemize}

\vspace{1.0cm}

\begin{Problem}[\textit{USDT} vs. \textit{W-PLT} als Berechnungsgrundlage für Fees, Staking etc.]
\vspace{0.2cm}

\todo{Was nehmen wir hier?}

\vspace{0.5cm}

\todo{Folgend übernommene alte Test-Passagen zu dem Thema:}

\vspace{0.5cm}

Ein weiterer sehr essenzieller Faktor für die Größe des zu stakenden Betrags könnte der Kurs des IPTs sein. Denn laut der \textbf{Bonding-Curves}-Implementierung würde der \textit{W-PLT}-Preis mit steigender Zirkulation steigen, was mit der Zunahme von existierende Pools geschähe. Damit wäre die Erstellung neuer Pools mit ihrer zahlen\-mäßigen Zunahme stets kapital-intensiver (aber nicht gleichbedeutend teurer). \textbf{Die Frage hierbei ist also, ob der zu erbringende Stake des Pool-Creators auf den \textit{Total-Supply des W-PLT} normiert werden sollte oder nicht}, die gänzlich mit der obigen Fragestellung einhergeht, ob der Pool-Creator eigentlich staken möchte oder das nur tun muss.
	
\begin{itemize}
	\item Gegen eine Normierung spricht die Annahme/Hoffnung, ein Pool-Creator sei gleichzeitig auch ein großer Supporter des gesamten Projekt und glaube daran. Wenn der \textit{W-PLT}-Preis steigt, ist dies gleichbedeutend mit der Zunahme an genutzten Pools, an denen der Pool-Creator als Staker, Besitzer von \textit{W-PLT} und damit Projekt-Investor auch selbst (finanziell) profitiert.
	\item Für eine Normierung spricht dagegen die potenzielle Gefahr, neue oder bestehende User durch eine zu hohe finanzielle Sicherheitseinlage davon abzuschrecken neue Pools zu erstellen.
\end{itemize}

\vspace{0.2cm}
	
Die Antwort auf diese Fragestellung könnte auch darin liegen, ob wir uns besonders viele oder lieber weniger aber besonders Teilnehmer-starke Pools wünschen.

\vspace{0.5cm}
	
Die \textit{Pool-Teilnehmer} (außer des Creators) können bei dieser Logik aber nicht wie nicht wie die Staker zusätzlich als Projekt-Investoren angesehen werden, weil sie \textit{W-PLT} kaufen, da die gekauften \textit{W-PLT} direkt als Gebühr entrichtet werden. Für die Pool-Teilnehmer stellt der \textit{W-PLT} also eher einen Utility- bzw. Purpose-Token dar weshalb die Höhe der zu entrichtenden Gebühr zweifelsfrei auf Basis von \textit{Total-Supply des W-PLT} normiert werden muss (die Gebühr darf keinesfalls mit Zunahme von Pools steigen).


\end{Problem}


\newpage

