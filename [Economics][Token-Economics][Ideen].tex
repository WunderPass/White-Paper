% !TEX root = paper.tex

\vspace{0.5cm}

\href{http://cdetr.io/smart-markets/}{Smart Markets for Stablecoins}

\href{https://medium.com/crypto3conomics/token-engineering-research-b6627add09ee}{Token Engineering Research}

\href{https://medium.com/@stephen_yo/a-token-engineering-process-16687f3b9a74}{A Token Engineering Process}

\vspace{0.5cm}

\begin{itemize}
  \item Staken von Sub-Projects. 
  \begin{itemize}
  	\item Teilprojekt wird nach \textbf{Bonding-Curves-Modell} implementiert und bekommt damit seinen eigenen Token.
  	\item Die Projekteinlage erfolge in WUNDER-Tokens (Staking).
  	\item Investoren von WUNDER hätten damit die Möglichkeit, die für sie besonders interessanten Projekte stärker zu unterstützen als lediglich das übergeordnete WunderPass-Projekt.
  	\item Der WUNDER bekäme damit einen intrinsischen Wert: Man braucht ihn, um sich an den Teilprojekten zu beteiligen.
  	\item Teilprojekte könnten outgesourcet und ausschließlich mittels einer größeren WUNDER-Einlage in den Contract des neuen Projekts incentiviert werden. Die Voraussetzung dafür wäre die Einbindung des neuen Projekts und seines Tokens in das WunderPass-Ökosystem und die Nutzung der Wunder-Identity.
  \end{itemize}
\end{itemize}

\vspace{0.5cm}
