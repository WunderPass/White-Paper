% !TEX root = [Economics][Quantifizierung].tex

\paragraph{Fazit} 
\label{sec:eco_zahlen_zustand_wp_fazit}
\textrm{ }

\vspace{0.3cm}

Ungeachtet des Werts der bisher erzielten erfolgreichen Ergebnisse hinsichtlich der quantitativen Einordnung des WunderPass-Fortschritts zu einem Zeitpunkt $t \in T$, besitzt der Zusatz "...simple Betrachtung" innerhalb der Überschrift des gegenwärtigen Kapitels durchaus seine Rechtfertigung.

Wir haben zwar die Größe $\Gamma(t)$ als sehr gut geeigneten Gradmesser für den Fortschritt WunderPasses herausgearbeitet und dieses ebenfalls in Abhängigkeit der intuitiven Erfolgsmesser $\widehat{n}$ und $\widehat{m}$ gesetzt sowie nach unten und oben abgeschätzt. Jedoch scheint unser Ökosystem zu komplex und unsere bisherige Betrachtungsweise zu global geprägt, als dass wir guten Gewissens den besagten Zusatz "...simple Betrachtung" in der Überschrift des gegenwärtigen Kapitels weglassen könnten. Den geäußerten Zweifel verdeutlicht folgendes 

\vspace{0.3cm}

\begin{Example*}
Wir nehmen den Zustand zum Zeitpunkt $t \in T$ mit $\widehat{m} = 5$ angebundenen Service-Providern und als durch $\Gamma(t) \approx 50.000$ beschrieben an und schauen uns drei Szenarien an, die allesamt die getroffene Annahme hergeben:

\vspace{0.3cm}

\begin{enumerate}
  \item Wir könnten von $\widehat{n} = 50.000$ angebundenen Usern ausgehen, von denen je 10.000 mit je einem einzigen der $\widehat{m} = 5$ Provider connectet wären und keinem anderem. 
  \item Genauso könnten dieselben $\widehat{n} = 50.000$ angebundene User so verteilt sein, dass 49.996 (quasi alle) mit demselben einzelnen Provider connectet sind, und die restlichen 4 (also quasi niemand) User mit je einem anderen der verbleibenden 4 Provider verbunden sind.
  \item Ein ganz anderes Szenario wäre der Fall von $\widehat{n} \approx 25.000$, von denen jeder mit denselben zwei unserer fünf Service-Providern connectet wäre (und keinem anderen) und zudem ein paar vereinzelte zusätzliche User mit je einem der verbleibenden drei unserer fünf Provider.
\end{enumerate}

\vspace{0.3cm}

Rein an den Größen $\widehat{n}$, $\widehat{m}$, $\Gamma(t)$ gemessen, scheint Fall (3) aufgrund von $\widehat{n} = 25.000$ der schlechteste zu sein. Rein intuitiv scheint genau dieser Fall aber der beste zu sein. Dies ist aber nur ein Gefühl. Es lassen sich ebenso gute Argumente finden, warum Fall (1) oder Fall (2) der beste sein könnten. Es kommt eben darauf an...Gleichwohl für alle der Fälle $\Gamma(t) = 50.000$ gilt, lässt sich zweifelsfrei entscheiden, welcher zwingend der beste sein soll.

Was sich jedoch objektiv beurteilen lässt, ist die Tatsache, dass in Fall (2) vier der fünf Service-Provider quasi "wertlos" sind. Und in Fall (3) immer noch drei von fünf!

\vspace{0.3cm}

Wir könnten also unsere Gegenüberstellung der drei angeführten Cases auch zur folgenden quantitativen Beurteilung stellen:

\vspace{0.3cm}

\begin{enumerate}
  \item $\Gamma_1(t) = 50.000$, $\widehat{n}_1 = 50.000$ und $\widehat{m}_1 = 5$
  \item $\Gamma_2(t) = 50.000$, $\widehat{n}_2 = 50.000$ und $\widehat{m}_2 = 1$
  \item $\Gamma_3(t) = 50.000$, $\widehat{n}_3 = 25.000$ und $\widehat{m}_3 = 2$
\end{enumerate}

\vspace{0.3cm}

Was ist also besser?

\end{Example*}

\vspace{0.6cm}

