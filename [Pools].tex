% !TEX root = paper.tex
\section{Project 'Pools'}
\label{sec:pools}

\subsection{Einleitung}

\vspace{0.3cm}

Die Idee hinter den sogenannten \textit{Wunder-Pools} ist das Bündeln von Liquidität mehrerer User/Teilnehmer bzw. eine Art 'Treuehandverwahrung' in einem gemeinsamen Pool. Die Anwendungsfälle solche Pools sind sehr zahlreich. Um in Folgenden nur einige Beispiele zu nennen:  

\begin{itemize}
  \item Gemeinsame Invests in (Crypto-)Assets.
  \item Pool für ein gemeinsames (Geburtstags-)Geschenk.
  \item Kicktipp-Pool (der über die gesamte Saison verwahrt werden muss).
  \item Wetten unter Freunden.
  \item Ausgleichspool für Auslagen von Geld an Freunde (Splitwise).
\end{itemize}

\vspace{0.2cm}

Das besondere an dem in den folgenden Abschnitten genauer zu beschreibenden Modell, ist sein sehr allgemein gehaltener Ansatz, mit dem sich gleichzeitig Cases umsetzen lassen, die auf den ersten Blick sehr verschieden zu sein scheinen. Genauer genommen lassen sich solche Pools mit speziellen \textit{DAO-Strukturen} umsetzen.

Abgesehen von der den Pools zugrundeliegenden Geschäftslogik, besteht der zentrale Ansatz unserer \textit{Wunder-Pools} darin, dem User ein rundes Produkt bereitzustellen - und zwar ganz unabhängig dessen, welcher der oben genannten Cases nun tatsächlich umgesetzt wird. An dieser Stelle möchten wir uns ganz explizit von dem Status quo der heute gängigen UX in der Web3-Welt abgrenzen.

\vspace{0.2cm}

Ganz grob beschrieben, streben wir in etwa folgende Geschäftslogik an:

\begin{itemize}
  \item Ein User erstellt einen Pool (in einer dafür implementierten Wunder-Pool-Applikation).
  \item Derselbe User definiert die Pool-Art, ein etwaiges dazugehöriges Regelwerk und fordert andere User auf, dem Pool beizutreten. Idealerweise erfolgt die Einladung mittels Suche nach der Wunder-ID des einzuladenden Teilnehmers (und nicht etwa anhand seiner Ethereum-Adresse oder sonstigem).
  \item Die eingeladenen Teilnehmer erhalten die Einladung (in der WunderPass-App oder der Wunder-Pool-Applikation) und können entscheiden, ob sie dem Pool betreten möchten oder nicht. 
  \item In der Regel ist sofort beim Beitritt des Pools der definierte Einsatz zu entrichten (der anschließend in die Pool-Treasury geht). In einigen Cases kann der Einsatz evtl. zu einem späteren Zeitpunkt erfolgen oder gar ganz entfallen (Splitwise).
  \item \todo{TODO}
\end{itemize}

\vspace{0.2cm}

\todo{Verallgemeinerung}

\todo{Product-Sicht}

\vspace{0.5cm}

\subsection{Pool-Erzeugung}

\vspace{0.3cm}

\todo{Pool erstellen (DAO)}

\todo{Pool konfigurieren}

\todo{User einladen -> mit WP verknüpfte Kontakte}

\vspace{0.5cm}

\subsection{Pool-Economics}

\vspace{0.3cm}

\todo{Staking von WUNDER}

\todo{Pool-Governance-Tokens}

\todo{Pass-NFT-Status-Rewards}

\todo{Pass-NFT-Wunder-Rewards}

\vspace{0.5cm}

\subsection{Pool-Lifetime}

\vspace{0.3cm}

\todo{zeitlich begrenzte Pools}

\todo{zeitlich unbegrenzte Pools}

\todo{Beispiele}

\vspace{0.5cm}

\subsection{Pool-Liquidierung}

\vspace{0.3cm}

\todo{Oracle entscheidet Auszahlungsschlüssel}

\vspace{0.5cm}
