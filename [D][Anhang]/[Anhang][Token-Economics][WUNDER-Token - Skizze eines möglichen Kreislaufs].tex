% !TEX root = paper.tex

\subsection{WUNDER-Token - Skizze eines möglichen Kreislaufs}
\vspace{0.3cm}

\begin{Solution}[möglicher Token-Flow]

\begin{itemize}
  \item Ein User nutzt einen Service-Provider A, der WunderPass unterstützt, und ist auch mit seinem WunderPass bei Provider A eingeloggt.
  \begin{itemize}
  	\item Beispiel 1: Der Service-Provider A ist ein Identity-Data-Management-Service, der die persönlichen Daten des Users verwaltet und bei Bedarf Dritten zur Verfügung stellen kann.
  	\item Beispiel 2: Der Service-Provider A ist EasyJet.
  \end{itemize}
  \item Der User und der Service-Provider A erzielen - wie auch immer - eine Übereinkunft darüber, dass die von Provider A verwalteten - den User betreffenden Daten - theoretisch mittels des WunderPass-Lookups mit Dritten geteilt werden können sollen. 
  \begin{itemize}
  	\item Beispiel 1: Die Daseinsberechtigung des Identity-Data-Management-Service beschränkt sich eigentlich ausschließlich auf das Teilen von Daten mit Dritten. Hierbei ist die obige Anforderung also trivialerweise unabdingbar.
  	\item Beispiel 2: Beim Beispiel von EasyJet könnten die besagten Daten z. B. gebuchte Flugtickets sein, die man mit Drittdiensten teilt, um daran ausgerichtet gezielte Werbeangebote im zugehörigen Ausland zu ermöglichen.
  \end{itemize}
  User und Provider einigen sich auf einen Preis/Preisformel für das Teilen dieser Daten - und zwar auf den konkreten Preis von \textbf{x WPT} (WunderPass-Utility-Token).
  \item Service-Provider B (der ebenfalls WunderPass unterstützt) möchte Userdaten des Service-Provider A nutzen, falls solche vorliegen.
  \begin{itemize}
  	\item Beispiel 1: Hierbei könnte Provider B so ziemlich jeder denkbare Online-Dienst sein, der irgendwelche persönlichen Userdaten benötigt (z. B. Adresse, Email, Kreditkarte etc.).
  	\item Beispiel 2: Hierbei könnte es sich z. B. um (schlecht ausgelastete) Hotels handeln, die anhand der EasyJet-Flugdaten über die Destination des Users wissend, besondere Angebote an ihn ausspielen wollen.
  \end{itemize}
  \item Service-Provider B callt der WunderPass-Lookup-Service, um die Existenz etwaiger Daten und deren \textbf{Preis x WPT} in Erfahrung zu bringen.
  \item Liegen Lookup-Daten vor, kann Provider B entscheiden, ob er diese zum angegebenen Preis beziehen möchte. 
  \item Möchte Service-Provider B Gebrauch vom Lookup machen, muss er in Vorleistung gehen und den Betrag von \textbf{2 * x WPT} in den Lookup-Contract einzahlen.
  \item Die eingebrachten \textbf{2 * x WPT} werden - abzüglich einer kleinen WunderPass-Fee - im Lookup-Contract gelockt. Service-Provider B erhält im Gegenzug einen \textit{Berechtigungs-Token} für den Abruf von entsprechenden Daten von Provider A (hierbei ist eher ein technischer Security-Token und kein Crypto-Token gemeint).
  \item Die Zugriffsberechtigung für das Abrufen der Daten von Provider A soll dabei einer \textbf{zeitlichen Beschränkung z} unterliegen (z. B. "eine Woche"). \textbf{z} ist hierbei ebenso individuell (Teilnehmer- und Daten-abhängig) zu sehen wie \textbf{x}.
  \item Service-Provider B fragt unter Vorlage des Berechtigungs-Token die gewünschten Daten beim Service-Provider A an.
  \begin{itemize}
  	\item Provider A muss den Berechtigungs-Token validieren (beim Lookup-Service).
  	\item A muss unter Umständen die Freigabe beim User einfordern (ggf. sollte der User in irgendeiner Weise "bestraft" werden, falls er den Datenzugriff verwehrt).
  	\item Provider A und B müssen einen gewissen "Handshake" implementieren, der A bescheinigt, wie vereinbart die korrekten Daten an B ausgeliefert zu haben. 
  \end{itemize}
  \item Provider A liefert die Daten an Provider B aus und erhält im Gegenzug ein Bestätigungszertifikat von B.
  \item Mit dem Bestätigungszertifikat kann Provider A seine Vergütung beim Lookup-Contract einlösen. Dabei wird die Hälfte der gelockten Einlage von Provider B (also an dieser Stelle die Hälfte von \textbf{2 * x WPT} - also \textbf{x WPT}) ausgeschüttet. Und zwar zur Hälfte an Provider A und zur andern Hälfte an den User.
  \item \textbf{x WPT} des ursprünglich eingezahlten Deposits von B bleiben weiterhin im Lookup-Contract gelockt. 
  \item Jede künftige Anfrage von B an A (bezüglich desselben Datensatzes) innerhalb des definierten Zeitraums \textbf{z} releast immer wieder die Hälfte des verbliebenen gelockten Deposits.
  \item Nach Ablauf des definierten \textbf{Zeitraums z}
  \begin{itemize}
  	\item bekommt B den verbliebenen (nicht ausgeschütteten) Teil seines Deposits zurückerstattet.
  	\item wird der \textit{Berechtigungs-Token} ungültig.
  	\item hat A keinen Anspruch mehr, für die Datenauslieferung an B vergütet zu werden (auch dann, falls er Daten ausgeliefert, ohne vorher die abgelaufene Gültigkeit des Berechtigungs-Tokens zu validieren).
  \end{itemize}
  \item Es ist denkbar, die an der User ausgezahlten Rewards in irgendeiner Weise (zeitlich) zu locken und deren Release an bestimmte Bedingungen zu knüpfen ($\rightarrow$ um den User zu incentivieren irgendetwas zu tun).
\end{itemize}

\vspace{0.5cm}

\underline{\textbf{Cashflow}}:

\begin{itemize}
  \item Provider B zahlt für den Lookup. Aber auch nur dann, falls er den Lookup nutzt. Andernfalls erhält er seinen getätigten Deposit (abzüglich einer kleinen Fee an WunderPass) zurück. Er zahlt in gleichem Teil an Provider A und den User. Aus der Verwertung der bezogenen Daten kreiert er einen Value (im Sinne seiner Dienstleistung). Einen Value, der auch durchaus im Sinne des Users sein könnte. Es ist also gut denkbar, dass Provider B eine Rechtfertigung besitzt, den User an seinen Kosten zu beteiligen (z. B. mittels einer Fee für die erbrachte Dienstleitung, die den gekauften Datensatz erforderte; idealerweise ebenfalls in \textbf{WPT} vom User zu erbringen).
  \item Provider A ist der klare Nutznießer des Datenaustauschs. Der "Daten-Trade" hat - direkt betrachtet - erst einmal gar nichts mit seinem Kerngeschäfts zu tun (es sei denn, A sei wie in Beispiel 1 ein Identity-Data-Management-Service, dessen Kerngeschäft ausschließlich darin besteht, Daten zur Verfügung zu stellen). In der perfekten WunderWelt kann Provider A in einem anderen Case, analog als Provider B auftreten, um seine erhaltenen Token-Rewards für für ihn relevante Daten auszugeben.
  \item Der User scheint hierbei auch der Nutznießer von etwaigen "Daten-Deals" zu sein. Seine Stellung als solcher ist aber weniger klar als diejenige von Provider A, da er von dem stattgefundenen Datenaustausch indirekt ebenso profitieren könnte, indem er z. B. auf Basis der Datennutzung eine bessere Dienstleistung von Provider B erhält. Der Pitch "der User monetarisiert seine Daten" kling zwar sehr attraktiv, muss man hierbei jedoch sehr aufpassen, den Bogen nicht zu überspannen. Denn - während die Rolle von Provider A als Profiteur unbestreitbar ist - wird die Zahlungsbereitschaft von Provider B von Fall zu Fall ganz unterschiedlich und nur bedingt vorhanden sein. Denn schließlich ist es alles andere als selbsterklärend, ein Online-Shop solle für Adressdaten des Users bezahlen, um seine Bestellung zustellen zu können, während der User davon profitiert. In diesem Fall wäre es eher nachvollziehbar, Provider B und der User würden sich die an Provider A zu entrichtenden Fees für die Bereitstellung der Adressdaten teilen. Hierbei ist die \textbf{Verteilung der Fees leider extrem heterogen}.
\end{itemize} 

\end{Solution}


\vspace{0.3cm}

\todo{WPT}

\vspace{0.5cm}
