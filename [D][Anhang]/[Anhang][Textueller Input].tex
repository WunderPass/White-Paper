% !TEX root = paper.tex

\subsection{Quellen und Inspiration}
\label{sec:inspiration}

\vspace{0.3cm}

Eine schöne Definition der Identität laut \href{https://link.springer.com/article/10.1007/s11612-001-0022-y}{Döring, N. (1999). Sozialpsychologie des Internet.}

\vspace{0.3cm}

\begin{Business-Def}[Identität laut Döring, N. (1999). Sozialpsychologie des Internet.]

Identität wird heute als komplexe Struktur aufgefasst, die aus einer Vielzahl einzelner Elemente besteht (Multiplizität), von denen in konkreten Situationen jeweils Teilmengen aktiviert sind oder aktiviert werden (Flexibilität). Eine Person hat aus dieser Perspektive nicht nur eine "wahre" Identität, sondern verfugt über eine Vielzahl an gruppen-, rollen-, raum-, körper- oder tätigkeitsbezogenen Teil-Identitäten.

\end{Business-Def}

\vspace{0.3cm}

Folgende hilfreiche Zitate, Aussagen und Formulierungen entstammen der \href{https://vsis-www.informatik.uni-hamburg.de/getDoc.php/thesis/47/DA_Gordian_Kaulbarsch.pdf}{Diplomarbeit "Identitäten und ihre Schnittstellen auf Basis von Ontologien in einer dezentralen Umgebung"}. 

\vspace{0.3cm}

\begin{Zitat}[Betrachtungsweise der digitalen Identität]

In der Informatik finden sowohl der rein mathematische Identitätsbegriff Verwendung
– ein Standardkonzept in den meisten Programmiersprachen –, als auch der sozialpsychologische Aspekt dieses Begriffs. In dieser Arbeit ist ein Identitätsbegriff der Betrachtungsgegenstand, der von beiden Seiten inspiriert ist. Der mathematische Identitätsbegriff bildet die Grundlage: Anhand eines Identifikators ist eine Identität eindeutig
bestimmbar. Dieser Identifikator wird angereichert durch eine beliebige Vielfalt an ergänzenden Attributen und ihre situationsbedingt eingeschränkte Verwendung.

\end{Zitat}

\vspace{0.3cm}


\begin{Zitat}[Avatar]

Diesem Vorbild der Newsgroup-Nutzer folgend unterstützen viele Foren-Systeme im ¨
World Wide Web von vornherein das Anlegen eines Identitäts-Profils. Neben diversen
Identitäts- und Nutzungsdaten kann hier oft ein \textbf{Bild als Stellvertreter und Wiedererkennungsmerkmal und emotionale Botschaft eingesetzt} werden, für welches der Begriff \textbf{"Avatar"} geprägt wurde. [...]

Einen Nachteil neben der mangelnden
Standardisierbarkeit und dem demzufolge bestehenden Mangel an automatischer Auswertbarkeit weist die Identitätsdarstellung im World Wide Web ebenfalls noch auf: Es
gibt keine praktikable Möglichkeit, zu bestimmen, wer auf diese Daten zugreifen kann
und wer nicht. Daten, die sich im World Wide Web befinden, sind im Allgemeinen für jeden einsehbar. 

\end{Zitat}

\vspace{0.3cm}


\begin{Zitat}[Web-Visitenkarte]

Im Rahmen des World Wide Webs hat sich eine Variante privater Homepages herausgebildet, die als Hauptmerkmal die Darstellung der eigenen Person aufweist. Die
starke Verbreitung dieser persönlichen Homepages oder Web-Visitenkarten ist [...] ein Indiz fur den starken Bedarf nach individueller Darstellung der eigenen Identität im virtuellen Raum.

\end{Zitat}

\vspace{0.3cm}


\begin{Zitat}[Skepsis hinsichtlich Datenerfassung]

Als Reaktion auf solche - in Abschnitt \ref{sec:einleitung_probleme_digitaler_identitaet} zitierte - oftmals ungefragt oder aber vom Anwender ungewünscht erfolgenden Datenerfassungsmethoden werden Gegenmaßnahmen eingesetzt: Bei der Dateneingabe werden \textbf{bewusst Falschangaben vorgenommen} oder Cookies und die Quellen von Web-Bugs werden blockiert. Dies geschieht insbesondere bei \textbf{Anbietern, bei denen die Daten nicht zwingend benötigt werden} oder deren Notwendigkeit zur Erfassung dem Nutzer nicht einsichtig ist. Insgesamt hat das Vorgehen vieler Anbieter zumindest bei kritischen Nutzern des Internets ein starkes Misstrauen gegenüber diesen Techniken geweckt. So finden die Gegenmaßnahmen – zum Beispiel die Blockade von Cookies – auch dann leicht statt, wenn sie unbegründet wäre und vielmehr ein echter Vorteil dadurch
ermöglicht wurde. Ein Beispiel für einen solchen Vorteil ist die Vereinfachung und Individualisierung von Informationsangeboten durch personalisierte Darstellung.

\end{Zitat}

\vspace{0.3cm}


\begin{Zitat}[Misstrauen vernichtet Value]

Insgesamt hat das Vorgehen vieler Anbieter zumindest bei kritischen Nutzern des Internets ein starkes Misstrauen gegenüber diesen Techniken geweckt. So finden die Gegenmaßnahmen – zum Beispiel die Blockade von Cookies – auch dann leicht statt, wenn sie unbegründet wäre und vielmehr ein echter Vorteil dadurch
ermöglicht wurde. Ein Beispiel für einen solchen Vorteil ist die Vereinfachung und Individualisierung von Informationsangeboten durch personalisierte Darstellung.

Weniger kritische Nutzer und solche, die sich ein differenziertes Bild über die Vor- und Nachteile dieser Techniken verschafft haben, erhalten für sie speziell zusammengestellte Inhalte, bekommen relevante Angebote unterbreitet oder haben die Möglichkeit, mit
anderen Nutzern mit ähnlichen Interessen oder mit entsprechend ähnlichen Fähigkeiten
in Kontakt zu treten. Dieser Nutzen gilt allerdings immer nur im eingeschränkten Bereich
innerhalb eines Angebotes.

\end{Zitat}

\vspace{0.3cm}


\begin{Zitat}[Synonymisierung]

[...] Dabei geht es nicht immer um eine der Wirklichkeit entsprechende Darstellung, sondern oftmals auch um \textbf{spielerische} oder die reale Identität \textbf{verschleiernde Pseudonyme} und Rollen-Repräsentationen. Manche Dienste – Online-Spiele beispielsweise – fordern dies sogar explizit ein, während andere – zum Beispiel Instant Messenger – dies problemlos ermöglichen. Wesentlich ist bei beiden die Kontinuität der Identifizierbarkeit. \textbf{Auch hier gilt die Beschränkung der Nutzbarkeit der Identitätsdaten auf einen Dienst}. Dies ist beim Beispiel des Online-Spiels wohl auch grundsätzlich sinnvoll – die erschaffene Spiel-Identität hat schließlich oftmals wenig mit der realen Identität gemein –, beim Instant Messaging aber schon \textbf{weniger gewünscht}.

\end{Zitat}

\vspace{0.3cm}


\begin{Zitat}[Identitätsdaten]

Identitätsdaten sind variantenreich und individuell und beschränken sich nicht auf einen
Kundendatensatz oder Anmeldedaten für Online-Dienste. Dies sind allerdings bisher die ¨
Hauptbereiche, in denen Identitätsdaten heute zum Einsatz kommen. Die Daten einer
Identität müssen aber alle Aspekte einer solchen abbilden können. Diese Vielzahl an persönlichen und auch personengebundenen Daten kann viele \textbf{Erleichterungen und Automatisierungen} mit sich bringen, birgt aber \textbf{auch Risiken} und \textbf{erschwert die Handhabung}. So ist bei einer Betrachtung von Konzepten zu einem Identitätsmanagement immer auch der
Blick zu richten auf die \textbf{Frage nach der Kontrolle der Daten} durch den Anwender, nach den \textbf{Verwendungsmöglichkeiten durch zur Nutzung dieser Daten} berechtigte Personen und nach Möglichkeiten des \textbf{unvorhergesehenen Missbrauchs}. Als noch entscheidenderes Kriterium für die \textbf{Akzeptanz durch die Anwender} ist aber sicherlich die Frage nach dem \textbf{Mehraufwand}: Kann ein Konzept, beziehungsweise seine Umsetzung in einer Anwendung gewisse Kriterien erfüllen, \textbf{dass es der Anwender als vorteilhaft und nicht als belastend wertet}? [...]

\end{Zitat}

\vspace{0.3cm}

Wichtige Aspekte aus dem letzten Zitat:

\begin{itemize}
  \item Pros
  \begin{itemize}
    \item Erleichterungen und Automatisierungen
    \item Kontrolle der Daten beim User
    \item Verwendungsmöglichkeiten durch Nutzung der Daten
  \end{itemize}
  \item Kontras
  \begin{itemize}
    \item Risiken
    \item erschwerte Handhabe
    \item Kontrolle der Daten beim Provider
    \item unvorhergesehener Missbrauch
  \end{itemize}
  \item Akzeptanz $\rightarrow$ Rechtfertigt der Nutzen den Mehraufwand?
\end{itemize}

\vspace{0.3cm}


\begin{Zitat}[E-Mail (auch als Identifier)]

Der E-Mail-Standard ist sicher kein Standard für ein Identiätsmanagement. Er sei hier
aber erwähnt, da es sich um den ältesten und am weitesten verbreiteten digitalen Standard handelt, der sich primär auf Individuen und somit Identitäten bezieht. [...]

Neben dem reinen Aspekt der gegenseitigen Erreichbarkeit weist eine E-Mail-Adresse
nur durch die – heute oft freie – Wahl der Kennung Individualität auf. Die meisten E-Mail-Systeme interpretieren ebenfalls zusätzliche Angaben des vollen Namens und der
Organisation. Neben der eher "seriösen" Variante, den eigenen Namen in voller oder teilweise abgekürzter Form zu verwenden, versuchen viele Personen eine bestimmte Geisteshaltung, Zuneigung oder Gruppenzugehörigkeit durch die Wahl der richtigen Kennung
auszudrücken. Genau hier ist aber auch schon die Grenze des E-Mail-Standards als Identitätskonzept erreicht: Weder lässt sich eine Namenswahl klar deuten – es sei denn, man ist mit dem weiteren Kontext des Anwenders vertraut –, noch lässt sich dieses durch automatische Prozesse sinnvoll auswerten. \textbf{Eine E-Mail-Adresse bleibt als Identitätskonzept das, was sie von Anfang an auch nur sein sollte: Ein eindeutiges  Identifizierungszeichen, um der damit verknüpften Identität Daten zukommen lassen zu können.}

\end{Zitat}

\vspace{0.3cm}


\begin{Zitat}[Workaround im Status quo]

Die wenigen bisherigen Standards und auch andere Konzepte ermöglichen nicht mehr
als die Speicherung der eigenen Kennung, Kontaktdaten oder Daten zum Bezahlen. [...]

Viele Anwender verwalten schon heute eine Reihe von Daten, die auf Basis einer
digitalen Identitäten-Infrastruktur zusammengefasst betrachtet werden könnten: Dazu
zählen solche Dinge wie das digitale Adressbuch, ein Kalender, die Lesezeichen, Verwaltungsdaten von Sammlungen (beispielsweise Fotos, Bücher oder Musik), ¨
Wunschlisten (beispielsweise bei Onlineshops), Lebensläufe, \textbf{Ergebnisstände von Computerspielen} und vieles mehr. All diese Daten liegen bisher in verschiedenen Strukturen vor, ohne Gesamtstruktur und \textbf{ohne, dass sich automatisierte Querbezüge bei Bedarf herstellen ließen, obwohl es alles Daten sind, die sich der Identität des Anwenders zuordnen ließen}. Diese fehlende Gesamtstruktur kann zur Folge unerwünschte \textbf{Redundanzen und auch Inkonsistenzen} mit sich bringen.

\end{Zitat}

\vspace{0.3cm}


\begin{Zitat}[Übergeordnete Struktur $\rightarrow$ "Querverweise"]

[...] Notwendig ist hierbei eine zusätzliche Struktur, die – ohne die bestehenden Daten und ihren eventuell aktuellen Bezug zueinander zu verändern – diesen Daten eine Gesamtstruktur verleiht: eine Identitätsdatengesamtstruktur. Auf diese Weise ließen sich Daten wie bisher speichern. Zusätzlich ließen sich mittels dieser Struktur aber auch Zusammenhänge herstellen, die unabhängig von der ursprünglichen Gebundenheit der Daten bestünden.

Derart ließen sich auch Identitätsdaten auf automatisierte Weise kontrolliert weitergeben. \textbf{"Kontrolliert" in diesem Zusammenhang bedeutet die Möglichkeit für den Anwender, selbst zu entscheiden, an wen er welche Daten wann und zu welchen Bedingungen übermittelt.} Dies ließe sich leicht bewerkstelligen, indem er Teile der Struktur mit entsprechenden Freigaben oder Einschränkungen versähe. Einen geeigneten Kommunikations- oder Kooperationsdienst vorausgesetzt, könnte der Anwender so bestimmten anderen Anwendern gezielt Daten über sich zukommen lassen, ohne sich selbst um die Zusammenstellung dieser Daten oder deren Übertragung kümmern zu müssen.

\end{Zitat}

\vspace{0.3cm}


\begin{Fazit}[Link zu WunderPass]

Insbesondere das letzte Zitat schreit förmlich nach WunderPass. Die "Struktur", von der dort abstrakt die Rede ist, heißt "WunderPass" (zumindest auf den Aspekt der "Querverweise" bezogen).

\end{Fazit}

\vspace{0.3cm}


\begin{Zitat}[Herausforderung]

Der Ansatz, einen Großteil der persönlichen Daten strukturell der Identität zuzuordnen,
bietet ein großes Potenzial für die Personalisierung und Individualisierung in Datennetzen. Es bestehen allerdings auch grunds¨atzliche Probleme, die ein solches Konzept überwinden muss: Wenn individuelle und umfassende Strukturen die Identitätsdaten in einen Gesamtzusammenhang bringen sollen, so müssen diese Strukturen erstellt werden. Wenn die Strukturen die Kommunikation unterstützen sollen, muss die individuelle ¨
Struktur auf Empfängerseite bekannt sein, um dort von Vorteil sein zu können.

\end{Zitat}

\vspace{0.3cm}


\begin{Fazit}[Link zu WunderPass]

Das letzte Zitat beschreibt nicht anderes als unser "Henne-Ei-Problem" hinsichtlich der Anbindung WunderPasses an Drittanbieter.

\end{Fazit}

\vspace{0.3cm}


\begin{Fazit}[Aufwand beim User]

Die Forderung nach oben zitierter Struktur (aka WunderPass) erfordert aber das Zutun des Users, welches mit nicht unerheblichem Aufwand einhergeht. Die Vorteile genannter Struktur werden dabei nicht zwangsläufig von Anfang an für den User ersichtlich sein. Das bedeutet im Umkehrschluss, er müsse zu einem Aufwand gedrängt werden, dessen Mehrwert sich für ihn kaum erschließt.

Es erfordert als eines Incentivierungs-Mechanismus (z. B. als Bestandteil etwaiger Token-Economics). Gleichzeitig ist es aus dem Blickwinkel des gesamten Ökosystems nicht zu rechtfertigen, der User werde ausschließlich aufgrund seiner Ignoranz - nämlich seine eigenen Vorteile aus obiger Struktur nicht erkennen zu können - Nutznießer von (vom Ökosystem gemeinschaftlich getragenen) Incentives. Daher wäre ein Hebel innerhalb der Token-Economics wünschenswert, der den User - ab Eintreten persönlicher Vorteile durch die "Querverweise" - die ausgeschütteten Incentives wieder zurückzahlen lässt.
 

\end{Fazit}

\vspace{0.3cm}