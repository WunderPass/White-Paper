% !TEX root = paper.tex

\subsection{Token-Economics (WPT)}
\label{sec:eco_wpt}

\todo{WIP}

\vspace{0.3cm}

\begin{Praemisse}[generelle Anforderungen an den Token]

\begin{itemize}
  \item Der Token soll ein \textbf{echter} Utility-Token sein. Er braucht also zwingend einen \textbf{intrinsischen Wert}.
\end{itemize}

\vspace{0.2cm}

Die Teilnehmer (User und Provider) müssen einen intrinsischen Vorteil am Besitz von Tokens innerhalb des Ökosystems erfahren. Sie müssen quasi "irgendwas mit dem Token machen können" - und zwar innerhalb des Ökosystems und nicht mittels "Verkaufs nach außen". Wenn man als Teilnehmer die Möglichkeit besitzt, Tokens für/durch irgendetwas zu erwerben, muss auch die Möglichkeit bestehen, diesen für irgendetwas (anderes; "nützliches") innerhalb des Ökosystems auszugeben. Idealerweise verhält sich unser Token zur Euro, wie sich der Euro zum nicht existenten "Weltall-Taler" verhält - also ohne Rechtfertigung zu besitzen, das Ökosystem verlassen zu müssen.

Falls die Schaffung einer solchen Ökonomie nicht gelingen sollte - weil z.B. die Service-Provider mehr Value generieren, als sie innerhalb des Ökosystems "konsumieren" können -  sollte diese Forderung zumindest für den Teilnehmer "User" sichergestellt werden. Denn der User partizipiert in seinem Dasein eher als Konsument innerhalb des Digital-Universums, als als Wertschöpfer, weshalb seinem intrinsischen Vorteil am Besitz von Tokens mit dem damit ermöglichten Konsum von digitalen Dienstleistungen Genüge getan sein sollte.

\vspace{0.3cm}

\begin{itemize}
  \item Der Token sollte natürlich auch einen \textbf{extrinsischen} Wert besitzen.
\end{itemize}

\vspace{0.2cm}

Nicht all zu laut (der Community ggü.) kommuniziert, wäre unsere ganze Unternehmung im Falle des Fehlen des extrinsischen Werts nichts anderes als ein kommunistischer Akt. Nur diese Beschaffenheit des Tokens liefert uns ein Monetarisierungs-Modell. Und auch deutet zudem vieles darauf hin, die Service-Provider-Teilnehmer kämen ohne einen extrinsischen Wert nicht aus.

\vspace{0.3cm}

\begin{itemize}
  \item Der Token soll \textbf{nicht inflationär} sein - also einen definierten Cap besitzen.
\end{itemize}

\vspace{0.2cm}

Mit voriger Forderung - laut der man "etwas mit dem Token innerhalb des Systems machen kann", verleiht die gegenständige Forderung das dieses "Etwas", was mittels des Tokens ermöglicht wird, einem gewissen Qualitätsanspruch genügen muss. Je größer die Qualität dieses besagten "Etwas" - also z. B. einer Dienstleistung, die mit ausschließlich mit dem Token bezahlt werden kann - ist, desto \textit{wertvoller} wird auch der Besitz des Tokens. Und damit auch sowohl sein intrinsischer als auch extrinsischer Wert. Schlichtweg deshalb, weil der Token und somit der mögliche Konsum besagter Dienstleistung gecappt ist.

\vspace{0.3cm}

\begin{itemize}
  \item \todo{Kreislauf}.
\end{itemize}

\vspace{0.2cm}

\todo{Kreislauf-Beschreibung}

\end{Praemisse}

\vspace{0.3cm}


\begin{Praemisse}[Daten haben einen Wert]

\vspace{0.2cm}

\todo{TODO: Evaluierung extrem schwierig. Folgende Aussagen/Antworten sind zu beweisen.}

\vspace{0.2cm}

\begin{itemize}
  \item Wer besitzt Daten/Informationen?
  \item Für wen sind diese Daten von "Wert" (Geld verdienen)?
  \item Wie kann der Wert der Daten maximiert werden? Wer profitiert im welchen Maße davon?
  \item Wer würde für diese Daten bezahlen und wie viel?
  \item Wie ist die (maximale) Wertschöpfung zu verteilen? Wer wird beteiligt? Wie wird die maximierende Rolle der Wertschöpfung belohnt?
  \item Wer trägt etwaige Risiken und in welchem Verhältnis?
  \item Wie ist das alles in die Token-Economics zu integrieren? 
\end{itemize}

\end{Praemisse}


\vspace{0.3cm}

\begin{Fazit}[unser Ökosystem generiert Value]

\begin{itemize}
  \item Wir schöpfen Mehrwert, indem wir Datenerfassung ermöglichen (die ja einen nachgewiesenen Value besitzen?
  \item Besitzer der Daten werden entlohnt
  \item Nutzer der Daten zahlen für Daten, generieren damit aber Value, der wiederum entlohnt wird.
  \item Am Ende haben alle Teilnehmer entweder Value generiert oder aber im Wert des values verkonsumiert
  \item Wir partizipieren am extrinsischen Wert des Tokens (Kurs-Entwicklung durch positive Wertschöpfung des gesamten Ökosystems).
  \item Incentives sind nötig, um das Henne-Ei-Problem zu lösen
  \item Incentives sollten nachträglich mit der dadurch geschaffenen Wertschöpfung verrechtet werden. 
\end{itemize}

\end{Fazit}

\vspace{0.3cm}

\todo{TODO}

\subsubsection{Einleitung}
\label{sec:wpt_einleitung}
\todo{TODO}

\subsubsection{Kreislauf}
\label{sec:wpt_kreislauf}
\todo{TODO}

\subsubsection{Token-Design}
\label{sec:wpt_design}
\todo{TODO}

\subsubsection{Incentivierung}
\label{sec:wpt_incent}
\todo{TODO}

\subsubsection{Milestones-Reward-Pool}
\label{sec:wpt_reward_pool}
\todo{TODO}

\subsubsection{WPT in Zahlen}
\label{sec:wpt_zahlen}
\todo{TODO}

\subsubsection{Fazit}
\label{sec:wpt_fazit}
\todo{TODO}